An orthogonal projection operator, projecting along a subspace~$V$ has
the properties that
\[ PV=0,\qquad x\perp V\Rightarrow Px=x \]
from which such properties as $P^2=P$ easily follow. The explicit form
of~$P$ is 
\begin{equation}
    P_V\colon x\rightarrow x-V(V^tV)\inv V^tx.
    \label{eq:simple-projection}\end{equation}

Just for reference, here is the basic theorem of projections.
\begin{theorem}\label{th:project-min}
The projected vector $P_Vx$ has minimum norm in the affine
space~$x+\nobreak V$.
\end{theorem}
\begin{proof}
Write $P_Vx=x-V\bar y$, where $\bar y=\nobreak (V^tV)\inv V^tx$,
and use the property that $x-V\bar y\perp V$.
Now, for any~$y$:
\begin{eqnarray*}
\|x-V(y-\bar y)\| &=&(x-V\bar y+Vy)^t(x-V\bar y+Vy)\\
&=&\|x-V\bar y\|^2+\|Vy\|^2 +2(Vy)^t(x-V\bar y)\\
&\geq&\|x-V\bar y\|^2
\end{eqnarray*}
\end{proof}

Note the similarities between the proof of this theorem and
that of theorem~\ref{th:petrov-galerkin}.

We can define a general projection along~$V$, and orthogonal to~$W$
under an inner product~$B$ as
\begin{equation}
    P_{V,W,B}\colon x\rightarrow x-V(W^tBV)\inv W^tBx,
    \label{eq:general-projection}\end{equation}
which satisfies
\[ P_{V,W,B}V=0,\qquad x\perp_BW\Rightarrow P_{V,W,B}x=x.\]
Trying to reproduce theorem~\ref{th:project-min} for this inner
product gives us that $\|x-V\bar y\|_B$,
where $\bar y=\nobreak(W^tBV)\inv W^tBx$,
is minimal if
\[ (Vy)^tB(x-V\bar y)+(Vy)^tB^t(x-V\bar y)\geq 0 \]
which is true (with equality to~zero) if $B$~is symmetric,
so that both terms are equal,
and~$V=\nobreak W$, so that we can use the $B$-orthogonality of
$x-\nobreak V\bar y$ to~$W$.

We repeat the observation of \eqref{eq:rn+1:by-projection} that
\[r_{n+1}h_{n+1n}=AM\inv r_n-R_n(S_n^tN\invt R_n)\inv S_n^tN AM\inv r_n,\]
which states that $r_{n+1}$ is obtained by projecting $AM\inv r_n$
along~$R_n$, and the result is $N$-orthogonal to~$S_n$.

\begin{question}
Generalise this for other inner products.
\end{question}
