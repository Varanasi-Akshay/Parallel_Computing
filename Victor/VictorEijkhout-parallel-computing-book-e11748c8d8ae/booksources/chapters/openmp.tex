% -*- latex -*-
%%%%%%%%%%%%%%%%%%%%%%%%%%%%%%%%%%%%%%%%%%%%%%%%%%%%%%%%%%%%%%%%
%%%%%%%%%%%%%%%%%%%%%%%%%%%%%%%%%%%%%%%%%%%%%%%%%%%%%%%%%%%%%%%%
%%%%
%%%% This text file is part of the source of 
%%%% `Parallel Programming in MPI and OpenMP'
%%%% by Victor Eijkhout, copyright 2012-6
%%%%
%%%% openmp.tex : topics that have not been rolled into their
%%%%              own chapter yet.
%%%%
%%%%%%%%%%%%%%%%%%%%%%%%%%%%%%%%%%%%%%%%%%%%%%%%%%%%%%%%%%%%%%%%
%%%%%%%%%%%%%%%%%%%%%%%%%%%%%%%%%%%%%%%%%%%%%%%%%%%%%%%%%%%%%%%%

\Level 0 {Runtime functions and internal control variables}
\label{ref:omp-environ}
\index{OpenMP!environment variables|(textbf}
\index{OpenMP!library routines|(textbf}
\index{Internal Control Variable (ICV)|(textbf}

OpenMP has a number of settings that can be set through \emph{environment variables},
and both queried and set through \emph{library routines}. These settings are called
\emph{\acfp{ICV}}: an OpenMP implementation behaves as if there is an internal variable
storing this setting.

The runtime functions are:
\begin{itemize}
\item \indextermttdef{omp_set_num_threads}
\item \indextermttdef{omp_get_num_threads}
\item \indextermttdef{omp_get_max_threads}
\item \indextermttdef{omp_get_thread_num}
\item \indextermttdef{omp_get_num_procs}
\item \indextermttdef{omp_in_parallel}
\item \indextermttdef{omp_set_dynamic}
\item \indextermttdef{omp_get_dynamic}
\item \indextermttdef{omp_set_nested}
\item \indextermttdef{omp_get_nested}
\item \indextermttdef{omp_get_wtime}
\item \indextermttdef{omp_get_wtick}
\item \indextermttdef{omp_set_schedule}
\item \indextermttdef{omp_get_schedule}
\item \indextermttdef{omp_set_max_active_levels}
\item \indextermttdef{omp_get_max_active_levels}
\item \indextermttdef{omp_get_thread_limit}
\item \indextermttdef{omp_get_level}
\item \indextermttdef{omp_get_active_level}
\item \indextermttdef{omp_get_ancestor_thread_num}
\item \indextermttdef{omp_get_team_size}
%\item \indextermttdef{omp_}
\end{itemize}

Here are the OpenMP \emph{environment variables}:
\begin{itemize}
\item \indextermttdef{OMP_CANCELLATION} Set whether cancellation is activated
\item \indextermtt{OMP_DISPLAY_ENV} Show OpenMP version and environment variables
\item \indextermttdef{OMP_DEFAULT_DEVICE} Set the device used in target regions
\item \indextermttdef{OMP_DYNAMIC} Dynamic adjustment of threads
\item \indextermttdef{OMP_MAX_ACTIVE_LEVELS} Set the maximum number of nested parallel
  regions
\item \indextermttdef{OMP_MAX_TASK_PRIORITY} Set the maximum task priority value
\item \indextermttdef{OMP_NESTED} Nested parallel regions
\item \indextermttdef{OMP_NUM_THREADS} Specifies the number of threads to use
\item \indextermttdef{OMP_PROC_BIND} Whether theads may be moved between CPUs
\item \indextermttdef{OMP_PLACES} Specifies on which CPUs the theads should be placed
\item \indextermttdef{OMP_STACKSIZE} Set default thread stack size
\item \indextermttdef{OMP_SCHEDULE} How threads are scheduled
\item \indextermttdef{OMP_THREAD_LIMIT} Set the maximum number of threads
\item \indextermttdef{OMP_WAIT_POLICY} How waiting threads are handled
\end{itemize}

There are 4 \acp{ICV} that behave as if each thread has its own copy of them.
The default is implementation-defined unless otherwise noted.
\begin{itemize}
  \item It may be possible to adjust dynamically the number of threads
    for a parallel region. Variable: \indextermtt{OMP_DYNAMIC};
    routines: \indextermtt{omp_set_dynamic},
    \indextermtt{omp_get_dynamic}.
  \item If a code contains \indextermsub{nested}{parallel regions},
    the inner regions may create new teams, or they may be executed by
    the single thread that encounters them. Variable:
    \indextermtt{OMP_NESTED}; routines \indextermtt{omp_set_nested},
    \indextermtt{omp_get_nested}. Allowed values are \n{TRUE} and
    \n{FALSE}; the default is false.
  \item The number of threads used for an encountered parallel region
    can be controlled. Variable: \indextermtt{OMP_NUM_THREADS};
    routines \indextermtt{omp_set_num_threads},
    \indextermtt{omp_get_max_threads}.
  \item The schedule for a parallel loop can be set. Variable:
    \indextermtt{OMP_SCHEDULE}; routines
    \indextermtt{omp_set_schedule}, \indextermtt{omp_get_schedule}.
\end{itemize}

Non-obvious syntax:
\begin{verbatim}
export OMP_SCHEDULE="static,100"
\end{verbatim}

Other settings:
\begin{itemize}
\item\indextermtt{omp_get_num_threads}: query the number of threads
  active at the current place in the code; this can be lower than what
  was set with \n{omp_set_num_threads}. For a meaningful answer, this
  should be done in a parallel region.
\item\indextermtt{omp_get_thread_num}
\item\indextermtt{omp_in_parallel}: test if you are in a parallel
  region (see for instance section~\ref{sec:parallelregion}).
\item\indextermtt{omp_get_num_procs}: query the physical number of cores available.
\end{itemize}

Other environment variables:
\begin{itemize}
\item \indextermtt{OMP_STACKSIZE} controls the amount of space that is
  allocated as per-thread \indexterm{stack}; the space for private
  variables.
\item \indextermtt{OMP_WAIT_POLICY} determines the behaviour of
  threads that wait, for instance for \indexterm{critical section}:
  \begin{itemize}
  \item \n{ACTIVE} puts the thread in a \indexterm{spin-lock}, where
    it actively checks whether it can continue;
  \item \n{PASSIVE} puts the thread to sleep until the \ac{OS} wakes
    it up.
  \end{itemize}
  The `active' strategy uses CPU while the thread is waiting; on the
  other hand, activating it after the wait is instantaneous. With the
  `passive' strategy, the thread does not use any CPU while waiting,
  but activating it again is expensive. Thus, the passive strategy
  only makes sense if threads will be waiting for a (relatively) long
  time.
\item \indextermtt{OMP_PROC_BIND} with values \n{TRUE} and \n{FALSE}
  can bind threads to a processor. On the one hand, doing so can
  minimize data movement; on the other hand, it may increase load
  imbalance.
\end{itemize}

\index{OpenMP!environment variables|)}
\index{OpenMP!library routines|)}
\index{Internal Control Variable (ICV)|)}

\Level 0 {Timing}
\label{sec:omp-timing}

OpenMP has a wall clock timer routine \indexcommand{omp_get_wtime}
\begin{verbatim}
double omp_get_wtime(void);
\end{verbatim}
The starting point is arbitrary and is different for each program run;
however, in one run it is identical for all threads.
This timer has a resolution given by \indexcommand{omp_get_wtick}.

\begin{exercise}
  Use the timing routines to demonstrate speedup from using
  multiple threads.
  \begin{itemize}
  \item Write a code segment that takes a measurable amount of time, that is,
    it should take a multiple of the tick time.
  \item Write a parallel loop and measure the speedup. You can for instance do this
\begin{verbatim}
for (int use_threads=1; use_threads<=nthreads; use_threads++) {
#pragma omp parallel for num_threads(use_threads)
    for (int i=0; i<nthreads; i++) {
        .....
    }
    if (use_threads==1)
      time1 = tend-tstart;
    else // compute speedup
\end{verbatim}
\item In order to prevent the compiler from optimizing your loop away, let
  the body compute a result and use a reduction to preserve these results.
  \end{itemize}
\end{exercise}

\Level 0 {Dependency analysis}
\index{data dependencies|(}
\index{flow dependency|see{data dependencies}}
\index{anti dependency|see{data dependencies}}
\index{output dependency|see{data dependencies}}

If two statements refer to the same data item,
we say that there is a \emph{data dependency} between
the statements. Such dependencies limit the extent to which
the execution of the statements can be  rearranged.
The study of this topic probably started in the 1960s,
when processors could execute statements \emph{out of order}\index{out-of-order execution}
to increase throughput. The re-ordering of statements
was limited by the fact that the execution
had to obey the \indexterm{program order} semantics:
the result had to be as if the statements were executed
strictly in the order in which they appear in the program.

These issues of statement ordering, and therefore of
data dependencies, arise in OpenMP in two main ways:
\begin{enumerate}
\item When a loop is parallelized, the iterations are no longer
  executed in their program order, so we have to check for dependencies.
\item The introduction of tasks also means that parts of a program
  can be executed in a different order from in which they appear
  in a sequential execution.
\end{enumerate}

The easiest case of dependency analysis is that of
detecting that loop iterations can be executed independently.
Iterations are of course independent if a data item
is read in two different iterations, but if the same
item is read in one iteration and written in another,
or written in two different iterations,
we need to do further analysis.

Analysis of \emph{data dependencies} can be performed
by a compiler, but compilers take, of necessity,
a conservative approach. This means that iterations
may be independent, but can not be recognized as such by
a compiler. Therefore, OpenMP shifts this responsibility
to the programmer; see for instance section~\ref{sec:omp-ordered}.

The three types of dependencies are:
\begin{itemize}
\item flow dependencies, or `read-after-write';
\item anti dependencies, or `write-after-read'; and
\item output dependencies, or `write-after-write'.
\end{itemize}

\begin{verbatim}
for (i) {
  y[i] = t;
  x[i+1] = y[i+1];
  t = x[i];
}
\end{verbatim}

\Level 1 {Flow dependencies}

Flow dependencies, or read-afer-write,
are not a problem if the read and write occur in the same
loop iteration:
\begin{verbatim}
for (i=0; i<N; i++) {
  x[i] = .... ;
  .... = ... x[i] ... ;
}
\end{verbatim}
On the other hand, if the read happens in a later iteration,
there is no simple way to parallelize the loop:
\begin{verbatim}
for (i=0; i<N; i++) {
  .... = ... x[i] ... ;
  x[i+1] = .... ;
}
\end{verbatim}
This usually requires rewriting the code.

\Level 1 {Anti dependencies}

The simplest case of write-after-read is a reduction:
\begin{verbatim}
for (i=0; i<N; i++) {
  t = t + .....
}
\end{verbatim}
This can be dealt with by explicit declaring the loop to be a reduction,
or to use any of the other strategies in section~\ref{sec:reduction}.

If the read and write are on an array the situation is more complicated.
The iterations in this fragment
\begin{verbatim}
for (i=0; i<N; i++) {
  x[i] = ... x[i+1] ... ;
}
\end{verbatim}
can not be executed in arbitrary order as such. However, conceptually there
is no dependency. We can solve this by introducing a temporary array:
\begin{verbatim}
for (i=0; i<N; i++)
  xtmp[i] = x[i];
for (i=0; i<N; i++) {
  x[i] = ... xtmp[i+1] ... ;
}
\end{verbatim}
This is an example of a transformation that a compiler is unlikely
to perform, since it can greatly affect the memory demands of the program.
Thus, this is left to the programmer.

\Level 1 {Output dependencies}

The case of write-after-write does not occur by itself:
if a variable is written twice in sequence without an intervening
read, the first write can be removed without changing the meaning of the program.
Thus, this case reduces to a flow dependency.

Other output dependencies can easily be removed. In the following code, \n{t}~can be
declared private, thereby removing the dependency.
\begin{verbatim}
for (i=0; i<N; i++) {
  t = f(i)
  s += t*t;
}
\end{verbatim}
If the final value of \n{t} is wanted, the \indexpragma{lastprivate} can be used.

\index{data dependencies|)}

\Level 0 {Thread safety}
\index{thread-safe|(}

With OpenMP it is relatively easy to take existing code and make
it parallel by introducing parallel sections. If you're careful
to declare the appropriate variables shared and private,
this may work fine. However, your code may include
calls to library routines that include a \indexterm{race condition};
such code is said not to be \emph{thread-safe}.

For example a routine
\begin{verbatim}
static int isave;
int next_one() {
 int i = isave;
 isave += 1;
 return i;
}

...
for ( .... ) {
  int ivalue = next_one();
}
\end{verbatim}
has a clear race condition, as the iterations of the loop
may get different \n{next_one} values, as they are supposed to,
or not. This can be solved by using an \indexpragma{critical}
pragma for the \n{next_one} call; another solution 
is to use an \indexpragma{threadprivate} declaration for \n{isave}.
This is for instance the right solution if  the \n{next_one}
routine implements a \indexterm{random number generator}.

\index{thread-safe|)}

\Level 0 {Performance and tuning}

%See the EPCC benchmark suite~\cite{epcc-ompbench}.

The performance of an OpenMP code can be influenced by the following.

\begin{description}
\item[Amdahl effects] Your code needs to have enough parts that are
  parallel (see~\HPSCref{sec:amdahl}). Sequential parts may be sped up
  by having them executed redundantly on each thread, since that keeps
  data locally.
\item[Dynamism] Creating a thread team takes time. In practice, a team
  is not created and deleted for each parallel region, but creating
  teams of different sizes, or recursize thread creation, may
  introduce overhead.
\item[Load imbalance] Even if your program is parallel, you need to
  worry about load balance. In the case of a parallel loop you can set
  the \indexclause{schedule} clause to \n{dynamic}, which evens out
  the work, but may cause increased communication.
\item[Communication] Cache coherence causes communication. Threads
  should, as much as possible, refer to their own data.
  \begin{itemize}
  \item Threads are likely to read from each other's data. That is
    largely unavoidable.
  \item Threads writing to each other's data should be avoided: it may
    require synchronization, and it causes coherence traffic.
  \item If threads can migrate, data that was local at one time is no
    longer local after migration.
  \item Reading data from one socket that was allocated on another
    socket is inefficient; see section~\ref{sec:first-touch}.
  \end{itemize}
\item[Affinity] Both data and execution threads can be bound to a
  specific locale to some extent. Using local data is more efficient
  than remote data, so you want to use local data, and minimize the extent to which data
  or execution can move.
  \begin{itemize}
  \item See the above points about phenomena that cause communication.
  \item Section~\ref{omp:threadbind} describes how you can specify the
    binding of threads to places. There can, but does not need, to be
    an effect on affinity. For instance, if an OpenMP thread can
    migrate between hardware threads, cached data will stay local.
    Leaving an OpenMP thread completely free to migrate can be
    advantageous for load balancing, but you should only do that if
    data affinity is of lesser importance.
  \item Static loop schedules have a higher chance of using data that
    has affinity with the place of execution, but they are worse for
    load balancing. On the other hand, the \indexclause{nowait} clause
    can aleviate some of the problems with static loop schedules.
  \end{itemize}
\item[Binding] You can choose to put OpenMP threads close together or
  to spread them apart. Having them close together makes sense if they
  use lots of shared data. Spreading them apart may increase
  bandwidth. (See the examples in section~\ref{sec:omp:bindeffect}.)
\item[Synchronization] Barriers are a form of synchronization. They
  are expensive by themselves, and they expose load
  imbalance. Implicit barriers happen at the end of worksharing
  constructs; they can be removed with \n{nowait}.

  Critical sections imply a loss of parallelism, but they are also
  slow as they are realized through \indexterm{operating system}
  functions. These are often quite costly, taking many thousands of
  cycles.  Critical sections should be used only if the parallel work
  far outweighs it.
\end{description}

\Level 0 {Accelerators}

In OpenMP 4.0 there is support for offloading work to an
\emph{accelerator}\index{OpenMP!accelerator support in}
or
\emph{co-processor}\index{OpenMP!co-processor support in}:
\begin{verbatim}
#pragma omp target [clauses]
\end{verbatim}
with clauses such as
\begin{itemize}
\item \n{data}: place data
\item \n{update}: make data consistent between host and device
\end{itemize}

\endinput

\Level 0 {Idioms}

\Level 1 {While loops}

\verbatimsnippet{linkedlist}

