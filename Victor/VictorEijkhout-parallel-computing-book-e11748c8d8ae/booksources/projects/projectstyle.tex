Here are some guidelines for how to submit assignments and projects.
As a general rule, consider programming as an experimental science,
and your writeup as a report on some tests you have done: explain
the problem you're addressing, your strategy, your results.

\paragraph*{\bf Structure of your writeup}

Most of the exercises in this book test whether you are able to code the solution to a certain
problem. That does not mean that turning in the code is sufficient, nor code plus sample output.
Turn in a writeup in pdf form that was generated from a text processing program such as Word or (preferably)
\LaTeX\ (for a tutorial, see~\HPSCref{tut:latex}). Your writeup should have 
\begin{itemize}
\item The relevant fragments of your code,
\item an explanation of your algorithms or solution strategy,
\item a discussion of what you observed,
\item graphs of runtimes and TAU plots; see~\ref{tut:tau}.
\end{itemize}

\paragraph*{Observe, measure, hypothesize, deduce}

In most applications of computing machinery we care about the efficiency with which
we find the solution. Thus, make sure that you do measurements. In general, make
observations that allow you to judge whether your program behaves the way you
would expect it to.

Quite often your program will display unexpected behaviour. It is important to observe
this, and hypothesize what the reason might be for your observed behaviour.

\paragraph*{Including code}

If you include code samples in your writeup, make sure they look good. For starters,
use a mono-spaced font. In \LaTeX, you can use the \n{verbatim} environment or the 
\n{verbatiminput} command. In that section option the source is included automatically,
rather than cut and pasted. This is to be preferred, since your writeup will
stay current after you edit the source file.

Including whole source files makes for a long and boring writeup. The code samples in this
book were generated as follows. In the source files, the relevant snippet was marked as
\begin{verbatim}
... boring stuff
#pragma samplex
  .. interesting! ..
#pragma end
... more boring stuff
\end{verbatim}
The files were then processed with the following command line (actually, included
in a makefile, which requires doubling the dollar signs):
\begin{verbatim}
for f in *.{c,cxx,h} ; do
  cat $x | awk 'BEGIN {f=0}
                   /#pragma end/ {f=0}
                   f==1 {print $0 > file}
                   /pragma/ {f=1; file=$2 }
                '
done
\end{verbatim}
which gives (in this example) a file \n{samplex}. Other solutions are of course possible.

\paragraph*{\bf Code formatting}

Code without proper indentation is very hard to read. Fortunately,
most editors have some knowledge of the syntax of the most popular
languages. The \indexterm{emacs} editor will, most of the time,
automatically activate the appropriate mode based on the file
extension. If this does not happen, you can activate a mode by \n{ESC
  x fortran-mode} et cetera, or by putting the string
\verb+-*- fortran -*-+ in a comment on the first line of your file.

The \indexterm{vi} editor also has syntax support: use the commands
\n{:synxtax on} to get syntax colouring, and \n{:set cindent} to get
automatic indentation while you're entering text. Some of the more
common questions are addressed in
\url{http://stackoverflow.com/questions/97694/auto-indent-spaces-with-c-in-vim}.

\paragraph*{\bf Running your code}

A single run doesn't prove anything. For a good report, you need to
run your code for more than one input dataset (if available) and in
more than one processor configuration. When you choose problem sizes,
be aware that an average processor can do a billion operations per
second: you need to make your problem large enough for the timings to
rise above the level of random variations and startup phenomena.

When you run a code in parallel, beware that on clusters the behaviour
of a parallel code will always be different between one node and
multiple nodes.  On a single node the MPI implementation is likely
optimized to use the shared memory. This means that results obtained
from a single node run will be unrepresentative. In fact, in timing
and scaling tests you will often see a drop in (relative) performance
going from one node to two.  Therefore you need to run your code in a
variety of scenarios, using more than one node.

\paragraph*{Reporting scaling}

If you do a scaling analysis, a graph reporting runtimes should not
have a linear time axis: a logarithmic graph is much easier to
read. A~speedup graph can also be informative.

Some algorithms are mathematically equivalent in their sequential and
parallel versions. Others, such as iterative processes, can take more
operations in parallel than sequentially, for instance because the
number of iterations goes up. In this case, report both the speedup of
a single iteration, and the total improvement of running the full
algorithm in parallel.

\paragraph*{Repository organization}

If you submit your work through a repository, make sure you organize
your submissions in subdirectories, and that you give a clear name to
all files. Object files and binaries should not be in a repository
since they are dependent on hardware and things like compilers.
