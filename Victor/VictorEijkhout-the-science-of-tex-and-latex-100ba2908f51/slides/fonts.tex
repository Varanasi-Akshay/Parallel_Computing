\documentclass{beamer}

\usepackage{beamerthemevictor,comment,verbatim,graphicx,amssymb}

\newcommand{\TbT}{\TeX\ by Topic}
\newcommand{\Metafont}{\textsc{Metafont}}
\let\metafont\Metafont
\newcommand\ascii{{\sc Ascii}}
\newcommand\ebcdic{{\sc Ebcdic}}
\newcommand\gnuplot{\protect\n{gnuplot}}
\def\mod{\mathbin{\mathrm{mod}}}
\def\eqref#1{equation~(\ref{#1})}
\def\Eqref#1{Equation~(\ref{#1})}
\def\inv{^{-1}}
\def\endproof{\unskip\penalty10000 \hskip10pt plus 1fill$\bullet$\par}

\def\etal{\textit{et al.}}
\def\lex{{\it lex}}
\def\yacc{{\it yacc}}
\def\make{{\it make}}
\def\web{{\textsc{Web}}}
\def\n{\bgroup
  \catcode`\#=12 \catcode`\_=12 \catcode`\^=12
  \catcode`\&=12 
  \tt \let\next=}
\newcommand{\cs}[1]{{\tt\char`\\#1}}
{\catcode`\.=13
 \gdef.{${}_\bullet$}
}
\def\parserule#1{%
    \ifmmode \n{#1}\,\rightarrow\relax
    \else $\n{#1}\,\rightarrow{}$\nobreak
    \fi
    \hbox\bgroup\catcode`\.=13 \tt\let\next=}

\newenvironment{remark}%
    {\medskip\noindent{\bf Remark.}}{}
\usepackage{bnf}
\input bnf.env

\generalcomment{inputwithcode}
  {\begingroup\def\ProcessCutFile{}}
  {\verbatiminput{\CommentCutFile}
   \endgroup
   \input{\CommentCutFile}
  }
\generalcomment{examplewithcode}
  {\begingroup
     \def\ProcessCutFile{}\def\CommentCutFile{example.tex}}
  {\verbatiminput{\CommentCutFile}
   Output:
   \begin{quote}
     \begin{minipage}[t]{3in}
        \everypar{}
        \input{\CommentCutFile} 
     \end{minipage}
   \end{quote}
   \endgroup
  }
\specialcomment{ttexamplewithcode}
  {\begingroup\def\ProcessCutFile{}}
  {\verbatiminput{\CommentCutFile}
   \endgroup
   Output:
   \begin{quote}\begin{ttfamily} \input{\CommentCutFile} 
     \end{ttfamily}\end{quote}
  }
\generalcomment{mathexamplewithcode}
  {\begingroup\def\ProcessCutFile{}}
  {\verbatiminput{\CommentCutFile}
   Output:
   \begin{equation} \input{\CommentCutFile} \end{equation}
   \endgroup
  }

\def\chaptertitle{\csname\chaptername title\endcsname}
\def\chaptershorttitle{\csname\chaptername shorttitle\endcsname}

\def\lambdatitle{Lambda calculus in \TeX}
\def\lambdashorttitle{Lambda calculus}

\def\latextitle{Yet another \LaTeX\ tutorial}
\def\latexshorttitle{\LaTeX}

\def\lextitle{A \lex\ tutorial}
\def\lexshorttitle{\lex}

\def\yacctitle{A \yacc\ tutorial}
\def\yaccshorttitle{\yacc}

\expandafter\def\csname tex1title\endcsname{\TeX\ -- macro programming}
\expandafter\let\csname tex1shorttitle\expandafter\endcsname
    \csname tex1title\endcsname
\expandafter\def\csname tex2title\endcsname{\TeX\ -- visuals}
\expandafter\let\csname tex2shorttitle\expandafter\endcsname
    \csname tex2title\endcsname

\def\parsingtitle{An introduction to parsing}
\def\parsingshorttitle{Parsing}

\def\hashingtitle{Hashing}
\let\hashingshorttitle\hashingtitle

\def\beziertitle{Curve approximation by splines}
\def\beziershorttitle{Splines}

\def\dynamictitle{Dynamic programming}
\let\dynamicshorttitle\dynamictitle

\def\paragraphtitle{Line breaking}
\let\paragraphshorttitle\paragraphtitle

\def\pagetitle{Page breaking in \TeX}
\def\pageshorttitle{Page breaking}

\def\completenesstitle{Introduction to NP-Completeness}
\def\completenesshortstitle{NP-Completeness}

\def\rastertitle{Introduction to Raster Graphics}
\def\rastershortstitle{Raster Graphics}

\def\pythontitle{Really quick Python tutorial}
\def\pythonshortstitle{Python}

\def\encodingtitle{Character encoding}
\let\encodingshorttitle\encodingtitle

\def\rastertitle{Raster Graphics}
\let\rastershorttitle\rastertitle

\def\gnuplottitle{Gnuplot}
\let\gnuplotshorttitle\gnuplottitle

\def\fontstitle{Fonts and font files}
\def\fontsshorttitle{Fonts}

\def\softwaretitle{\TeX\ and Software Engineering}
\def\softwareshorttitle{Software Engineering}

\renewcommand{\beamervictortitle}{CS-594 Eijkhout, Fall 2004}

\newcommand{\sectionframe}[1]%
  {\section{#1}
   \frame{\begin{center}\color{blue}\Large #1\end{center}}
  }
\newcommand{\subsectionframe}[1]%
  {\subsection{#1}
   \frame{\begin{center}\color{blue}\Large #1\end{center}}
  }

\input idxmacs

\begin{document}

\title{Fonts}
\author{Victor Eijkhout}
\date{Notes for CS 594 -- Fall 2004}

\frame{\titlepage}

\section{Introduction}

\frame{
  \frametitle{Terminology}
\begin{itemize}
\item Typeface: collection of size, weight, shape
\item Font: single instance
\item<2-> Not much consistency these days
\end{itemize}
}

\frame{
\pgfimage[height=3in]{Times0}
}

\frame{
\pgfimage[height=1.5in]{Plantin}
}

\frame{
\pgfimage[height=2in]{Palatino}
}

\frame{
\pgfimage[height=2.5in]{Johnston}
}

\frame{
\pgfimage[height=2.5in]{Gill}
}

\frame{
\pgfimage[height=2.2in]{Melior}
}

\frame{
\pgfimage[height=2.2in]{Optima}
}

\frame{
\pgfimage[height=2.2in]{Frutiger}
}

\sectionframe{Fonts do not contain characters}

\subsection{Terminology}

\frame{
  \frametitle{Character versus glyph}
\begin{itemize}
\item Abstract character: `Roman lowercase~a'
\item Glyphs: rendering
\pgfimage[height=.75in]{a-glyphs}
\end{itemize}
}

\frame{
  \frametitle{Abstract character}
\begin{itemize}
\item Definition:
\begin{quotation}
abstract character: a unit of information used for the organization,
control, or representation of textual data.
\end{quotation}
\item Consequence: no simple one-to-one mapping between character and
  glyph
\end{itemize}
}

\subsection{Tricky issues}

\frame{
  \frametitle{Ligatures}
\begin{itemize}
\item Ligature as `prettyfying' text: `\textrm{fi}'~instead
  of~`\textrm{f{}i}' \pgfimage[height=.75in]{fi-ligature}
\item One glyph for two characters
\item<2-> In Danish, `\ae' is a single character
\end{itemize}
}

\frame{
  \frametitle{Split characters}
\begin{itemize}
\item In Tamil:

\pgfimage[height=.75in]{Tamil-split}

to be placed around other characters
\item<2-> Now allow for ligatures\dots
\end{itemize}
}

\frame{
  \frametitle{Accents and diacritics}
\begin{itemize}
\item Is an accented letter one character? One glyph?
\pgfimage[height=1in]{o-accents}
\end{itemize}
}

\subsection{Design sizes}

\frame{
  \frametitle{Scaling vs design}
\begin{itemize}
\item With lead, scaling is not possible: different design at
  different sizes
\pgfimage[height=.25in]{design-size}
\item Adobe Multiple Masters \& Optical Masters, Apple TrueType GX \&
  Apple Advanced Typography
\pgfimage[height=.75in]{optical-masters}
\end{itemize}
}

\sectionframe{Font technologies}

\subsection{Type 1}

\frame{
  \frametitle{Type 1 essentials}
\begin{itemize}
\item Developed by Adobe
\item Adopted by Apple: Laserwriter
\item Two files: outline and font information (hinting)
\item (often three files: separate bitmaps for screen display)
\item Hinting is patented
\item Limit of 256 (active) characters 
\end{itemize}
}

\subsection{TrueType}

\frame{
  \frametitle{TrueType essentials}
\begin{itemize}
\item Developed by Apple and Microsoft
\item Quadratic splines instead of cubic
\item Large number of characters (Chinese, Japanese)
\item also patented hinting
\end{itemize}
}

\subsection{FreeType}

\frame{
  \frametitle{FreeType}
\begin{itemize}
\item Open Source
\item<2-> $\Rightarrow$ can not use hinting
\item<3-> $\Rightarrow$ automatic hinting
\end{itemize}
}

\subsection{OpenType}

\frame{
  \frametitle{OpenType}
\begin{itemize}
\item Adobe and Microsoft
\item Single file, cross-platform
\item Optical Masters
\end{itemize}
}

\sectionframe{Fonts in \TeX\ / \LaTeX}
\subsection{\TeX}

\frame{
  \frametitle{Tfm files}
\begin{itemize}
\item \TeX\ processor only needs metric information
\item \n{tfm}: `\TeX\ font metrics'
\item contains spacing information, character metrics, ligature and kerning
\end{itemize}
}

\frame[containsverbatim]{
  \frametitle{Font dimensions}
\begin{itemize}
\item \verb+\fontdimen1\tenrm=5pt+ et cetera
\item Natural space, stretch / shrink
\item Quad (width of `M'), x-height
\item Slant per point
\end{itemize}
}

\frame{
  \frametitle{Character metrics}
\begin{itemize}
\item Height, depth, width of each character
\item<2-> Italic correction
\begin{quote}\begin{rmfamily}
`\TeX\ \textup{has\dots}' versus `\TeX\/ \textup{has\dots}
  \end{rmfamily}
\end{quote}
\end{itemize}
}

\frame{
  \frametitle{Ligatures and kerning}
\begin{itemize}
\item Ligature:
\begin{quote}\begin{rmfamily}
`f{}i' or `f{}l' versus `fi' or~`fl'
  \end{rmfamily}
\end{quote}
\item Kerning
\begin{quote}\begin{rmfamily}`Von' versus `\hbox{V}\hbox{on}'
  \end{rmfamily}
\end{quote}
\end{itemize}
}

\subsectionframe{\LaTeX\ font selection}
\frame[containsverbatim]{
  \frametitle{}
\begin{itemize}
\item Load fonts with \verb+\usepackage[times]+
\item can be combination of typefaces: Times, and Helvetica for sans serif
\end{itemize}
}

\frame{
  \frametitle{Orthogonal combinations}
\begin{itemize}
\item Families: roman, sans serif, typewriter type
\item Font series: medium weight, bold
\item Shape: upright, slanted/italic, small caps
\end{itemize}
}

\end{document}
\frame{
  \frametitle{}
\begin{itemize}
\item 
\end{itemize}
}

\frame{
  \frametitle{}
\begin{itemize}
\item 
\end{itemize}
}

\frame{
  \frametitle{}
\begin{itemize}
\item 
\end{itemize}
}

\frame{
  \frametitle{}
\begin{itemize}
\item 
\end{itemize}
}

\frame{
  \frametitle{}
\begin{itemize}
\item 
\end{itemize}
}

\end{document}

