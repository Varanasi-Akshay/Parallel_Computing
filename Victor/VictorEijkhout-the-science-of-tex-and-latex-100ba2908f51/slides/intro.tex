\documentclass{beamer}

\usepackage{beamerthemevictor}

\input tutmacs

\begin{document}

\title{Introduction}
\author{Victor Eijkhout}
\date{Notes for CS 594 -- Fall 2004}

\frame{\titlepage}

\section{\TeX\ and \LaTeX}
\subsection{Markup}

\frame[containsverbatim]{
\frametitle{Ancient typesetting systems}
\begin{itemize}
\item Input was compiled to printable form; not `wysiwyg'
\item Sequential processing of text input file
\item Commands for font choice and other layout
\item Macros with replacement text
\begin{verbatim}
$ADAM$ --> From our correspondent in Amsterdam
\end{verbatim}
\end{itemize}
}

\frame[containsverbatim]{
\frametitle{Computer typesetting systems}
\begin{itemize}
\item Same idea: document compilation, macros for text replacement and
  formatting
\begin{verbatim}
\TeX => T\kern -.1667em\lower .5ex\hbox {E}\kern -.125emX
\end{verbatim}
gives `\TeX'.
\item Macro replacement language
\item Turing equivalent
\end{itemize}
}

\frame[containsverbatim]{
\frametitle{Logical markup}
\begin{itemize}
\item Use macro names to indicate structure:
\begin{verbatim}
\begin{theorem}
\TeX\ is pretty cool.
\end{theorem}
\begin{proof}
See for yourself
\end{proof}
\end{verbatim}
\begin{quote}
\renewenvironment{proof}{\emph{Proof:}\enskip\rmfamily}{%
    \unskip\nobreak\kern2em\nobreak\hfill$\bullet$\par\bigskip}
\begin{theorem}
\TeX\ is pretty cool.
\end{theorem}
\begin{proof}
See for yourself
\end{proof}
\end{quote}
\end{itemize}
}

\frame[containsverbatim]{
\frametitle{Logical markup (2)}
\begin{itemize}
\item Layout is determined by style declaration
\begin{verbatim}
\documentclass{article}
\documentclass{IEEEproc}
\documentclass[twoside,a4paper]{artikel1}
\end{verbatim}
\end{itemize}
}

\subsection{\LaTeX}

\frame{
\frametitle{The name of the games}
\begin{itemize}
\item `\TeX' is ``tek'', `\LaTeX' is ``lay-tek'' or ``lah-tek''
\item \TeX\ is the basic system
\item \LaTeX\ is a macro package on top of it
\end{itemize}
}

\frame[containsverbatim]{
\frametitle{Aims of \LaTeX}
\begin{itemize}
\item Foremost: scientific documents
\item But also classes for letters, vitae, plays
\item Excellent math typesetting
\item Customizable, extendable
\item Not all that great for fancy layouts
\end{itemize}
}

\frame[containsverbatim]{
\frametitle{\LaTeX\ styles}
\begin{itemize}
\item Commands for document structuring
\begin{verbatim}
\section{Introduction}
\subsection{Prior research}
\end{verbatim}
\item Tools for making new structure constructs
\begin{verbatim}
\newtheorem{corollary}
\end{verbatim}
\item Lower level tools
\end{itemize}
}

\frame[containsverbatim]{
\frametitle{Math typesetting}
\begin{itemize}
\item Sophisticated algorithms
\item Math fonts with many parameters
\[
    \sqrt[3]{\frac{B}{1-A^2_{j_1,j_2}}}+\cdots+1=
    \int_1^\infty\widehat{\sin t}\mathrm{d}t
\]
\item Large number of weird symbols
\end{itemize}
}

\frame[containsverbatim]{
\frametitle{You will learn}
\begin{itemize}
\item \LaTeX\ commands for everyday use
\item Ways of customizing \LaTeX
\end{itemize}
}

\subsection{\TeX}

\frame[containsverbatim]{
\frametitle{Short history of \TeX}
\begin{itemize}
\item Thought up by Donald Knuth in 1978
\item as a summer project for his students
\item First real implementation in 1981
\item \TeX 2 in 1985, revision \TeX 3 in 1991, now frozen
\item Omega, pdftex
\end{itemize}
}

\frame[containsverbatim]{
\frametitle{Features of the \TeX\ language}
\begin{itemize}
\item Macro language
\begin{verbatim}
\def\theorem{\newline \bold Theorem \theoremcounter}
\theorem This is good
\end{verbatim}
\item Dynamically changable syntax
\item Many low-level constructs
\end{itemize}
}

\frame[containsverbatim]{
\frametitle{You will learn}
\begin{itemize}
\item Boxes, glue, paragraph parameters
\item Fancy macro programming
\end{itemize}
}

\section{Languages and parsing}

\frame[containsverbatim]{
\frametitle{Lexical analysis}
\begin{itemize}
\item Recognize words, numbers and such
\item Used as basic blocks for grammar
\item Finite State Automaton usually sufficient
\end{itemize}
}

\frame[containsverbatim]{
\frametitle{Syntactical analysis}
\begin{itemize}
\item Recognize statements and constructs
\item Translate `meaning' into internal representation
\item Pushdown Automaton usually sufficient
\end{itemize}
}

\frame[containsverbatim]{
\frametitle{You will learn}
\begin{itemize}
\item Recap of automata theory (FSA, PDA)
\item Applications of automata in programming language parsing
\item \lex\ and \yacc\ unix tools
\item Hashing
\end{itemize}
}

\section{Dynamic programming}

\frame[containsverbatim]{
\frametitle{Paragraph breaking}
\begin{itemize}
\item For right-justified paragraph:
\item Compress some lines, stretch others, use hyphenation
\item Aim for even `colour', avoid consecutive hyphens, rivers, et cetera
\item With $n$ words, $2^n$ breakpoints: efficient algorithm needed
\end{itemize}
}

\frame[containsverbatim]{
\frametitle{Naive `first fit' breaking}
\pgfimage{firstfit-pic}
}

\frame[containsverbatim]{
\frametitle{Sophisticated line breaking}
\pgfimage{knuthfit-pic}
}

\frame[containsverbatim]{
\frametitle{\TeX's paragraph algorithm}
\begin{itemize}
\item Dynamic programming
\item Small number of possibilities considered
\item Fast running time
\end{itemize}
}

\frame[containsverbatim]{
\frametitle{You will learn}
\begin{itemize}
\item Dynamic programming
\item NP-completeness
\item Python
\end{itemize}
}

\section{Font and graphics matters}

\frame[containsverbatim]{
\frametitle{Metafont}
\begin{itemize}
\item Knuth also wrote a program to design fonts with: Metafont
\item Based on splines
\end{itemize}
}

\frame{
\pgfimage{spline-pic}
}

\frame[containsverbatim]{
\frametitle{Raster graphics}
\pgfimage{discretization-pic}
}

\frame[containsverbatim]{
\frametitle{You will learn}
\begin{itemize}
\item Interpolation theory
\item Splines
\item Issues in raster graphics
\end{itemize}
}

\section{Lambda Calculus}

\frame[containsverbatim]{
\frametitle{\TeX's expansion mechanism}
\begin{itemize}
\item \TeX\ commands are of two kinds: expansion and execution
\item The expansion mechanism is strong enough to implement lambda calculus
\end{itemize}
}

\frame[containsverbatim]{
\frametitle{You will learn}
\begin{itemize}
\item Some foundations of mathematics
\end{itemize}
}

\section{Software engineering}
\subsection{Character encoding}

\frame[containsverbatim]{
\frametitle{What's in an input file}
\begin{itemize}
\item Plain Ascii?
\item The problem with funny languages
\end{itemize}
}

\frame[containsverbatim]{
\frametitle{You will learn}
\begin{itemize}
\item A history of character encodings
\item Unicode
\item Font organization
\end{itemize}
}

\subsection{Literate programming}

\frame[containsverbatim]{
\frametitle{Yet another Knuth product}
\begin{itemize}
\item The \web\ system for literate programming
\item Write code and documentation together
\end{itemize}
}

\frame[containsverbatim]{
\frametitle{Source pretty printing}
\pgfimage[height=3in]{literate-pic}
}

\frame[containsverbatim]{
\frametitle{You will learn}
\begin{itemize}
\item Literate programming
\item \web\ and \n{noweb}
\end{itemize}
}

\subsection{Code development and testing}

\frame[containsverbatim]{
\frametitle{You will learn}
\begin{itemize}
\item History of \TeX
\item Knuth's notions of development
\item The `torture test' idea
\item Competing notions
\end{itemize}
}

\end{document}
