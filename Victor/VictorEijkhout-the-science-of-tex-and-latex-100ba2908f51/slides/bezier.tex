\documentclass{beamer}

\usepackage{beamerthemevictor,comment,verbatim,graphicx,amssymb}

\newcommand{\TbT}{\TeX\ by Topic}
\newcommand{\Metafont}{\textsc{Metafont}}
\let\metafont\Metafont
\newcommand\ascii{{\sc Ascii}}
\newcommand\ebcdic{{\sc Ebcdic}}
\newcommand\gnuplot{\protect\n{gnuplot}}
\def\mod{\mathbin{\mathrm{mod}}}
\def\eqref#1{equation~(\ref{#1})}
\def\Eqref#1{Equation~(\ref{#1})}
\def\inv{^{-1}}
\def\endproof{\unskip\penalty10000 \hskip10pt plus 1fill$\bullet$\par}

\def\etal{\textit{et al.}}
\def\lex{{\it lex}}
\def\yacc{{\it yacc}}
\def\make{{\it make}}
\def\web{{\textsc{Web}}}
\def\n{\bgroup
  \catcode`\#=12 \catcode`\_=12 \catcode`\^=12
  \catcode`\&=12 
  \tt \let\next=}
\newcommand{\cs}[1]{{\tt\char`\\#1}}
{\catcode`\.=13
 \gdef.{${}_\bullet$}
}
\def\parserule#1{%
    \ifmmode \n{#1}\,\rightarrow\relax
    \else $\n{#1}\,\rightarrow{}$\nobreak
    \fi
    \hbox\bgroup\catcode`\.=13 \tt\let\next=}

\newenvironment{remark}%
    {\medskip\noindent{\bf Remark.}}{}
\usepackage{bnf}
\input bnf.env

\generalcomment{inputwithcode}
  {\begingroup\def\ProcessCutFile{}}
  {\verbatiminput{\CommentCutFile}
   \endgroup
   \input{\CommentCutFile}
  }
\generalcomment{examplewithcode}
  {\begingroup
     \def\ProcessCutFile{}\def\CommentCutFile{example.tex}}
  {\verbatiminput{\CommentCutFile}
   Output:
   \begin{quote}
     \begin{minipage}[t]{3in}
        \everypar{}
        \input{\CommentCutFile} 
     \end{minipage}
   \end{quote}
   \endgroup
  }
\specialcomment{ttexamplewithcode}
  {\begingroup\def\ProcessCutFile{}}
  {\verbatiminput{\CommentCutFile}
   \endgroup
   Output:
   \begin{quote}\begin{ttfamily} \input{\CommentCutFile} 
     \end{ttfamily}\end{quote}
  }
\generalcomment{mathexamplewithcode}
  {\begingroup\def\ProcessCutFile{}}
  {\verbatiminput{\CommentCutFile}
   Output:
   \begin{equation} \input{\CommentCutFile} \end{equation}
   \endgroup
  }

\def\chaptertitle{\csname\chaptername title\endcsname}
\def\chaptershorttitle{\csname\chaptername shorttitle\endcsname}

\def\lambdatitle{Lambda calculus in \TeX}
\def\lambdashorttitle{Lambda calculus}

\def\latextitle{Yet another \LaTeX\ tutorial}
\def\latexshorttitle{\LaTeX}

\def\lextitle{A \lex\ tutorial}
\def\lexshorttitle{\lex}

\def\yacctitle{A \yacc\ tutorial}
\def\yaccshorttitle{\yacc}

\expandafter\def\csname tex1title\endcsname{\TeX\ -- macro programming}
\expandafter\let\csname tex1shorttitle\expandafter\endcsname
    \csname tex1title\endcsname
\expandafter\def\csname tex2title\endcsname{\TeX\ -- visuals}
\expandafter\let\csname tex2shorttitle\expandafter\endcsname
    \csname tex2title\endcsname

\def\parsingtitle{An introduction to parsing}
\def\parsingshorttitle{Parsing}

\def\hashingtitle{Hashing}
\let\hashingshorttitle\hashingtitle

\def\beziertitle{Curve approximation by splines}
\def\beziershorttitle{Splines}

\def\dynamictitle{Dynamic programming}
\let\dynamicshorttitle\dynamictitle

\def\paragraphtitle{Line breaking}
\let\paragraphshorttitle\paragraphtitle

\def\pagetitle{Page breaking in \TeX}
\def\pageshorttitle{Page breaking}

\def\completenesstitle{Introduction to NP-Completeness}
\def\completenesshortstitle{NP-Completeness}

\def\rastertitle{Introduction to Raster Graphics}
\def\rastershortstitle{Raster Graphics}

\def\pythontitle{Really quick Python tutorial}
\def\pythonshortstitle{Python}

\def\encodingtitle{Character encoding}
\let\encodingshorttitle\encodingtitle

\def\rastertitle{Raster Graphics}
\let\rastershorttitle\rastertitle

\def\gnuplottitle{Gnuplot}
\let\gnuplotshorttitle\gnuplottitle

\def\fontstitle{Fonts and font files}
\def\fontsshorttitle{Fonts}

\def\softwaretitle{\TeX\ and Software Engineering}
\def\softwareshorttitle{Software Engineering}

\renewcommand{\beamervictortitle}{CS-594 Eijkhout, Fall 2004}

\newcommand{\sectionframe}[1]%
  {\section{#1}
   \frame{\begin{center}\color{blue}\Large #1\end{center}}
  }
\newcommand{\subsectionframe}[1]%
  {\subsection{#1}
   \frame{\begin{center}\color{blue}\Large #1\end{center}}
  }

\input idxmacs

\begin{document}

\title{Bezier curves}
\author{Victor Eijkhout}
\date{Notes for CS 594 -- Fall 2004}

\frame{\titlepage}

\section{Introduction}
\frame{
  \frametitle{Fonts}
\begin{itemize}
\item Use of Bezier in fonts
\item Bitmap vs outline (vector)
\item Curve descriptions are scalable, smaller
\item<2-> More processing
\item<2-> More intelligence in the rasterizer
\end{itemize}
}

\frame{
  \frametitle{Requirements}
\begin{itemize}
\item Description unique, simple to compute
\item Easy to change shape (design)
\item Well behaved: small change in parameter gives small change in
  shape
\item Smoothly composable
\end{itemize}
}

\frame{
  \frametitle{Two basic problems:}
\begin{itemize}
\item Points known, smooth curve required
\item Function known, approximation required
\end{itemize}
}

\subsectionframe{Interpolation}

\frame{
  \frametitle{Lagrange interpolation}
\begin{itemize}
\item Given points $(x_1,f_1)\ldots(x_n,f_n)$, draw curve
\item $n$ points: polynomial of degree~$n-1$ (actually, order)
\[ p(x)=p_{n-1}x^{n-1}+\cdots+p_1x+p_0 \]
\item equations $p(x_i)=f_i$ to solve
\end{itemize}
}

\frame{
\begin{itemize}
\item System of equations
\begin{eqnarray*}
p_{n-1}x_1^{n-1}+\cdots+p_1x_1+p_0&=&f_1\\
&\ldots\\
p_{n-1}x_n^{n-1}+\cdots+p_1x_n+p_0&=&f_n
\end{eqnarray*}
written as $X\bar p=\bar f$, where
\[ X=\bigl(x_i^j\bigr),\quad
   \bar p=\begin{pmatrix}p_1\\ \vdots\\ p_{n-1}
   \end{pmatrix}
   ,\quad
   \bar f=\begin{pmatrix}f_1-p_0\\ \vdots\\ f_n-p_0
   \end{pmatrix}
   \]
\item not stable
\end{itemize}
}

\frame{
  \frametitle{Lagrange polynomials}
\begin{itemize}
\item $p^{(k)}$:
\[ p^{(k)}(x) = c_k
    (x-x_1)\cdots(x-x_{k-1})\,(x-x_{k+1})\cdots(x-x_n) \]
where $c_k$ is chosen so that~$p^{(k)}(x_k)=1$.
\item<2-> $p^{(i)}(x_j)=\delta_{ij}$, so
\begin{equation} p(x) = \sum_i f_ip^{(i)}(x),\qquad
    p^{(i)}(x)=\prod_{j\not=i}{x-x_j\over x_i-x_j}
    \label{eq:lagrange}\end{equation}
\end{itemize}
}

\frame{
  \frametitle{Hermite interpolation}
\begin{itemize}
\item Dictate function values and derivatives
\item Hermite polynomials
\[ q^{(k)} = c_k
    (x-x_1)^2\cdots(x-x_{k-1})^2\cdot (x-x_k)\cdot
    (x-x_{k+1})^2\cdots(x-x_n)^2 \]
where $c_k$ is chosen so that $q^{(k)'}(x_k)=1$.
\item<2-> Combination of $p^{(k)}$ and $q^{(k)}$ polynomials:\\
match values and derivatives
\end{itemize}
}

\subsectionframe{Approximation}

\frame{
  \frametitle{Distance, convergence}
\begin{itemize}
\item Approximate known curve
\item cheaper computation, use uniform computations
\item distance?
\item<2-> Family of approximating curves $f_n$
\item<2-> Does $|f-f_n|\rightarrow0$?
\end{itemize}
}

\frame{
  \frametitle{Convergence}
\begin{itemize}
\item Pointwise convergence
\[ \forall_{x\in I,\epsilon}\exists_N\forall_{n\geq N}:
    \left|f_n(x)-f(x)\right|\leq\epsilon.
\]
\item<2-> Uniform convergence
\[ \forall_\epsilon\exists_N\forall_{x\in I,n\geq N}:
    \left|f_n(x)-f(x)\right|\leq\epsilon.
\]
\end{itemize}
}

\frame{
  \frametitle{non-uniform convergence}
\pgfimage{hat-function}
}

\frame{
  \frametitle{Problems with high degree polynomials}
\pgfimage[width=2.5in]{runge}
\begin{itemize}
\item Better location of points
\item Better polynomials
\item<2-> Not use high degree polynomial
\end{itemize}
}

\subsectionframe{Computations}

\frame{
  \frametitle{Lagrange polynomials}
\begin{itemize}
\item $n$ terms of $n$ multiplications each: 
\[ p(x) = \sum_i f_ip^{(i)}(x),\qquad
    p^{(i)}(x)=\prod_{j\not=i}{x-x_j\over x_i-x_j}
\]
\item quadratic cost
\item<2-> Down to linear:
\[ p(x) = \prod_i(x-t_i)\cdot\sum_i{y_i\over x-t_i},
    \qquad y_i=f_i/\prod_{j\not=i}(x_i-x_j),
\]
with $y_i$ precomputed
\item<3-> complicated story with derivatives
\end{itemize}
}

\subsectionframe{Divided differences}
\newcommand\kdd[1]{\kddl{1}{#1}}
\newcommand\kddl[2]{[\tau_{#1},\ldots,\tau_{#2}]}

\frame{
  \frametitle{Definition}
\begin{itemize}
\item $n$-th divided difference $\kdd{n+1}g$ is leading coefficient of
  $n+1$-st order ($\Pi_{<n+1}$, degree~$n$ or lower) polynomial that
  agrees with~$g$ in $\tau_1,\ldots,\tau_{n+1}$
\item<2-> Zeroeth divided difference:
constant polynomial, match $g(\tau_1)=g_1$:
 $[\tau_1]g=g_1$
\item<3-> First divided difference: linear function, leading
  coefficient is slope
\[ [\tau_1,\tau_2]g={g(\tau_2)-g(\tau_1)\over \tau_2-\tau_1}
    ={[\tau_2]g-[\tau_1]g\over \tau_2-\tau_1}
\]
\item<3-> recurrence?
\end{itemize}
}

\frame{
  \frametitle{Relation}
\begin{itemize}
\item Let $p_{k+1}\in\prod_{<k+1}$ agree with~$g$ in
$\tau_1\ldots\tau_{k+1}$, and $p_k\in\prod_{<k}$ with~$g$ in
$\tau_1\ldots\tau_k$, then
\begin{equation}
    p_{k+1}(x)-p_k(x)=\kdd{k+1}g\prod_{i=1}^k(x-\tau_i).
    \label{eq:pk-diff}
\end{equation}
\item<2->
Proof. $p_k$ is of a lower order, so
\[ p_{k+1}-p_k=\kdd{k+1}gx^k+cx^{k-1}+\cdots.\]
\item $p_{k+1}-p_k$ is zero in~$t_i$ for~$i\leq\nobreak k$:
\[ p_{k+1}-p_k=C\prod_{i=1}^k(x-\tau_i). \]
\item therefore $C=\kdd{k+1}g$.
\end{itemize}
}

\frame{
\begin{itemize}
\item Repeat this:
\begin{equation}
    p_{k+1}(x)=\sum_{m=1}^{k+1}\kdd{m}g\prod_{i=1}^{m-1}(x-\tau_i),
    \label{eq:p-sum-diffs}
\end{equation}
\item can be evaluated as
\begin{footnotesize}
\[
\begin{array}{rcl}
 p_{k+1}(x) &=&
 \kdd{k+1}g\prod^k(x-\tau_i)+\kdd{k}g\prod^{k-1}(x-\tau_i)\\
    &=&\kdd{k+1}g(x-\tau_k)\bigl(c_k
                +\kdd{k}g(x-\tau_{k-1})\bigl(c_{k-1}+\cdots
\end{array}
\]
\end{footnotesize}
where $c_k=\kdd{k}g/\kdd{k+1}g$
\item Horner's rule
\end{itemize}
}

\frame{
  \frametitle{Construction of divided differences}
\begin{itemize}
\item Recursive:
\[
    \kdd{n+1}g=\left(\kdd{n}g-\kddl{2}{n+1}g\right)/(\tau_1-\tau_{n+1}).
\]
\item Proof. Three polynomials given:\\
 $p^{(1)}_n\in\prod_{<n}$ agrees with~$g$ on
  $\tau_1\ldots\tau_n$;\\
 $p^{(2)}_n\in\prod_{<n}$ agrees with~$g$ on
  $\tau_2\ldots\tau_{n+1}$;\\
 $p_{n-1}\in\prod_{<n-1}$ agrees with~$g$ on
  $\tau_2\ldots\tau_n$.
\item then
\[ \begin{array}{rcl}
  p^{(1)}_n-p_{n-1}&=&\kdd{n}g\prod_{j=2}^n(x-\tau_j)\\
  p^{(2)}_n-p_{n-1}&=&\kddl{2}{n+1}g\prod_{j=2}^n(x-\tau_j)
\end{array}
\]
\end{itemize}
}

\frame{
  \frametitle{}
\begin{itemize}
\item $p_{n+1}$ agrees with~$g$ on $\tau_1\ldots\tau_{n+1}$:
\[ \begin{array}{rcl}
  p_{n+1}-p^{(1)}&=&\kdd{n+1}g\prod_{j=1}^n(x-\tau_j)\\
  p_{n+1}-p^{(2)}&=&\kdd{n+1}g\prod_{j=2}^{n+1}(x-\tau_j)
\end{array} \]
\item Subtracting gives for $p^{(1)}_n-p^{(2)}_n$:
\begin{footnotesize}
\[ \left(\kdd{n}g-\kddl{2}{n+1}g\right)\prod_{j=2}^n(x-\tau_j)
  = \kdd{n+1}g\left(\prod_{j=2}^{n+1}-\prod_{j=1}^n\right)
        (x-\tau_j)
\] 
\end{footnotesize}
\item Fill in $\tau_2\ldots\tau_n$: zero\\
With $x=\nobreak\tau_1$
\begin{footnotesize}
\[ \left(\kdd{n}g-\kddl{2}{n+1}g\right)\prod_{j=2}^n(\tau_1-\tau_j)
  = \kdd{n+1}g\prod_{j=2}^{n+1}(\tau_1-\tau_j)
\]
\end{footnotesize}
\end{itemize}
}

\sectionframe{Parametric curves}

\frame{
  \frametitle{Functions are not enough}
\begin{itemize}
\item no unique mapping $x\mapsto y$: circle
\item implicit $f(x,y)=0$ trouble with half circle
\item parametric: \[ P=P(t) = \begin{pmatrix}x(t)\\ y(t)
    \end{pmatrix}.\]
\item parametric interpolation: $P=tP_2+(1-t)P_1$:\\
$P(0)=P_1$, $P(1)=P_2$;\\
\[ P(t)=(\cos 2\pi t,\sin 2\pi t) \]
\end{itemize}
}

\subsection{Cubics}

\frame{
  \frametitle{Cubics}
\begin{itemize}
\item Flexible enough: 4 degrees of freedom\\
location and direction in two points\\
(Hermite strategy)
\item Higher degrees harder to control
\end{itemize}
}

\frame{
  \frametitle{Formal description}
\begin{itemize}
\item Matrix/Vector description: $Q(t)=C\cdot T$
\item coefficient matrix
\begin{footnotesize}
\[ C=\begin{pmatrix}c_{11}&c_{12}&c_{13}&c_{14}\\
    c_{21}&c_{22}&c_{23}&c_{24}\\ c_{31}&c_{32}&c_{33}&c_{34}
    \end{pmatrix}
    ,\qquad 
    T=\begin{pmatrix}1\\ t\\ t^2\\ t^3
    \end{pmatrix}
\]
\end{footnotesize}
\item  Direction:
\begin{footnotesize}
\[ {dQ(t)\over dt}=C\cdot {dT\over dt}=
    C\cdot\begin{pmatrix}0\\ 1\\ 2t\\ 3t^2
    \end{pmatrix}
\]
\end{footnotesize}
\item Hermite example: $P_1=Q(0)$,
$R_1=Q'(0)$, $P_2=Q(1)$, and~$R_2=\nobreak Q'(1)$ 
\begin{footnotesize}
\[ C\cdot \begin{pmatrix}1&0&1&0\\ 0&1&1&1\\ 0&0&1&2\\ 0&0&1&3
    \end{pmatrix}
    =[P_1,R_1,P_2,R_2],
\]
\end{footnotesize}
\end{itemize}
}

\frame{
  \frametitle{Blending functions}
\begin{itemize}
\item Description $C=G\cdot M$ 
in terms of basis polynomials and geometry matrix:
\begin{equation}
  Q(t) = G\cdot M\cdot T = 
    \begin{pmatrix}g_{11}&g_{12}&g_{13}&g_{14}\\
        g_{21}&g_{22}&g_{23}&g_{24}\\ g_{31}&g_{32}&g_{33}&g_{34}
    \end{pmatrix}
    \cdot \begin{pmatrix}m_{11}&\ldots&m_{14}\\ \vdots&&\vdots\\
        m_{41}&\ldots&m_{44}
    \end{pmatrix}
    \cdot T
\label{eq:geom}
\end{equation}
If we introduce new basis polynomials $\pi_i(t)=M_{i*}\cdot T$, then we
see that $Q_x=G_{11}\pi_1+G_{12}\pi_2+G_{13}\pi_3+G_{14}\pi_4$,
$Q_y=G_{21}\pi_1+\cdots$, et cetera.
\end{itemize}
}

\frame{
  \frametitle{Computing the basis matrix $M$}
Hermite case: geometric constraints $[Q(0),Q'(0),Q(1),Q'(1)]$\\
From $Q=G\cdot M\cdot T$:
\[ Q(t)= G_H\cdot M_H\cdot T(t),\qquad
   Q'(t) = G_H\cdot M_H\cdot T'(t). \]
Applying to $t=0$ and $t=1$:
\[ Q_H\equiv[Q(0),Q'(0),Q(1),Q'(1)] = G_H\cdot M_H\cdot T_H \]
where
\begin{footnotesize}
\[ T_H=[T(0),T'(0),T(1),T'(1) ]
     =\begin{pmatrix}1&0&1&0\cr 0&1&1&1\cr 0&0&1&2\cr 0&0&1&3
     \end{pmatrix}
\]
\end{footnotesize}
\begin{footnotesize}
But now $G_H=Q_H$. It now follows that
\[ M_H=T_H^{-1}
    =\begin{pmatrix}1&0&-3&2\cr 0&1&-2&1\cr 0&0&3&-2\cr 0&0&-1&1
    \end{pmatrix}
\]
\end{footnotesize}
}

\frame{
  \frametitle{}
\begin{itemize}
\item $Q=G\cdot M\cdot T$ with
\[ M_H=T_H^{-1}
    =\begin{pmatrix}1&0&-3&2\\ 0&1&-2&1\\ 0&0&3&-2\\ 0&0&-1&1
    \end{pmatrix}.
\]
\item Writing this out, we find the cubic Hermite polynomials
\[ P_1(t)=2t^3-3t^2+1,\quad P_2(t)=t^3-2t^2+t,\quad
    P_3(t)=-2t^3+3t^2,\quad P_1(t)=t^3-t^2
\]

\end{itemize}
}

\newcommand\picandsrc[2]
{\leavevmode\pgfimage[height=1.3in]{#1}
\begin{minipage}{2in}
\begin{footnotesize}
\verbatiminput{#2.gp}
\end{footnotesize}
\end{minipage}
}

\frame[containsverbatim]{
  \frametitle{Hermite basis functions}
\picandsrc{hermite}{hermite}
}

\frame{
  \frametitle{Example of Hermite interpolation}
Curve $.3P_1-2P_2+P_3-2P4$:\\
through $(0,.3)$ and~$(1,1)$, with slope~$-2$ in
both~$x=0,1$.

\pgfimage[height=1.5in]{herminp}
}

\frame{
  \frametitle{Hermite parametric curves}
Both coordinates parametric:
\[ Q = \left({.1\atop .2}\right)P_1+\left({1\atop 0}\right)P_2
    +\left({.9\atop .3}\right)P_3+\left({0\atop -1}\right)P_4.
\]
\picandsrc{hermpara}{hermpara}
}

\frame{
  \frametitle{Smoothness}
\picandsrc{hermpara2}{hermpara2}
}

\subsectionframe{Splines}

\frame{
  \frametitle{Bernstein polynomials}
\begin{itemize}
\item $z(t)=(1-t)^3z_1+3(1-t)^2tz_2+3(1-t)t^2z_3+t^3z_4$
\pgfimage[height=2in]{bernshtein}
\end{itemize}
}

\frame{
  \frametitle{Derivation from Hermite basis}
\begin{itemize}
\item Hermite direction vectors~$R_1,R_2$,\\
replace by control points~$P_1',P_2'$:
\[ R_1=3(P_1'-P1), \quad R_2=3(P_2-P_2') \]
\item Geometry matrix
\[ G_B=[P_1,P_1',P_2',P_2]\quad
   G_H=[P_1,R_1,P_2,R_2]
    = [P_1,P_1',P_2',P_2] M_{BH} = G_B\cdot M_{BH}
\]
with 
\[
     M_{BH}=\begin{pmatrix}1&-3&0&0\\ 0&3&0&0\\ 0&0&0&-3\\ 0&0&1&3
     \end{pmatrix}
\]
\end{itemize}
}

\frame{
\begin{itemize}
\item Define
\[ M_B=M_{BH}\cdot M_H=
   \begin{pmatrix}1&-3&3&-1\\ 0&3&-6&3\\ 0&0&3&-3\\ 0&0&0&1
   \end{pmatrix}
\]
the Bezier curves:
\[ Q(t)=G_H\cdot M_H\cdot T(t)=G_B\cdot M_{BH}\cdot M_H\cdot T(t)
    =G_B\cdot M_B\cdot T(t)
\]
\item Per component: $Q_x(t)=g_{11}B_1(t)+g_{12}B_2(t)+\cdots$
where
\[ \begin{array}{rll}
B_1(t) &= 1-3t+3t^2 -t^3&=(1-t)^3\\
B_2(t) &= 3t-6t^2+3t^3&=3t(1-t)^2\\
B_3(t) &= 3t^2-3t^3&=3t^2(1-t)\\
B_4(t) &=          &=t^3
\end{array}
\]
\end{itemize}
}

\frame{
  \frametitle{Enclosing hull}
\begin{itemize}
\item Bernshtein polynomials
\[ z(t)=(1-t)^3z_1+3(1-t)^2tz_2+3(1-t)t^2z_3+t^3z_4 \]
\item With all $z_i\equiv1$:
\[ z(t)=(t+(1-t))^3\equiv1 \]
\item $\Rightarrow$ Bezier curve components are weighted averages
\item $\Rightarrow$ curve containes in convex hull of control points
\end{itemize}
}

\frame{
  \frametitle{Calculation of Bezier curves}
\begin{itemize}
\item The simple way: $Q(t)=G\cdot M\cdot T(t)$
\begin{itemize}
\item 2 multiplications to form the terms $t^2$ and~$t^3$ in~$T$;
\item 16 multiplications and 12 additions forming~$M\cdot T$;
\item 12 multiplications and 9 additions forming $G\cdot(M\cdot T)$.
\end{itemize}
\item Improvement: store $\tilde M=G\cdot M$:
\begin{itemize}
\item two multiplications for~$T$;
\item 12 multiplications and 9 additions for forming $\tilde M\cdot
  T$.
\end{itemize}
\end{itemize}
}

\frame{
  \frametitle{Calculation with divided differences}
\begin{footnotesize}
\begin{itemize}
\item $Q(t)=G\cdot M\cdot T(t)$, $x$ component:
\[ x(t) = \sum_kc_kt^{k-1}\qquad c_k=\sum_jG_{1j}M_{jk}. \]
\item  recall
\[  M_B=\begin{pmatrix}1&-3&3&-1\\ 0&3&-6&3\\ 0&0&3&-3\\ 0&0&0&1
        \end{pmatrix}
\]
\item  writing $g_j\equiv G_{1j}$:
\[ c_1=g_1,\quad c_2=3(g_2-g_1),\quad
    c_3=3(g_3-2g_2+g_1),\quad c_4=g_4-3g_3+3g_2-g_1. \]
\item These are divided differences:
\[ \begin{array}{lcl}\relax [2,1]g&=&g_2-g_1,\\
 \relax [3,2,1]g&=&[3,2]g-[2,1]g=(g_3-g_2)-(g_2-g_1)\\
         &=&g_3-2g_2+g_1\\
 \relax [4,3,2,1]&=&[4,3,2]g-[3,2,1]g=(g_4-2g_3+g_2)-(g_3-2g_2+g_1)\\
         &=&g_4-3g_3+3g_2-g_1
    \end{array}
\]
\end{itemize}
\end{footnotesize}
}

\sectionframe{Practical use of splines}

\frame{
  \frametitle{Piecewise curves}
\pgfimage[height=2in]{bezier2}
}

\frame{
  \frametitle{Spline drawing}
\begin{itemize}
\item Divided difference algorithm is good for single points
\item<2-> Whole curve: compute step by step $Q(n\delta)$, $n=0,1,\ldots$
\item<3-> Other tricks: recursive bisection; use line drawing when curve
  is flat enough
\item<3-> (flatness test through control points for Bezier)
\end{itemize}
}

\end{document}

\frame{
  \frametitle{}
\begin{itemize}
\item 
\end{itemize}
}

