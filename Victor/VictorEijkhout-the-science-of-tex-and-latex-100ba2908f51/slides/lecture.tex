\documentclass{beamer}

\usepackage{beamerthemevictor,comment,verbatim,graphicx,amssymb}

\newcommand{\TbT}{\TeX\ by Topic}
\newcommand{\Metafont}{\textsc{Metafont}}
\let\metafont\Metafont
\newcommand\ascii{{\sc Ascii}}
\newcommand\ebcdic{{\sc Ebcdic}}
\newcommand\gnuplot{\protect\n{gnuplot}}
\def\mod{\mathbin{\mathrm{mod}}}
\def\eqref#1{equation~(\ref{#1})}
\def\Eqref#1{Equation~(\ref{#1})}
\def\inv{^{-1}}
\def\endproof{\unskip\penalty10000 \hskip10pt plus 1fill$\bullet$\par}

\def\etal{\textit{et al.}}
\def\lex{{\it lex}}
\def\yacc{{\it yacc}}
\def\make{{\it make}}
\def\web{{\textsc{Web}}}
\def\n{\bgroup
  \catcode`\#=12 \catcode`\_=12 \catcode`\^=12
  \catcode`\&=12 
  \tt \let\next=}
\newcommand{\cs}[1]{{\tt\char`\\#1}}
{\catcode`\.=13
 \gdef.{${}_\bullet$}
}
\def\parserule#1{%
    \ifmmode \n{#1}\,\rightarrow\relax
    \else $\n{#1}\,\rightarrow{}$\nobreak
    \fi
    \hbox\bgroup\catcode`\.=13 \tt\let\next=}

\newenvironment{remark}%
    {\medskip\noindent{\bf Remark.}}{}
\usepackage{bnf}
\input bnf.env

\generalcomment{inputwithcode}
  {\begingroup\def\ProcessCutFile{}}
  {\verbatiminput{\CommentCutFile}
   \endgroup
   \input{\CommentCutFile}
  }
\generalcomment{examplewithcode}
  {\begingroup
     \def\ProcessCutFile{}\def\CommentCutFile{example.tex}}
  {\verbatiminput{\CommentCutFile}
   Output:
   \begin{quote}
     \begin{minipage}[t]{3in}
        \everypar{}
        \input{\CommentCutFile} 
     \end{minipage}
   \end{quote}
   \endgroup
  }
\specialcomment{ttexamplewithcode}
  {\begingroup\def\ProcessCutFile{}}
  {\verbatiminput{\CommentCutFile}
   \endgroup
   Output:
   \begin{quote}\begin{ttfamily} \input{\CommentCutFile} 
     \end{ttfamily}\end{quote}
  }
\generalcomment{mathexamplewithcode}
  {\begingroup\def\ProcessCutFile{}}
  {\verbatiminput{\CommentCutFile}
   Output:
   \begin{equation} \input{\CommentCutFile} \end{equation}
   \endgroup
  }

\def\chaptertitle{\csname\chaptername title\endcsname}
\def\chaptershorttitle{\csname\chaptername shorttitle\endcsname}

\def\lambdatitle{Lambda calculus in \TeX}
\def\lambdashorttitle{Lambda calculus}

\def\latextitle{Yet another \LaTeX\ tutorial}
\def\latexshorttitle{\LaTeX}

\def\lextitle{A \lex\ tutorial}
\def\lexshorttitle{\lex}

\def\yacctitle{A \yacc\ tutorial}
\def\yaccshorttitle{\yacc}

\expandafter\def\csname tex1title\endcsname{\TeX\ -- macro programming}
\expandafter\let\csname tex1shorttitle\expandafter\endcsname
    \csname tex1title\endcsname
\expandafter\def\csname tex2title\endcsname{\TeX\ -- visuals}
\expandafter\let\csname tex2shorttitle\expandafter\endcsname
    \csname tex2title\endcsname

\def\parsingtitle{An introduction to parsing}
\def\parsingshorttitle{Parsing}

\def\hashingtitle{Hashing}
\let\hashingshorttitle\hashingtitle

\def\beziertitle{Curve approximation by splines}
\def\beziershorttitle{Splines}

\def\dynamictitle{Dynamic programming}
\let\dynamicshorttitle\dynamictitle

\def\paragraphtitle{Line breaking}
\let\paragraphshorttitle\paragraphtitle

\def\pagetitle{Page breaking in \TeX}
\def\pageshorttitle{Page breaking}

\def\completenesstitle{Introduction to NP-Completeness}
\def\completenesshortstitle{NP-Completeness}

\def\rastertitle{Introduction to Raster Graphics}
\def\rastershortstitle{Raster Graphics}

\def\pythontitle{Really quick Python tutorial}
\def\pythonshortstitle{Python}

\def\encodingtitle{Character encoding}
\let\encodingshorttitle\encodingtitle

\def\rastertitle{Raster Graphics}
\let\rastershorttitle\rastertitle

\def\gnuplottitle{Gnuplot}
\let\gnuplotshorttitle\gnuplottitle

\def\fontstitle{Fonts and font files}
\def\fontsshorttitle{Fonts}

\def\softwaretitle{\TeX\ and Software Engineering}
\def\softwareshorttitle{Software Engineering}

\renewcommand{\beamervictortitle}{CS-594 Eijkhout, Fall 2004}

\newcommand{\sectionframe}[1]%
  {\section{#1}
   \frame{\begin{center}\color{blue}\Large #1\end{center}}
  }
\newcommand{\subsectionframe}[1]%
  {\subsection{#1}
   \frame{\begin{center}\color{blue}\Large #1\end{center}}
  }

\input idxmacs

\begin{document}

\title{Lecture note macros}
\author{Victor Eijkhout}
\date{Notes for CS 594 -- Fall 2004}

\frame{\titlepage}

\frame{
  \frametitle{Basic problem}
\begin{itemize}
\item Disabling piece of input (under construction)
\item Selectively inluding parts of input (teacher/student versions)
\item Multiple processing of input (input and output)
\end{itemize}
}

\frame{
  \frametitle{Structure of the solution}
\begin{itemize}
\item Use environment
\item Read input, line by line, verbatim
\item Output to file
\item Maybe process file.
\end{itemize}
}

\frame[containsverbatim]{
  \frametitle{API}
\begin{itemize}
\item Define comment by 
\begin{verbatim}
\excludecomment{comment}
\end{verbatim}
or
\begin{verbatim}
\includecomment{comment}
\end{verbatim}
\item Then write
\begin{verbatim}
\begin{comment}
maybe in, maybe out
\end{comment}
\end{verbatim}
\item (file \n{comment.sty})
\end{itemize}
}

\sectionframe{Tools}

\frame[containsverbatim]{
  \frametitle{File handling in \TeX}
\begin{itemize}
\item Define stream \verb+\newwrite\CommentStream+
\item Open stream \verb+\openout\CommentStream=<file>+
\item Write to stream \verb+\write\CommentStream{#1}+
\item Close stream \verb+\closeout\CommentStream+
\item (\TeX nical bit: use \verb+\immediate+ for open/close/write)
\end{itemize}
}

\frame[containsverbatim]{
  \frametitle{Verbatim handling}
\begin{verbatim}
*\show\dospecials
> \dospecials=macro:
->\do \ \do \\\do \{\do \}\do \$\do \&\do \#\do \^\do \_\do \%\do \~.

\def\makeinnocent#1{\catcode`#1=12 }

<inside the verbatim macro:>
    \let\do\makeinnocent \dospecials 
    \makeinnocent\^^L% and whatever other special cases
    \endlinechar`\^^M \catcode`\^^M=12
\end{verbatim}
}

\frame[containsverbatim]{
  \frametitle{\LaTeX\ environments}
Environment
\begin{verbatim}
\begin{foo}
...
\end{foo}
\end{verbatim}
is really nothing but
\begin{verbatim}
\begingroup
  \foo
  ...
  \endfoo % if it exists
\endgroup
\end{verbatim}
}

\sectionframe{Putting the algorithm together}

\frame[containsverbatim]{
  \frametitle{Comment environment definition}
Call \verb+\excludecomment{foo}+
defines \verb+\foo+ as start of processing;
end command is more tricky.
}

\frame[containsverbatim]{
 \frametitle{Full implementation}
\begin{verbatim}
\def\excludecomment
 #1{\csarg\def{#1}{\endgroup
        \begingroup
           \def\ProcessCutFile{}%
           \def\ThisComment####1{}\ProcessComment{#1}}%
    \csarg\def{After#1Comment}{\CloseAndInputCutFile \endgroup}
    \CommentEndDef{#1}}
\end{verbatim}
}

\frame[containsverbatim]{
  \frametitle{Recursive processing (simplified)}
\begin{verbatim}
\def\ProcessComment#1% start it all of
   {\begingroup
    <all that makeinnocent stuff goes here>
    \xComment}
{\escapechar=-1\relax
 \global\edef\endcommenttest{\string\\end\string\{comment\}}
}
{\catcode`\^^M=12 \endlinechar=-1 %
 \gdef\xComment#1^^M{\ProcessCommentLine}
 \gdef\ProcessCommentLine#1^^M{\def\test{#1}
      \ifx\endcommenttest\test
          \edef\next{\endgroup\noexpand\EndOfComment}%
      \else \immediate\write\CommentStream{#1}
          \let\next\ProcessCommentLine
      \fi \next}
}
\end{verbatim}
}

\frame[containsverbatim]{
  \frametitle{The utility stuff}
\begin{verbatim}
\def\CommentCutFile{comment.cut}
\def\SetUpCutFile
   {\immediate\openout\CommentStream=\CommentCutFile
    \let\ThisComment\WriteCommentLine}
\def\WriteCommentLine#1{\immediate\write\CommentStream{#1}}
\def\CloseAndInputCutFile
   {\immediate\closeout\CommentStream
    \ProcessCutFile
    }%
\def\ProcessCutFile
   {\input{\CommentCutFile}\relax}
\end{verbatim}
}

\sectionframe{Nifty extension}

\frame[containsverbatim]{
  \frametitle{More general macro}
\begin{verbatim}
\long\def\generalcomment
 #1#2#3{\message{General comment '#1'}%
    \csarg\def{#1}{\endgroup % counter the environment open of LaTeX
          #2\relax \SetUpCutFile \ProcessComment{#1}}%
    \csarg\def{After#1Comment}{\CloseAndInputCutFile#3}%
    \CommentEndDef{#1}}
% Use \#1
\generalcomment{inputwithcode}
  {\begingroup\def\ProcessCutFile{}}
  {\verbatiminput{\CommentCutFile}
   \endgroup
   \input{\CommentCutFile}
  }
\end{verbatim}
}

\frame[containsverbatim]{
  \frametitle{more examples}
\footnotesize
\begin{verbatim}
\generalcomment{mathexamplewithcode}
  {\begingroup\def\ProcessCutFile{}}
  {\verbatiminput{\CommentCutFile}
   Output:
   \begin{equation} \input{\CommentCutFile} \end{equation}
   \endgroup
  }
\generalcomment{examplewithcode}
  {\begingroup
     \def\ProcessCutFile{}\def\CommentCutFile{example.tex}}
  {\verbatiminput{\CommentCutFile}
   Output:
   \begin{quote}
     \begin{minipage}[t]{3in}
        \everypar{} \input{\CommentCutFile} 
     \end{minipage}
   \end{quote}
   \endgroup
  }
\end{verbatim}
}

\frame[containsverbatim]{
  \frametitle{Exercise and answer macros}
Use:
\begin{verbatim}
\begin{594exercise}
Show that ...
\end{594exercise}
\begin{answer}
Given ....
\end{answer}
\end{verbatim}
in the input file of the chapter.
}

\frame[containsverbatim]{
  \frametitle{Implementation}
\footnotesize
\begin{verbatim}
\generalcomment{594exercise}
  {\refstepcounter{excounter}%
   \begingroup\def\ProcessCutFile{}\par
   \edef\tmp{\def\noexpand\CommentCutFile
                 {\chaptername-ex\arabic{excounter}.tex}}\tmp
   }
  {\begin{quote}
   \textbf{Exercise \arabic{excounter}.}\hspace{1em}\ignorespaces
   \input{\CommentCutFile}
   \end{quote}
   \endgroup}
\generalcomment{answer}
  {\begingroup
   \edef\tmp{\def\noexpand\CommentCutFile
                 {\chaptername-an\noexpand\arabic{excounter}.tex}}\tmp
   \def\ProcessCutFile{}}
  {\endgroup}
\end{verbatim}
}

\frame[containsverbatim]{
  \frametitle{exercise sheet}
\footnotesize
Chapter file has:
\begin{verbatim}
\input{\chaptername}
\newwrite\nx
\openout\nx=\chaptername-nx.tex
\write\nx{\arabic{excounter}}
\closeout\nx
\end{verbatim}
Exercise file has
\begin{verbatim}
\newread\nx
\openin\nx=\chaptername-nx.tex
\read\nx to \nex
\closein\nx

\begin{enumerate}
\repeat \for{nx} \to{\nex}
  \do{\item \edef\tmp{\noexpand\input \chaptername-ex\number\nx.tex}\tmp}
\end{enumerate}
\end{verbatim}
}

\end{document}

\frame[containsverbatim]{
  \frametitle{}
\begin{itemize}
\item 
\end{itemize}
}

\frame[containsverbatim]{
  \frametitle{}
\begin{itemize}
\item 
\end{itemize}
}

