\documentclass{beamer}

\usepackage{beamerthemevictor,comment,verbatim,graphicx,amssymb}

\newcommand{\TbT}{\TeX\ by Topic}
\newcommand{\Metafont}{\textsc{Metafont}}
\let\metafont\Metafont
\newcommand\ascii{{\sc Ascii}}
\newcommand\ebcdic{{\sc Ebcdic}}
\newcommand\gnuplot{\protect\n{gnuplot}}
\def\mod{\mathbin{\mathrm{mod}}}
\def\eqref#1{equation~(\ref{#1})}
\def\Eqref#1{Equation~(\ref{#1})}
\def\inv{^{-1}}
\def\endproof{\unskip\penalty10000 \hskip10pt plus 1fill$\bullet$\par}

\def\etal{\textit{et al.}}
\def\lex{{\it lex}}
\def\yacc{{\it yacc}}
\def\make{{\it make}}
\def\web{{\textsc{Web}}}
\def\n{\bgroup
  \catcode`\#=12 \catcode`\_=12 \catcode`\^=12
  \catcode`\&=12 
  \tt \let\next=}
\newcommand{\cs}[1]{{\tt\char`\\#1}}
{\catcode`\.=13
 \gdef.{${}_\bullet$}
}
\def\parserule#1{%
    \ifmmode \n{#1}\,\rightarrow\relax
    \else $\n{#1}\,\rightarrow{}$\nobreak
    \fi
    \hbox\bgroup\catcode`\.=13 \tt\let\next=}

\newenvironment{remark}%
    {\medskip\noindent{\bf Remark.}}{}
\usepackage{bnf}
\input bnf.env

\generalcomment{inputwithcode}
  {\begingroup\def\ProcessCutFile{}}
  {\verbatiminput{\CommentCutFile}
   \endgroup
   \input{\CommentCutFile}
  }
\generalcomment{examplewithcode}
  {\begingroup
     \def\ProcessCutFile{}\def\CommentCutFile{example.tex}}
  {\verbatiminput{\CommentCutFile}
   Output:
   \begin{quote}
     \begin{minipage}[t]{3in}
        \everypar{}
        \input{\CommentCutFile} 
     \end{minipage}
   \end{quote}
   \endgroup
  }
\specialcomment{ttexamplewithcode}
  {\begingroup\def\ProcessCutFile{}}
  {\verbatiminput{\CommentCutFile}
   \endgroup
   Output:
   \begin{quote}\begin{ttfamily} \input{\CommentCutFile} 
     \end{ttfamily}\end{quote}
  }
\generalcomment{mathexamplewithcode}
  {\begingroup\def\ProcessCutFile{}}
  {\verbatiminput{\CommentCutFile}
   Output:
   \begin{equation} \input{\CommentCutFile} \end{equation}
   \endgroup
  }

\def\chaptertitle{\csname\chaptername title\endcsname}
\def\chaptershorttitle{\csname\chaptername shorttitle\endcsname}

\def\lambdatitle{Lambda calculus in \TeX}
\def\lambdashorttitle{Lambda calculus}

\def\latextitle{Yet another \LaTeX\ tutorial}
\def\latexshorttitle{\LaTeX}

\def\lextitle{A \lex\ tutorial}
\def\lexshorttitle{\lex}

\def\yacctitle{A \yacc\ tutorial}
\def\yaccshorttitle{\yacc}

\expandafter\def\csname tex1title\endcsname{\TeX\ -- macro programming}
\expandafter\let\csname tex1shorttitle\expandafter\endcsname
    \csname tex1title\endcsname
\expandafter\def\csname tex2title\endcsname{\TeX\ -- visuals}
\expandafter\let\csname tex2shorttitle\expandafter\endcsname
    \csname tex2title\endcsname

\def\parsingtitle{An introduction to parsing}
\def\parsingshorttitle{Parsing}

\def\hashingtitle{Hashing}
\let\hashingshorttitle\hashingtitle

\def\beziertitle{Curve approximation by splines}
\def\beziershorttitle{Splines}

\def\dynamictitle{Dynamic programming}
\let\dynamicshorttitle\dynamictitle

\def\paragraphtitle{Line breaking}
\let\paragraphshorttitle\paragraphtitle

\def\pagetitle{Page breaking in \TeX}
\def\pageshorttitle{Page breaking}

\def\completenesstitle{Introduction to NP-Completeness}
\def\completenesshortstitle{NP-Completeness}

\def\rastertitle{Introduction to Raster Graphics}
\def\rastershortstitle{Raster Graphics}

\def\pythontitle{Really quick Python tutorial}
\def\pythonshortstitle{Python}

\def\encodingtitle{Character encoding}
\let\encodingshorttitle\encodingtitle

\def\rastertitle{Raster Graphics}
\let\rastershorttitle\rastertitle

\def\gnuplottitle{Gnuplot}
\let\gnuplotshorttitle\gnuplottitle

\def\fontstitle{Fonts and font files}
\def\fontsshorttitle{Fonts}

\def\softwaretitle{\TeX\ and Software Engineering}
\def\softwareshorttitle{Software Engineering}

\renewcommand{\beamervictortitle}{CS-594 Eijkhout, Fall 2004}

\newcommand{\sectionframe}[1]%
  {\section{#1}
   \frame{\begin{center}\color{blue}\Large #1\end{center}}
  }
\newcommand{\subsectionframe}[1]%
  {\subsection{#1}
   \frame{\begin{center}\color{blue}\Large #1\end{center}}
  }

\input idxmacs

\begin{document}

\title{Character encoding}
\author{Victor Eijkhout}
\date{Notes for CS 594 -- Fall 2004}

\frame{\titlepage}

\section{Introduction}
\subsectionframe{Prehistory: Ascii}

\frame{
  \frametitle{Ascii vs Ebcdic}
\begin{itemize}
\item \ascii\ good, \ebcdic\ IBM\n{\char`^H\char`^H\char`^H}bad
\item<2-> \ascii: alphabet contiguous; \ebcdic\ not
\item<3-> \ascii\ has unprintable or `control codes'
\item<4-> High bit always off
\end{itemize}
}

\frame{
\frametitle{ISO 646, \ascii}
\pgfimage[height=2in]{ascii}
}

\subsection{Code pages}
\frame{
  \frametitle{`8-bit \ascii}
\begin{itemize}
\item Languages other than English need accents
\item Some languages need completely differt alphabets
\item<2-> Code page: way of using the characters over 128
\item<2-> Standards ISO 646-DE et cetera.
\end{itemize}
}

\frame{
  \frametitle{Dos: 437}
\pgfimage[height=2in]{dos437}
}

\frame{
  \frametitle{Microsoft Windows Latin 1}
\pgfimage[height=2.3in]{winlatin1}
}

\subsectionframe{Recent history: ISO 8859}

\frame{
  \frametitle{ISO 8859}
\begin{itemize}
\item Set of standards: 1~for Latin, 2~east European, 5~Cyrillic
\item first 32 positions over 128 left open for vendor extension
\end{itemize}
}

\frame{
  \frametitle{8859-1: Latin 1}
\pgfimage[height=2in]{latin1}
}

\frame{
  \frametitle{Still trouble}
\begin{itemize}
\item Asian alphabets can be really large
\item DBCS: Double Byte Character Set
\item<2-> fun to code: no longer \n{s++} and such
\end{itemize}
}

\sectionframe{Unicode}
\subsection{Character content}

\frame{
  \frametitle{ISO 10646}
\begin{itemize}
\item UCS: Universal Character Set (ISO 10646)
\item Unicode adds lots of goodies
\item Over a million positions
\end{itemize}
}

\frame{
  \frametitle{Subsets}
\begin{itemize}
\item First 128: \ascii
\item First 256: ISO 8859-1
\item First 2-byte plane: BMP (Basic Multilingual Plane)
\end{itemize}
}

\subsectionframe{Encodings}

\frame{
  \frametitle{The idea of encoding}
\begin{itemize}
\item Unicode is a list: how do you access elements?
\item Six bytes is a waste for plain \ascii, incompatible
\item Encodings of subsets, encodings for the whole caboodle
\end{itemize}
}

\frame{
  \frametitle{Examples}
\begin{itemize}
\item UCS-2: two byte
\item UTF-16: BMP
\item UTF-8: one byte access to every character
\item UTF-7: id, but based on 0--127 positions
\end{itemize}
}

\frame{
  \frametitle{UTF-8}
\begin{itemize}
\item 0--127 rendered `as such'
\item higher positions take 2--6 bytes:
\begin{itemize}
\item first byte in the range 0xC0--0xFD (192--252)
\item next up to~5 bytes in the range 0x80--0xBF (128--191)
\item $\n{8}=\n{1000}$ and $\n{B}=\n{1011}$, so  bit pattern starting with~\n{10})
\item $\Rightarrow$ six bits left for encoding
\end{itemize}
\item (UTF-8 is standardized as RFC~3629.)
\end{itemize}
}

\frame{
\begin{footnotesize}
\begin{ttfamily}\begin{tabular}{|l|llll|}
\hline
U-00000000 - U-0000007F&\textrm{7 bits}&
0xxxxxxx&&\\
U-00000080 - U-000007FF&$11=5+6$&
110xxxxx&10xxxxxx&\\
U-00000800 - U-0000FFFF&$16=4+2\times6$&
1110xxxx&10xxxxxx&10xxxxxx\\
U-00010000 - U-001FFFFF&$21=3+3\times6$&
11110xxx&\multicolumn{2}{l|}{10xxxxxx (3 times)}\\
U-00200000 - U-03FFFFFF&$26=2+4\times6$&
111110xx&\multicolumn{2}{l|}{10xxxxxx (4 times)}\\
U-04000000 - U-7FFFFFFF&$31=1+5\times6$&
1111110x&\multicolumn{2}{l|}{10xxxxxx (5 times)}\\\hline
  \end{tabular}
\end{ttfamily}
\end{footnotesize}
}

\sectionframe{Remaining stuff}

\subsection{Precise definitions}

\frame{
  \frametitle{Character sets}
\begin{itemize}
\item Mapping from sequences of bytes to characters
\item (reverse may not be unique)
\item ISO 2022 defines switching between character sets; escape sequences
\end{itemize}
}

\frame{
  \frametitle{Character $\leftrightarrow$ encoding}
\begin{itemize}
\item Abstract Character Repertoire: list, unordered
\item<2-> Coded Character Set (code page, code points): numbers
  assigned
\item<3-> Character Encoding Form: mapping number to sequences of code
  units (UCS-2: one 16-bit unit, UTF-8: several 8-bit units)
\item<4-> Character Encoding Scheme: mapping to sequence of bytes
  (byte order, escape sequences)
\end{itemize}
}

\subsection{Character sets in use}

\frame[containsverbatim]{
  \frametitle{Bootstrap problem}
\begin{itemize}
\item How do you tell the receiver what character set you are using?
\item MIBenum (RFC 1759, RFC 3808): unique names and numbers (IANA)
\item Example: ascii
\begin{itemize}
\item[name] \n{ANSI_X3.4-1968}
\item[reference] RFC1345,KXS2
\item[MIBenum] 3
\item[source] ECMA registry
\item[aliases] \n{iso-ir-6}, \n{ANSI_X3.4-1986}, \n{ISO_646.irv:1991},
  \n{ASCII}, \n{ISO646-US}, \n{US-ASCII (preferred MIME name)},
  \n{us}, \n{IBM367}, \n{cp367}, \n{csASCII}
\end{itemize}
\end{itemize}
}

\frame[containsverbatim]{
  \frametitle{Charactersets in HTML}
\begin{itemize}
\item Decimal or hex numerical code: \verb+&#32;+
\item Symbolic name:
  \verb+&copy;+ is the copyright symbol.
\item Use UTF-8 encoding; server states
\begin{verbatim}
Content-type: text/html;charset=utf-8
\end{verbatim}
or file starts with
\begin{verbatim}
<META HTTP-EQUIV="Content-Type"
    CONTENT="text/html;charset=utf-8">
\end{verbatim}
\end{itemize}
}

\frame[containsverbatim]{
  \frametitle{More places}
\begin{itemize}
\item Ftp: knows nothing, depends on client (line end, code page
  translation)
\item Email: mime encoding
\item Editors: emacs supports UTF-8
\item Programming languages: Windows NT/2000/XP, (inc Visual Basic),
  uses UCS-2 natively:\\
Strings are declared \n{wchar_t} instead of \n{char},\\
use \n{wcslen} instead of \n{strlen}\\
string is created as \verb+L"Hello world"+.
\end{itemize}
}

\sectionframe{Character issues in \TeX\ / \LaTeX}

\frame[containsverbatim]{
  \frametitle{Diacritics / accents}
\begin{itemize}
\item original \TeX: accent placed over character; raised/lowered, shifted to
  center, extra shift for italic\\
$\Rightarrow$ because of shifts, no hyphenation possible
\item 8-bit support in \TeX: possibility of fonts with accents,
  directly addressed
\end{itemize}
}

\frame[containsverbatim]{
  \frametitle{Old problems revisited}
\begin{itemize}
\item Input file allows 8-bit: dependence on code page
\begin{verbatim}
\usepackage[code]{applemac}
\end{verbatim}
\item (unprintable characters made active: definition adjusted
  dynamically)
\item Dependence on font organization left:
\begin{verbatim}
\usepackage[T1]{fontenc}
\end{verbatim}
\end{itemize}
}

\end{document}
\frame[containsverbatim]{
  \frametitle{}
\begin{itemize}
\item 
\end{itemize}
}

\frame[containsverbatim]{
  \frametitle{}
\begin{itemize}
\item 
\end{itemize}
}

\frame[containsverbatim]{
  \frametitle{}
\begin{itemize}
\item 
\end{itemize}
}

\frame[containsverbatim]{
  \frametitle{}
\begin{itemize}
\item 
\end{itemize}
}

\frame{
  \frametitle{}
\begin{itemize}
\item 
\end{itemize}
}

\frame{
  \frametitle{}
\begin{itemize}
\item 
\end{itemize}
}

\frame{
  \frametitle{}
\begin{itemize}
\item 
\end{itemize}
}

\frame{
  \frametitle{}
\begin{itemize}
\item 
\end{itemize}
}

\frame{
  \frametitle{}
\begin{itemize}
\item 
\end{itemize}
}

\frame{
  \frametitle{}
\begin{itemize}
\item 
\end{itemize}
}

\end{document}

\frame{
  \frametitle{}
\begin{itemize}
\item 
\end{itemize}
}

