\documentclass{beamer}

\usepackage{beamerthemevictor,comment,verbatim,graphicx,amssymb}

\newcommand{\TbT}{\TeX\ by Topic}
\newcommand{\Metafont}{\textsc{Metafont}}
\let\metafont\Metafont
\newcommand\ascii{{\sc Ascii}}
\newcommand\ebcdic{{\sc Ebcdic}}
\newcommand\gnuplot{\protect\n{gnuplot}}
\def\mod{\mathbin{\mathrm{mod}}}
\def\eqref#1{equation~(\ref{#1})}
\def\Eqref#1{Equation~(\ref{#1})}
\def\inv{^{-1}}
\def\endproof{\unskip\penalty10000 \hskip10pt plus 1fill$\bullet$\par}

\def\etal{\textit{et al.}}
\def\lex{{\it lex}}
\def\yacc{{\it yacc}}
\def\make{{\it make}}
\def\web{{\textsc{Web}}}
\def\n{\bgroup
  \catcode`\#=12 \catcode`\_=12 \catcode`\^=12
  \catcode`\&=12 
  \tt \let\next=}
\newcommand{\cs}[1]{{\tt\char`\\#1}}
{\catcode`\.=13
 \gdef.{${}_\bullet$}
}
\def\parserule#1{%
    \ifmmode \n{#1}\,\rightarrow\relax
    \else $\n{#1}\,\rightarrow{}$\nobreak
    \fi
    \hbox\bgroup\catcode`\.=13 \tt\let\next=}

\newenvironment{remark}%
    {\medskip\noindent{\bf Remark.}}{}
\usepackage{bnf}
\input bnf.env

\generalcomment{inputwithcode}
  {\begingroup\def\ProcessCutFile{}}
  {\verbatiminput{\CommentCutFile}
   \endgroup
   \input{\CommentCutFile}
  }
\generalcomment{examplewithcode}
  {\begingroup
     \def\ProcessCutFile{}\def\CommentCutFile{example.tex}}
  {\verbatiminput{\CommentCutFile}
   Output:
   \begin{quote}
     \begin{minipage}[t]{3in}
        \everypar{}
        \input{\CommentCutFile} 
     \end{minipage}
   \end{quote}
   \endgroup
  }
\specialcomment{ttexamplewithcode}
  {\begingroup\def\ProcessCutFile{}}
  {\verbatiminput{\CommentCutFile}
   \endgroup
   Output:
   \begin{quote}\begin{ttfamily} \input{\CommentCutFile} 
     \end{ttfamily}\end{quote}
  }
\generalcomment{mathexamplewithcode}
  {\begingroup\def\ProcessCutFile{}}
  {\verbatiminput{\CommentCutFile}
   Output:
   \begin{equation} \input{\CommentCutFile} \end{equation}
   \endgroup
  }

\def\chaptertitle{\csname\chaptername title\endcsname}
\def\chaptershorttitle{\csname\chaptername shorttitle\endcsname}

\def\lambdatitle{Lambda calculus in \TeX}
\def\lambdashorttitle{Lambda calculus}

\def\latextitle{Yet another \LaTeX\ tutorial}
\def\latexshorttitle{\LaTeX}

\def\lextitle{A \lex\ tutorial}
\def\lexshorttitle{\lex}

\def\yacctitle{A \yacc\ tutorial}
\def\yaccshorttitle{\yacc}

\expandafter\def\csname tex1title\endcsname{\TeX\ -- macro programming}
\expandafter\let\csname tex1shorttitle\expandafter\endcsname
    \csname tex1title\endcsname
\expandafter\def\csname tex2title\endcsname{\TeX\ -- visuals}
\expandafter\let\csname tex2shorttitle\expandafter\endcsname
    \csname tex2title\endcsname

\def\parsingtitle{An introduction to parsing}
\def\parsingshorttitle{Parsing}

\def\hashingtitle{Hashing}
\let\hashingshorttitle\hashingtitle

\def\beziertitle{Curve approximation by splines}
\def\beziershorttitle{Splines}

\def\dynamictitle{Dynamic programming}
\let\dynamicshorttitle\dynamictitle

\def\paragraphtitle{Line breaking}
\let\paragraphshorttitle\paragraphtitle

\def\pagetitle{Page breaking in \TeX}
\def\pageshorttitle{Page breaking}

\def\completenesstitle{Introduction to NP-Completeness}
\def\completenesshortstitle{NP-Completeness}

\def\rastertitle{Introduction to Raster Graphics}
\def\rastershortstitle{Raster Graphics}

\def\pythontitle{Really quick Python tutorial}
\def\pythonshortstitle{Python}

\def\encodingtitle{Character encoding}
\let\encodingshorttitle\encodingtitle

\def\rastertitle{Raster Graphics}
\let\rastershorttitle\rastertitle

\def\gnuplottitle{Gnuplot}
\let\gnuplotshorttitle\gnuplottitle

\def\fontstitle{Fonts and font files}
\def\fontsshorttitle{Fonts}

\def\softwaretitle{\TeX\ and Software Engineering}
\def\softwareshorttitle{Software Engineering}

\renewcommand{\beamervictortitle}{CS-594 Eijkhout, Fall 2004}

\newcommand{\sectionframe}[1]%
  {\section{#1}
   \frame{\begin{center}\color{blue}\Large #1\end{center}}
  }
\newcommand{\subsectionframe}[1]%
  {\subsection{#1}
   \frame{\begin{center}\color{blue}\Large #1\end{center}}
  }

\input idxmacs

\begin{document}

\title{Dynamic Programming}
\author{Victor Eijkhout}
\date{Notes for CS 594 -- Fall 2004}

\frame{\titlepage}

\section{Introduction}

\frame[containsverbatim]{
  \frametitle{What is dynamic programming?}
\begin{itemize}
\item Solution technique for minization problems
\item Often lower complexity than naive techniques
\item Sometimes equivalent to analytical techniques
\end{itemize}
}

\frame[containsverbatim]{
  \frametitle{What is it \emph{not}?}
\begin{itemize}
\item Black box that will solve your problem
\item Way of finding lowest complexity
\end{itemize}
}

\frame[containsverbatim]{
  \frametitle{When dynamic programming?}
\begin{itemize}
\item Minimization problems
\item constraints; especially integer
\item sequence of decisions
\end{itemize}
}

\section{Examples}
\subsectionframe{Decision timing}

\frame[containsverbatim]{
  \frametitle{Description}
\begin{itemize}
\item Occasions for deciding yes/no
\item items with attractiveness $\in[0,1]$
\item Finite set
\item no reconsidering
\item Question: at any given step, do you choose or pass?
\item Objective: maximize expectation
\end{itemize}
}

\frame{
  \frametitle{crucial idea}
\begin{itemize}
\item start from the end:
\item step $N$: no choice left
\item expected yield: $.5$
\item<2-> in step $N-1$: pick if better than $.5$
\end{itemize}
}

\frame{
  \frametitle{yield from $N-1$}
\begin{itemize}
\item pick if $>.5$: done in $.5$ of the cases
\item<2-> expected yield $.75$
\item<3-> go on in $.5$ of the cases
\item<4-> expected yield then $.5$
\item<5-> total expected yield: $.5\times.75+.5\times.5=.625$
\end{itemize}
}

\frame{
  \frametitle{at $N-2$}
\begin{itemize}
\item pick if better than $.625$
\item<2-> happens in $.375$ of the cases,
\item<3-> yield in that case $1.625/2$
\item<4-> otherwise, $.625$ yield from later choice
\item<4-> et cetera
\end{itemize}
}

\frame[containsverbatim]{
  \frametitle{Essential features}
\begin{itemize}
\item Stages: more or less independent decisions
\item Global minimization; solving by subproblems
\item Principle of optimality: sub part of total solution is optimal
  solution of sub problem
\end{itemize}
}

\subsectionframe{Manufacturing problem}
\frame{
  \frametitle{Statement}
\begin{itemize}
\item Total to be produced in given time, variable cost in each time
  period
\item wanted: scheduling
\item
\[ \min_{\sum p_k=S}\sum w_kp_k^2.\]
\end{itemize}
}

\frame[containsverbatim]{
  \frametitle{define concepts}
\begin{itemize}
\item amount of work to produce~$s$ in $n$~steps:
\[ v(s|n)=\min_{\sum_{k>N-n} p_k=s}\sum w_kp_k^2 \]
\item optimal amount $p(s|n)$ at $n$~months from the end
\end{itemize}
}

\frame[containsverbatim]{
  \frametitle{principle of optimality}
\begin{eqnarray*}
    v(s|n)&=&\min_{p_n\leq s}\left\{w_np_n^2
        +\sum_{{k>N-n+1\atop\sum p_k=s-p_n}}w_kp_k^2\right\} \\
        &=&\min_{p_n\leq s}\left\{ w_np_n^2+v(s-p_n|n-1)\right\}
\end{eqnarray*}
}

\frame{
  \frametitle{start from the end}
\begin{itemize}
\item In the last period: $p(s|1)=s$, and $v(s|1)=w_1s^2$
\item<2-> period before:
\[ v(s|2)=\min_{p_2}\{w_2p_2^2+v(s-p_2|1)\}=\min_{p_2}c(s,p_2) \]
where $c(s,p_2)=w_2p_2^2+w_1(s-p_2)^2$.
\item<3-> Minimize: $\delta c(s,p_2)/\delta p_2=0$,
\item<3-> then
$p(s|2)=w_1s/(w_1+w_2)$ and $v(s|2)=w_1w_2s^2/(w_1+w_2)$.
\end{itemize}
}

\frame{
  \frametitle{general form}
\begin{itemize}
\item Inductively
 \[ p(s|n)={1/w_n\over \sum_{i=1}^n 1/w_i}s,\qquad
    v(s|n)=s^2\sum_{i=1}^n1/w_i.\]
\item<2-> Variational approach:
\[ \sum_kw_kp_k^2+\lambda(\sum_kp_k-S) \]
Constrained minimization
\item<3-> Solve by setting derivatives to $p_n$ and~$\lambda$ to zero.
\end{itemize}
}

\frame{
  \frametitle{characteristics}
\begin{itemize}
\item Stages: time periods
\item State: amount of good left to be produced
\item Principle of optimality
\item<2-> Can be solved analytically
\item<3-> Analytical approach can not deal with integer constraints
\end{itemize}
}

\subsectionframe{Stagecoach problem}

\frame[containsverbatim]{
  \frametitle{Statement}
Several routes from beginning to end\\
Cost of travel insurance:

\convertMPtoPDF{stages.1}{1}{1}

Objective: minimize cost
}

\frame[containsverbatim]{
  \frametitle{data in python}
\begin{verbatim}
table = [ [0, 5, 4, 0, 0, 0, 0, 0, 0], # first stage: 0
          [0, 0, 0, 1, 3, 4, 0, 0, 0], # second: 1 & #2
          [0, 0, 0, 4, 2, 2, 0, 0, 0],
          [0, 0, 0, 0, 0, 0, 5, 1, 0], # third: 3, #4, #5
          [0, 0, 0, 0, 0, 0, 2, 4, 0],
          [0, 0, 0, 0, 0, 0, 4, 3, 0],
          [0, 0, 0, 0, 0, 0, 0, 0, 5], # fourth: 6 & #7
          [0, 0, 0, 0, 0, 0, 0, 0, 2]
          ]
final = len(table); 
\end{verbatim}
}

\frame{
  \frametitle{recursive formulation}
\begin{itemize}
\item In the final city, the cost is zero;
\item Otherwise minimum over all cities reachable\\
  of the cost of the next leg plus the minimum cost from that city.\\
(principle of optimality)
\item<2-> wrong to code it recursively
\end{itemize}
}

\frame[containsverbatim]{
  \frametitle{recursive solution in python}
\footnotesize
\hypertarget{rec-coach}{}
\begin{verbatim}
def cost_from(n):
    # if you're at the end, it's free
    if n==final: return 0
    # otherwise range over cities you can reach
    # and keep minimum value
    val = 0
    for m in range(n+1,final+1):
        local_cost = table[n][m]
        if local_cost>0:
            # if there is a connection from here,
            # compute the minimum cost
            local_cost += cost_from(m)
            if val==0 or local_cost<val:
                val = local_cost
    return val
print "recursive minimum cost is",cost_from(0)
\end{verbatim}
\hyperlink{back-coach}{\beamerbutton{backward}}
}

\frame[containsverbatim]{
  \frametitle{characteristic}
\begin{itemize}
\item Overlapping subproblems\\
\convertMPtoPDF{123.1}{1}{1}
\item \n{cost_from(1)} computed twice
\item Cost: $N$~cities, $S$~stages of $L$~each: $O(L^S)$
\end{itemize}
}

\frame[containsverbatim]{
  \frametitle{dynamic programming}
\begin{itemize}
\item Compute minimum cost $f_n(x_n)$\\
of traveling from step~$n$, starting in $x_n$ (state variable)
\item 
Formally, $f_k(s)$ minimum cost for traveling from stage~$k$\\
starting in city~$s$. Then
\[ f_{k-1}(s)=\min_t\{ c_{st}+f_k(t) \]
where $c_{st}$ cost of traveling from city~$s$ to~$t$.
\end{itemize}
}

\frame[containsverbatim]{
  \frametitle{backward dynamic solution in python}
\footnotesize
\hypertarget{back-coach}{}
\begin{verbatim}
cost = (final+1)*[0] # initialization
# compute cost backwards
for t in range(final-1,-1,-1):
    # computing cost from t
    for i in range(final+1):
        local_cost = table[t][i]
        if local_cost==0: continue
        local_cost += cost[i]
        if cost[t]==0 or local_cost<cost[t]:
            cost[t] = local_cost
print "minimum cost:",cost[0]
\end{verbatim}
\hyperlink{rec-coach}{\beamerbutton{recursive}}
\hyperlink{forw-coach}{\beamerbutton{forward}}
}

\frame[containsverbatim]{
  \frametitle{analysis}
\begin{itemize}
\item Running time $O(N\cdot L)$ or~$O(L^2S)$
\item compare $L^S$ for recursive
\end{itemize}
}

\frame{
  \frametitle{Forward solution}
\begin{itemize}
\item Backward: $f_n(x)$ cost from $x$ in $n$ steps to the end
\item Forward: $f_n(x)$ cost in $n$ steps to $x$
\[ f_n(t) =\min_{s<t}\{c_{st}+f_{n-1}(s)\} \]
\item sometimes more appropriate
\item same complexity
\end{itemize}
}

\frame[containsverbatim]{
  \frametitle{forward dynamic solution in python}
\footnotesize
\hypertarget{forw-coach}{}
\begin{verbatim}
cost = (final+1)*[0]
for t in range(final):
    for i in range(final+1):
        local_cost = table[t][i]
        if local_cost == 0: continue
        cost_to_here = cost[t]
        newcost = cost_to_here+local_cost
        if cost[i]==0 or newcost<cost[i]:
            cost[i] = newcost
print "cost",cost[final]
\end{verbatim}
\hyperlink{back-coach}{\beamerbutton{backward}}
}

\subsectionframe{Traveling salesman}

\frame{
  \frametitle{Problem statement}
\begin{itemize}
\item Cities as stages?
\item<2-> No ordering
\item<3-> Stage $n$ : having $n$ cities left
\item<4-> State: combination of cities to visit plus current city
\item<4-> Cost formula to~0 (both start/end)
\begin{eqnarray*}
C(\{\,\},f)&=&a_{f0}\quad\hbox{for $f=1,2,3,\ldots$}\\
C(S,f)&=&\min_{m\in S}a_{fm}+\left[C(S-m,m)]\right]
\end{eqnarray*}
\end{itemize}
}

\frame[containsverbatim]{
  \frametitle{backward implementation}
\footnotesize
\begin{verbatim}
def shortest_path(start,through,lev):
    if len(through)==0:
        return table[start][0]
    l = 0
    for dest in through:
        left = through[:]; left.remove(dest)
        ll = table[start][dest]+shortest_path(dest,left,lev+1)
        if l==0 or ll<l:
            l = ll
    return l
to_visit = range(1,ntowns);
s = shortest_path(0,to_visit,0)
\end{verbatim}
(recursive; need to be improved)
}

\sectionframe{Discussion}

\frame[containsverbatim]{
  \frametitle{Characteristics}
\begin{description}
\item[Stages] sequence of choices
\item[Stepwise solution] solution by successive subproblems
\item[State] cost function has a state parameter,\\
description of work left~\&c
\item[Overlapping subproblems]
\item[Principle of optimality] This is the property that the
  restriction of a global solution to a subset of the stages is also
  an optimal solution for that subproblem.
\end{description}
}

\frame[containsverbatim]{
  \frametitle{Principle of optimality}
\begin{quote}
An optimal policy has the property that whatever the initial state and
initial decision are, the remaining decisions must be an optimal
policy with regard to the state resulting from the first decision.
\end{quote}
}

\frame[containsverbatim]{
  \frametitle{derivation}
Maximize~$\sum^N_ig_i(x_i)$ under~$\sum_ix_i=\nobreak X$, $x_i\geq0$\\
Call this~$f_N(X)$, then
\begin{eqnarray*}
f_N(X)&=&\max_{\sum_i^Nx_i=X}\sum_i^Ng_i(x_i)\\
&=&\max_{x_N<X}\left\{g_N(x_N)+\max_{\sum_i^{N-1}x_i=X-x_N}\sum_i^{N-1}g_i(x_i)\right\}\\
&=&\max_{x_N<X}\left\{g_N(x_N)+f_{N-1}(X-x_N)\right\}
\end{eqnarray*}
}

\end{document}
