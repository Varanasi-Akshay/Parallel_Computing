\documentclass{beamer}

\usepackage{beamerthemevictor,comment,verbatim,graphicx,amssymb}

\newcommand{\TbT}{\TeX\ by Topic}
\newcommand{\Metafont}{\textsc{Metafont}}
\let\metafont\Metafont
\newcommand\ascii{{\sc Ascii}}
\newcommand\ebcdic{{\sc Ebcdic}}
\newcommand\gnuplot{\protect\n{gnuplot}}
\def\mod{\mathbin{\mathrm{mod}}}
\def\eqref#1{equation~(\ref{#1})}
\def\Eqref#1{Equation~(\ref{#1})}
\def\inv{^{-1}}
\def\endproof{\unskip\penalty10000 \hskip10pt plus 1fill$\bullet$\par}

\def\etal{\textit{et al.}}
\def\lex{{\it lex}}
\def\yacc{{\it yacc}}
\def\make{{\it make}}
\def\web{{\textsc{Web}}}
\def\n{\bgroup
  \catcode`\#=12 \catcode`\_=12 \catcode`\^=12
  \catcode`\&=12 
  \tt \let\next=}
\newcommand{\cs}[1]{{\tt\char`\\#1}}
{\catcode`\.=13
 \gdef.{${}_\bullet$}
}
\def\parserule#1{%
    \ifmmode \n{#1}\,\rightarrow\relax
    \else $\n{#1}\,\rightarrow{}$\nobreak
    \fi
    \hbox\bgroup\catcode`\.=13 \tt\let\next=}

\newenvironment{remark}%
    {\medskip\noindent{\bf Remark.}}{}
\usepackage{bnf}
\input bnf.env

\generalcomment{inputwithcode}
  {\begingroup\def\ProcessCutFile{}}
  {\verbatiminput{\CommentCutFile}
   \endgroup
   \input{\CommentCutFile}
  }
\generalcomment{examplewithcode}
  {\begingroup
     \def\ProcessCutFile{}\def\CommentCutFile{example.tex}}
  {\verbatiminput{\CommentCutFile}
   Output:
   \begin{quote}
     \begin{minipage}[t]{3in}
        \everypar{}
        \input{\CommentCutFile} 
     \end{minipage}
   \end{quote}
   \endgroup
  }
\specialcomment{ttexamplewithcode}
  {\begingroup\def\ProcessCutFile{}}
  {\verbatiminput{\CommentCutFile}
   \endgroup
   Output:
   \begin{quote}\begin{ttfamily} \input{\CommentCutFile} 
     \end{ttfamily}\end{quote}
  }
\generalcomment{mathexamplewithcode}
  {\begingroup\def\ProcessCutFile{}}
  {\verbatiminput{\CommentCutFile}
   Output:
   \begin{equation} \input{\CommentCutFile} \end{equation}
   \endgroup
  }

\def\chaptertitle{\csname\chaptername title\endcsname}
\def\chaptershorttitle{\csname\chaptername shorttitle\endcsname}

\def\lambdatitle{Lambda calculus in \TeX}
\def\lambdashorttitle{Lambda calculus}

\def\latextitle{Yet another \LaTeX\ tutorial}
\def\latexshorttitle{\LaTeX}

\def\lextitle{A \lex\ tutorial}
\def\lexshorttitle{\lex}

\def\yacctitle{A \yacc\ tutorial}
\def\yaccshorttitle{\yacc}

\expandafter\def\csname tex1title\endcsname{\TeX\ -- macro programming}
\expandafter\let\csname tex1shorttitle\expandafter\endcsname
    \csname tex1title\endcsname
\expandafter\def\csname tex2title\endcsname{\TeX\ -- visuals}
\expandafter\let\csname tex2shorttitle\expandafter\endcsname
    \csname tex2title\endcsname

\def\parsingtitle{An introduction to parsing}
\def\parsingshorttitle{Parsing}

\def\hashingtitle{Hashing}
\let\hashingshorttitle\hashingtitle

\def\beziertitle{Curve approximation by splines}
\def\beziershorttitle{Splines}

\def\dynamictitle{Dynamic programming}
\let\dynamicshorttitle\dynamictitle

\def\paragraphtitle{Line breaking}
\let\paragraphshorttitle\paragraphtitle

\def\pagetitle{Page breaking in \TeX}
\def\pageshorttitle{Page breaking}

\def\completenesstitle{Introduction to NP-Completeness}
\def\completenesshortstitle{NP-Completeness}

\def\rastertitle{Introduction to Raster Graphics}
\def\rastershortstitle{Raster Graphics}

\def\pythontitle{Really quick Python tutorial}
\def\pythonshortstitle{Python}

\def\encodingtitle{Character encoding}
\let\encodingshorttitle\encodingtitle

\def\rastertitle{Raster Graphics}
\let\rastershorttitle\rastertitle

\def\gnuplottitle{Gnuplot}
\let\gnuplotshorttitle\gnuplottitle

\def\fontstitle{Fonts and font files}
\def\fontsshorttitle{Fonts}

\def\softwaretitle{\TeX\ and Software Engineering}
\def\softwareshorttitle{Software Engineering}

\renewcommand{\beamervictortitle}{CS-594 Eijkhout, Fall 2004}

\newcommand{\sectionframe}[1]%
  {\section{#1}
   \frame{\begin{center}\color{blue}\Large #1\end{center}}
  }
\newcommand{\subsectionframe}[1]%
  {\subsection{#1}
   \frame{\begin{center}\color{blue}\Large #1\end{center}}
  }

\input idxmacs

\begin{document}

\title{Line breaking}
\author{Victor Eijkhout}
\date{Notes for CS 594 -- Fall 2004}

\frame{\titlepage}

\section{Basic problem}

\frame[containsverbatim]{
  \frametitle{}
\begin{itemize}
\item Problem: break horizontal list into lines
\item $n$ words: $2^n$ possibilities
\item (much) better is possible
\item Aim: visually even `colour'
\end{itemize}
}

\frame[containsverbatim]{
  \frametitle{What is a paragraph?}
\begin{itemize}
\item Boxes: words, math, other solid objects
\item Glue: white space, possibly with stretch/shrink
\item Penalties: prevent or force breaks
\end{itemize}
}

\frame{
  \frametitle{Boxes}
\begin{itemize}
\item Words, formulas, actual boxes
\item described by height, width,depth
\item (hyphenation)
\end{itemize}
}

\frame[containsverbatim]{
  \frametitle{Penalties}
\begin{itemize}
\item Inserted by user, macro, automatically
\item Example: \verb+\def~{\penalty10000 \ }+
\item Automatic: paragraph ends with \verb+\unskip\penalty10000\hfill+
\item also: penalties for two consecutive hyphenated lines; last full
  line hyphenated
\end{itemize}
}

\frame[containsverbatim]{
  \frametitle{Glue}
\begin{itemize}
\item Denotations: \verb+\hskip 2cm plus 1cm minus .5cm+
\item Automatically inserted: \cs{leftskip}, \cs{abovedisplayskip}
\item Adding glue together: \n{2cm plus 1cm} and \n{2cm plus -1cm}
  total no stretch
\item Infinite glue: \cs{hfill} or \verb+\hskip 0pt plus 1fil+
  (\n{fill}, \n{filll})
\item infinite glue present: all other stretch/shrink ignored
\end{itemize}
}

\frame[containsverbatim]{
  \frametitle{Line break locations}
\begin{itemize}
\item Foremost: at a glue, but only if preceeded by a non-discardable
\item At a hyphen or hyphenation location
\item At a penalty (this can override the above clauses)
\item (complicated rules for defining extent of a word)
\end{itemize}
}

\sectionframe{Examples}

\frame[containsverbatim]{
  \frametitle{Centered text}
\begin{examplewithcode}
\begin{minipage}{4cm}
\leftskip=0pt plus 1fil \rightskip=0pt plus 1fil
\parfillskip=0pt
This paragraph puts infinitely stretchable glue at 
the left and right of each line.
The effect is that the lines will be centered.
\end{minipage}
\end{examplewithcode}
}

\frame[containsverbatim]{
  \frametitle{Centered last line}
\begin{examplewithcode}
\begin{minipage}{5cm}
\leftskip=0pt plus 1fil \rightskip=0pt plus -1fil
\parfillskip=0pt plus 2fil
This paragraph puts infinitely stretchable glue at 
the left and right of each line, but the amounts cancel out.
The parfillskip on the last line changes that.
\end{minipage}
\end{examplewithcode}
}

\frame[containsverbatim]{
  \frametitle{Hanging punctuation}
\begin{itemize}
\item Put punctuation in the right margin
\item Make right margin look more straight/solid
\pgfimage{hang}
\end{itemize}
}

\frame[containsverbatim]{
  \frametitle{hanging punctuation code}
\begin{verbatim}
\newbox\pbox \newbox\cbox
\setbox\pbox\hbox{.} \wd\pbox=0pt
\setbox\cbox\hbox{,} \wd\cbox=0pt
\newdimen\csize \csize=\wd\cbox
\newdimen\psize \psize=\wd\pbox

\catcode`,=13 \catcode`.=13
\def,{\copy\cbox \penalty0 \hskip\csize\relax}
\def.{\copy\pbox \penalty0 \hskip\psize\relax}
\end{verbatim}
}

\frame[containsverbatim]{
  \frametitle{Mathematical Reviews}
\begin{itemize}
\item Reviewer signature separated from review text
\item but if possible on the same line
\pgfimage{mr}
\end{itemize}
}

\frame[containsverbatim]{
  \frametitle{mathematical reviews code}
\begin{verbatim}
\def\signed#1{\unskip
  \penalty10000 \hskip 40pt plus 1fill
  \penalty0
  \hbox{}\penalty10000
                \hskip 0pt plus 1fill
  \hbox{#1}%
                \par
  }
\end{verbatim}
}

\sectionframe{\TeX's line breaking algorithm}

\subsection{Glue setting}

\frame[containsverbatim]{
  \frametitle{Glue ratio}
\begin{itemize}
\item Consider a line, and desired length
\item compare to natural width, stretch, shrink
\[ \rho=\begin{cases}
    0&\ell=L\\ (\ell-L)/X&\mbox{(stretch:) $\ell>L$ and $X>0$}\\
    (\ell-L)/Y&\mbox{(shrink:) $\ell<L$ and $Y>0$}\\
    \mathrm{undefined}&\mathrm{otherwise}
        \end{cases}
\]
\item note: negative for shrink
\end{itemize}
}

\frame[containsverbatim]{
  \frametitle{Badness}
\begin{itemize}
\item Stretching/shrinking too far is `bad':
\[ b = \begin{cases}
          10\,000&\mbox{$\rho<-1$ or undefined}\\
          \min\left\{10\,000, 100|\rho|^3\right\}&\mbox{otherwise}
       \end{cases}
\]
\item Stretch beyond 100\% possible, shrink not.
\item Categories:
\begin{itemize}
\item[\textit{tight (3)}] if it has shrunk with $b\geq 13$
\item[\textit{decent (2)}] if $b\leq12$
\item[\textit{loose (1)}] if it has stretched with $100>b\geq 13$
\item[\textit{very loose (0)}] if it has stretched with $b\geq100$
\end{itemize}
Note $100\times(1/2)^3=12.5$
\item discourage `visually incompatible' lines
\end{itemize}
}

\frame[containsverbatim]{
  \frametitle{Demerits}
\begin{itemize}
\item Add together badness and penalties
\item \cs{linepenalty}, \cs{doublehyphendemerits}, \cs{finalhyphendemerits}
\end{itemize}
}

\subsectionframe{Breaking strategies}

\frame[containsverbatim]{
  \frametitle{First fit}
\begin{itemize}
\item Wait until word crosses the margin, then
\item if the line can be shrunk: shrink
\item otherwise, if it can be stretched: stretch
\item otherwise, try hyphenating
\item otherwise, really stretch
\end{itemize}
}

\frame[containsverbatim]{
  \frametitle{Best fit and total fit}
\begin{itemize}
\item (Best fit) Decide between stretch/shrink/hyphenate based on badness
\item (Total fit) Add all badnesses together; minimize over whole paragraph
\end{itemize}
}

\subsection{Implementation}

\frame{
  \frametitle{Program structure}
\begin{itemize}
\item Loop over all words, check if feasible breakpoint
\item<2-> Inner loop: over all previous words, check if feasible start
  of line
\item<3-> Optimization: cutoff on previous words
\item<4-> `active list'
\end{itemize}
}

\frame[containsverbatim]{
  \frametitle{Main program}
\footnotesize
\begin{verbatim}
active = [0]; nwords = len(paragraph)
for w in range(1,nwords):
    # compute the cost of breaking after word w
    print "Recent word",w
    for a in active:
        line = paragraph[a:w+1]
        if w==nwords-1:
            ratio = 0 # last line will be set perfect
        else:
            ratio = compute_ratio(line)
            print "..line=",line,"; ratio=",ratio
        if ratio<-1:
            active.remove(a)
            print "active point",a,"removed"
        else:
            update_cost(a,w,ratio)
    report_cost(w)
    active.append(w)
    print
\end{verbatim}
}

\subsection{First fit, best fit}
\frame[containsverbatim]{
  \frametitle{Data structure}
\begin{verbatim}
def init_costs():
    global cost
    cost = len(paragraph)*[0]
    for i in range(len(paragraph)):
        cost[i] = {'cost':0, 'from':0}
    cost[0] = {'cost':10000, 'from':-1}
\end{verbatim}
}

\frame[containsverbatim]{
  \frametitle{Cost function; first fit}
\begin{verbatim}
def update_cost(a,w,ratio):
    global cost
    if a>0 and cost[a-1]['cost']<10000:
        if ratio<=1 and ratio>=-1:
            to_here = abs(ratio)
        else: to_here = 10000
        if cost[w]['cost']==0 or to_here<cost[w]['cost']:
            cost[w]['cost'] = to_here
            cost[w]['from'] = a-1
\end{verbatim}
}

\frame[containsverbatim]{
  \frametitle{First fit, dynamic version}
\begin{verbatim}
def update_cost(a,w,ratio):
    global cost
    if ratio<=1 and ratio>=-1:
        to_here = abs(ratio)
    else: to_here = 10000
    if a>0:
        from_there = cost[a-1]['cost']
        to_here = to_here+from_there
    else: from_there = 0
    if cost[w]['cost']==0 or to_here<cost[w]['cost']:
        cost[w]['cost'] = to_here
        cost[w]['from'] = a-1
\end{verbatim}
}

\frame[containsverbatim]{
  \frametitle{example}
\begin{footnotesize}
\begin{verbatim}
You may  never  have  thought  of  it,  but  fonts  (better:   -0.111111111111
typefaces) usually have a mathematical  definition  somehow.   -0.666666666667
If   a   font   is   given   as   bitmap,   this   is  often   0.888888888889
a  result  originating  from  a  more  compact  description.   0.0
Imagine the situation that you have bitmaps at  300dpi,  and   -0.777777777778
you   buy   a  600dpi  printer.  It  wouldn't  look  pretty.   0.25
There   is   then   a   need   for  a  mathematical  way  of   0.555555555556
describing  arbitrary  shapes.  These  shapes  can  also  be   0.0
three-dimensional; in fact,  a~lot  of  the  mathematics  in   -0.285714285714
this  chapter  was  developed  by  a  car  manufacturer  for   0.0
modeling   car   body  shapes.  But  let  us  for  now  only   0.222222222222
look   in  two  dimensions,  which  means  that  the  curves   0.125
are  lines,  rather  than  planes.
\end{verbatim}
\end{footnotesize}
}

\frame[containsverbatim]{
  \frametitle{Best fit cost function}
\begin{verbatim}
def update_cost(a,w,ratio):
    global cost
    to_here = 100*abs(ratio)**2
    if a>0:
        from_there = cost[a-1]['cost']
        to_here = to_here+from_there
    else: from_there = 0
    if cost[w]['cost']==0 or to_here<cost[w]['cost']:
        cost[w]['cost'] = to_here; cost[w]['from'] = a-1
\end{verbatim}
}

\frame[containsverbatim]{
  \frametitle{example}
\begin{footnotesize}
\begin{verbatim}
You may  never  have  thought  of  it,  but  fonts  (better:   -0.111111111111
typefaces) usually have a mathematical  definition  somehow.   -0.666666666667
If   a   font   is   given   as  bitmap,  this  is  often  a   0.5
result   originating   from   a  more  compact  description.   0.5
Imagine the situation that you have bitmaps at  300dpi,  and   -0.777777777778
you   buy   a  600dpi  printer.  It  wouldn't  look  pretty.   0.25
There   is   then   a   need   for  a  mathematical  way  of   0.555555555556
describing  arbitrary  shapes.  These  shapes  can  also  be   0.0
three-dimensional; in fact,  a~lot  of  the  mathematics  in   -0.285714285714
this  chapter  was  developed  by  a  car  manufacturer  for   0.0
modeling   car   body  shapes.  But  let  us  for  now  only   0.222222222222
look   in  two  dimensions,  which  means  that  the  curves   0.125
are  lines,  rather  than  planes.
\end{verbatim}
\end{footnotesize}
}

\subsection{Total fit}
\frame[containsverbatim]{
  \frametitle{Total fit, data structure}
\begin{verbatim}
def init_costs():
    global cost
    nul = [0,0,0]
    cost = len(paragraph)*[ 0 ]
    for i in range(len(paragraph)):
        cost[i] = nul[:]
        for j in range(3):
            cost[i][j] = {'cost':10000, 'from':-2}
    for j in range(3):
        cost[0][j] = {'cost':10000, 'from':-1}
\end{verbatim}
}

\frame[containsverbatim]{
  \frametitle{cost function}
\begin{verbatim}
def update_cost(a,w,ratio):
    global cost
    type = stretch_type(ratio)
    to_here = 100*abs(ratio)**2
    if a>0:
        [from_there,from_type] = minimum_cost_and_type(a-1)
        to_here += from_there
    else: from_there = 0
    if cost[w][type]['cost']==0 or\
       to_here<cost[w][type]['cost']:
        cost[w][type]['cost'] = to_here;
        cost[w][type]['from'] = a-1
\end{verbatim}
}

\frame[containsverbatim]{
  \frametitle{example}
\begin{footnotesize}
\begin{verbatim}
You may  never  have  thought  of  it,  but  fonts  (better:   -0.111111111111
typefaces)   usually   have   a   mathematical   definition    1.2
somehow. If a font is given  as  bitmap,  this  is  often  a   -0.454545454545
result   originating   from   a  more  compact  description.   0.5
Imagine   the   situation   that   you   have   bitmaps   at   1.0
300dpi, and you buy  a  600dpi  printer.  It  wouldn't  look   -0.333333333333
pretty. There is then a  need  for  a  mathematical  way  of   -0.4
describing  arbitrary  shapes.  These  shapes  can  also  be   0.0
three-dimensional; in fact,  a~lot  of  the  mathematics  in   -0.285714285714
this  chapter  was  developed  by  a  car  manufacturer  for   0.0
modeling   car   body  shapes.  But  let  us  for  now  only   0.222222222222
look   in  two  dimensions,  which  means  that  the  curves   0.125
are  lines,  rather  than  planes.
\end{verbatim}
\end{footnotesize}
}

\end{document}
