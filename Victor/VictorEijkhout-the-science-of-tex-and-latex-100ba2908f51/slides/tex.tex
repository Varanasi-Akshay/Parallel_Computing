\documentclass{beamer}

\usepackage{beamerthemevictor,comment,verbatim,graphicx,rotating}
\newcommand{\sectionframe}[1]%
  {\section{#1}
   \frame{\begin{center}\color{blue}\Large #1\end{center}}
  }
\newcommand{\subsectionframe}[1]%
  {\subsection{#1}
   \frame{\begin{center}\color{blue}\Large #1\end{center}}
  }

\newcommand{\TbT}{\TeX\ by Topic}
\newcommand{\Metafont}{\textsc{Metafont}}
\let\metafont\Metafont
\newcommand\ascii{{\sc Ascii}}
\newcommand\ebcdic{{\sc Ebcdic}}
\newcommand\gnuplot{\protect\n{gnuplot}}
\def\mod{\mathbin{\mathrm{mod}}}
\def\eqref#1{equation~(\ref{#1})}
\def\Eqref#1{Equation~(\ref{#1})}
\def\inv{^{-1}}
\def\endproof{\unskip\penalty10000 \hskip10pt plus 1fill$\bullet$\par}

\def\etal{\textit{et al.}}
\def\lex{{\it lex}}
\def\yacc{{\it yacc}}
\def\make{{\it make}}
\def\web{{\textsc{Web}}}
\def\n{\bgroup
  \catcode`\#=12 \catcode`\_=12 \catcode`\^=12
  \catcode`\&=12 
  \tt \let\next=}
\newcommand{\cs}[1]{{\tt\char`\\#1}}
{\catcode`\.=13
 \gdef.{${}_\bullet$}
}
\def\parserule#1{%
    \ifmmode \n{#1}\,\rightarrow\relax
    \else $\n{#1}\,\rightarrow{}$\nobreak
    \fi
    \hbox\bgroup\catcode`\.=13 \tt\let\next=}

\newenvironment{remark}%
    {\medskip\noindent{\bf Remark.}}{}
\usepackage{bnf}
\input bnf.env

\generalcomment{inputwithcode}
  {\begingroup\def\ProcessCutFile{}}
  {\verbatiminput{\CommentCutFile}
   \endgroup
   \input{\CommentCutFile}
  }
\generalcomment{examplewithcode}
  {\begingroup
     \def\ProcessCutFile{}\def\CommentCutFile{example.tex}}
  {\verbatiminput{\CommentCutFile}
   Output:
   \begin{quote}
     \begin{minipage}[t]{3in}
        \everypar{}
        \input{\CommentCutFile} 
     \end{minipage}
   \end{quote}
   \endgroup
  }
\specialcomment{ttexamplewithcode}
  {\begingroup\def\ProcessCutFile{}}
  {\verbatiminput{\CommentCutFile}
   \endgroup
   Output:
   \begin{quote}\begin{ttfamily} \input{\CommentCutFile} 
     \end{ttfamily}\end{quote}
  }
\generalcomment{mathexamplewithcode}
  {\begingroup\def\ProcessCutFile{}}
  {\verbatiminput{\CommentCutFile}
   Output:
   \begin{equation} \input{\CommentCutFile} \end{equation}
   \endgroup
  }

\def\chaptertitle{\csname\chaptername title\endcsname}
\def\chaptershorttitle{\csname\chaptername shorttitle\endcsname}

\def\lambdatitle{Lambda calculus in \TeX}
\def\lambdashorttitle{Lambda calculus}

\def\latextitle{Yet another \LaTeX\ tutorial}
\def\latexshorttitle{\LaTeX}

\def\lextitle{A \lex\ tutorial}
\def\lexshorttitle{\lex}

\def\yacctitle{A \yacc\ tutorial}
\def\yaccshorttitle{\yacc}

\expandafter\def\csname tex1title\endcsname{\TeX\ -- macro programming}
\expandafter\let\csname tex1shorttitle\expandafter\endcsname
    \csname tex1title\endcsname
\expandafter\def\csname tex2title\endcsname{\TeX\ -- visuals}
\expandafter\let\csname tex2shorttitle\expandafter\endcsname
    \csname tex2title\endcsname

\def\parsingtitle{An introduction to parsing}
\def\parsingshorttitle{Parsing}

\def\hashingtitle{Hashing}
\let\hashingshorttitle\hashingtitle

\def\beziertitle{Curve approximation by splines}
\def\beziershorttitle{Splines}

\def\dynamictitle{Dynamic programming}
\let\dynamicshorttitle\dynamictitle

\def\paragraphtitle{Line breaking}
\let\paragraphshorttitle\paragraphtitle

\def\pagetitle{Page breaking in \TeX}
\def\pageshorttitle{Page breaking}

\def\completenesstitle{Introduction to NP-Completeness}
\def\completenesshortstitle{NP-Completeness}

\def\rastertitle{Introduction to Raster Graphics}
\def\rastershortstitle{Raster Graphics}

\def\pythontitle{Really quick Python tutorial}
\def\pythonshortstitle{Python}

\def\encodingtitle{Character encoding}
\let\encodingshorttitle\encodingtitle

\def\rastertitle{Raster Graphics}
\let\rastershorttitle\rastertitle

\def\gnuplottitle{Gnuplot}
\let\gnuplotshorttitle\gnuplottitle

\def\fontstitle{Fonts and font files}
\def\fontsshorttitle{Fonts}

\def\softwaretitle{\TeX\ and Software Engineering}
\def\softwareshorttitle{Software Engineering}

\input idxmacs

\begin{document}

\title{\TeX--slides in progress}
\author{Victor Eijkhout}
\date{Notes for CS 594 -- Fall 2004}

\frame{\titlepage}

\sectionframe{The input processor}

\frame[containsverbatim]{
  \frametitle{From file to lines}
\begin{itemize}
\item Lines lifted from file, minus line end
\item Trailing spaces removed
\item \cs{endlinechar} appended, if~0--255, default~13
\item accessing all characters: \verb+^+$n$ with~$n<128$ replaced by
  character $\mod(n+64,128)$; or \verb+^^xy+ with \n{x},~\n{n}
  lowercase hex replaced by character \n{xy}.
\end{itemize}
}

\frame[containsverbatim]{
  \frametitle{Category codes}
\begin{itemize}
\item Special characters are dynamic: character code to category code
  mapping during scanning of the line
\item example: \verb+\catcode36=3+, or \verb+\catcode`\$=3+
\item Assignment holds immediately!
\begin{examplewithcode}
Normal math $n=1$, \catcode`\/=3 /x^2+y^2/.
\end{examplewithcode}
\end{itemize}
}

\frame[containsverbatim]{
  \frametitle{Usual catcode assignments}
\begin{itemize}
\item 0:~escape character~\verb+/+,
 1/2:~beginning/end of group~\verb+{}+,
 3:~math shift~\verb+$+,
  4:~alignment tab~\verb+&+,
 5:~line end, 6:~parameter~\verb+#+
\item 7/8:~super/subscript~\verb+^_+, 9:~ignored~\n{NULL}
\item 10:~space, 11:~letter, 12:~other
\item 13:~active~\verb+~+, 14:~comment~\verb+%+, 15:~invalid~\n{DEL}
\end{itemize}
}

\frame[containsverbatim]{
  \frametitle{Token building}
\begin{itemize}
\item Backslash (really: escape character) plus letters (really:
  catcode~11) $\Rightarrow$ control word, definable, many primitives given
\item backslash plus space: control space (hardwired command)
\item backslash plus any other character: control symbol; many default
  definitions, but programmable
\item \verb+#+$n$ replaced by `parameter token~$n$',
  \verb+##+~replaced by macro parameter character
\item Anything else: character token (character plus category)
\end{itemize}
}

\frame[containsverbatim]{
  \frametitle{Some simple catcode tinkering}
\begin{itemize}
\item
\begin{verbatim}
\catcode`\@=11
\def\@InternalMacro{...}
\def\UserMacro{ .... \@InternalMacro ... }
\catcode`\@=12
\end{verbatim}
\end{itemize}
}

\frame[containsverbatim]{
  \frametitle{States}
\begin{itemize}
\item Every line starts in state $N$
\item in state $N$: spaces ignored, newline gives \cs{par}, with
  anything else go to~$M$ (middle of line)
\item State $S$ entered after control word, control space, or space in
  state~$M$; in this state ignore spaces and line ends
\item State $M$: entered after almost everything. In state~$M$, line
  end gives space token.
\end{itemize}
}

\frame[containsverbatim]{
  \frametitle{How many levels down are we?}
\begin{enumerate}
\item Lifting lines from file, appending EOL
\item translating \verb+^^xy+ to characters
\item catcode assignment
\item tokenization
\item state transitions
\end{enumerate}
}

\frame[containsverbatim]{
  \frametitle{What does this give us?}
\begin{itemize}
\item \TeX\ is now looking at a stream of tokens: mostly control
  sequences and characters
\item Actions depend on the nature of the token: expandable tokens get
  expanded, assignments and such get executed, text and formulas go to
  output processing.
\item Read chapters 1,2,3 of \TbT.
\end{itemize}
}

\sectionframe{Macros and expansion}

\frame[containsverbatim]{
  \frametitle{Expansion}
\begin{itemize}
\item Expansion takes command, gives replacement text.
\item Macros: replace command plus arguments by replacement text
\item Conditionals: yield true or false branch
\item Various tools
\item Read chapters 11,12 of \TbT.
\end{itemize}
}

\subsectionframe{The basics of macro programming}

\frame[containsverbatim]{
  \frametitle{Macro definitions}
\begin{itemize}
\item Simplest form: \verb+\def\foo#1#2#3{ .. #1 ... }+
\item Max 9 parameters, each one token or group:
\begin{examplewithcode}
\def\a#1#2{1:(#1) 2:(#2)}
\a b{cde}
\end{examplewithcode}
\end{itemize}
}

\frame[containsverbatim]{
  \frametitle{Delimited macro definitions}
\begin{itemize}
\item Delimited macro arguments:
\begin{examplewithcode}
\def\a#1 {Arg: `#1'}
\a stuff stuff
\end{examplewithcode}
\item Delimited and undelimited:
\begin{examplewithcode}
\def\Q#1#2?#3!{Question #1: #2?\par Answer: #3.}
\Q {5.2}Why did the chicken cross
        the Moebius strip?Eh\dots!
\end{examplewithcode}
\end{itemize}
}

\frame[containsverbatim]{
  \frametitle{Conditionals}
\begin{itemize}
\item General form \verb+\if<something> ... \else ... \fi+
\item \cs{ifx} equality of macro definition (also char, catcode)
\item 
\begin{examplewithcode}
\ifnum\value{section}<3 Just getting started.
\else On our way\fi
\end{examplewithcode}
\item Chapter 13 of \TbT
\end{itemize}
}

\frame[containsverbatim]{
  \frametitle{Grouping}
\begin{itemize}
\item Groups induced by
  \verb+{} \bgroup \egroup \begingroup \endgroup+
\item \cs{bgroup}, \cs{egroup} can sometimes replace \verb+{}+
\item \cs{begingroup}, \cs{endgroup} independent
\item funky stuff:
\begin{verbatim}
\def\open{\begingroup} \def\close{\endgroup}
\open ... \close
\end{verbatim}
\begin{examplewithcode}
\newenvironment{mybox}{\hbox\bgroup}{\egroup}
A \begin{mybox}B\end{mybox} C
\end{examplewithcode}
\item Chapter 10 of \TbT.
\end{itemize}
}

\frame[containsverbatim]{
  \frametitle{More tools}
\begin{itemize}
\item Counters:
\begin{verbatim}
\newcount\MyCounter \MyCounter=12
\advance\MyCounter by -3 \number\MyCounter
\end{verbatim}
also \cs{multiply}, \cs{divide}
\item Test numbers by
\begin{verbatim}
\ifnum\MyCounter<3 <then part>\else <else part> \fi
\end{verbatim}
available relations: \n{> < =}; also \cs{ifodd}, and
\begin{verbatim}
\ifcase\MyCounter <case 0>\or <case 1> ...
    \else <other> \fi
\end{verbatim}
\end{itemize}
}

\frame[containsverbatim]{
Only a finite number of counters in \TeX;\\
use \verb+\def\Constant{42}+ instead of
\begin{verbatim}
\newcount\Constant \Constant=24
\end{verbatim}
}

\frame[containsverbatim]{
  \frametitle{Conditionals}
\begin{itemize}
\item Already mentioned \cs{ifnum}, \cs{ifcase}
\item Programming tools: \cs{iftrue}, \cs{iffalse}
\begin{verbatim}
\iftrue {\else }\fi \iffalse {\else }\fi
\end{verbatim}
\item \cs{ifx} equality of character (char code and cat code);
  equality of macro definition
\item \cs{if} equality of character code after expansion.
\end{itemize}
}

\subsectionframe{Bunch of examples}

\frame[containsverbatim]{
  \frametitle{Grouping trickery}
\begin{verbatim}
\def\narrowbox{%
   \vbox\bgroup \addtolength{\textwidth}{-1in}
   \let\next=}
\end{verbatim}
Use this as
\begin{verbatim}
\narrowbox{ <bunch of text> }
% becomes
\vbox\bgroup 
   \addtolength{\textwidth}{-1in}
   \let\next={ <bunch of text> }
\end{verbatim}
}

\frame[containsverbatim]{
  \frametitle{Use of delimited arguments}
\begin{verbatim}
\def\FakeSC#1#2 {%
   {\uppercase{#1}\footnotesize\uppercase{#2}\ }%
    \FakeSC}
\end{verbatim}
\def\periodstop{.}
\def\FakeSC#1#2 {\def\tmp{#1}%
   \ifx\tmp\periodstop 
     \def\next{.}
   \else
     \def\next{%
         {\uppercase{#1}\footnotesize\uppercase{#2}\ }%
         \FakeSC}%
   \fi \next}
\begin{examplewithcode}
\FakeSC This sentence is fake small-caps .
\end{examplewithcode}
}

\frame[containsverbatim]{
  \frametitle{How did I stop that recursion?}
\begin{verbatim}
\def\periodstop{.}
\def\FakeSC#1#2 {\def\tmp{#1}%
   \ifx\tmp\periodstop 
     \def\next{.}
   \else
     \def\next{%
         {\uppercase{#1}\footnotesize\uppercase{#2}\ }%
         \FakeSC}%
   \fi \next}
\end{verbatim}
}

\frame[containsverbatim]{
  \frametitle{Two-step macros}
\begin{itemize}
\item Wanted:
\begin{verbatim}
\PickToEOL This text is the macro argument
and this is not
\end{verbatim}
\item Basic idea:
\begin{verbatim}
\def\PickToEOL
   {\begingroup\catcode`\^^M=12 \xPickToEOL}
\def\xPickToEOL#1^^M{ ...#1... \endgroup\par}
\end{verbatim}
\item The \cs{xPickToEOL} definition is not right
\end{itemize}
}

\frame[containsverbatim]{
%  \frametitle{Conditions at the definition count}
\begin{itemize}
\item Better:
\begin{examplewithcode}
\def\PickToEOL
   {\begingroup\catcode`\^^M=12 \xPickToEOL}
{\catcode`\^^M=12 %
 \gdef\xPickToEOL#1^^M{ \textbf{#1}\endgroup\par}
}
\PickToEOL This text is the macro argument
and this is not
\end{examplewithcode}
\end{itemize}
}

\frame[containsverbatim]{
  \frametitle{Optional arguments}
\begin{itemize}
\item Example: \verb+\section[Short]{Long title}+
\begin{examplewithcode}
\let\brack[
\def\section{\futurelet\next\xsection}
\def\xsection
   {\ifx\next\brack
          \let\next\sectionwithopt
    \else \let\next\sectionnoopt \fi \next}
\def\sectionnoopt#1{\sectionwithopt[#1]{#1}}
\def\sectionwithopt[#1]#2{Arg: `#2'; Opt `#1'}
\section[short]{Long}\par
\section{One}
\end{examplewithcode}
\end{itemize}
}

\frame[containsverbatim]{
  \frametitle{More expansion trickery}
\begin{itemize}
\item The \LaTeX\ command \verb+\newcounter{my}+ executes a command
  \verb+\countdef\mycounter+. How is that name formed?
\item Form control sequence names with
  \verb+\csname stuff\endcsname+. However
\begin{verbatim}
\def\newcounter#1{\countdef\csname #1\endcsname}
\end{verbatim}
would define a counter name \cs{csname}.
\item Use \cs{expandafter}:
\begin{verbatim}
\def\newcounter#1{%
    \expandafter\countdef\csname #1\endcsname}
\end{verbatim}
\end{itemize}
}

\frame[containsverbatim]{
  \frametitle{More expanding after}
\begin{itemize}
\item Suppose
\begin{verbatim}
\def\a#1#2{Arg1: #1, arg2: #2.}
\def\b{{one}{two}}
\end{verbatim}
How do you give the contents of \cs{b} to~\cs{a}?
\item Wrong: \verb+\a\b+\par
Solution:
\begin{verbatim}
\expandafter\a\b
\end{verbatim}
\item Suppose \verb+\def\c{\b}+, how would you get \verb+\a\c+ to work?
\end{itemize}
}

\frame[containsverbatim]{
  \frametitle{Expanding definition}
\begin{itemize}
\item \verb+\edef\foo{ .... }+ first expands the body, before doing
  the definition.
\item Defining based on current conditions:
\begin{verbatim}
\edef\restoreatcat{\catcode`\@=\the\catcode`\@\relax}
\catcode`\@=11
\def\@foo{...}
\restoreatcat
\end{verbatim}
\item Use as tool (above example revisited)
\begin{verbatim}
\def\a#1#2{Arg1: #1, arg2: #2.}
\def\b{{one}{two}}
\edef\tmp{\noexpand\a\b}\tmp
\end{verbatim}
\item Ponder this:
\begin{verbatim}
\edef\foo
   {\expandafter\noexpand\csname bar\endcsname}
\end{verbatim}
\end{itemize}
}

\frame[containsverbatim]{
  \frametitle{Nested macro definitions}
\begin{itemize}
\item Wrong: \verb+\def\a{\def\b#1{}}+
\item Also wrong: \verb+\def\a#1{\def\b#1{}}+
\item Remember that \verb+##+ is replaced by~\verb+#+:
\begin{examplewithcode}
\def\a#1{\def\b##1#1{Arg: `##1'}}
\a ! \b word words!\par
\a s \b word words!
\end{examplewithcode}
\end{itemize}
}

\frame[containsverbatim]{
  \frametitle{To summarize your toolbox}
\begin{itemize}
\item \cs{def}, \cs{edef}
\item \cs{expandafter}, \cs{noexpand}
\item \cs{csname}, \cs{endcsname}
\item \cs{let}, \cs{futurelet}
\end{itemize}
}

\frame[containsverbatim]{
  \frametitle{How do you debug this stuff?}
\begin{itemize}
\item The \TeX\ equivalent of \n{printf}\dots
\item \verb+\tracingmacros=2+
\item output goes into the log file
\end{itemize}
}

\frame[containsverbatim]{
  \frametitle{More to come}
so if you downloaded this version, you'll have to do it again
}

\end{document}

\section{Text handling}

\frame[containsverbatim]{
  \frametitle{Text handling}
}

\frame[containsverbatim]{
  \frametitle{}
\begin{itemize}
\item
\end{itemize}
}

\end{document}
