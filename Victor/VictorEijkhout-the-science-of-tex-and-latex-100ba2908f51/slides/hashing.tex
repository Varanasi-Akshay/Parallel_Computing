\documentclass{beamer}

\usepackage{beamerthemevictor,comment,verbatim,graphicx,amssymb}
\usepackage[noeepic]{qtree}

\newcommand{\TbT}{\TeX\ by Topic}
\newcommand{\Metafont}{\textsc{Metafont}}
\let\metafont\Metafont
\newcommand\ascii{{\sc Ascii}}
\newcommand\ebcdic{{\sc Ebcdic}}
\newcommand\gnuplot{\protect\n{gnuplot}}
\def\mod{\mathbin{\mathrm{mod}}}
\def\eqref#1{equation~(\ref{#1})}
\def\Eqref#1{Equation~(\ref{#1})}
\def\inv{^{-1}}
\def\endproof{\unskip\penalty10000 \hskip10pt plus 1fill$\bullet$\par}

\def\etal{\textit{et al.}}
\def\lex{{\it lex}}
\def\yacc{{\it yacc}}
\def\make{{\it make}}
\def\web{{\textsc{Web}}}
\def\n{\bgroup
  \catcode`\#=12 \catcode`\_=12 \catcode`\^=12
  \catcode`\&=12 
  \tt \let\next=}
\newcommand{\cs}[1]{{\tt\char`\\#1}}
{\catcode`\.=13
 \gdef.{${}_\bullet$}
}
\def\parserule#1{%
    \ifmmode \n{#1}\,\rightarrow\relax
    \else $\n{#1}\,\rightarrow{}$\nobreak
    \fi
    \hbox\bgroup\catcode`\.=13 \tt\let\next=}

\newenvironment{remark}%
    {\medskip\noindent{\bf Remark.}}{}
\usepackage{bnf}
\input bnf.env

\generalcomment{inputwithcode}
  {\begingroup\def\ProcessCutFile{}}
  {\verbatiminput{\CommentCutFile}
   \endgroup
   \input{\CommentCutFile}
  }
\generalcomment{examplewithcode}
  {\begingroup
     \def\ProcessCutFile{}\def\CommentCutFile{example.tex}}
  {\verbatiminput{\CommentCutFile}
   Output:
   \begin{quote}
     \begin{minipage}[t]{3in}
        \everypar{}
        \input{\CommentCutFile} 
     \end{minipage}
   \end{quote}
   \endgroup
  }
\specialcomment{ttexamplewithcode}
  {\begingroup\def\ProcessCutFile{}}
  {\verbatiminput{\CommentCutFile}
   \endgroup
   Output:
   \begin{quote}\begin{ttfamily} \input{\CommentCutFile} 
     \end{ttfamily}\end{quote}
  }
\generalcomment{mathexamplewithcode}
  {\begingroup\def\ProcessCutFile{}}
  {\verbatiminput{\CommentCutFile}
   Output:
   \begin{equation} \input{\CommentCutFile} \end{equation}
   \endgroup
  }

\def\chaptertitle{\csname\chaptername title\endcsname}
\def\chaptershorttitle{\csname\chaptername shorttitle\endcsname}

\def\lambdatitle{Lambda calculus in \TeX}
\def\lambdashorttitle{Lambda calculus}

\def\latextitle{Yet another \LaTeX\ tutorial}
\def\latexshorttitle{\LaTeX}

\def\lextitle{A \lex\ tutorial}
\def\lexshorttitle{\lex}

\def\yacctitle{A \yacc\ tutorial}
\def\yaccshorttitle{\yacc}

\expandafter\def\csname tex1title\endcsname{\TeX\ -- macro programming}
\expandafter\let\csname tex1shorttitle\expandafter\endcsname
    \csname tex1title\endcsname
\expandafter\def\csname tex2title\endcsname{\TeX\ -- visuals}
\expandafter\let\csname tex2shorttitle\expandafter\endcsname
    \csname tex2title\endcsname

\def\parsingtitle{An introduction to parsing}
\def\parsingshorttitle{Parsing}

\def\hashingtitle{Hashing}
\let\hashingshorttitle\hashingtitle

\def\beziertitle{Curve approximation by splines}
\def\beziershorttitle{Splines}

\def\dynamictitle{Dynamic programming}
\let\dynamicshorttitle\dynamictitle

\def\paragraphtitle{Line breaking}
\let\paragraphshorttitle\paragraphtitle

\def\pagetitle{Page breaking in \TeX}
\def\pageshorttitle{Page breaking}

\def\completenesstitle{Introduction to NP-Completeness}
\def\completenesshortstitle{NP-Completeness}

\def\rastertitle{Introduction to Raster Graphics}
\def\rastershortstitle{Raster Graphics}

\def\pythontitle{Really quick Python tutorial}
\def\pythonshortstitle{Python}

\def\encodingtitle{Character encoding}
\let\encodingshorttitle\encodingtitle

\def\rastertitle{Raster Graphics}
\let\rastershorttitle\rastertitle

\def\gnuplottitle{Gnuplot}
\let\gnuplotshorttitle\gnuplottitle

\def\fontstitle{Fonts and font files}
\def\fontsshorttitle{Fonts}

\def\softwaretitle{\TeX\ and Software Engineering}
\def\softwareshorttitle{Software Engineering}

\renewcommand{\beamervictortitle}{CS-594 Eijkhout, Fall 2004}

\newcommand{\sectionframe}[1]%
  {\section{#1}
   \frame{\begin{center}\color{blue}\Large #1\end{center}}
  }
\newcommand{\subsectionframe}[1]%
  {\subsection{#1}
   \frame{\begin{center}\color{blue}\Large #1\end{center}}
  }

\input idxmacs

\begin{document}

\title{Hashing}
\author{Victor Eijkhout}
\date{Notes for CS 594 -- Fall 2004}

\frame{\titlepage}

\section{Introduction}

\frame{
  \frametitle{The basic problem}
Storing names and information about them:\\
associative storage
}

\frame{
\frametitle{Issues}
\begin{itemize}
\item<2-> Insertion
\item<3-> Retrieval
\item<4-> Deletion
\end{itemize}
}

\frame{
  \frametitle{Simple strategies}
\begin{itemize}
\item List in order of creation
\item<2-> $\Rightarrow$ Cheap to create, linear search time, linear deletion
\item<3-> Sorted list
\item<4-> $\Rightarrow$ Creation in linear time, search logarithmic,
  deletion linear
\item<5-> Linear list
\item<6-> $\Rightarrow$ all linear time
\end{itemize}
}

\frame{
  \frametitle{one more strategy}
\begin{quote}
\Tree [.$\bullet$ [.B ART [.E GIN LL ] ] [.E LSE ND ] ]
\end{quote}
\begin{itemize}
\item<2-> $\Rightarrow$ all linear in length of string
\end{itemize}
}

\sectionframe{Hash functions}

\frame[containsverbatim]{
  \frametitle{}
\begin{itemize}
\item Mapping from space of words to space of indices
\item Source: unbounded; in practice not extremely large
\item Target: array (static/dynamic)
\end{itemize}
}

\frame{
  \frametitle{Requirements}
\begin{itemize}
\item<2-> Function determined only by input data
\item<3-> Determined by as much of the data as possible\\
\n{key1}, \n{key2},\dots
\item<4-> Uniform distribution (clustering bad, collisions really bad)
\item<5-> Similar data, mapped far apart
\end{itemize}
}

\subsection{Modulo operations}

\frame[containsverbatim]{
  \frametitle{Good idea: prime numbers}
With $M$ size of the hash table:
\begin{equation}
    h(K) = K\mod M, \end{equation}
or:
\begin{equation}
    h(K) = aK\mod M,\end{equation}
}

\frame[containsverbatim]{
  \frametitle{Bad examples:}
\begin{itemize}
\item $M$~is even, say~$M=2M'$,\\ $r=K\mod M$ say~$K=nM+r$
then
\begin{eqnarray*}
K=2K'&\Rightarrow& r=2(nM'-K')\\
K=2K'+1&\Rightarrow& r=2(nM'-K')+1
\end{eqnarray*}
so key even iff number $\Rightarrow$~dependence on last digit
\item $M$ multiple of three: anagrams map to same key (sum of digits)
\item $\Rightarrow$ $M$ prime, far away from powers of 2
\end{itemize}
}

\frame[containsverbatim]{
  \frametitle{Multiplication instead of division}
\begin{itemize}
\item $r=K\mod M = M\big((K/M)\mod 1\big)$ 
\item $A\approx w/M$, where $w$ maxint
\item Then $1/M=A/w$, ($A$ with decimal point to its left).
\item from %K\mod M=M(K/M\mod 1)$:
\[ h(K)=\lfloor M\left(\left({A\over w}K\right)\mod
1\right)\rfloor. \]
\end{itemize}
}

\frame[containsverbatim]{
  \frametitle{Example: Bible}
\begin{itemize}
\item 42,829 unique words,
\item into a hash table with 30,241 elements (prime): 76.6\% used
\item table of size: 30,240 (divisible by 2--9): 60.7\% used
\item (collisions discussed later)
\end{itemize}
}

\subsection{Character hashing}

\frame[containsverbatim]{
  \frametitle{Two-step hashing}
\begin{itemize}
\item Mix up characters of the key
\item then modulo with table size
\end{itemize}
}

\frame[containsverbatim]{
  \frametitle{Character based hashing}
\begin{verbatim}
  h = <some value>
  for (i=0; i<len(var); i++)
    h = h + <byte i of string>;
\end{verbatim}
prevent anagram problem:
\begin{verbatim}
  h = <some value>
  for (i=0; i<len(var); i++)
    h = Rand( h + <byte i of string> );
\end{verbatim}
with table of random numbers; also function possible
}

\frame[containsverbatim]{
  \frametitle{ELF hash}
\begin{verbatim}
/* UNIX ELF hash
 * Published hash algorithm used in the UNIX ELF format
 * for object files
 */
unsigned long hash(char *name)
{
    unsigned long h = 0, g;

    while ( *name ) {
        h = ( h << 4 ) + *name++;
        if ( g = h & 0xF0000000 )
          h ^= g >> 24;
        h &= ~g;
    }
}
\end{verbatim}
}

\frame[containsverbatim]{
  \frametitle{Another hash function}
\begin{verbatim}
/* djb2
 * This algorithm was first reported by Dan Bernstein
 * many years ago in comp.lang.c
 */
unsigned long hash(unsigned char *str)
{
    unsigned long hash = 5381;
    int c; 
    while (c = *str++) hash = ((hash << 5) + hash) + c;
    return hash;
}
\end{verbatim}
}

\sectionframe{Hash tables: collisions}

\frame[containsverbatim]{
  \frametitle{So far so good}
\pgfimage[height=2in]{hash-direct}
}

\frame[containsverbatim]{
  \frametitle{Collisions}
\begin{itemize}
\item $k_1\not=k_2$, $h(k_1)=h(k_2)$
\item several strategies; all analysis statistical in nature
\item open hash table: solve conflict outside the table
\item closed hash table: solve by moving around in the table
\end{itemize}
}

\frame[containsverbatim]{
  \frametitle{Separate chaining}
\pgfimage[height=2in]{hash-separate}
}

\subsection{Open hash table}

\frame[containsverbatim]{
  \frametitle{}
\begin{itemize}
\item Pro: no need for searching through hash table
\item Con: dynamic storage
\item Also: $M$ large to prevent collisions $\Rightarrow$~wasted space
\end{itemize}
}

\subsection{Closed hash table}

\frame[containsverbatim]{
  \frametitle{Linear probing}
\pgfimage[height=2in]{hash-linear}

Location occupied: search linearly from first hash
}

\frame[containsverbatim]{
\footnotesize
\begin{verbatim}
addr = Hash(K);
if (IsEmpty(addr)) Insert(K,addr);
else {
    /* see if already stored */
  test:
    if (Table[addr].key == K) return;
    else {
      addr = Table[addr].link; goto test;}
    /* find free cell */
    Free = addr;
    do { Free--; if (Free<0) Free=M-1; }
    while (!IsEmpty(Free) && Free!=addr)
    if (!IsEmpty(Free)) abort;
    else {
      Insert(K,Free); Table[addr].link = Free;}
}
\end{verbatim}
}

\frame{
  \frametitle{Merging blocks in linear probing}
\pgfimage[height=1in]{probing1}
\pgfimage[height=1in]{probing2}
\pgfimage[height=1in]{probing3}
}

\frame[containsverbatim]{
  \frametitle{Linear probing analysis}
\begin{itemize}
\item  Clusters forming
\item Particularly bad: merging clusters
\item Ratio occupied/total: $\alpha=N/M$\\
expected search time
\[ T\approx\begin{cases}
    {1\over 2}\left(1+\left({1\over 1-\alpha}\right)^2\right)&
        unsuccessful\\
    {1\over 2}\left(1+{1\over 1-\alpha}\right)&
        successful
    \end{cases}
\]
\item $\Rightarrow$ increasing as table fills up
\end{itemize}
}

\subsection{Chaining}

\frame[containsverbatim]{
\frametitle{Chaining}
\pgfimage[height=1in]{hash-chain0}
\pgfimage[height=1in]{hash-chain}
\pgfimage[height=1in]{hash-hash2}

If location occupied, search from top of table
}

\frame[containsverbatim]{
\footnotesize
\begin{verbatim}
addr = Hash(K); Free = M-1;
if (IsEmpty(addr)) Insert(K,addr);
else {
    /* see if already stored */
  test:
    if (Table[addr].key == K) return;
    else {
      addr = Table[addr].link; goto test;}
    /* find free cell */
    do { Free--; }
    while (!IsEmpty(Free)
    if (Free<0) abort;
    else {
      Insert(K,Free); Table[addr].link = Free;}
}
\end{verbatim}
}

\frame[containsverbatim]{
  \frametitle{Chaining analysis}
\begin{itemize}
\item No clusters merging
\item Coalescing lists
\item Search time ($\alpha$~occupied fraction)
\[ T\approx\begin{cases}
        1+(e^{2\alpha}-1-2\alpha)/4&unsuccessful\\
        1+(e^{2\alpha}-1-2\alpha)/8\alpha+\alpha/4&successful
           \end{cases}
\]
\end{itemize}
}

\frame[containsverbatim]{
  \frametitle{Nonlinear rehashing}
\begin{itemize}
\item `Random probing': Try $(h(m)+p_i)\mod s$, where $p_i$~is a
  sequence of random numbers (stored)\\
  prevent secondary collisions
\item `Add the hash': Try $(i\times h(m))\mod s$. ($s$~prime)
\item Pro: scattered hash keys
\item Con: more calculations, worse memory locality
\end{itemize}
}

\section{Other}
\subsection{Deletion}

\frame[containsverbatim]{
  \frametitle{Deleting keys}
\begin{itemize}
\item Simple in direct chaining
\item Very hard in closed hash table methods: can only mark `unused'
\end{itemize}
}

\subsection{Examples}

\frame[containsverbatim]{
  \frametitle{Search in chess programs}
\begin{itemize}
\item Problem: evaluation board positions
\item if position arrived in two ways, no two calculations
\item Solution: hash the board, use as key in table of evaluations
\item Collisions?
\end{itemize}
}

\frame{
  \frametitle{String searching}
\begin{itemize}
\item Problem: does string (length~$M$) occur in document (length~$N$)
\item naive: $N$ comparisons, giving $O(MN)$ complexity
\item solution: hash the strings, compare hash values
\item (hash function does not distinguish between anagrams)
\[ h(k)=\left\{\sum_ik[i]\right\}\mathop{\mathrm{mod}} K\]
\item<2-> cheap updating of the document hash key:
\[ h(t[2\ldots n+1]) = h(t[1\ldots n])+t[n+1]-t[1] \]
(with addition/subtraction modulo~$K$)
\item string comparison in $O(1)$, $\Rightarrow$~total cost~$O(M+N)$
\end{itemize}
}

\subsectionframe{Discussion}

\frame{
  \frametitle{Hash table vs trees}
\begin{itemize}
\item<2-> Best case search time can be equal: harder to implement in
  trees
\item<3-> Trees can become unbalanced: considerable time and effort to
  balance
\item<4-> Threes have dynamic storage: harder to code optimally; worse
  memory locality
\end{itemize}
}

\frame{
  \frametitle{Open vs closed hash tables}
\begin{itemize}
\item<2-> Approximately equal performance until the table fills up
\item<3-> Open: much simpler storage management, especially deletion
\end{itemize}
}

\end{document}
