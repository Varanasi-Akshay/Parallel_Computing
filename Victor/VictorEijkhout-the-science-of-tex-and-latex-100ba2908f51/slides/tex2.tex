\documentclass{beamer}

\usepackage{beamerthemevictor,comment,verbatim,graphicx,rotating}

\newcommand{\TbT}{\TeX\ by Topic}
\newcommand{\Metafont}{\textsc{Metafont}}
\let\metafont\Metafont
\newcommand\ascii{{\sc Ascii}}
\newcommand\ebcdic{{\sc Ebcdic}}
\newcommand\gnuplot{\protect\n{gnuplot}}
\def\mod{\mathbin{\mathrm{mod}}}
\def\eqref#1{equation~(\ref{#1})}
\def\Eqref#1{Equation~(\ref{#1})}
\def\inv{^{-1}}
\def\endproof{\unskip\penalty10000 \hskip10pt plus 1fill$\bullet$\par}

\def\etal{\textit{et al.}}
\def\lex{{\it lex}}
\def\yacc{{\it yacc}}
\def\make{{\it make}}
\def\web{{\textsc{Web}}}
\def\n{\bgroup
  \catcode`\#=12 \catcode`\_=12 \catcode`\^=12
  \catcode`\&=12 
  \tt \let\next=}
\newcommand{\cs}[1]{{\tt\char`\\#1}}
{\catcode`\.=13
 \gdef.{${}_\bullet$}
}
\def\parserule#1{%
    \ifmmode \n{#1}\,\rightarrow\relax
    \else $\n{#1}\,\rightarrow{}$\nobreak
    \fi
    \hbox\bgroup\catcode`\.=13 \tt\let\next=}

\newenvironment{remark}%
    {\medskip\noindent{\bf Remark.}}{}
\usepackage{bnf}
\input bnf.env

\generalcomment{inputwithcode}
  {\begingroup\def\ProcessCutFile{}}
  {\verbatiminput{\CommentCutFile}
   \endgroup
   \input{\CommentCutFile}
  }
\generalcomment{examplewithcode}
  {\begingroup
     \def\ProcessCutFile{}\def\CommentCutFile{example.tex}}
  {\verbatiminput{\CommentCutFile}
   Output:
   \begin{quote}
     \begin{minipage}[t]{3in}
        \everypar{}
        \input{\CommentCutFile} 
     \end{minipage}
   \end{quote}
   \endgroup
  }
\specialcomment{ttexamplewithcode}
  {\begingroup\def\ProcessCutFile{}}
  {\verbatiminput{\CommentCutFile}
   \endgroup
   Output:
   \begin{quote}\begin{ttfamily} \input{\CommentCutFile} 
     \end{ttfamily}\end{quote}
  }
\generalcomment{mathexamplewithcode}
  {\begingroup\def\ProcessCutFile{}}
  {\verbatiminput{\CommentCutFile}
   Output:
   \begin{equation} \input{\CommentCutFile} \end{equation}
   \endgroup
  }

\def\chaptertitle{\csname\chaptername title\endcsname}
\def\chaptershorttitle{\csname\chaptername shorttitle\endcsname}

\def\lambdatitle{Lambda calculus in \TeX}
\def\lambdashorttitle{Lambda calculus}

\def\latextitle{Yet another \LaTeX\ tutorial}
\def\latexshorttitle{\LaTeX}

\def\lextitle{A \lex\ tutorial}
\def\lexshorttitle{\lex}

\def\yacctitle{A \yacc\ tutorial}
\def\yaccshorttitle{\yacc}

\expandafter\def\csname tex1title\endcsname{\TeX\ -- macro programming}
\expandafter\let\csname tex1shorttitle\expandafter\endcsname
    \csname tex1title\endcsname
\expandafter\def\csname tex2title\endcsname{\TeX\ -- visuals}
\expandafter\let\csname tex2shorttitle\expandafter\endcsname
    \csname tex2title\endcsname

\def\parsingtitle{An introduction to parsing}
\def\parsingshorttitle{Parsing}

\def\hashingtitle{Hashing}
\let\hashingshorttitle\hashingtitle

\def\beziertitle{Curve approximation by splines}
\def\beziershorttitle{Splines}

\def\dynamictitle{Dynamic programming}
\let\dynamicshorttitle\dynamictitle

\def\paragraphtitle{Line breaking}
\let\paragraphshorttitle\paragraphtitle

\def\pagetitle{Page breaking in \TeX}
\def\pageshorttitle{Page breaking}

\def\completenesstitle{Introduction to NP-Completeness}
\def\completenesshortstitle{NP-Completeness}

\def\rastertitle{Introduction to Raster Graphics}
\def\rastershortstitle{Raster Graphics}

\def\pythontitle{Really quick Python tutorial}
\def\pythonshortstitle{Python}

\def\encodingtitle{Character encoding}
\let\encodingshorttitle\encodingtitle

\def\rastertitle{Raster Graphics}
\let\rastershorttitle\rastertitle

\def\gnuplottitle{Gnuplot}
\let\gnuplotshorttitle\gnuplottitle

\def\fontstitle{Fonts and font files}
\def\fontsshorttitle{Fonts}

\def\softwaretitle{\TeX\ and Software Engineering}
\def\softwareshorttitle{Software Engineering}

\input slidemacs
\input idxmacs

\begin{document}

\title{\TeX\ -- visual matters}
\author{Victor Eijkhout}
\date{Notes for CS 594 -- Fall 2004}

\frame{\titlepage}

\section{Text handling}

\subsectionframe{Paragraphs}

\frame[containsverbatim]{
  \frametitle{Paragraph start}
\begin{itemize}
\item Paragraph starts triggered by text, math, certain commands
\item Vertical space added: \cs{parskip}; horizontal indentation:
  \cs{parindent}
\item Also inserted \cs{everypar} token list
\tracingmacros=2 \tracingcommands=2
\begin{examplewithcode}
\everypar{\onebold} \def\onebold#1{\textbf{#1}}
First paragraph\par Second one\par
\end{examplewithcode}
\end{itemize}
}

\frame[containsverbatim]{
  \frametitle{}
\tiny
\begin{examplewithcode}
\newcounter{vcount}
\def\Header#1{\medskip
  \hbox{\bfseries #1}
  \setcounter{vcount}{1}
  \everypar{\arabic{vcount}\stepcounter{vcount}\ }
  }

\Header{The Title}

One line of text that is long enough to wrap as a paragraph
that is long enough to wrap as a paragraph

two lines of text that are long enough to wrap as a paragraph
that is long enough to wrap as a paragraph

more lines of text that are long enough to wrap as a paragraph
that is long enough to wrap as a paragraph
\end{examplewithcode}
}

\frame[containsverbatim]{
  \frametitle{Paragraph end}
\begin{itemize}
\item Paragraph ends because of \cs{par} (empty line), display math,
  other vertical commands
\item End of paragraph: \verb+\unskip\penalty10000\hskip\parfillskip+
\begin{examplewithcode}
\parfillskip=0pt
lots of lots of lots of lots of lots of lots
of lots of text
\end{examplewithcode}
\item Normal \cs{parfillskip} is \n{0pt plus 1fil}
\end{itemize}
}

\frame[containsverbatim]{
\begin{examplewithcode}
\renewenvironment{proof}
  {Proof.\ }
  {\hfill$\bullet$\par}
\begin{proof}
This is a long long long long long long long 
long long long proof
\end{proof}
\end{examplewithcode}
}

\frame[containsverbatim]{
  \frametitle{Paragraph shape}
\begin{examplewithcode}
\begin{minipage}{2in}
\parindent=0pt \hangindent=15pt \hangafter=-3 
This paragraph has several lines of text 
so that it can show off the `hanging indentation'
of \TeX, which can be used for all sorts of purposes.
\end{minipage}
\end{examplewithcode}
}

\frame[containsverbatim]{
  \frametitle{Margin parameters}
\begin{itemize}
\item \cs{leftskip}, \cs{rightskip} at the margins
\item \cs{parindent} start of first line
\item \cs{parfillskip} end of last
\item \cs{hangindent} extra shift
\item \cs{parshape}
\end{itemize}
}

\frame[containsverbatim]{
  \frametitle{Margin tricks}
\begin{examplewithcode}
\leftskip=0cm plus 0.5fil \rightskip=0cm plus -0.5fil
\parfillskip=0cm plus 1fil
This style of paragraph setting is rather old fashioned, 
typically used for the last paragraph of a chapter.
\end{examplewithcode}
}

\frame[containsverbatim]{
  \frametitle{Line breaking}
\begin{itemize}
\item Global minimization of `badness' from glue setting and other
  penalties
\item Badness from glue setting: Line is `decent' is less than half
  the stretch or shrink is used; `loose' and `tight' if more than
  used; `very loose' if more than the stretch is used. Add
  \cs{adjdemerits} if adjacent line not of same or adjacent
  classification
\item also \cs{doublehyphendemerits}, \cs{finalhyphendemerits}
\end{itemize}
}

\frame[containsverbatim]{
\begin{itemize}
\item First pass: without hyphenation; maximum badness
  \cs{pretolerance}
\item Second pass: with hyphenation; maximum allowed is \cs{tolerance}
\item Third pass: add \cs{emergencystretch}
\end{itemize}
}

\frame[containsverbatim]{
  \frametitle{Line break problems}
\small
\begin{examplewithcode}
\tolerance500 \emergencystretch=0pt
Paragraphs with words such as the German `Weltschmerz'
can be hard to set, even if 
anti-disestablishmentarianism comes into play.
Other topics can also give
superduperhyperbig problems. As you can see.
\end{examplewithcode}
}

\frame[containsverbatim]{
\small
\begin{examplewithcode}
\tolerance500 \emergencystretch=20pt
Paragraphs with words such as the German `Weltschmerz'
can be hard to set, even if 
anti-disestablishmentarianism comes into play.
Other topics can also give
superduperhyperbig problems. As you can see.
\end{examplewithcode}
}

\section{More text handling}
\subsectionframe{Boxes}

\frame[containsverbatim]{
  \frametitle{Horizontal Boxes}
 Horizontal: \cs{hbox}
\begin{examplewithcode}
A \raise 2pt \hbox{B c d E} F
\lower -7pt \hbox{G} H
\end{examplewithcode}
Tight fit: one line.
}

\frame[containsverbatim]{
  \frametitle{Vertical boxes}
 Vertical: \cs{vbox}, \cs{vtop}
\begin{examplewithcode}
A \vbox{\hsize=3cm Lots of text, organised in one paragraph.\par
And one paragraph more, with lots of text text text}
B C \vtop{\hsize=3cm Lots of text in one paragraph.}
D E
\end{examplewithcode}
Acts like normal text, page width
}

\frame[containsverbatim]{
  \frametitle{Boxes and skips}
\begin{examplewithcode}
A \hbox{B\hskip 1cm} C D \hbox{\hskip-5mm E F\hskip 3mm} G
\end{examplewithcode}
\begin{examplewithcode}
A \hbox to 20pt{B\hfill} C D \hbox to 0pt{E F\hss}G H
\end{examplewithcode}
}

\subsectionframe{Modes}

\frame[containsverbatim]{
  \frametitle{Horizontal mode}
\begin{itemize}
\item Starts with letter, math, commands like \cs{hskip}
\item Material lines up horizontally
\item Inner horizontal mode: inside \cs{hbox} -- one line, no
  paragraph building.
\item Example
\begin{examplewithcode}
A \hbox{b} \raise 2pt \vbox{\hsize=20pt c} d
\end{examplewithcode}
\end{itemize}
}

\frame[containsverbatim]{
  \frametitle{Vertical mode}
\begin{itemize}
\item After paragraph, display math, vertical commands like \cs{vskip}
\item Material stacked vertically
\item Inner vertical mode: inside \cs{vbox} -- this \emph{does} build
  paragraphs
\item Example
\begin{examplewithcode}
A b

\hbox{b} c d
\end{examplewithcode}
\end{itemize}
}

\subsection{Rules}

\frame[containsverbatim]{
  \frametitle{Rules}
\begin{itemize}
\item \cs{hrule} is vertical command, \cs{vrule} horizontal
\item rules extend to fill surrounding box
\begin{examplewithcode}
\par
\hbox{\vrule\ ab\ \vrule}
\vbox{\hsize=3cm \hrule 
  Here is a paragraph that is completely
  inside this vbox \hrule}
\end{examplewithcode}
\item Horizontal lines in horizontal mode are a bit more tricky
\end{itemize}
}

\sectionframe{Math}

\frame[containsverbatim]{
  \frametitle{Math styles}
\begin{itemize}
\item Styles: display, text, script, scripscript
\begin{examplewithcode}
\[ {\displaystyle\sum_i^\infty 1/i},\qquad
          {\textstyle\sum_i^\infty 1/i} \]
\[ x^{x^x}, {\textstyle x},{\scriptstyle x},
           {\scriptscriptstyle x} \]
\end{examplewithcode}
\item Display math starts in display style, inline math in text style
\end{itemize}
}

\frame[containsverbatim]{
  \frametitle{Math character codes}
\begin{itemize}
\item $\mathop{\cs{mathcode}}n=\n{"xyzz}$: class, font position
\item $\mathop{\cs{delcode}}n=\n{"uvvxyy}$: two font positions, small
  and large
\item $\mathop{\cs{mathaccent}}n\n{<expr>}$
\item $\mathop{\cs{mathchardef}}\cs{sum}=\n{"1350}$
\end{itemize}
}

\frame[containsverbatim]{
  \frametitle{Math spacing}
\begin{itemize}
\item Spaces are ignored, any spacing inserted automatically
\item Three sizes of spaces: thick, med, thin
\begin{examplewithcode}
$a=b$ vs $a{=}b$\par % thick
$a+b$ vs $a{+}b$\par % med
$a,b$ vs $a{,}b$
\end{examplewithcode}
\end{itemize}
}

\frame[containsverbatim]{
  \frametitle{Math object classes}
\begin{itemize}
\item Spacing depends on function of an object (`class')
\item Binary operators: \verb+$x\mathrm{e}y$+ is~`$x\mathrm{e}y$'\\
    \verb+$x\mathbin{\mathrm{e}}y$+ is~`$x\mathbin{\mathrm{e}}y$'
\item Similar: \cs{mathop} for large operators, \cs{mathrel}~for
  binary relations (equals~\&c); \cs{mathopen}, \cs{mathclose},
  \cs{mathord}, \cs{mathpunct}
\end{itemize}
}

\sectionframe{Output}

\frame[containsverbatim]{
  \frametitle{Vertical list}
\begin{itemize}
\item Objects are added to vertical list: lines from paragraph,
  display math
\item Various penalties: \cs{abovedisplaypenalty}, \cs{widowpenalty}
\item Page breaking algorithm minimizes balance of penalties and
  stretch/shrink
\end{itemize}
}

\frame[containsverbatim]{
  \frametitle{Output routine}
\begin{verbatim}
\output={
  \setbox255=\vbox
     {\headline \box255 \footline}
  \shipout\box255
}
\end{verbatim}
}

\frame[containsverbatim]{
  \frametitle{Marks}
\begin{itemize}
\item Remember \cs{markright} and \cs{markboth} in \LaTeX.
\item Basic: \cs{mark} in \TeX
\item During output: \cs{firstmark} is first mark on this page\\
\cs{botmark} is last mark on this page\\
\cs{topmark} is last mark of previous page
\item If no marks on this page, all three equal to \cs{botmark} of
  last page
\end{itemize}
}

\end{document}

\frame[containsverbatim]{
  \frametitle{}
\begin{itemize}
\item
\end{itemize}
}

