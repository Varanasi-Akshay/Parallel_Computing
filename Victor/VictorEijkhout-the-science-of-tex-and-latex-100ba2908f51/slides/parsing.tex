\documentclass{beamer}

\usepackage{beamerthemevictor,comment,verbatim,graphicx,amssymb}
\usepackage[noeepic]{qtree}

\newcommand{\TbT}{\TeX\ by Topic}
\newcommand{\Metafont}{\textsc{Metafont}}
\let\metafont\Metafont
\newcommand\ascii{{\sc Ascii}}
\newcommand\ebcdic{{\sc Ebcdic}}
\newcommand\gnuplot{\protect\n{gnuplot}}
\def\mod{\mathbin{\mathrm{mod}}}
\def\eqref#1{equation~(\ref{#1})}
\def\Eqref#1{Equation~(\ref{#1})}
\def\inv{^{-1}}
\def\endproof{\unskip\penalty10000 \hskip10pt plus 1fill$\bullet$\par}

\def\etal{\textit{et al.}}
\def\lex{{\it lex}}
\def\yacc{{\it yacc}}
\def\make{{\it make}}
\def\web{{\textsc{Web}}}
\def\n{\bgroup
  \catcode`\#=12 \catcode`\_=12 \catcode`\^=12
  \catcode`\&=12 
  \tt \let\next=}
\newcommand{\cs}[1]{{\tt\char`\\#1}}
{\catcode`\.=13
 \gdef.{${}_\bullet$}
}
\def\parserule#1{%
    \ifmmode \n{#1}\,\rightarrow\relax
    \else $\n{#1}\,\rightarrow{}$\nobreak
    \fi
    \hbox\bgroup\catcode`\.=13 \tt\let\next=}

\newenvironment{remark}%
    {\medskip\noindent{\bf Remark.}}{}
\usepackage{bnf}
\input bnf.env

\generalcomment{inputwithcode}
  {\begingroup\def\ProcessCutFile{}}
  {\verbatiminput{\CommentCutFile}
   \endgroup
   \input{\CommentCutFile}
  }
\generalcomment{examplewithcode}
  {\begingroup
     \def\ProcessCutFile{}\def\CommentCutFile{example.tex}}
  {\verbatiminput{\CommentCutFile}
   Output:
   \begin{quote}
     \begin{minipage}[t]{3in}
        \everypar{}
        \input{\CommentCutFile} 
     \end{minipage}
   \end{quote}
   \endgroup
  }
\specialcomment{ttexamplewithcode}
  {\begingroup\def\ProcessCutFile{}}
  {\verbatiminput{\CommentCutFile}
   \endgroup
   Output:
   \begin{quote}\begin{ttfamily} \input{\CommentCutFile} 
     \end{ttfamily}\end{quote}
  }
\generalcomment{mathexamplewithcode}
  {\begingroup\def\ProcessCutFile{}}
  {\verbatiminput{\CommentCutFile}
   Output:
   \begin{equation} \input{\CommentCutFile} \end{equation}
   \endgroup
  }

\def\chaptertitle{\csname\chaptername title\endcsname}
\def\chaptershorttitle{\csname\chaptername shorttitle\endcsname}

\def\lambdatitle{Lambda calculus in \TeX}
\def\lambdashorttitle{Lambda calculus}

\def\latextitle{Yet another \LaTeX\ tutorial}
\def\latexshorttitle{\LaTeX}

\def\lextitle{A \lex\ tutorial}
\def\lexshorttitle{\lex}

\def\yacctitle{A \yacc\ tutorial}
\def\yaccshorttitle{\yacc}

\expandafter\def\csname tex1title\endcsname{\TeX\ -- macro programming}
\expandafter\let\csname tex1shorttitle\expandafter\endcsname
    \csname tex1title\endcsname
\expandafter\def\csname tex2title\endcsname{\TeX\ -- visuals}
\expandafter\let\csname tex2shorttitle\expandafter\endcsname
    \csname tex2title\endcsname

\def\parsingtitle{An introduction to parsing}
\def\parsingshorttitle{Parsing}

\def\hashingtitle{Hashing}
\let\hashingshorttitle\hashingtitle

\def\beziertitle{Curve approximation by splines}
\def\beziershorttitle{Splines}

\def\dynamictitle{Dynamic programming}
\let\dynamicshorttitle\dynamictitle

\def\paragraphtitle{Line breaking}
\let\paragraphshorttitle\paragraphtitle

\def\pagetitle{Page breaking in \TeX}
\def\pageshorttitle{Page breaking}

\def\completenesstitle{Introduction to NP-Completeness}
\def\completenesshortstitle{NP-Completeness}

\def\rastertitle{Introduction to Raster Graphics}
\def\rastershortstitle{Raster Graphics}

\def\pythontitle{Really quick Python tutorial}
\def\pythonshortstitle{Python}

\def\encodingtitle{Character encoding}
\let\encodingshorttitle\encodingtitle

\def\rastertitle{Raster Graphics}
\let\rastershorttitle\rastertitle

\def\gnuplottitle{Gnuplot}
\let\gnuplotshorttitle\gnuplottitle

\def\fontstitle{Fonts and font files}
\def\fontsshorttitle{Fonts}

\def\softwaretitle{\TeX\ and Software Engineering}
\def\softwareshorttitle{Software Engineering}

\renewcommand{\beamervictortitle}{CS-594 Eijkhout, Fall 2004}

\newcommand{\sectionframe}[1]%
  {\section{#1}
   \frame{\begin{center}\color{blue}\Large #1\end{center}}
  }
\newcommand{\subsectionframe}[1]%
  {\subsection{#1}
   \frame{\begin{center}\color{blue}\Large #1\end{center}}
  }

\input idxmacs

\begin{document}

\title{Parsing}
\author{Victor Eijkhout}
\date{Notes for CS 594 -- Fall 2004}

\frame{\titlepage}

\section{Introduction}

\frame[containsverbatim]{
  \frametitle{What is parsing?}
\begin{itemize}
\item Check for correctness: is this a legal program
\item Uncover meaning: convert to internal representation
\end{itemize}
}

\frame[containsverbatim]{
  \frametitle{Levels of parsing}
\begin{itemize}
\item Check for illegal characters
\item Build tokens (identifiers, numbers, operators~\&c) from
  characters (lexical analysis)
\item Statements tokens (syntactical analysis)
\item Semantical restrictions: define/use~\&c
\end{itemize}
}

\frame[containsverbatim]{
\begin{verbatim}
    my_array[ii] = 3+sin(1.0);
\end{verbatim}
\begin{itemize}
\item Lexical analysis: `\n{my_array}', `\verb+[+', `\n{ii}'~\&c.
\item Syntactical: this is an assignment; lhs is something you can
  assign to, rhs is arithmetic expression
\item Semantics: \n{my_array} is array, \n{ii}~is integer, \n{sin}~is
  defined function
\end{itemize}
}

\frame[containsverbatim]{
  \frametitle{Mixing of levels}
In Fortran:
\begin{verbatim}
    X = SOMENAME( Y+3 )
\end{verbatim}
\begin{itemize}
\item Lexical analysis simple
\item Syntax unclear: rhs can be function call or array element
\item Solution: give lexer access to symbol table
\end{itemize}
}

\frame[containsverbatim]{
  \frametitle{Correctness}
\begin{itemize}
\item Lexical analysis finds identifiers: \n{5ab} is illegal
\item Syntactical analysis finds expressions: \n{array[ii)} is illegal
\item In \TeX?
\end{itemize}
}

\frame[containsverbatim]{
  \frametitle{Parsing by automaton}
\begin{itemize}
\item Lexical analysis by Finite State Automaton
\item Syntactical analysis by Pushdown Automaton
\item In practice some mixing of levels
\end{itemize}
}

\subsection{Automata theory}

\frame[containsverbatim]{
  \frametitle{Terminology}
\begin{itemize}
\item Language: a set of words \[\{a^n|\hbox{$n$ is prime}\}\]
\item Grammar: set of rules that produces a language
\item Automaton: abstract device that can recognize a language
\item Derivation: actual sequency of rules or transitions used to
  derive a string
\item Parse tree: 2D way of writing derivation
\end{itemize}
}

\frame[containsverbatim]{
  \frametitle{to be precise}
\begin{itemize}
\item Grammar:
\begin{itemize}
\item Start symbol $S$
\item Terminal symbols $a,b,c,...$ from the alphabet
\item Non-terminals $A,B,C,...$, ultimately to be replaced
\item Rules $\alpha\rightarrow\beta$ where $\alpha,\beta$ strings of
  terminals and non-terminals
\end{itemize}
\item Automaton:
\begin{itemize}
\item Starting state
\item Accepting state
\item Work storage
\item Transition diagram
\end{itemize}
\end{itemize}
}

\frame[containsverbatim]{
  \frametitle{Types of languages}
\begin{itemize}
\item Languages differ in types of grammar rules
  $\alpha\rightarrow\beta$
\item Automata differ in amount of workspace
\item Four levels Chomsky hierarchy; many other types
\end{itemize}
}

\frame{
  \frametitle{Type 0}
\begin{itemize}
\item Regular languages
\item<2-> Turing machine: infinite tape
\item<2-> No restriction on grammar rules
\end{itemize}
}

\frame{
  \frametitle{Type 1}
\begin{itemize}
\item Context-sensitive languages
\item<2-> Linear-bounded automata
\item<2-> No rules $\alpha\rightarrow\epsilon$
\item<3-> Normal form: $AB\rightarrow BA$,
  $AB\rightarrow A\beta$
\end{itemize}
}

\frame{
  \frametitle{Type 2}
\begin{itemize}
\item Context-free languages
\item<2-> Push Down Automata
\item<2-> Only rules $A\rightarrow\alpha$
\item<3-> Normal form: $A\rightarrow b\alpha$ or $A\rightarrow b$
\end{itemize}
}

\frame{
  \frametitle{Type 3}
\begin{itemize}
\item Regular languages
\item<2-> Finite State Automata
\item<2-> Only rules $A\rightarrow bC$, $A\rightarrow b$
\end{itemize}
}

\sectionframe{Lexical analysis}

\frame[containsverbatim]{
  \frametitle{Function of a lexer}
\begin{itemize}
\item Recognize identifiers, numbers
\item Also side effects: store names of functions
\end{itemize}
}

\subsection{Regular languages}

\frame[containsverbatim]{
  \frametitle{Definition}
 Inductively, through regular expressions
\begin{itemize}
\item $\epsilon$ is the empty language
\item `$a$' denotes the language~$\{a\}$ ($a$~in alphabet)
\item if $\alpha,\beta$ denote languages~$A,\nobreak B$, then
\begin{itemize}
\item $\alpha\beta$ or $\alpha\cdot\beta$ denotes $\{xy|x\in A,y\in B\}$
\item $\alpha|\beta$ denotes the language $A\cup B$.
\item $\alpha^*$ denotes the language $\cup_{n\geq 0}A^n$.
\end{itemize}
\end{itemize}
}

\frame{
  \frametitle{Finite state automata}
\begin{itemize}
\item Starting state $S_0$
\item other states $S_i$; subset: accepting states
\item input alphabet~$I$; output alphabet~$O$
\item transition diagram $I\times S\rightarrow S$
\item<2-> non-deterministic: $I\cup\{\epsilon\}\times S\rightarrow S$
\item<3-> String is accepted if (any) sequence of transitions it
  causes leads to an accepting state
\end{itemize}
}

\frame[containsverbatim]{
  \frametitle{The NFA of a regular language}
Automaton that accepts~$\epsilon$:\\
\convertMPtoPDF{eps.1}{1}{1}

Automaton that accepts~$a$:\\
\convertMPtoPDF{a.1}{1}{1}
}

\frame[containsverbatim]{
  \frametitle{The NFA of a regular language}
Automaton that accepts~$A\cdot B$:\\
\convertMPtoPDF{AB.1}{1}{1}
}

\frame[containsverbatim]{
  \frametitle{The NFA of a regular language}
Automaton that accepts~$A\cup B$:\\
\convertMPtoPDF{AvB.1}{1}{1}
}

\frame[containsverbatim]{
  \frametitle{The NFA of a regular language}
Automaton that accepts~$A^*$:\\
\convertMPtoPDF{Astar.1}{1}{1}
}

\frame[containsverbatim]{
  \frametitle{Characterization}
\begin{itemize}
\item Any sufficiently long string $\alpha=uvw$
\item then $uv^nw$ also in the language
\end{itemize}
}

\subsection{DFAs and NFAs}

\frame{
  \frametitle{Example}
Language $a^*|b^*$:

\leavevmode
\convertMPtoPDF{nfa1.1}{1}{1}
$\quad\Rightarrow\quad$
\convertMPtoPDF{dfa1.1}{1}{1}
}

\frame[containsverbatim]{
  \frametitle{Example: keywords}
A bit like what happens in lexical analysis:

\convertMPtoPDF{begin.1}{1}{1}
}

\frame[containsverbatim]{
  \frametitle{}
Deterministic version

\convertMPtoPDF{begind.1}{1}{1}
}

\frame{
  \frametitle{Converting NFA to DFA}
\begin{itemize}
\item Introduce new states
\item<2-> new state is set of old states
\item<3-> new states closed under $\epsilon$-transitions
\end{itemize}
}

\frame{
\convertMPtoPDF{begin.1}{1}{1}

New $S_0=\{0,1,6\}$
\begin{itemize}
\item<2->$S_0+\n{B}\Rightarrow S_1=\{2,6,7\}$,
\item<3->$S_0+\neg\n{B}\Rightarrow S_6=\{6,7\}$
\item<4->$S_1+\n{E}\Rightarrow S_2=\{3,6,7\}$, et cetera
\end{itemize}
}

\frame[containsverbatim]{
  \frametitle{NFA for lexical analysis}
\convertMPtoPDF{compound.1}{1}{1}
}

\frame[containsverbatim]{
  \frametitle{small problems}
\begin{itemize}
\item Careful with the $\epsilon$-transition back:
\begin{verbatim}
  printf("And then he said ""Boo!""");
\end{verbatim}
final state reached three times: only transition when maximum string
recognized
\item Not always:
\begin{verbatim}
X = 4.E3
IF (4.EQ.VAR) THEN
\end{verbatim}
$\Rightarrow$ look-ahead needed
\end{itemize}
}

\subsectionframe{\lex}

\frame[containsverbatim]{
  \frametitle{A tool for lexical analysis}
\begin{itemize}
\item You write regular expressions, and \lex\ reports if it finds any
\item Three sections: definitions, rules, code
\end{itemize}
}

\frame[containsverbatim]{
  \frametitle{Example}
\footnotesize
\begin{verbatim}
%{
 int charcount=0,linecount=0;
%}

%%

. charcount++;
\n {linecount++; charcount++;}

%%
int main()
{
  yylex();
  printf("There were %d characters in %d lines\n",
         charcount,linecount);
  return 0;
}
\end{verbatim}
}

\frame[containsverbatim]{
  \frametitle{Running lex}
\lex\ code gets translated to C:

\begin{verbatim}
lex -t count.l > count.c
cc -c -o count.o count.c
cc -o counter count.o -ll
\end{verbatim}

Executable uses stdin/out, can be changed
}

\frame[containsverbatim]{
  \frametitle{Definitions section}
\begin{itemize}
\item C code: between \verb+%{ ... %}+ copied to top of C file
\item Definitions: `\verb+letter [a-zA-Z]+' (like \n{#define})
\item State definitions (later)
\end{itemize}
}

\frame[containsverbatim]{
  \frametitle{Example 2}
\begin{verbatim}
%{
 int charcount=0,linecount=0,wordcount=0;
%}
letter [^ \t\n]

%%

{letter}+ {wordcount++; charcount+=yyleng;}
.         charcount++;
\n        {linecount++; charcount++;}
\end{verbatim}
}

\frame[containsverbatim]{
  \frametitle{Rules section}
\begin{itemize}
\item Input is matched by character
\item Actions of longest match are taken, earliest if equal length
\item Matched text is \verb+char *yytext+, length \verb+int yyleng+
\end{itemize}
}

\frame[containsverbatim]{
  \frametitle{Example 2'}
\begin{verbatim}
{letter}+ {wordcount++; charcount+=yyleng;}
[ \t]     spacecount++;
.         charcount++;
\n        linecount++;
\end{verbatim}
}

\frame[containsverbatim]{
  \frametitle{Example 3}
\begin{verbatim}
[0-9]+          process_integer();
[0-9]+\.[0-9]*  |
\.[0-9]+        process_real();
\end{verbatim}
}

\frame[containsverbatim]{
  \frametitle{Regular expressions}
\def\titem#1{\item[{\tt #1}]}
\footnotesize
\begin{description}
\titem{.} Match any character except newlines.
\titem{\char`\\n} A newline character.
\titem{\char`\\t} A tab character.
\titem{\char`\^} The beginning of the line.
\titem{\$} The end of the line.
\titem{<expr>*} Zero or more occurrences of the expression.
\titem{<expr>+} One or more occurrences of the expression.
\titem{<expr>?} Zero or one occurrences of the expression.
\titem{(<expr1>|<expr2>)} One expression of another.
\titem{{[<set>]}} A set of characters or ranges, such as \verb+[,.:;]+
  or \verb+[a-zA-Z]+.
\titem{{[\char`\^<set>]}} The complement of the set, for instance
\verb+[^ \t]+.
\end{description}
}

\frame[containsverbatim]{
  \frametitle{Example: filtering comments}
\begin{verbatim}
%%
"/*".*"*/"  ;
.           |
\n          ECHO;
\end{verbatim}
works on 
\begin{verbatim}
This text /* has a */ comment
in it
\end{verbatim}
}

\frame[containsverbatim]{
Does not work on
\begin{verbatim}
This text /* has */ a /* comment */ in it
\end{verbatim}
}

\frame[containsverbatim]{
  \frametitle{Context}
\begin{itemize}
\item Match in context
\item Left context implemented through states:
\begin{verbatim}
<STATE>(some pattern) {...
\end{verbatim}
State switching:
\begin{verbatim}
<STATE>(some pattern) {some action; BEGIN OTHERSTATE;}
\end{verbatim}
Initial state is \n{INITIAL}, other states defined
\begin{verbatim}
%s MYSTATE
%x MYSTATE
\end{verbatim}
\end{itemize}
}

\frame[containsverbatim]{
  \frametitle{Use of states}
\begin{verbatim}
%x COMM

%%

.          |
\n         ECHO;
"/*"       BEGIN COMM;
<COMM>"*/" BEGIN INITIAL;
<COMM>.    |
<COMM>\n   ;

%%
\end{verbatim}
}

\frame[containsverbatim]{
  \frametitle{Context'}
\begin{itemize}
\item Right context:
\begin{verbatim}
abc/de {some action}
\end{verbatim}
\item context tokens not in \n{yytext}/\n{yyleng}
\end{itemize}
}

\frame[containsverbatim]{
  \frametitle{Example: text cleanup}
Input:
\begin{verbatim}
This    text (all of it  )has occasional lapses , in 
 punctuation(sometimes pretty bad) ,( sometimes not so).


(Ha! ) Is this : fun?Or what!
\end{verbatim}
Solution with context more compact than without.

Define:
\begin{verbatim}
punct [,.;:!?]
text [a-zA-Z]
\end{verbatim}
}

\frame[containsverbatim]{
  \frametitle{need for context}
\begin{itemize}
\item Consider `\n{),}' `\n{) ,}' `\n{)a}' `\n{) a}'
\item Rules \verb!")" " "+ {printf(")");}! depend on context
\end{itemize}
}

\frame[containsverbatim]{
  \frametitle{right context solution}
\footnotesize
\begin{verbatim}
")"" "+/{punct}       {printf(")");}
")"/{text}            {printf(") ");}
{text}+" "+/")"       {while (yytext[yyleng-1]==' ') yyleng--; ECHO;}

({punct}|{text}+)/"(" {ECHO; printf(" ");}
"("" "+/{text}        {while (yytext[yyleng-1]==' ') yyleng--; ECHO;}

{text}+" "+/{punct}   {while (yytext[yyleng-1]==' ') yyleng--; ECHO;}

^" "+                 ;
" "+                  {printf(" ");}
.                     {ECHO;}
\n/\n\n               ;
\n                    {ECHO;}
\end{verbatim}
}

\frame[containsverbatim]{
  \frametitle{left context solution}
Use defined states:
\begin{verbatim}
punct [,.;:!?]
text [a-zA-Z]

%s OPEN
%s CLOSE
%s TEXT
%s PUNCT
\end{verbatim}
}

\frame[containsverbatim]{
  \frametitle{left context solution, cont'd}
\footnotesize
\begin{verbatim}
" "+ ;

<INITIAL>"(" {ECHO; BEGIN OPEN;}
<TEXT>"("    |
<PUNCT>"("   {printf(" "); ECHO; BEGIN OPEN;}

")" {ECHO ; BEGIN CLOSE;}

<INITIAL>{text}+ |
<OPEN>{text}+    {ECHO; BEGIN TEXT;}
<CLOSE>{text}+   |
<TEXT>{text}+    |
<PUNCT>{text}+   {printf(" "); ECHO; BEGIN TEXT;}

{punct}+ {ECHO; BEGIN PUNCT;}

\n {ECHO; BEGIN INITIAL;}
\end{verbatim}
}

\sectionframe{Syntactical analysis}

\frame[containsverbatim]{
  \frametitle{Function of syntactical analysis}
\begin{itemize}
\item Recognize statements: loops, assignments~\&c
\item Convert to internal representation: parse trees
\begin{quote}
\hbox{%
\Tree [.* [.+ $2\quad$ $5\quad$ ] $3\quad$ ]
\Tree [.+ $2\quad$ [.* $5\quad$ $3\quad$ ] ]
}
\end{quote}
\item Semantics: define/use sequence~\&c
\end{itemize}
}

\frame[containsverbatim]{
  \frametitle{Grammars}
\begin{itemize}
\item Backus Naur, or other formalism
\item In \LaTeX: \n{bnf.sty}
\begin{examplewithcode}
\begin{bnf}
Expr: number Tail.
Tail: $\epsilon$ ; + number Tail; * number Tail
\end{bnf}
\end{examplewithcode}
(use my \n{bnf.env})
\item most language constructs are context-free
\end{itemize}
}

\frame[containsverbatim]{
  \frametitle{Concepts}
\begin{itemize}
\item Grammar rules $A\rightarrow x\alpha$
\item Derivations $abcA\gamma\Rightarrow abc x\alpha\gamma$
\item Parse tree
\begin{quote}
\Tree [.. abc [.A $x\quad$ $\alpha\quad$ ] $\gamma$ ]
\end{quote}
\end{itemize}
}

\subsectionframe{Context-free languages}

\frame[containsverbatim]{
  \frametitle{Definition}
\begin{itemize}
\item Grammatical: only rules $A\rightarrow \alpha$
\item From automata: pushdown automata
\end{itemize}
}

\frame[containsverbatim]{
  \frametitle{Pumping lemma}
\begin{itemize}
\item For every language there is an $n$ such that
\begin{itemize}
\item strings longer than~$n$ can be written $uvwxy$
\item and for all~$k$: $uv^kwx^ky$ also in the language
\end{itemize}
\item Proof: \begin{quote}
\Tree [.S $u$ [.A $v$ [.A $w$ ] $x$ ] $y$ ]
\end{quote}
\item Non-\{context-free\} language: $\{a^nb^nc^n\}$
\end{itemize}
}

\frame[containsverbatim]{
  \frametitle{Deterministic and non-deterministic}
\begin{itemize}
\item No equivalence
\item Deterministic: $L_c=\{\alpha c\alpha^R|c\not\in\alpha\}$
\item Non-deterministic: $L=\nobreak\{\alpha\alpha^R\}$
\end{itemize}
}

\def\mbx{\mathbf{x}}
\def\mba{\mathbf{a}}
\def\mbb{\mathbf{b}}
\def\mbA{\mathbf{A}}
\def\mbB{\mathbf{B}}
\def\mby{\mathbf{y}}
\def\mbf{\mathbf{f}}
\frame[containsverbatim]{
  \frametitle{Algebra of languages}
\begin{itemize}
\item Expressions $\mbx$ and $\mby$ denote languages, then
\begin{itemize}
\item union: $\mbx+\mby=\mbx\cup\mby$
\item concatenation: $\mbx\mby=\{w=xy|x\in\mbx,y\in\mby\}$
\item repetition: $\mbx^*=\{w=x^n|x\in\mbx,n\geq0\}$ 
\end{itemize}
\end{itemize}
}

\frame{
  \frametitle{Algebra: solving equations}
\begin{itemize}
\item Equation: $\mbx=\mba+\mbx\mbb$
\item Interpretation: $\mbx=\mba\cup\{w=xb|x\in\mbx,b\in\mbb\}$
\item Solving:
\begin{itemize}
\item first of all $\mbx\supset\mba$
\item then also $\mbx\supset\mba\cdot\mbb$
\item continuing: $\mbx\supset\mba\mbb\mbb$,\dots
\end{itemize}
\item verify: $\mbx=\mba\mbb^*$
\item<2> Numerically: $x=a/(1-b)$
\end{itemize}
}

\frame{
  \frametitle{Derive normal form}
\begin{itemize}
\item Normal form: $A\rightarrow a\alpha$
\item Write grammar of context-free language as $\mbx^t=\mbx^t\mbA+\mbf^t$,
where $\mbx$~non-terminals, $\mbf$~rhs that are of normal form,
$\mbx^t\mbA$~describes normal form rhs
\item<2> Example:\\
\hbox{%
$\displaystyle\begin{array}{l}
S\rightarrow aSb|XY|c\\
X\rightarrow YXc|b\\
Y\rightarrow XS
  \end{array}$\ \ %
  $\displaystyle [S,X,Y]=[S,X,Y]\left[
                \begin{matrix}\phi&\phi&\phi\\
                  Y&\phi&S\\ \phi&Xc&\phi \end{matrix}\right]
                +[aSb+c,b,\phi]$
}
\item<3->  Solution:
\[ \mbx^t = \mbf^t\mbA^* \]
\item<4-> Needed: more explicit expression for~$\mbA^*$.
\end{itemize}
}

\frame{
\begin{itemize}
\item Note $\mbA^*=\lambda+\mbA\mbA^*$
\item then normal form: \[
 \mbx^t = \mbf^t+\mbf^t\mbA\mbA^*
  =  \mbf^t+\mbf^t\mathbf{B} \]
where $\mathbf{B}=\mbA\mbA^*$.
\item<2-> $\mbB$:
\[ \mbB=\mbA\mbA^*=
  \mbA+\mbA\mbA\mbA^* =
  \mbA+\mbA\mbB \]
not necessarily normal form
\item<3-> Elements of~$\mbA$ that start with a
nonterminal can only start with nonterminals in~$\mbx$. Hence
substitute a rule from equation above.
\end{itemize}
}

\subsectionframe{Parsing strategies}

\frame[containsverbatim]{
  \frametitle{Top-down parsing}
\begin{itemize}
\item Start with $S$ on the stack, replace by appropriate rule, guided
  by input
\item Example: expression \n{2*5+3},
which is produced by the grammar
\begin{bnf}
Expr: number Tail.
Tail: $\epsilon$ ; + number Tail; * number Tail
\end{bnf}
\end{itemize}
}

\frame[containsverbatim]{
\begin{tabbing}
start symbol on stack:$\quad$\=${}2*5+3{}\quad$\=\kill
initial queue:\>$2*5+3$\\
start symbol on stack:\>\>Expr\\
replace\>\>number Tail\\
match\>${}*5+3$\>Tail\\
replace\>\>* number Tail\\
match\>$5+3$\>number Tail\\
match\>${}+3$\>Tail\\
replace\>\>+ number Tail\\
match\>$3$\>number Tail\\
match\>$\epsilon$\>Tail\\
match
\end{tabbing}
\[ E\Rightarrow n\, T\Rightarrow n*n\,T\Rightarrow
    n*n+n\,T\Rightarrow n*n+n \]
$LL(1)$
}

\frame[containsverbatim]{
  \frametitle{}
Equivalent grammar:
\begin{bnf}
Expr: number; number + Expr; number * Expr
\end{bnf}
assuming one more token look-ahead:
\begin{tabbing}
start symbol on stack:$\quad$\=${}2*5+3{}\quad$\=\kill
initial queue:\>$2*5+3$\\
start symbol on stack:\>\>Expr\\
replace\>\>number * Expr\\
match\>${}5+3$\>Tail\\
replace\>\>number + Expr\\
match\>$3$\>Expr\\
replace\>$3$\>number\\
match\>$\epsilon$
\end{tabbing}
$LL(2)$
}

\frame[containsverbatim]{
  \frametitle{$LL$ is recursive descent}
Finding of proper rule:
\begin{verbatim}
define FindIn(Sym,NonTerm)
  for all expansions of NonTerm:
    if leftmost symbol == Sym
      then found
    else if leftmost symbol is nonterminal
      then FindIn(Sym,that leftmost symbol)
FindIn(symbol,S);
\end{verbatim}
}

\frame[containsverbatim]{
  \frametitle{Problems with $LL(k)$}
\begin{itemize}
\item Some grammars are not $LL(k)$ for any~$k$:\\
if \n{A<B} and \n{A<B>} both legal
\item Infinite loop:
\begin{bnf}
Expr: number; Expr + number; Expr * number
\end{bnf}
\end{itemize}
}

\frame[containsverbatim]{
  \frametitle{Bottom-up: Shift-reduce}
\begin{itemize}
\item Recognize productions from terminals
\item Example: expression $2*5+3$ produced by
\begin{bnf}
E: number; E + E; E * E
\end{bnf}
\end{itemize}
}

\frame[containsverbatim]{
\begin{tabbing}
shift, shift, reduce: abcd\=initial state:initial\=\kill
\>stack\>queue\\
initial state:\>\>$2*5+3$\\
shift\>2\>*5+3\\
reduce\>E\>*5+3\\
shift\>E*\>5+3\\
shift\>E*5\>+3\\
reduce\>E*E\>+3\\
reduce\>E\>+3\\
shift, shift, reduce\>E+E\\
reduce\>E\\
\end{tabbing}
\[ E\Rightarrow E+E\Rightarrow E+3\Rightarrow E*E+3
    \Rightarrow E*5+3\Rightarrow 2*5+3 \]
$LR(0)$
}

\frame[containsverbatim]{
  \frametitle{Where to start reducing?}
\begin{itemize}
\item `Greedy' reducing is not always best
\item Grammar:
\begin{bnf}
S:aAcBe.
A:bA;b.
B:d.
\end{bnf}
and string \n{abbcde}.
\item Derivation 1:
\[ \n{abbcde}\Leftarrow\n{abAcde}\Leftarrow\n{aAcde}\Leftarrow
    \n{aAcBe}\Leftarrow\n{S}. \]
\item Derivation 2:
\[ \n{abbcde}\Leftarrow\n{aAbcde}\Leftarrow aAAcde \Leftarrow ? \]
\end{itemize}
}

\frame[containsverbatim]{
  \frametitle{Handle}
\begin{quote} If $S\Rightarrow^*\alpha Aw\Rightarrow\alpha\beta w$ is a
  right-most derivation, then $A\rightarrow\beta$ at the position
  after~$\alpha$ is a handle of~$\alpha Aw$.
\end{quote}
Question: how to find handles
}

\frame[containsverbatim]{
  \frametitle{Operator-precedence grammars}
\begin{itemize}
\item Operator grammar: `expr-op-expr'
\item Formally: never two consecutive non-terminals, and no
  rules~$A\rightarrow\nobreak\epsilon$.
\item Declare precedences (and associativity)
\begin{tabular}{r|ccc}
&\hbox{number}&$+$&$\times$\\\hline
\hbox{number}&&$\gtrdot$&$\gtrdot$\\
$+$&$\lessdot$&$\gtrdot$&$\lessdot$\\
$\times$&$\lessdot$&$\gtrdot$&$\gtrdot$
\end{tabular}\\
\end{itemize}
}

\frame[containsverbatim]{
\begin{itemize}
\item Annotate expression: $5+2*3$ becomes
$\lessdot5\gtrdot+\lessdot2\gtrdot*\lessdot3\gtrdot$
\item Reducing: $E+E*E$
\item Insert precedences: ${\lessdot}+{\lessdot}*{\gtrdot}$
\item Scan forward to closing, back to open:
  ${\lessdot}E*E{\gtrdot}$~is handle
\item Reduce: $E+E$
\item $\Rightarrow$ precendences correctly observed
\item (note: no global scanning; still shift-reduce like)
\end{itemize}
}

\frame[containsverbatim]{
  \frametitle{Definition of $LR$ parser}
An LR parser has the following components
\begin{itemize}
\item Stack and input queue; stack will also contain states
\item Actions `shift', `reduce', `accept', `error'
\item Functions \n{Action} and \n{Goto}
\begin{itemize}
\item With input symbol~$a$ and state on top of the stack~$s$:
\item If \n{Action}$(a,s)$ is `shift', then $a$ and a new state
  $s'=\n{Goto}(a,s)$ are pushed on the stack.
\item If \n{Action}$(a,s)$ is `reduce $A\rightarrow\beta$' where
  $|\beta|=r$, then $2r$ symbols are popped from the stack, a new
  state $s'=\n{Goto}(a,s'')$ is computed based on the newly exposed
  state on the top of the stack, and $A$ and~$s'$ are pushed. The
  input symbol~$a$ stays in the queue.
\end{itemize}
\end{itemize}
More powerful than simple shift/reduce; states much more complicated
}

\frame[containsverbatim]{
  \frametitle{motivating example}
\begin{itemize}
\item Grammar
\begin{bnf}
E: E + E; E * E
\end{bnf}
input string $1+2*3+4$.
\item Define precedences:
\def\op{\mathop{\mathbf{op}}}
$\op(+)=1, \op(\times)=2$
\item Define states as; initially state~0
\item Transitions: push operator precedence, do not change state for numbers
\item Shift/reduce strategy: reduce if precedence of input lower than
  of stack top
\end{itemize}
}

\frame[containsverbatim]{
\def\op{\mathop{\mathbf{op}}}
\footnotesize
\begin{tabbing}
1 $S_0$ + $S_1$ 2 $S_1$ * $S_2$ 3 $S_2$ \= $1+2*3+4$ \=\kill
\>$1+2*3+4$\> push symbol; highest precedence is 0\\
1 $S_0$\>$+2*3+4$\>highest precedence now becomes 1\\
1 $S_0$ + $S_1$\>$2*3+4$\\
1 $S_0$ + $S_1$ 2 $S_1$\>$*3+4$\>highest precedence becoming 2\\
1 $S_0$ + $S_1$ 2 $S_1$ * $S_2$\>$3+4$\\
1 $S_0$ + $S_1$ 2 $S_1$ * $S_2$ 3 $S_2$\>$+4$\>reduce because $\op(+){}<2$\\
1 $S_0$ + $S_1$ 6 $S_1$\>$+4$\>the highest exposed precedence is 1\\
1 $S_0$ + $S_1$ 6 $S_1$ + $S_1$\>$4$\\
1 $S_0$ + $S_1$ 6 $S_1$ + $S_1$ 4 $S_1$ \>\>at the end of the queue we reduce\\
1 $S_0$ + $S_1$ 10 $S_1$ \\
11
\end{tabbing}
}

\frame[containsverbatim]{
  \frametitle{Parser states}
\begin{description}
\item[item] Grammar rule with location indicated.\\
 From \parserule{A}{B C} items:
 \parserule{A}{.B C}, \parserule{A}{B .C}, \parserule{A}{B C.}\\
(stack is left of dot, queue right)
\item[closure of an item]   The smallest set that
\begin{itemize}
\item Contains that item;
\item If $I$ in closure and $I={}$\parserule{A}{$\alpha$ .B $\beta$}
 with~\n{B} nonterminal, then $I$~contains all items \parserule{B}{.$\gamma$}.
\end{itemize}
\item[state] Set of items.
\item[follow] of~\n{A}: set of all terminals that can follow $A$'s~expansions
\end{description}
}

\frame[containsverbatim]{
  \frametitle{Motivation: valid items}
\begin{itemize}
\item  Recognized so far: $\alpha\beta_1$
\item Consider  item \parserule{A}{$\beta_1$.$\beta_2$}
\item Item is called \emph{valid}, if rightmost derivation
\[ S\Rightarrow^*\alpha Aw\Rightarrow \alpha\beta_1\beta_2w \]
\item Case: $\beta_2=\epsilon$, then $A\rightarrow\beta_1$ 
  handle: reduce
\item Case: $\beta_2\not=\epsilon$, so shift~$\beta_2$.
\end{itemize}
}

\frame[containsverbatim]{
  \frametitle{example of valid items}
\begin{itemize}
\item String~\n{E+T*} in grammar:
\begin{bnf}
E:E+T; T.
T:T*F; F.
F:(E); id
\end{bnf}
\item Derivations
\[ E\Rightarrow E+T\Rightarrow E+T*F \]
\[ E\Rightarrow E+T\Rightarrow E+T*F\Rightarrow E+T*(E) \]
\[ E\Rightarrow E+T\Rightarrow E+T*F\Rightarrow E+T*\mathrm{id} \]
give items \parserule{T}{T*.F} \parserule{F}{.(E)} \parserule{F}{.id}
\end{itemize}
}

\frame[containsverbatim]{
  \frametitle{States and transitions}
\begin{itemize}
\item New start symbol~\n{S'}, added production \parserule{S'}{S}.
\item  Starting state is closure of \parserule{S'}{.S}.
\item Transition $d(s,\n{X})$: the closure of
\[ \{ \parserule{A}{$\alpha$ X. $\beta$}
          | \hbox{\parserule{A}{$\alpha$ .X $\beta$} is in $s$} \} \]
\item The initial state is the closure of \parserule{S'}{.S}.
\end{itemize}
}

\frame{
\footnotesize
Grammar: $S\rightarrow(S)S\mid|\epsilon$
%\begin{bnf}S:(S)S;$\epsilon$\end{bnf}

States (after adding \parserule{S'}{.S}):
\begin{enumerate}\setcounter{enumi}{-1}
\item<2-> $\{ \parserule{S'}{.S}, \parserule{S}{.(S)S}, \parserule{S}{.} \}$ 
\item<3-> $\{ \parserule{S'}{S.} \}$ 
\item<4-> $\{ \parserule{S}{(.S)S}, \parserule{S}{.(S)S}, \parserule{S}{.} \}$ 
\item<5-> $\{ \parserule{S}{(S.)S} \}$ 
\item<6-> $\{ \parserule{S}{(S).S}, \parserule{S}{.(S)S}, \parserule{S}{.} \}$ 
\item<7-> $\{ \parserule{S}{(S)S.} \}$
\end{enumerate}
with transitions ($\{A\rightarrow\alpha_\bullet X\beta\in s\Rightarrow
\mathop(cl)(A\rightarrow\alpha X_\bullet\beta)$):
\begin{itemize}
\item<8->$d(0,S) = 1 $
\item<9->$d(0,'(') = 2 $
\item<10->$d(2,S) = 3 $
\item<11->$d(2,'(') = 2 $
\item<12->$d(3,')') = 4 $
\item<13->$d(4,S) = 5 $
\item<14->$d(4,'(') = 2$
\end{itemize}
}

\frame[containsverbatim]{
  \frametitle{Stack handling}
\begin{tabbing}
\textbf{Loop}:\\
(1) \textbf{else} \=\kill
(1) \textbf{if} \>th\=e current state contains \parserule{S'}{S.}\\
\>\>accept the string\\
(2) \textbf{else} \>\textbf{if} the current state %
    contains any other final item \parserule{A}{$\alpha$.}\\
\>\>pop all the tokens in $\alpha$ from the stack,\\
\>\>\hspace{20pt}along with the corresponding states; \\
\>\>let $s$ be the state left on top of the stack:\\
\>\>\hspace{20pt}    push \n{A}, push \n{d($s$,A)}\\
(3) \textbf{else} \>\textbf{if} the current state contains any item %
    \parserule{A}{$\alpha$ .x $\beta$},\\
\>\>$\qquad$ where x is the next input token\\
\>\>let $s$ be the state on top of the stack: %
    push \n{x}, push \n{d($s$,x)}\\
(1) \=\kill
\>\textbf{else} report failure
\end{tabbing}
}

\subsectionframe{Ambiguity and conflicts}

\frame[containsverbatim]{
  \frametitle{Shift/reduce conflict}
\begin{itemize}
\item Grammar for $2+5*3$:
\begin{bnf}
<expr>: <number>; <expr> + <expr>; <expr> $\times$ <expr>.
\end{bnf}
\item interpretations:
\hbox{%
\Tree [.* [.+ $2\quad$ $5\quad$ ] $3\quad$ ]
\Tree [.+ $2\quad$ [.* $5\quad$ $3\quad$ ] ]
}
\item Parse: reduce $2+5$ to \n{<expr> + <expr>},\\
then reduce to~\n{<expr>}, or~shift the minus?
\end{itemize}
}

\frame[containsverbatim]{
  \frametitle{solutions}
\begin{itemize}
\item Reformulate the grammar as
\begin{bnf}
<expr>: <mulex>; <mulex> + <mulex>.
<mulex>: <term>; <term> $\times$ <term>.
<term>: number.
\end{bnf}
\item new parse:
\begin{quote}
\Tree [.expr [.mulex [.term 2 ] ] + [.mulex [.term 5 ] * [.term 3 ] ] ]
\end{quote}
\item Introduce precedence of operators.\\ Possibly more efficient if
  large number of operators.
\end{itemize}
}

\frame[containsverbatim]{
  \frametitle{`Dangling else'}
\begin{itemize}
\item Consider the grammar
\begin{bnf}
<statement>: if <clause> then <statement>; if <clause> then <statement> else <statement>
\end{bnf}
\item  string
\begin{quote}
\tt if c$_1$ then if c$_2$ then s$_1$ else s$_2$
\end{quote}
\item Interpretations:
\hbox{\tiny
\Tree [.S If Then [.S If Then ] Else ]
\hskip-1cm
\Tree [.S If Then [.S If Then Else ] ]
}
\end{itemize}
}

\frame[containsverbatim]{
  \frametitle{Reduce/reduce conflict}
\begin{itemize}
\item Grammar for \n{x y c}
\begin{bnf}
A : B c d ; E c f.
  B : x y.
  E : x y.
\end{bnf}
\item $LR(1)$ parser: shift \n{x y},\\ then reduceto \n{B} or~\n{E}?
\item $LR(2)$ parser: sees \n{d} or~\n{f}
\item An $LL$ parser: ambiguity in the first 3 tokens\\
  $LL(4)$ parser can sees \n{d} or~\n{f}.
\end{itemize}
}

\frame[containsverbatim]{
  \frametitle{}
\begin{itemize}
\item Grammar for $x\,y\,c^n\,\{d|f\}$:
\begin{bnf}
  A : B C d ; E C f.
  B : x y .
  E : x y .
  C : c ; C c.
\end{bnf}
\item confusing for any $LR(n)$ or $LL(n)$ parser with a
fixed amount of look-ahead
\item rewrite:
\begin{bnf}
  A    : BorE c d ; BorE c f.
  BorE : x y.
\end{bnf}
or (for an $LL(n)$ parser):
\begin{bnf}
  A    : BorE c tail.
  tail : d ; f.
  BorE : x y.
\end{bnf}
\end{itemize}
}


\subsectionframe{\yacc}

\frame[containsverbatim]{
  \frametitle{\yacc\ and \lex}
\begin{itemize}
\item \lex\ produces tokens
\item \yacc\ analyzes sequences of tokens
\item lexer returns on recognizing a token
\item main program in \yacc\ code
\end{itemize}
}

\frame[containsverbatim]{
  \frametitle{File structure}
\begin{verbatim}
  ...definitions...
%%
  ...rules...
%%
  ...code...
\end{verbatim}
\begin{itemize}
\item Default main calls \n{yyparse}
\end{itemize}
}

\frame[containsverbatim]{
  \frametitle{Example: \yacc\ code header}
\footnotesize
File name \n{words.y}
\begin{verbatim}
%{

#include <stdlib.h>
#include <string.h>
  int yylex(void);
#include "words.h"
  int nwords=0;
#define MAXWORDS 100
  char *words[MAXWORDS];
%}

%token WORD

%%
\end{verbatim}
}

\frame[containsverbatim]{
  \frametitle{include file}
Generated by running \yacc:
\begin{verbatim}
%% cat words.h
#define WORD 257
\end{verbatim}
}

\frame[containsverbatim]{
  \frametitle{Example: \lex\ code}
\footnotesize
\begin{verbatim}
%{

#include "words.h"
int find_word(char*);
extern int yylval;
%}

%%

[a-zA-Z]+ {yylval = find_word(yytext);
	   return WORD;}
.         ;
\n        ;

%%
\end{verbatim}
}

\frame[containsverbatim]{
\footnotesize
\begin{verbatim}
text : ;
       | text WORD  ; {
            if ($2<0) printf("new word\n");
            else printf("matched word %d\n",$2);
                      }
%%

int find_word(char *w)
{ int i;
  for (i=0; i<nwords; i++)
    if (strcmp(w,words[i])==0) return i;
  words[nwords++] = strdup(w); return -1;
}

int main(void)
{
  yyparse();
  printf("there were %d unique words\n",nwords);
}
\end{verbatim}
}

\frame[containsverbatim]{
  \frametitle{Running \lex\ and \yacc}
\begin{verbatim}
/* create and compile yacc C file */
yacc -d -t -o YACCFILE.c YACCFILE.y
cc -c -o YACCFILE.o YACCFILE.c

/* create and compile lex C file */
lex -t LEXFILE.l > LEXFILE.c
cc -c -o LEXFILE.o LEXFILE.c

/* link together */
cc YACCFILE.o LEXFILE.o -o YACCPROGRAM -ly -ll
\end{verbatim}
}

\frame[containsverbatim]{
  \frametitle{Make with suffix rules}
\footnotesize
\begin{verbatim}
# disable normal rules
.SUFFIXES:
.SUFFIXES: .l .y .o
# lex rules
.l.o :
        lex -t $*.l > $*.c
        cc -c $*.c -o $*.o
# yacc rules
.y.o :
        if [ ! -f $*.h ] ; then touch $*.h ; fi
        yacc -d -t -o $*.c $*.y 
        cc -c -o $*.o $*.c ;
        rm $*.c
# link lines
lexprogram : $(LEXFILE).o
        cc $(LEXFILE).o -o $(LEXFILE) -ll
yaccprogram : $(YACCFILE).o $(LEXFILE).o
        cc $(YACCFILE).o $(LEXFILE).o -o $(YACCFILE) -ly -ll
\end{verbatim}
}

\frame[containsverbatim]{
  \frametitle{\yacc\ definitions section}
\begin{itemize}
\item C code in between \verb+%{ ... %}+
\item Token definitions: the \lex\ return tokens
\item Associativity rules (later)
\end{itemize}
}

\frame[containsverbatim]{
  \frametitle{Tokens}
\begin{itemize}
\item Definition: \verb+%token FOO+
\item In \n{.h} file: \verb+#define FOO 257+ (or so)
\item \lex\ code: \verb+return FOO+
\end{itemize}
}

\frame[containsverbatim]{
  \frametitle{Returning values over the stack}
\begin{itemize}
\item \lex\ assigns to \n{yylval}
\item value is put on top of stack
\item if a \yacc\ rule is matched: \verb+$1, $2, $3+ are assigned\\
(as many as elements in rhs)
\item replace stack top: assign to \verb+$$+
\end{itemize}
}

\frame[containsverbatim]{
  \frametitle{Calculator example: \lex\ code}
\footnotesize
\begin{verbatim}
%{
#include "calc1.h"
void yyerror(char*);
extern int yylval;
%}

%%
[ \t]+ ;
[0-9]+     {yylval = atoi(yytext);
            return INTEGER;}
[-+*/]     {return *yytext;}
"("        {return *yytext;}
")"        {return *yytext;}
\n         {return *yytext;}
.          {char msg[25];
            sprintf(msg,"%s <%s>","invalid character",yytext);
            yyerror(msg);}
\end{verbatim}
}

\frame[containsverbatim]{
  \frametitle{Calculator example: \yacc\ code}
\footnotesize
\begin{verbatim}
%{
int yylex(void);
#include "calc1.h"
%}

%token INTEGER

%%

program:
        line program
        | line
line:
        expr '\n'           { printf("%d\n",$1); }
        | 'n'
\end{verbatim}
}

\frame[containsverbatim]{
  \frametitle{Calculator example: \yacc\ code, cont'd}
\footnotesize
\begin{verbatim}
expr:
        expr '+' mulex      { $$ = $1 + $3; }
        | expr '-' mulex    { $$ = $1 - $3; }
        | mulex             { $$ = $1; }
mulex:
        mulex '*' term      { $$ = $1 * $3; }
        | mulex '/' term    { $$ = $1 / $3; }
        | term              { $$ = $1; }
term:
        '(' expr ')'        { $$ = $2; }
        | INTEGER           { $$ = $1; }

\end{verbatim}
}

\frame[containsverbatim]{
  \frametitle{Calculator with variables}
\begin{itemize}
\item Simple case: single letter variables
\item more complicated: names
\item Extra rule: assignments
\item \lex\ returns
\begin{itemize}
\item \n{double} values \item \n{int} index of variable
\end{itemize}
\end{itemize}
}

\frame[containsverbatim]{
  \frametitle{Multiple return types}
\footnotesize
\begin{itemize}
\item Declare possible return types:
\begin{verbatim}
%union {int ival; double dval;}
\end{verbatim}
\item Connect types to return tokens:
\begin{verbatim}
%token <ival> NAME
%token <dval> NUMBER
\end{verbatim}
\item The types of non-terminals need to be given:
\begin{verbatim}
%type <dval> expr
%type <dval> mulex
%type <dval> term
\end{verbatim}
\item In \n{.h}~file will now have
\begin{verbatim}
#define name 258
#define NUMBER 259
typedef union {int ival; double dval;} YYSTYPE;
extern YYSTYPE yylval;
\end{verbatim}
\end{itemize}
}

\frame[containsverbatim]{
  \frametitle{Multiple return types: \lex\ code}
\begin{verbatim}
[ \t]+ ;
(([0-9]+(\.[0-9]*)?)|([0-9]*\.[0-9]+)) {
            yylval.dval = atof(yytext);
            return DOUBLE;}
[-+*/=]    {return *yytext;}
"("        {return *yytext;}
")"        {return *yytext;}
[a-z]      {yylval.ivar = *yytext - 'a';
            return NAME;} /* more later */
\n         {return *yytext;}
.          {char msg[25];
            sprintf(msg,"%s <%s>","invalid character",yytext);
            yyerror(msg);}
\end{verbatim}
}

\frame[containsverbatim]{
  \frametitle{Example: calculator with variables}
Tokens are \n{double} numbers, or variables (\n{int} index in table)
\begin{verbatim}
%{
#define NVARS 100
char *vars[NVARS]; double vals[NVARS]; int nvars=0;
%}
%union { double dval; int ivar; }
%token <dval> DOUBLE
%token <ivar> NAME
%type <dval> expr
%type <dval> mulex
%type <dval> term
\end{verbatim}
}

\frame[containsverbatim]{
  \frametitle{Symbol table handling}
\begin{itemize}
\item \lex\ parses variable names:
\begin{verbatim}
[a-z][a-z0-9]* {
            yylval.ivar = varindex(yytext);
            return NAME;}
\end{verbatim}
\item names are dynamically stored:
\begin{verbatim}
int varindex(char *var)
{
  int i;
  for (i=0; i<nvars; i++)
    if (strcmp(var,vars[i])==0) return i;
  vars[nvars] = strdup(var);
  return nvars++;
}
\end{verbatim}
\end{itemize}
}

\frame[containsverbatim]{
  \frametitle{Arithmetic}
Largely as before:
\begin{verbatim}
expr:
        expr '+' mulex      { $$ = $1 + $3; }
        | expr '-' mulex    { $$ = $1 - $3; }
        | mulex             { $$ = $1; }
mulex:
        mulex '*' term      { $$ = $1 * $3; }
        | mulex '/' term    { $$ = $1 / $3; }
        | term              { $$ = $1; }
term:
        '(' expr ')'        { $$ = $2; }
        | NAME              { $$ = vals[$1]; }
        | DOUBLE            { $$ = $1; }
\end{verbatim}
}

\frame[containsverbatim]{
  \frametitle{Assignments}
\begin{verbatim}
line:
        expr '\n'             { printf("%g\n",$1); }
        | NAME '=' expr '\n'  { vals[$1] = $3; }
\end{verbatim}
}

\frame[containsverbatim]{
  \frametitle{Operator precedence and associativity}
Increasing precedence order:
\begin{verbatim}
%left '+' '-'
%left '*' '/'
%right '^'
%%
expr:
        expr '+' expr ;
        expr '-' expr ;
        expr '*' expr ;
        expr '/' expr ;
        expr '^' expr ;
        number ;
\end{verbatim}
}

\frame[containsverbatim]{
  \frametitle{Unary operators}
Declare non-associative;\\
indicate presence in rule
\begin{verbatim}
%left '-' '+'
%nonassoc UMINUS
%
expression : expression '+' expression
    | expression '-' expression
    | '-' expression %prec UMINUS
\end{verbatim}
}

\frame[containsverbatim]{
  \frametitle{Error handling}
\footnotesize
\begin{itemize}
\item Default: \n{yyerror} prints \n{syntax error}
\item Better:\\
\lex\ code:
\begin{verbatim}
\n    lineno++;
\end{verbatim}
\yacc\ code:
\begin{verbatim}
void yyerror(char *s)
{
  printf("Parsing failed in line %d because of %s\n",lineno,s);
  return;
}
\end{verbatim}
\item Your own error messages:
\begin{verbatim}
expr : name '[' name ']'
        {if (!is_array($1) yyerror("array name expected");
\end{verbatim}
\end{itemize}
}

\frame[containsverbatim]{
  \frametitle{Error recovery}
\begin{itemize}
\item Use of \n{error} token:
\begin{verbatim}
foo : bar baz ;
    | error baz printf("Hope for the best\n");
\end{verbatim}
\end{itemize}
}

\end{document}

\frame[containsverbatim]{
  \frametitle{}
\begin{itemize}
\item
\end{itemize}
}

