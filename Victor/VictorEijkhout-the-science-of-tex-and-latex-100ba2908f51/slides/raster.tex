\documentclass{beamer}

\usepackage{beamerthemevictor,comment,verbatim,graphicx,amssymb}

\newcommand{\TbT}{\TeX\ by Topic}
\newcommand{\Metafont}{\textsc{Metafont}}
\let\metafont\Metafont
\newcommand\ascii{{\sc Ascii}}
\newcommand\ebcdic{{\sc Ebcdic}}
\newcommand\gnuplot{\protect\n{gnuplot}}
\def\mod{\mathbin{\mathrm{mod}}}
\def\eqref#1{equation~(\ref{#1})}
\def\Eqref#1{Equation~(\ref{#1})}
\def\inv{^{-1}}
\def\endproof{\unskip\penalty10000 \hskip10pt plus 1fill$\bullet$\par}

\def\etal{\textit{et al.}}
\def\lex{{\it lex}}
\def\yacc{{\it yacc}}
\def\make{{\it make}}
\def\web{{\textsc{Web}}}
\def\n{\bgroup
  \catcode`\#=12 \catcode`\_=12 \catcode`\^=12
  \catcode`\&=12 
  \tt \let\next=}
\newcommand{\cs}[1]{{\tt\char`\\#1}}
{\catcode`\.=13
 \gdef.{${}_\bullet$}
}
\def\parserule#1{%
    \ifmmode \n{#1}\,\rightarrow\relax
    \else $\n{#1}\,\rightarrow{}$\nobreak
    \fi
    \hbox\bgroup\catcode`\.=13 \tt\let\next=}

\newenvironment{remark}%
    {\medskip\noindent{\bf Remark.}}{}
\usepackage{bnf}
\input bnf.env

\generalcomment{inputwithcode}
  {\begingroup\def\ProcessCutFile{}}
  {\verbatiminput{\CommentCutFile}
   \endgroup
   \input{\CommentCutFile}
  }
\generalcomment{examplewithcode}
  {\begingroup
     \def\ProcessCutFile{}\def\CommentCutFile{example.tex}}
  {\verbatiminput{\CommentCutFile}
   Output:
   \begin{quote}
     \begin{minipage}[t]{3in}
        \everypar{}
        \input{\CommentCutFile} 
     \end{minipage}
   \end{quote}
   \endgroup
  }
\specialcomment{ttexamplewithcode}
  {\begingroup\def\ProcessCutFile{}}
  {\verbatiminput{\CommentCutFile}
   \endgroup
   Output:
   \begin{quote}\begin{ttfamily} \input{\CommentCutFile} 
     \end{ttfamily}\end{quote}
  }
\generalcomment{mathexamplewithcode}
  {\begingroup\def\ProcessCutFile{}}
  {\verbatiminput{\CommentCutFile}
   Output:
   \begin{equation} \input{\CommentCutFile} \end{equation}
   \endgroup
  }

\def\chaptertitle{\csname\chaptername title\endcsname}
\def\chaptershorttitle{\csname\chaptername shorttitle\endcsname}

\def\lambdatitle{Lambda calculus in \TeX}
\def\lambdashorttitle{Lambda calculus}

\def\latextitle{Yet another \LaTeX\ tutorial}
\def\latexshorttitle{\LaTeX}

\def\lextitle{A \lex\ tutorial}
\def\lexshorttitle{\lex}

\def\yacctitle{A \yacc\ tutorial}
\def\yaccshorttitle{\yacc}

\expandafter\def\csname tex1title\endcsname{\TeX\ -- macro programming}
\expandafter\let\csname tex1shorttitle\expandafter\endcsname
    \csname tex1title\endcsname
\expandafter\def\csname tex2title\endcsname{\TeX\ -- visuals}
\expandafter\let\csname tex2shorttitle\expandafter\endcsname
    \csname tex2title\endcsname

\def\parsingtitle{An introduction to parsing}
\def\parsingshorttitle{Parsing}

\def\hashingtitle{Hashing}
\let\hashingshorttitle\hashingtitle

\def\beziertitle{Curve approximation by splines}
\def\beziershorttitle{Splines}

\def\dynamictitle{Dynamic programming}
\let\dynamicshorttitle\dynamictitle

\def\paragraphtitle{Line breaking}
\let\paragraphshorttitle\paragraphtitle

\def\pagetitle{Page breaking in \TeX}
\def\pageshorttitle{Page breaking}

\def\completenesstitle{Introduction to NP-Completeness}
\def\completenesshortstitle{NP-Completeness}

\def\rastertitle{Introduction to Raster Graphics}
\def\rastershortstitle{Raster Graphics}

\def\pythontitle{Really quick Python tutorial}
\def\pythonshortstitle{Python}

\def\encodingtitle{Character encoding}
\let\encodingshorttitle\encodingtitle

\def\rastertitle{Raster Graphics}
\let\rastershorttitle\rastertitle

\def\gnuplottitle{Gnuplot}
\let\gnuplotshorttitle\gnuplottitle

\def\fontstitle{Fonts and font files}
\def\fontsshorttitle{Fonts}

\def\softwaretitle{\TeX\ and Software Engineering}
\def\softwareshorttitle{Software Engineering}

\renewcommand{\beamervictortitle}{CS-594 Eijkhout, Fall 2004}

\newcommand{\sectionframe}[1]%
  {\section{#1}
   \frame{\begin{center}\color{blue}\Large #1\end{center}}
  }
\newcommand{\subsectionframe}[1]%
  {\subsection{#1}
   \frame{\begin{center}\color{blue}\Large #1\end{center}}
  }

\input idxmacs

\begin{document}

\title{Raster Graphics}
\author{Victor Eijkhout}
\date{Notes for CS 594 -- Fall 2004}

\frame{\titlepage}

\frame{
  \frametitle{From mathematics to pixels}
\begin{itemize}
\item Shapes and curves described mathematically (bezier)
\item Screen has pixels
\item different arithmetic
\item rounding behaviour
\item Vector graphics vs Bitmap, Raster
\end{itemize}
}

\section{Basic raster algorithms}
\subsection{Lines}

\frame{
  \frametitle{Line drawing}
\begin{itemize}
\item Symmetry: limit to slope~$\leq 1$
\pgfimage{one-per-column}
\item one pixel on per column
\end{itemize}
}

\frame{
  \frametitle{Incremental drawing}
\begin{itemize}
\item Line $y=mx+B$, slope $m=\delta y/\delta x$.
\item Pixels: $\delta x\equiv1$, so $\delta y=m$
\[y_{i+1}=y_i+\delta y.\]
\item implementation
\begin{tabbing}
let $x_0,y_0$ and $m$ be given, then\\
for \=$i=0\ldots n-1$\\
\>\n{WritePixel}$(x_i,\mathop{\textrm{Round}}(y_i))$\\
\>$x_{i+1}=x_i+1$\\
\>$y_{i+1}=y_i+m$
\end{tabbing}
\item roundoff, cost
\end{itemize}
}

\subsection{Midpoint algorithm}

\frame{
  \frametitle{Midpoint algorithm}
\begin{itemize}
\item Given `on' pixel, choices are 1~right, 2~right-and-up
\pgfimage[height=1.5in]{line-midpoint}
\item Write
\[ y={dy\over dx}x+B, \qquad F(x,y)=ax+by+c=0.\]
then $a=dy$, $b=-dx$, $c=B$
\item derive $dx,dy$ from the end points
\end{itemize}
}

\frame{
  \frametitle{Midpoint location}
\begin{itemize}
\item Does the midpoint~$M$ lie above or under the line?
\item  use~$F(\cdot,\cdot)$: evaluate the `decision value' of the midpoint:
\[ d=F(x_p+1,y_p+1/2). \]
\item The two cases to consider then are
\begin{description}
\item[$d<0$:] $M$ lies over the line, so we take $y_{p+1}=y_p$;
\item[$d\geq0$:] $M$ lies under the line, so we take $y_{p+1}=y_p+1$.
\end{description}
\end{itemize}
}

\frame{
  \frametitle{Use of $d$}
\begin{itemize}
\item Use $d$ instead of midpoint:
\[ d'=F(x_{p+1}+1,y_{p+1}+1/2). \]
\item Two cases:
\begin{footnotesize}
\[ \begin{array}{rll}
    d'=&a(x_{p+1}+1)+b(y_{p+1}+1/2)+c=\\
    d<0:&= a(x_p+2)+b(y_p+1/2)&=d+a=d+dy\\
    d\geq0:&= a(x_p+2)+b(y_p+3/2)+c&=d+a+b=d+dy-dx
\end{array} \]
\end{footnotesize}
\item Update~$d$ with $dy$ or~$dy-dx$ depending on
whether it's negative or non-negative.
\end{itemize}
}

\frame{
  \frametitle{Final refinement}
\begin{itemize}
\item Start off
\[ d_0=F(x_0+1,y_0+1/2)=F(x_0,y_0)+a+b/2=0+dy-dx/2.\]
\item Get rid of the division by~2:\\
$\tilde F(x,y)=2F(x,y)$;\\
 update~$d$ with $2dy$ and~$2(dy-dx)$ in the two cases.
\item Digital Differential Analyzers (DDA)
\end{itemize}
}

\subsectionframe{Circle drawing}
\frame{
  \frametitle{}
\pgfimage[height=1.5in]{circle-midpoint}
\begin{itemize}
\item Circle: \[ F(x,y) = x^2+y^2-R^2,\]
\item  decision value in the midpoint~$M$ is
\[ d=F(x_p+1,y_p+1/2)=x^2+2x+y^2+y+5/4. \]
\end{itemize}
}

\frame{
  \frametitle{}
\begin{itemize}
\item Cases
\begin{description}
\item[$d<0$:] $M$ lies in the circle, so we take $y_{p+1}=y_p$;
\item[$d\geq0$:] $M$ lies outside the circle, so we take $y_{p+1}=y_p+1$.
\end{description}
\item Updating:
\[ \begin{array}{rll}
    d'=&F(x_{p+1}+1,y_{p+1}+1/2)=\\
    d<0:&= x^2+4x+y^2+y+4\,1/4&=d+2x+3\\
    d\geq0:&= x^2+4x+y^2+3y+6\,1/4&=d+2(x+y)+5
\end{array} \]
\item Construct $2x$,$2y$ by shift
\end{itemize}
}

\subsectionframe{Cubics}

\frame{
  \frametitle{Stepwise computation}
\begin{itemize}
\item Cubic function $f(t)=at^3+bt^2+ct+d$
\item Strategy: compute the value $f(t+\delta)$ by updating
\[ f(t+\delta)=f(t)+\Delta f(t). \]
\item (alternatives: Horner, midpoint)
\item Difference:
\[ \begin{array}{rcl}
    \Delta f(t)&=&f(t+\delta)-f(t)\\
        &=&a(3t^2\delta+3t\delta^2+\delta^3)+b(2t\delta+\delta^2)+c\delta\\
        &=&3a\delta
    t^2+(3a\delta^2+2b\delta)t+a\delta^3+b\delta^2+c\delta
\end{array} \]
\item Quadratic term left
\end{itemize}
}

\frame{
\begin{itemize}
\item Define
\[ \begin{array}{rcl}
    \Delta^2f(t)&=&\Delta f(t+\delta)-\Delta f(t)\\
        &=&3a\delta(2t\delta+\delta^2)+(3a\delta^2+3b\delta)\delta\\
        &=&6a\delta^2t+6a\delta^3+2b\delta^2
\end{array} \]
\item Third difference:
$\Delta^3f(t)=\Delta^2f(t+\delta)-\Delta^2f(t)=6a\delta^2$
\item Together: compute $f_{n+1}\equiv f((n+1)\delta)$ by
\[ \Delta^3f_0=6a\delta^2,\quad
    \Delta^2f_0=6a\delta^3+2b\delta^2,\quad
    \Delta f_0=a\delta^3+b\delta^2+c\delta
\]
and computing by update
\[  f_{n+1}=f_n+\Delta f_n,\quad
    \Delta f_{n+1}=\Delta f_n+\Delta^2f_n,\quad
    \Delta^2f_{n+1}=\Delta^2f_n+\Delta^3f_0
\]
\end{itemize}
}

\sectionframe{Rasterizing type}

\frame{
  \frametitle{}
Type is tricky: lots of features in small objects\\
everyone immediately sees when it's wrong\\
\pgfimage[height=1.3in]{raster-problems}
\pgfimage[height=1.3in]{illegible}
}

\frame{
  \frametitle{Badly rasterized characters}
Obvious algorithm: pixel on if center in the contour\\
\pgfimage[height=1.3in]{e-bad}
\pgfimage[height=1.3in]{e-good}\\
\begin{itemize}
\item Problems with curves tangent to $n+1/2$ lines
\item Different scalings, different raster
\item Variable placement
\end{itemize}
}

\subsection{Basic algorithms}

\frame{
  \frametitle{Scaling and rasterizing}
\parbox[b]{1in}{Original character is on internal raster:}
\pgfimage[height=1in]{fig2-1-1}
\parbox[b]{1in}{Scale to target raster:}
\pgfimage[height=1in]{fig2-1-2}

\parbox[b]{1in}{Round to target raster:}
\pgfimage[height=1in]{fig2-1-3}
\parbox[b]{1in}{Set pixels:}
\pgfimage[height=1in]{fig2-1-4}
}

\frame{
  \frametitle{Scaling vs design size}
\pgfimage[height=.25in]{design-size}
\begin{itemize}
\item Scaling is a compromise
\item Different design sizes
\item Adobe Multiple Master
\end{itemize}
}

\frame{
  \frametitle{Filling in}
\begin{itemize}
\item Precisely what does `pixel lies within the contour' mean?
\item Complications: letters with `bowls'; multiple contours
\pgfimage[height=1in]{overlapping-contours}
\pgfimage[height=1in]{winding}
\item Winding rules
\end{itemize}
}

\frame{
  \frametitle{Winding rules}
\pgfimage[height=1.2in]{winding_ar4win_100}
\pgfimage[height=1.2in]{winding_ar4win_150}
}

\frame{
  \frametitle{Dropouts}
\pgfimage[height=2.5in]{dropout}
}

\subsectionframe{Hinting / instructing}

\frame{
  \frametitle{}
\pgfimage[height=1.3in]{o-horizontal}
\pgfimage[height=1.3in]{o-vertical}
\begin{itemize}
\item Small programs per font~/ character
\item Give constraints on placement, relations, distance
\end{itemize}
}

\end{document}
\sectionframe{Anti-aliasing}

\frame{
  \frametitle{}
\begin{itemize}
\item 
\end{itemize}
}

\frame{
  \frametitle{}
\begin{itemize}
\item 
\end{itemize}
}

\frame{
  \frametitle{}
\begin{itemize}
\item 
\end{itemize}
}

\frame{
  \frametitle{}
\begin{itemize}
\item 
\end{itemize}
}

\frame{
  \frametitle{}
\begin{itemize}
\item 
\end{itemize}
}

\end{document}

\frame{
  \frametitle{}
\begin{itemize}
\item 
\end{itemize}
}

