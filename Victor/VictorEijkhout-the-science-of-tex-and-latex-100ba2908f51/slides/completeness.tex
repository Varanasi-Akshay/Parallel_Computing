\documentclass{beamer}

\usepackage{beamerthemevictor,comment,verbatim,graphicx,amssymb}

\newcommand{\TbT}{\TeX\ by Topic}
\newcommand{\Metafont}{\textsc{Metafont}}
\let\metafont\Metafont
\newcommand\ascii{{\sc Ascii}}
\newcommand\ebcdic{{\sc Ebcdic}}
\newcommand\gnuplot{\protect\n{gnuplot}}
\def\mod{\mathbin{\mathrm{mod}}}
\def\eqref#1{equation~(\ref{#1})}
\def\Eqref#1{Equation~(\ref{#1})}
\def\inv{^{-1}}
\def\endproof{\unskip\penalty10000 \hskip10pt plus 1fill$\bullet$\par}

\def\etal{\textit{et al.}}
\def\lex{{\it lex}}
\def\yacc{{\it yacc}}
\def\make{{\it make}}
\def\web{{\textsc{Web}}}
\def\n{\bgroup
  \catcode`\#=12 \catcode`\_=12 \catcode`\^=12
  \catcode`\&=12 
  \tt \let\next=}
\newcommand{\cs}[1]{{\tt\char`\\#1}}
{\catcode`\.=13
 \gdef.{${}_\bullet$}
}
\def\parserule#1{%
    \ifmmode \n{#1}\,\rightarrow\relax
    \else $\n{#1}\,\rightarrow{}$\nobreak
    \fi
    \hbox\bgroup\catcode`\.=13 \tt\let\next=}

\newenvironment{remark}%
    {\medskip\noindent{\bf Remark.}}{}
\usepackage{bnf}
\input bnf.env

\generalcomment{inputwithcode}
  {\begingroup\def\ProcessCutFile{}}
  {\verbatiminput{\CommentCutFile}
   \endgroup
   \input{\CommentCutFile}
  }
\generalcomment{examplewithcode}
  {\begingroup
     \def\ProcessCutFile{}\def\CommentCutFile{example.tex}}
  {\verbatiminput{\CommentCutFile}
   Output:
   \begin{quote}
     \begin{minipage}[t]{3in}
        \everypar{}
        \input{\CommentCutFile} 
     \end{minipage}
   \end{quote}
   \endgroup
  }
\specialcomment{ttexamplewithcode}
  {\begingroup\def\ProcessCutFile{}}
  {\verbatiminput{\CommentCutFile}
   \endgroup
   Output:
   \begin{quote}\begin{ttfamily} \input{\CommentCutFile} 
     \end{ttfamily}\end{quote}
  }
\generalcomment{mathexamplewithcode}
  {\begingroup\def\ProcessCutFile{}}
  {\verbatiminput{\CommentCutFile}
   Output:
   \begin{equation} \input{\CommentCutFile} \end{equation}
   \endgroup
  }

\def\chaptertitle{\csname\chaptername title\endcsname}
\def\chaptershorttitle{\csname\chaptername shorttitle\endcsname}

\def\lambdatitle{Lambda calculus in \TeX}
\def\lambdashorttitle{Lambda calculus}

\def\latextitle{Yet another \LaTeX\ tutorial}
\def\latexshorttitle{\LaTeX}

\def\lextitle{A \lex\ tutorial}
\def\lexshorttitle{\lex}

\def\yacctitle{A \yacc\ tutorial}
\def\yaccshorttitle{\yacc}

\expandafter\def\csname tex1title\endcsname{\TeX\ -- macro programming}
\expandafter\let\csname tex1shorttitle\expandafter\endcsname
    \csname tex1title\endcsname
\expandafter\def\csname tex2title\endcsname{\TeX\ -- visuals}
\expandafter\let\csname tex2shorttitle\expandafter\endcsname
    \csname tex2title\endcsname

\def\parsingtitle{An introduction to parsing}
\def\parsingshorttitle{Parsing}

\def\hashingtitle{Hashing}
\let\hashingshorttitle\hashingtitle

\def\beziertitle{Curve approximation by splines}
\def\beziershorttitle{Splines}

\def\dynamictitle{Dynamic programming}
\let\dynamicshorttitle\dynamictitle

\def\paragraphtitle{Line breaking}
\let\paragraphshorttitle\paragraphtitle

\def\pagetitle{Page breaking in \TeX}
\def\pageshorttitle{Page breaking}

\def\completenesstitle{Introduction to NP-Completeness}
\def\completenesshortstitle{NP-Completeness}

\def\rastertitle{Introduction to Raster Graphics}
\def\rastershortstitle{Raster Graphics}

\def\pythontitle{Really quick Python tutorial}
\def\pythonshortstitle{Python}

\def\encodingtitle{Character encoding}
\let\encodingshorttitle\encodingtitle

\def\rastertitle{Raster Graphics}
\let\rastershorttitle\rastertitle

\def\gnuplottitle{Gnuplot}
\let\gnuplotshorttitle\gnuplottitle

\def\fontstitle{Fonts and font files}
\def\fontsshorttitle{Fonts}

\def\softwaretitle{\TeX\ and Software Engineering}
\def\softwareshorttitle{Software Engineering}

\renewcommand{\beamervictortitle}{CS-594 Eijkhout, Fall 2004}

\newcommand{\sectionframe}[1]%
  {\section{#1}
   \frame{\begin{center}\color{blue}\Large #1\end{center}}
  }
\newcommand{\subsectionframe}[1]%
  {\subsection{#1}
   \frame{\begin{center}\color{blue}\Large #1\end{center}}
  }

\input idxmacs

\begin{document}

\title{NP-completeness}
\author{Victor Eijkhout}
\date{Notes for CS 594 -- Fall 2004}

\frame{\titlepage}

\section{Introduction}

\frame[containsverbatim]{
  \frametitle{Complexity theory}
\begin{itemize}
\item What is the running time of an algorithm as function of its input?
\item What is the memory need of an algorithm as function of its
  input?
\item Best, worst, average, expected.
\item Notation: $f(n)=O(g(n))$, $f(n)=\Theta(g(n))$
\end{itemize}
}

\frame[containsverbatim]{
  \frametitle{Tractable and intractable}
\begin{itemize}
\item Tractable: polymomial complexity
\[ f(n)=O(n^k)\quad\hbox{some $k$} \]
\item Intractable: exponential complexity
\[ f(n)=O(2^n) \quad\hbox{or something like that} \]
\end{itemize}
}

\frame{
  \frametitle{Intractable problems}
\begin{itemize}
\item They exist
\item<2-> What do you tell your boss?
\pgfimage[width=3.5in]{np-prove}
\end{itemize}
}

\frame{
  \frametitle{Bounds versus reality}
\begin{itemize}
\item Ideal: algorithm and bound are similar
\item NP: all known algorithms are intractable, but not provably
\item<2-> Wrong way to approach your boss:
\pgfimage[width=3.5in]{np-dumb}
\end{itemize}
}

\frame{
  \frametitle{Some consolation}
\begin{itemize}
\item Problems in NP are equivalent: solve one, solve them all
\item You're not alone: no one knows an efficient algorithm
\item<2-> What to tell your boss:
\pgfimage[width=3.5in]{np-famous}
\end{itemize}
}

\sectionframe{Decision problems}
\subsection{Optimization}

\frame[containsverbatim]{
  \frametitle{Optimization vs decision}
\begin{itemize}
\item Many problems are optimization type:
\item traveling salesman: what is the shortest route
\item For complexity analysis more convenient decision problem
\item translation needed: given a bound $B$, is there a route shorter
  than~$B$
\end{itemize}
}

\subsection{Language of a problem}

\frame{
  \frametitle{Language of a problem}
\begin{itemize}
\item For a problem, there are `instances'
\item example: in traveling salesman problem, set of numbered cities
  $\{(i,c_i)\}_i$
\item<2-> `Yes set' of a problem: $Y_\Pi$ contains the instances that
  give a `yes' decision
\item<2-> Encoding~$e$ gives `language of a problem':
\[ L[\Pi,e]=\{ \hbox{the instances in $Y_\Pi$ encoded under $e$} \} \]
\item<3-> for traveling salesman: ordered list of cities $\langle
  c_1,c_2,\ldots\rangle$
\end{itemize}
}

\subsectionframe{Turing machines}

\frame{
  \frametitle{Accepting strings}
\begin{itemize}
\item A Turing machine can halt in `yes' state~$q_y$, `no'
  state~$q_n$, or not halt.
\item The TM \emph{accepts} a string if it halts in~$q_y$
\item The \emph{language} is the set of all accepted strings
\end{itemize}
}

\frame{
  \frametitle{`Solving' a problem}
 A Deterministic Turing Machine (DTM) $M$ \emph{solves} a problem~$\Pi$\\
 (equivalently: recognizes $L[\Pi,e]$) iff
\begin{itemize}
\item The DTM for all strings over the input alphabet
\item $L_M=L[\Pi,e]$
\item<2-> Example: traveling salesman problem for network of cities
  and bound~$B$,
\item<2-> $M$ always returns yes or no
\item<2-> input is list of cities
\item<2-> $M$ returns yes if list satisfies bound
\end{itemize}
}

\frame{
  \frametitle{Solution checking vs generating}
\begin{itemize}
\item In the traveling salesman problem, `solving' means checking a
  solution
\item<2-> Other example: primality testing
\item<2-> Input is number, output is yes/no (this number is prime~/
  not prime)
\item<2-> $\Rightarrow$ more intuitive notion of solving
\end{itemize}
}

\subsectionframe{Complexity classes}

\frame{
  \frametitle{Class $P$}
\begin{itemize}
\item Definition in terms of languages:
\[ P=\{L:\hbox{there is DTM that recognizes $L$ in polynomial
  time}\} \]
\item<2-> in terms of problems:
\begin{eqnarray*}
 \Pi\in P&\equiv& L[\Pi,e]\in P\quad\hbox{for some encoding $e$}\\
        &\equiv&\hbox{there is a polynomial time DTM
                 that recognizes $L[\Pi,e]$}
\end{eqnarray*}
\item<3-> Interpretation:\\
$\Pi\in P$ means is a polynomial time Turing machine~$M$\\
that recognizes~$Y_\Pi$\\
for strings not in~$Y_\Pi$: it halts in state~$q_N$
\end{itemize}
}

\frame{
  \frametitle{Class $NP$}
\begin{itemize}
\item \emph{Certificate}: guessed solution\\
(e.g.\ list of cities for traveling salesman)
\item Non-Deterministic Turing Machine:\\
first generate certificate by $\epsilon$-moves\\
then check correctness
\item<2-> Let NDTM $M$, then $L_M$ is the set of decision problems for
  which $M$ can generate a certificate
\item<2-> Certificate exists $\Leftrightarrow$ decision problem has
  `true' answer
\item<3-> $\Pi\in NP$ iff there is NDTM~$M$ that recognizes~$L[\Pi]$ in
  polynomial time
\item<4-> $\Pi$ is in~NP iff
there is a polynomial time function~$A(\cdot,\cdot)$ such that
\[ w\in Y_\Pi \Leftrightarrow \exists_C:A(w,C)=\mathrm{true}.\]
\end{itemize}
}

\frame{
  \frametitle{Examples}
\begin{itemize}
\item Trivially in $P$: given $a,b,n$, check whether $a\times b=n$
\item<2-> Non-trivially in $P$: given $n$, check whether $a,b$ exist
\item<3-> Not in $P$: actually find $a,b$\\
\[ \Theta(\exp((n\cdot 64/9)^{1/3})(\log n)^{2/3}) \]
\item<4-> Provably exponential time: best move in chess
\item<5-> Unclear: traveling salesman
\end{itemize}
}

\sectionframe{NP-completeness}
\subsection{Polynomial transformations}
\frame[containsverbatim]{
  \frametitle{Definition of Polynomial transformation}
\begin{itemize}
\item $L_1$~and~$L_2$ languages over alphabets $\sum_1^*$~and~$\sum_2^*$
\item $f:\sum_1^*\mapsto\sum_2^*$ is called polynomial transformation
  of problem~1 into problem~2 if
\begin{itemize}
\item There is DTM that computes~$f(x)$ in time~$T_f(x)\leq p_f(|x|)$ for
  polynomial~$p_f$, and
\item For all $x\in\sum_1^*$, $x\in L_1$ iff $f(x_1)\in L_2$.
\end{itemize}
\item `many-to-one polynomial transformation'
\end{itemize}
}

\frame{
  \frametitle{Relations}
\begin{itemize}
\item Let $f$ polynomial transformation from $L_1$ to~$L_2$,
\item then \[ L_2\in P\Rightarrow L_1\in P \]
\item<2-> Proof: Let $M_2:L_2\rightarrow\{0,1\}$ DTM that
  recognizes~$L_2$,\\ then $M_2\circ f$~is DTM that recognizes~$L_1$,
\item<3-> composite recognizer runs in time 
\[ T_{M_2\circ f}(x)\leq p_{T_2}(|p_f(|x|)|) \]
which is polynomial
\end{itemize}
}

\frame{
  \frametitle{Relations'}
\begin{itemize}
\item Notation: $L_1$ transforms to $L_2$ (in polynomial time):
\[ L_1\leq L_2 \]
\item (suggested reading: $L_1$ is easier than~$L_2$.)
\item<2-> Transitivity:
\[ L_1\leq L_2 \wedge L_2\leq L_3 \Rightarrow L_1\leq L_3, \]
\end{itemize}
}

\subsectionframe{NP-complete}

\frame{
  \frametitle{Definition}
\begin{itemize}
\item $L$ is NP-complete if
\begin{itemize}
\item $L\in NP$, and
\item for all $L'\in NP$: $L'\leq L$
\end{itemize}
\item<2-> First condition: NDTM has to halt.
\item<2-> Without that: NP-hard
\item<2-> Example: halting problem
\end{itemize}
}

\frame{
  \frametitle{Relations''}
\begin{itemize}
\item If $L_1,L_2\in NP$, $L_1$ is NP-complete, and $L_1\leq
  L_2$,\\ then $L_2$ is NP-complete.
\item<2-> Proof: need to check $\forall_{L_2\in NP}:L'\leq L_2$
\item<2-> $L_1$ is NP-complete, so $L'\leq L_1$.\\ Now use transitivity
\end{itemize}
}

\subsectionframe{Proof of NP-completeness}

\frame{
  \frametitle{Bootstrap problem}
\begin{itemize}
\item Have one NP-complete problem, construct translation, have
  another
\item Where to begin? How to check that every other problem is easier?
\item<2-> Strategy: for NP-complete problem, NDTM exists
\item<2-> encode that NDTM in some specific problem
\item<3-> First done for \textsc{Sat}, boolean satisfiability\\ Cook, 1971
\end{itemize}
}

\frame{
  \frametitle{Satisfiability}
\begin{itemize}
\item Problem statement: given boolean variables $x_1\ldots x_n$ and a
  logical formula $F(x_1,\ldots,x_n)$\\ is there is an assignment
  $x_i\mapsto \{0,1\}$ such that $F(x_1,\ldots)=1$.
\item<2-> Examples: $x_1\vee\neq x_1$ is always true;\\ $x_1\wedge\neq
  x_1$ is always false\\ $x_1\wedge \neq x_2$ is only true for
  $(x_1=T,x_2=F)$.
\item<2-> second example not satisfiable
\item<3-> \textsc{Sat} is in NP
\end{itemize}
}

\frame[containsverbatim]{
  \frametitle{Transformations into \textsc{Sat}}
\begin{itemize}
\item Let $\Pi$ be NP-complete; $M$~an NDTM that solves it
\item construct logical formula such that
\[ \hbox{formula true}\Leftrightarrow\hbox{successful TM run} \]
\end{itemize}
}

\frame[containsverbatim]{
  \frametitle{The Turing machine}
\[ M = \langle Q, s, \Sigma, F, \delta\rangle
\]
where
\begin{description}
\item[$Q$] is the set of states, and $s\in Q$ the initial
       state,
\item[$\Sigma$] the alphabet of
     tape symbols,
\item[$F\subset Q$] the set of accepting states, and
\item[$\delta\subset Q\times\Sigma\times Q\times\Sigma\times\{-1,+1\}$] the
  set of transitions,
\end{description}
Assume $M$ accepts or rejects instances in time
$p(n)$ where $n$~is the size of the instance and $p(\cdot)$ is a
polynomial.
}

\frame[containsverbatim]{
  \frametitle{Variables}
For $q\in Q$, $-p(n)\leq i\leq p(n)$, $j\in\Sigma$, $0\leq k\leq p(n)$:
\begin{tabular}{|lp{2in}l|}
\hline
Variables&
Intended interpretation&
How many\\
\hline
$T_{ijk}$&
True iff tape cell $i$ contains symbol~$j$ at step~$k$ of the computation&
$O(p(n)^2)$\\
$H_{ik}$&
True iff the $M$'s read/write head is at tape cell~$i$ at step~$k$ of the computation.&
$O(p(n)^2)$\\
$Q_{qk}$&
True iff~$M$ is in state~$q$ at step~$k$ of the computation.&
$O(p(n))$\\
\hline
\end{tabular}
}

\frame[containsverbatim]{
  \frametitle{Expression}
Conjunction of whole bunch of clauses\\
($-p(n) \leq i\leq p(n)$, $j\in\Sigma$, and~$0\leq k\leq p(n)$)

\emph{Initial conditions}
\begin{tabular}{|p{.8in}p{1.2in}p{1.3in}p{.7in}|}
\hline
For all:&
Add the clauses&
Interpretation&
How many clauses?\\
\hline
Tape cell~$i$ of the input~$I$ contains symbol~$j$.&
$T_{ij0}$&
Initial contents of the tape.&
$O(p(n))$
\\
&
$Q_{s0}$&
Initial  state of~$M$&
$O(1)$
\\
&
$H_{00}$&
Initial position of read/write head.&
$O(1)$
\\
\hline
\end{tabular}
}

\frame[containsverbatim]{
  \frametitle{Constraint clauses}
\begin{tabular}{|p{.8in}p{1.2in}p{1.3in}p{.7in}|}
\hline
For all:&
Add the clauses&
Interpretation&
How many clauses?\\
\hline
symbols\newline $j \not=j'$&
$T_{ijk} \rightarrow \neg T_{ij'k}$&
One symbol per tape cell.&
$O(p(n)^2)$
\\
states\newline $q \not=q'$&
$Q_{qk} \rightarrow  \neg  Q_{q'k}$&
Only one state at a time.&
$O(p(n))$
\\
cells\newline $i \not=i'$&
$H_{ik} \rightarrow  \neg  H_{i'k}$&
Only one head position at a time.&
$O(p(n))$
\\
\hline
\end{tabular}
}

\frame[containsverbatim]{
  \frametitle{Turing machine clauses}
\begin{tabular}{|p{.8in}p{1.2in}p{1.3in}p{.7in}|}
\hline
For all:&
Add the clauses&
Interpretation&
How many clauses?\\
\hline
$i,j,k$&
$T_{ijk} =T_{ij(k+1)} \vee H_{ik}$&
Tape remains unchanged unless written.&
$O(p(n)^2)$
\\
$f\in F$&
The disjunction of the clauses $Q_{f,p(n)}$&
Must finish in an accepting state.&
$O(1)$\\
\\
\hline
\end{tabular}
}

\frame[containsverbatim]{
  \frametitle{Transition table clauses}
\begin{tabular}{|p{.8in}p{1.2in}p{1.3in}p{.7in}|}
\hline
For all:&
Add the clauses&
Interpretation&
How many clauses?\\
\hline
$(q, \sigma ,q', \sigma', d)\allowbreak \in\delta$&
The disjunction of the clauses\newline
$(H_{ik} \wedge  Q_{qk} \wedge T_{i\sigma k})
\rightarrow  (H_{(i+d)(k+1)} \wedge Q_{q'(k+1)} \wedge T_{i\sigma' (k+1)})$&
Possible transitions at computation step~$k$ when head is at position~$i$.&
$O(p(n)^2)$
\\
\hline
\end{tabular}
}

\frame[containsverbatim]{
  \frametitle{Equivalence}
If there is an accepting computation for~$M$ on input~$I$,\\
then~$B$ is satisfiable, by assigning $T_{ijk}$, $H_{ik}$ and~$Q_{ik}$

\medskip
If~$B$ is satisfiable,\\
then there is an accepting computation for~$M$ on input~$I$:\\
follows the steps indicated by the assignments to the variables.
}

\frame[containsverbatim]{
  \frametitle{Polynomial time}
\begin{itemize}
\item $O(p(n)^2)$ Boolean variables, each encodable in space $O(\log
  p(n))$
\item  Number of clauses $O(p(n)^2)$
\item Total size of $B$ is $O((\log p(n))p(n)^2)$
\end{itemize}
}

\end{document}
\frame[containsverbatim]{
  \frametitle{}
\begin{itemize}
\item
\end{itemize}
}

\frame[containsverbatim]{
  \frametitle{}
\begin{itemize}
\item
\end{itemize}
}

\frame[containsverbatim]{
  \frametitle{}
\begin{itemize}
\item
\end{itemize}
}

\frame[containsverbatim]{
  \frametitle{}
\begin{itemize}
\item
\end{itemize}
}

\end{document}
