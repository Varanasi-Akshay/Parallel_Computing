\begin{exercises}
\project Do a literature study of code/documentation development. Here are some
places to start:
\begin{description}
\item[POD] Plain Old Documentation; used for
  Perl. \url{http://www.perl.com/pub/a/tchrist/litprog.html}
\item[JavaDoc] \url{http://java.sun.com/j2se/javadoc/}
\item[Doxygen] \url{http://www.stack.nl/~dimitri/doxygen/}
\item[Fitnesse] \url{http://fitnesse.org/}
\item[Leo] \url{http://webpages.charter.net/edreamleo/front.html}
\end{description}
What schools of
thought are there about developing medium size codes such as \TeX? How
does Knuth's philosophy relate to the others?
\project
Design a general model for tables and write software that formats
tables. You can output \TeX\ macros, or argue why they aren't powerful
enough, and design a better language for describing tables.
\project Compare the TeX "way" to MS Word, PageMaker, FrameMaker,
Lout, Griff, previewLaTeX
\project \TeX\ has been criticized for its arcane programming
language. Would a more traditional programming language work for the
purpose of producing text output? Compare \TeX\ to other systems, in
particular lout, \url{http://www.pytex.org/}, ant
\url{http://www-mgi.informatik.rwth-aachen.de/~blume/Download.html}
and write an essay on the possible approaches. Design a system of your own.
\project \TeX\ and HTML were designed primarily with output in
mind. Later systems (XML, DocBook) were designed so that output would
be possible, but to formalize the structure of a document better.
However, XML is impossible to write by hand. What would be a way out?
Give your thoughts about a better markup system, conversion between
one tool and another, et cetera.
\project Several improvements on \TeX\ and \LaTeX\ have been developed
or are under development. Investigate NTS, LaTeX3, Context, Lollipop
\TeX\, describe their methodologies, and evaluate relative merits.
\project Knuth has pretty liberal ideas about publishing software;
somewhat against the spirit of the times. Report on software patents,
the difference between patents and copyright, the state of affairs in
the world. Read \url{http://swpat.ffii.org/gasnu/knuth/index.en.html}
\end{exercises}
