\documentclass{artikel3}
\usepackage{times}

\usepackage{comment}
\input tutmacs

\begin{document}
\pagestyle{empty}
\title{Quiz over the parsing chapters}
\author{}\date{}\maketitle
\hbox{}\vskip -1.2in\hbox{}
\section*{Your name:}

\vskip.5cm

Is the grammar
\begin{bnf}
S:a; (T).
T:T,S;S
\end{bnf}
$LL(1)$ parsable? If yes, why; if no, what can you do about this?
\vskip 0pt plus 1fill

A FSA recognizes right-recursive languages, that is, languages
with grammar rules of the form (uppercase: nonterminal, lowercase: terminal)
\begin{bnf}
A: bC.
D: e
\end{bnf}
Equivalently, it recognizes left-recursive languages, generated by
grammar rules of the form
\begin{bnf}
A: Bc.
D: e
\end{bnf}
What happens if you mix right-recursive and left-recursive rules in
one grammar?
\vskip 0pt plus 1fill

Consider the hash function
\begin{verbatim}
  h = <some value>
  for (i=0; i<len(var); i++)
    h = Rand( h XOR <byte i of string> );
\end{verbatim}
and assume that the random function is a permutation of the
values~0--255. Show that two words that differ by one letter map to
different hash keys. How about two words where the second is the first
with an additional letter (for example \n{key} and \n{key0})?

\end{document}
