The \n{gnuplot} utility can be used for plotting sets of
points. However, here we will only discuss drawing curves.

\Level 0 {Introduction}

The two modes for running \n{gnuplot} are \emph{interactive} and
\emph{from file}. In interactive mode, you call \n{gnuplot} from the
command line, type commands, and watch output appear (see next
paragraph); in the second case you call \n{gnuplot <your file>}.

The output of \n{gnuplot} can be a picture on your screen, or drawing
instructions in a file. Where the output goes depends on the setting
of the \emph{terminal}. By default, \n{gnuplot} will try to draw a
picture. This is equivalent to declaring
\begin{verbatim}
set terminal x11
\end{verbatim}
or \n{aqua}, \n{windows}, or any choice of graphics hardware.

For output to file, declare
\begin{verbatim}
set terminal pdf
\end{verbatim}
or \n{fig}, \n{latex}, \n{pbm}, et cetera.

\Level 0 {Plotting}

The basic plot command is \n{plot}. By specifying
\begin{verbatim}
plot x**2
\end{verbatim}
you get a plot of $f(x)=x^2$; \n{gnuplot} will decide on the range
for~$x$.
With
\begin{verbatim}
set xrange [0:1]
plot 1-x title "down", x**2 title "up"
\end{verbatim}
you get two graphs in one plot, with the $x$~range limited to~$[0,1]$,
and the appropriate legends for the graphs. The variable~\n{x} is the
default for plotting functions.

Plotting one function against another --~or equivalently, plotting a
parametric curve~-- goes like this:
\begin{verbatim}
set parametric
plot [t=0:1.57] cos(t),sin(t)
\end{verbatim}
which gives a quarter circle.

To get more than one graph in a plot, use the command \n{set multiplot}.

\Level 1 {Styles}

You can change the default drawing style with
\begin{verbatim}
set style function dots
\end{verbatim}
(\n{lines}, \n{dots}, \n{points}, et cetera), or
change on a single plot with
\begin{verbatim}
plot f(x) with points
\end{verbatim}
