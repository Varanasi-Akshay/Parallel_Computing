\begin{594exercise}
Use the \cs{everypar} command so that the first paragraph after a
heading (define your own heading command) will have a bullet
(\verb+$\bullet$+) in the left margin.
\end{594exercise}
\begin{answer}
\begin{examplewithcode}
\def\Header#1{\medskip
  \hbox{\bfseries #1}
  \setcounter{vcount}{1}
  \everypar{\hbox to 0pt{\hskip-10pt $\bullet$\hss}%
            \everypar{}}
  }
\Header{the title}

Paragraph one

and two
\end{examplewithcode}
\end{answer}


%\tracingmacros=2 \tracingonline=1
\begin{594exercise}
Set \TeX\ up so that every paragraph starts in mediaeval `initial'
style: the first letter of the paragraph is set in a large type size,
and takes the first two or three lines.
Use the following auxiliary macro:
\verbatiminput{hangmacro.tex}
\input hangmacro
\begin{examplewithcode}
% small test:
A \Hang{$\bullet$} B \Hang{\Huge B} C. \bigskip
\end{examplewithcode}
Also, set \verb+\parindent=0pt+. The result should look like this. Input:
\begin{verbatim}
This is an old-fashioned mediaeval paragraph that has lots 
of text and...

Also, the second paragraph is an old-fashioned mediaeval
paragraph that...
\end{verbatim}
\smallskip with output:\par\medskip
\begin{minipage}{3in}
\def\biginitial#1{\hskip-20pt \Hang{\huge #1}\hskip 20pt}
\parindent=0pt
\everypar{\hangafter=-2 \hangindent=20pt \biginitial}
This is an old-fashioned mediaeval paragraph that has lots of text and
a very long first sentence. The second sentence is also long, and only
serves the purpose to make this more than 2 or so lines long. For good
measure we throw in a third line which should make this four lines
long, if not five with a little luck.

Also, the second paragraph is an old-fashioned mediaeval paragraph that
has lots of text and a very long first sentence. The second sentence
is also long, and only serves the purpose to make this more than 2 or
so lines long. For good measure we throw in a third line which should
make this four lines long, if not five with a little luck.
\end{minipage}
\end{594exercise}

\begin{answer}
Recall that macro:
\verbatiminput{hangmacro.tex}
\begin{examplewithcode}
\input hangmacro
\begin{minipage}[t]{3in}
\def\biginitial#1{\hskip-20pt \Hang{\huge #1}\hskip 20pt}
\parindent=0pt

\everypar{\hangafter=-2 \hangindent=20pt \biginitial}
This is an old-fashioned mediaeval paragraph that has lots of text and
a very long first sentence. The second sentence is also long, and only
serves the purpose to make this more than 2 or so lines long. For good
measure we throw in a third line which should make this four lines
long, if not five with a little luck.

Also, the second paragraph is an old-fashioned mediaeval paragraph that
has lots of text and a very long first sentence. The second sentence
is also long, and only serves the purpose to make this more than 2 or
so lines long. For good measure we throw in a third line which should
make this four lines long, if not five with a little luck.
\end{minipage}
\end{examplewithcode}
\end{answer}

\endinput

\begin{594exercise}
\end{594exercise}

\begin{594exercise}
\end{594exercise}

\begin{594exercise}
\end{594exercise}

\begin{594exercise}
\end{594exercise}

