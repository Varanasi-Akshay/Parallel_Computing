\def\DefineWithDelims#1%
   {\edef\tmp{\def\expandafter\noexpand\csname#1\endcsname
                  \LeftDelim ####1\RightDelim}
    \tmp}
Make this work:
\begin{examplewithcode}
\def\LeftDelim{(}\def\RightDelim{)}
\DefineWithDelims{foo}{My argument is `#1'.}
\def\LeftDelim{<}\def\RightDelim{>}
\DefineWithDelims{bar}{But my argument is `#1'.}
\foo(one)\par
\bar<two>
\end{examplewithcode}
In other words, \cs{DefineWithDelims} defines a macro --~in
this case \cs{foo}~-- and this macro has one argument, delimited by
custom delimiters. The delimiters can be specified for each macro separately.

Hint: \cs{DefineWithDelims} is actually a macro with only one argument.
Consider this code snippet:
\begin{verbatim}
\Define{foo}{ ... #1 ...}
\def\Define#1{
  \expandafter\def\csname #1\endcsname##1}
\end{verbatim}
