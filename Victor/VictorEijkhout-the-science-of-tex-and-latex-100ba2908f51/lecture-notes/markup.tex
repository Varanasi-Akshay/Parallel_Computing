\Level 0 {History}

The idea of markup has been invented several times. The ideas can be
traced by to William
Tunnicliffe, chairman of the Graphic Communications Association (GCA)
Composition Committee, who presented a talk on the separation of
information content of documents from their format in 1967.
He called this `generic coding'.
Simultaneously, a New York book designer named Stanley
Rice was publishing articles about "Standardized Editorial
Structures", parameterized style macros based on the structural
elements of publications.

\Level 1 {Development of markup systems}

Probably the first software incorporating these ideas comes out of
IBM. \index{Goldfarb, Charles}Charles Goldfarb recounts
(\url{http://www.sgmlsource.com/history/roots.htm}) how in 1969 he
invented \index{GML}GML with Ed Mosher and Ray Lorie. They were
tackling the problem of having a document storage system, an editor,
and a document printing system talk to each other, and found that each
was using different `\index{Markup!procedural}procedural markup' for
its own purposes. Gradually the idea grew to use markup for a logical
description of the document. The project was called `Integrated Text
Processing', and the first prototype `Integrated Textual Information
Management Experiment': InTIME.

GML was officially published in 1973, and by 1980 an extension,
\index{SGML}SGML, was under development. This was published in 1986 as
ISO 8879. Actually, SGML is a standard for defining markup languages,
rather than a language itself. Markup languages are defined in SGML
through `\index{DTD}Document Type Definitions' (DTDs). 

The most famous application of SGML is \index{HTML}HTML. However, HTML
quickly began violating the separation of function and presentation
that was the core idea of markup languages. A~renewed attempt was made
to introduce a system for markup languages that truly defined content,
not form, and this led to the `\index{XML}eXtensible Markup
Language'. \index{XHTML}XHTML is a realization of HTML as an XML
`\index{schema}schema'. While SGML made some attempts at readability
(and saving keystrokes for poor overworked typists), XML is aimed
primarily at being generated and understood by software, not by
humans.

Another well-known application of SGML is
`\index{DocBook}DocBook'. However, this has also been defined as an
XML DTD\footnote{XML has both DTDs, which are SGML-like, and Schemas,
  which themselves XML. Schemas are the more powerful mechanism, but
  also newer, so there are established DTDs that may not be redefined
  as Schemas. Furthermore, Schemas are more complicated to transmit
  and parse.}, and this seems to be the current definition. DocBook is
a good illustration of the separation of content and presentation:
there are XSL style sheets that render DocBook files as Pdf, Rtf,
HTML, or man pages.

\Level 1 {Typesetting with markup}

In the early 1970s, nroff/troff was written at Bell Labs, at first in
PDP assembler and targetting a specific photo typesetter for producing
Unix documentation. Later it was recoded in~C, with device independent
output. Various tasks such as tables and equations were hard in
nroff/troff, so preprocessors existed: eqn for formulas, tbl for
tables, and refer for bibliographies.

Brian Reid's thesis of 1980 descibed a markup system called
Scribe. Scribe source files can be compiled to several target
languages. For instance, recent versions Scribe can compile to \LaTeX,
HTML, or man pages.

\endinput
\Level 0 {Document typesetting systems}

Scribe, troff, Lout, \TeX.

