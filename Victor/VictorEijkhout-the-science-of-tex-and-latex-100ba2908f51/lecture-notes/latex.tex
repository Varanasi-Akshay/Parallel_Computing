\Level 0 {Document markup}

If you are used to `wysiwyg' (what you see is what you get) text
processors, \LaTeX\ may seem like a strange beast, primitive, and
probably out-dated. While it is true that there is a long history
behind \TeX\ and \LaTeX, and the ideas are indeed based on much more
primitive technology than what we have these days, these ideas have
regained surprising validity in recent times.

\Level 1 {A little bit of history}

Document markup dates back to the earliest days of computer
typesetting. In those days, terminals were strictly character-based:
they could only render mono-spaced built-in fonts. Graphics terminals
were very expensive. (Some terminals could switch to a graphical
character set, to get at least a semblance of graphics.) As a result,
compositors had to key in text on a terminal -- or using punched cards
in even earlier days -- and only saw the result
when it would come out of the printer.

Any control of the layout, therefore, also had to be through character
sequences. To set text in bold face, you may have had to surround it
with \n{<B> .. the text .. </B>}. Doesn't that look like something you
still encounter every day?

Such `control sequences' had a second use: they could serve a template
function, expanding to often used bits of text. For instance, you
could imagine \verb+$ADAM$+ expanding to `From our correspondent in
Amsterdam:'.

\LaTeX\ works exactly the same. There are command control sequences;
for instance, you get bold type by specifying~{\verb+\bf+}, et cetera.
There are also control sequences that expand to bits of text:
you have to type \verb+\LaTeX+ to get the
characters `LATEX' plus the control codes for all that shifting up and
down and changes in font size.
\begin{verbatim}
\TeX => T\kern -.1667em\lower .5ex\hbox {E}\kern -.125emX
\LaTeX => L\kern -.36em {\sbox \z@ T\vbox to\ht \z@ {\hbox
  {\check@mathfonts \fontsize \sf@size \z@ \math@fontsfalse
     \selectfont A} \vss }}\kern -.15em\TeX 
\end{verbatim}

\Level 1 {Macro packages}

The old typesetting systems were limited in their control sequences:
they had a fixed repertoire of commands related to typesetting, and
there usually was some mechanism to defining `\index{macro}macros'
with replacement text. Formally, a macro is a piece of the input that
gets replaced by its definition text, which can be a combination of
literal text and more macros or typesetting commands.

\begin{quote}
An important feature of many composition programs is the ability to
designate by suitable input instructions the use of specified
formats. Previously stored sequences of commands or text replace the
instructions, and the expanded input is then processed. In more
sophisticated systems, formats may summon other formats, including
themselves ["System/360 Text Processor Pagination/360, Application
  Description Manual," Form No. GE20-0328, IBM Corp., White Plains,
  New York.].
\end{quote}

That was the situation with commercial systems by manufacturers of
typesetting equipment such as Linotype. Systems developed by (and
for!) computer scientists, such Scribe or nroff/troff, were much more
customizable. In fact, they sometimes would have the equivalent of a
complete programming language on board. This makes it possible to take
the basic language, and design a new language of commands on top of
it. Such a repertoire of commands is called a macro package.

In our case, \TeX\ is the basic package with the strange macro
programming language, and \LaTeX\ is the macro package\footnote{In
  this tutorial I will say `\TeX' when a statement applies to the
  basic system, and `\LaTeX' if it \emph{only} applies to the macro
  package.}. \LaTeX\ was designed for typesetting scientific articles
and books: it offers a number of styles, each with slightly different
commands (for instance, there are no chapters in the article style)
and slightly different layout (books need a title page, articles
merely a title on the first page of the text). Styles can also easily
be customized.  For different purposes (art books with fancy designs)
it is often better to write new macros in \TeX, rather than to bend
the existing \LaTeX\ styles.

However, if you use an existing \LaTeX\ style, the whole of the
underlying \TeX\ programming language is still available, so many
extensions to \LaTeX\ have been written. The best place to find them is
through CTAN~\url{http://wwww.ctan.org/}.

%\tracingmacros=2
\begin{594exercise}
 Discuss the difference between a macro and a function or
  procedure in a normal programming language. In a procedural
  language, looping is implemented with \n{goto} instructions. How
  would you do that in a macro language? Is there a difference in
  efficiency?
\end{594exercise}
\begin{answer}
A macro is a piece of replacement text. Therefore it does not induce a
group, like a function does. It does not even have to be a complete
instruction.  Looping in a macro language is done with tail
recursion. Since tail recursion does not need to maintain a stack,
there is no difference in efficiency.
\end{answer}

\Level 1 {Logical markup}

Macro packages are initially motivated as a labour-saving device: a
macro abbreviates a commonly used sequence of commands. However, they
have another important use: a well designed macro package lets you use
commands that indicate the structure of a document rather than the
formatting. This means that you would write
\verb+\section{Introduction}+ and not worry about the layout. The
layout would be determined by a statement elsewhere as to what macros
to load\footnote{Compare this to the use of CSS versus HTML in web
  pages.}. In fact, you could take the same input and format it two
different ways. This is convenient in cases such as an article being
reprinted as part of a collection, or a book being written before the
final design is commissioned.

In a well written document, there will be few explicit typesetting
commands. Almost all macros should be of the type that indicates the
structure, and any typesetting is taken care of in the definition of
these. Further control of the layout of the document should be done
through global parameter settings in the preamble.

\pagebreak[2]
\Level 0 {The absolute basics of \LaTeX}

Here is the absolute minimum you need to know to use \LaTeX.

\Level 1 {Different kinds of characters}
\label{sec:character}

A \TeX\ input file mostly contains characters that you want to
typeset. That is, \TeX\ passes them on from input to output without
any action other than placement on the page and font choice. Now, in
your text there can be commands of different sorts. So \TeX\ has to
know how to recognize commands. It does that by making a number of
characters special. In this section you will learn which characters
have special meaning.

\begin{itemize}
\item Anything that starts with a backslash is a command or `control
  sequence'. A~control sequence consists of the backslash and the
  following sequence of letters -- no digits, no underscores allowed
  either -- or one single non-letter character.
\item Spaces at the beginning and end of a line are
  ignored. Multiple spaces count as one space.
\item Spaces are also ignored after control sequences, so writing
  \verb+\LaTeX is fun+ comes out as `\LaTeX is fun'. To force a space
  there, write \verb+\LaTeX{} is fun+ or \verb+\LaTeX\ is fun+. Spaces
  are {\em not} ignored after control symbols such as \verb+\$+, but
  they are again after the \index{control space}`control
  space'~\verb*+\ +\footnote{The funny bucket character here is how we
  visualize the space character.}.
\item A single newline or return in the input file has no meaning,
  other than giving a space in the input. You can use newlines to improve
  legibility of the input. Two newlines (leading to one empty line) or
  more cause a paragraph to end. You can also attain this paragraph
  end by the \cstoidx{par} command.
\item Braces \{,\} are mostly used for delimiting the arguments of a
  control sequence. The other use is for grouping. Above you saw an
  example of the use of an empty group; similarly
  \verb+\TeX{}ing is fun+ comes out as `\TeX{}ing is fun'.
\item Letters, digits, and most punctuation can be typed
  normally. However, a~bunch of characters mean something special to
  \LaTeX: {\tt\char`\%}\verb-$&^_#~{}-.
  Here are their functions:
\begin{description}
\item[\tt\char`\%] comment: anything to the end of line is ignored.
\item[\tt\char`\$,\char`\_,\char`\^] inline math (toggle), subscript,
  superscript. See section~\ref{sec:math}.
\item[\tt\char`\&] column separator in tables.
\item[\tt\char`\~] nonbreaking space. (This is called an
  `\index{active character}active character')
\item[\tt\char`\{\char`\}] Macro arguments and grouping.
\end{description}
  In order to type these characters, you need to precede
  them with a backslash, for instance~\verb+\%+ to
  get~`\%'.
  This is called a
  \index{control symbol}`control symbol'. Exception:
  use~\verb+$\backslash$+ to get~`$\backslash$'.
\item Some letters do not exist in all styles. As the most commonly
  encountered example, angle brackets~\n{<>} do not exist in the roman
  text font (note that they are in the typewriter style here, in roman
  you would get~`<>'), so you
  need to write, somewhat laboriously \verb+$\langle$S$\rangle$+ to
  get~`$\langle$S$\rangle$'\footnote{That's what macros are good for.}.
\end{itemize}

\begin{594exercise}
You read in a document `This happens only in 90rest of the time
  it works fine.' What happened here? There are articles in print
  where the word `From' has an upside down question mark in front of
  it. Try to think of an explanation.
\end{594exercise}
\begin{answer}
In the first case, clearly there was a percent sign after the 90 that
was not escaped. The remaining text on the line was then ignored by
\LaTeX.
The word `From' has special meaning in email. If this word appears at
the beginning of a line, mail software escapes it with~\verb+>+.
\end{answer}

\pagebreak[2]
\Level 1 {\LaTeX\ document structure}

Every \LaTeX\ document has the following structure\index{preamble}:
\begin{verbatim}
\documentclass[ <class options> ]{ <class name> }
    <preamble>
\begin{document}
    <text>
\end{document}
\end{verbatim}
Typical document classes are \n{article}, \n{report}, \n{book},
and~\n{letter}. As you may expect, that last one has rather different
commands from the others. The class options are optional; examples
would be \n{a4paper}, \n{twoside}, or~\n{11pt}.

The preamble is where additional packages get loaded, for instance 
\begin{verbatim}
\usepackage{times}
\end{verbatim}
switches the whole document to the Times Roman typeface. This is also
the place to define new commands yourself (section~\ref{sec:newcommand}).

\Level 2 {Title} 

To list title and author, a document would start with
\begin{verbatim}
\title{My story}
\author{B.C. Dull}
\date{2004} %leave this line out to get today's date
\maketitle
\end{verbatim}
After the title of an article and such, there is often an
abstract. This can be specified with
\begin{verbatim}
\begin{abstract}
... The abstract text ...
\end{abstract}
\end{verbatim}
The stretch of input from \cs{begin} to \cs{end} is called an
`environment'; see section~\ref{sec:environment}.

\Level 2 {Sectioning}
\label{sec:section}

The document text is usually segmented by calls
\begin{verbatim}
\section{This}
\subsection{That}
\subsection{The other}
\paragraph{one}
\subparagraph{two}
\chapter{Ho}
\part{Hum}
\end{verbatim}
which all get numbered automatically. Chapters only exist in the
\n{report} and \n{book} styles. Paragraphs and subparagraphs are not
numbered. To prevent sections et cetera from being numbered, use
\cstoidx{section*}\verb+{...}+ and such\footnote{This also makes the title not
  go into the table of contents. See section~\ref{sec:toc} on how to
  remedy that.}.

\Level 2 {Frontmatter, backmatter}

You can use commands \cstoidx{frontmatter}, \cstoidx{mainmatter},
\cstoidx{backmatter} -- in \n{book} class only -- to switch page
numbering from roman to arabic, and, for the back matter, section
numbering from numbers to letters.

\Level 1 {Running \LaTeX}

With the last two sections you should know enough to write a
\LaTeX\ document. Now how do you get to see the final output?
This takes basically two steps: formatting and viewing.

You need to know that \TeX's original output format is slightly
unusual. It is called `\index{DVI file format}DVI' for DeVice
Independent. There are viewers for this format, but usually you need
another step to print it.

Traditionally, you would run an executable called \n{latex}
(or~\n{tex}), which gives you a \n{dvi} file, which you then view with
a previewer such as \n{xtex} or~\n{xdvi}. To print this file, you
would use
\begin{verbatim}
dvips -Pcmz foo.dvi -o foo.ps
\end{verbatim}
to generate a \n{ps} file. This can be
printed, or converted to~\n{pdf}.

There are version of the \n{latex} executable that output to other
formats, for instance \n{pdflatex} (there is also a~\n{pdftex}) goes
straight to \n{pdf}, which you can view with the Adobe Acrobat Reader,
or \n{xpdf}. The big advantage of this approach is that you can get
hyperlinks in your pdf file; see section~\ref{sec:hyperref}.

\begin{594exercise}
Set up a document that will have the answers to your homework
exercises of the whole course.
\end{594exercise}

\Level 0 {The \TeX\ conceptual model of typesetting}

In \TeX, the question `on what page does this character
appear' is hard to answer. That is because \TeX\ typesets all material
for a page, sort of on a long scroll, before cutting a page off that
scroll. That means that when a piece of text is set, you do not know
if it falls before or after a certain page break.

A similar story holds for paragraph breaking, but the question on what
line something occurs is not usually interesting.

\Level 0 {Text elements}

Here are the main elements that make up a \LaTeX\ document.

%\begin{multicols}{2}
\Level 1 {Large scale text structure}

We already discussed sectioning commands in
section~\ref{sec:section}. Here are more major text elements in a
\LaTeX\ document.

\Level 2 {Input files}

Use \cstoidx{include}\verb+{<file>}+ to input a file beginning on a
new page, and \cstoidx{input} for just plain input. With
\indexcs{includeonly}
\begin{verbatim}
\includeonly{file1,file2}
\end{verbatim}
you can save processing time -- provided the files are \cs{include}d
to begin with.

The \n{.tex} extension can usually be left off; because of the way
\TeX\ works, be careful with funny characters in the file name.

On Unix installations, input files can be in other directories,
specified by the \n{TEXINPUTS} environment variable.

\Level 2 {Environments}
\label{sec:environment}

If a certain section of text needs to be treated in a different way
from the surrounding text, it can be segmented off by
\begin{verbatim}
\begin{<environment name>}
... text ...
\end{<environment name>}
\end{verbatim}
An environment defines a group, so you can have local definitions and 
changes of parameters.

Some predefined environments are
\begin{description}
\item[flushleft (flushright)]
  \index{environment!flushleft}\index{environment!flushright}for text
  that is left (right) aligned, but not right (left).
\item[center] \index{environment!center}for text that is centered.
\item[quote, quotation]
  \index{environment!quote}\index{environment!quotation}for text that
  needs to be set apart, by indenting both margins. The \n{quote}
  environment is for single paragraphs, the \n{quotation} for multiple.
\item[abstract] \index{environment!abstract}for the abstract that
  starts an article, report, or book. In the report and book style it
  is set on a separate page. In the article style it is set in a
  smaller type size, and with indented margins.
\item[verbatim] \index{environment!verbatim}see section~\ref{sec:verbatim}.
\end{description}

\Level 2 {Verbatim text}
\label{sec:verbatim}

As we have observed already, \TeX\ has a number of special characters,
which can be printed by prefixing them with a backslash, but that is a
hassle. Good thing that there is a mechanism for printing input
completely verbatim. For inline text, use
\indexcs{verb}\verb.\verb+&###\text+. to get `\verb+&###\text+'. The
essential command here is \cstoidx{verb}. Unlike with other commands
that have arguments, the argument is not delimited by braces, but by
two occurrences of a character that does not appear in the verbatim
text. A~plus sign is a popular choice. The \cstoidx{verb*} variant
makes spaces visible: \verb.\verb*+{ }+. gives~`\verb*+{ }+'.

For longer verbatim text there is a
\index{environment!verbatim}\n{verbatim} environment. The
\index{environment!verbatim*}\n{verbatim*} version prints each space
as a~\verb*+ + symbol.
To input whole files verbatim, use
\indexcs{verbatiminput}\verb+\verbatiminput{file}+, which is defined
in the \n{verbatim} package.

For \TeX{}nical reasons, verbatim text can not appear in some
locations such as footnotes or command definitions.

\begin{594exercise}
Why does the \cs{verb} command not have its argument in braces?
\end{594exercise}
\begin{answer}
The \cs{verb} command should be able to have braces, in particular a
closing brace, in its argument.
\end{answer}

\Level 2 {Lists}

Lists in \LaTeX\ are a special case of an environment; they are specified by
\begin{verbatim}
\begin{<list type>}
\item ...
\item ...
\end{<list type>}
\end{verbatim}
The three main list types are unnumbered lists,
\index{list!itemize}\n{itemize}, numbered lists,
\index{list!enumerate}\n{enumerate}, and definition or description lists,
\index{list!description}\n{description}.

In the case of a description list, it is mandatory to give the item
label:
\begin{verbatim}
\begin{description}
\item[Do] A deer, a female deer.
\item[Re] A drop of golden sun.
...
\end{description}
\end{verbatim}
You can give item labels in the other list types too.

Putting a list inside a list item will change the style of the item
labels and numbers in a way that is dictated by the document class.

You can put a \index{label!after \cs{item}}\cs{label} command after an
item to be able to refer to the item number.
\begin{examplewithcode}
\begin{enumerate}
\item\label{first:item} One
\item Two comes after \ref{first:item}
\end{enumerate}
\end{examplewithcode}
This only makes sense with
enumerate environments.

\Level 2 {Tabbing}

The \n{tabbing} environment is useful for setting pieces of text, such
as source code, that use a small number of `tab stops'. Tab stops
(a~term deriving from old mechanical typewriters) are locations on the
line that one can `tab~to', no matter how much material is currently
on the line.

Example:
\begin{examplewithcode}
\begin{tabbing}
The first line sets this: \=point;\\
the second jumps\>there
\end{tabbing}
\end{examplewithcode}
The \cs{=} command in the first line defines a tab stop; in every
subsequent line a~\cs{>} command will jump to that position, if it has
not been reached yet. There can be multiple tab stops, not necessarily
defined in the same line, and tab stops can be redefined.

A more interesting case is where the tab stop is used before the line
that defines it. For this case there is the \cs{kill} command, which
prevents a line from being displayed. Example:
\begin{examplewithcode}
\begin{tabbing}
while \=\kill
do\>\{\\
\>$i_1\leftarrow{}$\=1\\
\>$\ldots$\>2\\
\>\}\\
while (1)
\end{tabbing}
\end{examplewithcode}

\Level 2 {Tabular material}
\label{sec:tabular}

The \index{environment!tabular}\n{tabular} environment generates a
table. Tables are often placed independently of the text, at the top
or bottom of the page; see section~\ref{sec:float} for details.
The table itself is generated by
\begin{verbatim}
\begin{tabular}{<alignment>}
... material ...
\end{tabular}
\end{verbatim}
Each line of the table has items separated by \index{&@\tt\char`\&}
characters, and \verb+\\+ at the end of each line but the last.

In its simplest form, the alignment directions are a combination of
the letters \n{l,r,c}:
\begin{examplewithcode}
\begin{tabular}{rl}
"Philly" Joe & Jones\\ Dizzie & Gillespie\\ Art&Tatum
\end{tabular}
\end{examplewithcode}

Vertical rules are inserted by placing a \verb+|+~character in the
alignment specification; horizontal lines you get
from~\cstoidx{hline}.

\begin{examplewithcode}
\begin{tabular}{|r|rl|}
\hline
instrument&name&\\ \hline
drums: &"Philly" Joe & Jones\\
trumpet:& Dizzie & Gillespie\\ 
piano: &Art&Tatum\\ \hline
\end{tabular}
\end{examplewithcode}

Some more tricks:
\begin{itemize}
\item In headings you often want to span one item over several
  columns. Use
\begin{examplewithcode}
\begin{tabular}{|r|rl|}
\hline
instrument&\multicolumn{2}{|c|}{name}\\ \hline
drums: &"Philly" Joe & Jones\\
trumpet:& Dizzie & Gillespie\\ 
piano: &Art&Tatum\\ \hline
\end{tabular}
\end{examplewithcode}
\item \LaTeX\ inserts a standard amount of space between columns. You
  can override this with~\verb+@{<stuff>}+: 
\begin{verbatim}
\begin{tabular}{r@{.}l}
2&75\\ 800&1
\end{tabular}
\end{verbatim}
\medskip gives\ %
\begin{tabular}{r@{.}l}
2&75\\ 800&1
\end{tabular}
\item A column specification of~\verb+p{<size>}+ (where \verb+<size>+
  is something like \n{5.08cm} or~\n{2in}) makes the items in that
  column formatted as paragraphs with the width as specified.
\end{itemize}

\Level 2 {Footnotes}

Use the command \cstoidx{footnote}. The numbering style is determined
by the document class. The kinds of footnote to denote affiliation of the
author of a paper and such (these often use asterisk symbols and such,
even if footnotes are numbered in the rest of the document) are given
by the command~\cstoidx{thanks}.

There are two common cases where want more detailed control over
footnotes:
\begin{itemize}
\item You want to dictate the label yourself, for instance using the
  same label again (this often happens with author affiliations)
\item You want to place a footnote in a table or such; \LaTeX\ has
  trouble with that.
\end{itemize}
In such cases you can use \cstoidx{footnotemark} to place
the marker, and \cstoidx{footnotetext} for the text. You can also
set or change the \n{footnote} counter explicitly with counter functions
(section~\ref{sec:counter}), or use
\begin{verbatim}
\footnote[<label>]{<text>}
\end{verbatim}
where the label is used instead, and the counter is not increased.

\Level 2 {Text boxes}
\label{sec:textbox}

Sometimes you want a small amount of text to behave like one or more
paragraphs, except not as part of the main page. The main commands for
that are 
\begin{verbatim}
\parbox[pos]{width}{text}
\begin{minipage}[pos]{width}  text  \end{minipage}
\end{verbatim}
The optional \n{pos} parameter specifies whether the top (\n{t}) or
bottom (\n{b}) line of the box should align with surrounding text:
top-aligned box of text:
\begin{examplewithcode}
Chapter 1. \parbox[t]{2in}{\slshape Introduction. First easy
  lessons. Exercises. More about things to come. Conclusions}
\end{examplewithcode}
The default is a vertically centered position.

The \n{minipage} environment is meant for longer pieces of text; it
can also handle other environments in the text.

The \cstoidx{mbox} command is for text (or other objects) that need to
stay on one line.

\Level 1 {Minor text issues}

\Level 2 {Text styles}

You can switch between \index{roman}\index{text!roman}roman (the style
for this text), \index{italic}\index{text!italic}{\it italic} (also
called \index{cursive}`cursive'),
\index{slanted}\index{text!slanted}{\sl slanted} (in some typefaces,
italic and slanted may be identical), and
\index{bold}\index{text!bold}{\bf bold} with the commands
\cstoidx{texrm},
\cstoidx{textit}, \cstoidx{textsl}, and \cstoidx{textbf} respectively,
used as 
\begin{verbatim}
Text is stated \textbf{boldly} or \textsl{with a slant}.
\end{verbatim}
These combinations are not independent: nesting the commands can give
you \textbf{bold \textsl{slanted}} text.

The above commands are mostly for short bits of text. See
section~\ref{sec:nfss} for commands to change font parameters in a
more global manner.

If you are using italics for emphasis, consider using \cstoidx{emph}
instead, which works better, especially if you emphasize something
in text that is already italic.

\Level 2 {Fonts and typefaces}
\label{sec:nfss}

You already saw commands such as \cs{textrm} and \cs{textit} for
switching from one type style to another. These commands hide a more
complicated reality: \LaTeX\ handles its fonts as combination of
three parameters. These individual switches can be used inside a group,
or as an environment:
\begin{verbatim}
{\ttfamily This is typewriter text}
\begin{mdseries}
 This text is set in medium weight.
\end{mdseries}
\end{verbatim}
Here are the categories and possible values.
\begin{description}
\item[family] roman, sans serif, typewriter type: \cstoidx{rmfamily},
  \cstoidx{sffamily}, \cstoidx{ttfamily}.
\item[series] medium and bold: \cstoidx{mdseries}, \cstoidx{bfseries}.
\item[shape] upright, italic, slanted, and small caps:
  \cstoidx{upshape}, \cstoidx{itshape}, \cstoidx{slshape},
  \cstoidx{scshape}.
\end{description}

\Level 2 {Comments}

Anything from {\tt\char`\%} to the end of line is ignored. For
multiline comments, load either
\begin{verbatim}
\usepackage{comment}
\end{verbatim}
or
\begin{verbatim}
\usepackage{verbatim}
\end{verbatim}
and in both cases surround text with
\begin{verbatim}
\begin{comment}
to be ignored
\end{comment}
\end{verbatim}
where the closing line {\em has to be} on a line of its own.

\Level 2 {Hyphenation}
\label{sec:hyphen}

Sometimes \TeX\ has a hard time finding a breakpoint in a word. When
you are fixing the final layout of a document, you can help it with
\verb+helico\-pter+. If \TeX\ consistently breaks your name wrong, do
\begin{verbatim}
\hyphenation{Eijk-hout}
\end{verbatim}
in the preamble.

This is not the mechanism for telling \TeX\ about a new language; see
section~\ref{sec:babel}.

To keep text together, write \cstoidx{mbox}\verb+{do not break}+. You could
also write this with a non-breaking space
as \index{\char`\~}\verb+do~not~break+. (See also
section~\ref{sec:tilde}.) It is a good idea to write
\begin{verbatim}
A~note on...
increase by~$1$.
\end{verbatim}
to prevent single characters at the beginning of a line (first example), or
the end of a line (second example). The second example could
even give a line with a single character on it if it were to occur at
the end of a paragraph.

\Level 2 {Tilde}
\label{sec:tilde}

The tilde character has special meaning as a nonbreaking space; see
section~\ref{sec:hyphen}. To get a tilde accent, use \cstoidx{\char`\~}. To
get a literal tilde, do \verb+\~{}+, \verb+$\sim$+, or
\verb+\char`\~+. If you need a tilde in URLs, consider using the
\n{url} or \n{hyperref} package; see section~\ref{sec:hyperref}.

\Level 2 {Accents}

In basic \TeX, accents are formed by putting a control symbol of that accent in
front of the letter:
\begin{verbatim}
Sch\"on b\^et\'e
\end{verbatim}
for `Sch\"on b\^et\'e'. If you have an occasional foreign word in
English text this works fine. However, if your terminal allows you to
input accented characters, you can use them in \LaTeX\ with the
\n{inputenc} package.

Standard \TeX\ (or \LaTeX) does not understand Unicode encodings such
as UTF-8.

\pagebreak[3]
\Level 2 {Line/page breaking}

In general, you should leave line and page breaking to \TeX, at most
adjusting parameters. However, should you really need it,\linebreak
you can use the commands
\cstoidx{linebreak}\discretionary{}{}{}\n{[<num>]} and
\cstoidx{pagebreak}\discretionary{}{}{}\n{[<num>]}, where the number
is \n{1,2,3,4}, with 4~the highest\newline urgency. There is also
\cstoidx{nolinebreak} and \cs{nopagebreak} with a similar urgency
parameter.

In this last paragraph there was a \cs{linebreak} after
`need it'. You notice that \TeX\ still tried to fill out to the right
margin, with ugly consequences. After `highest' there was
a~\cstoidx{newline}, which really breaks then and there.
Similarly, there is \cstoidx{newpage}.

There is also \cstoidx{nolinebreak} and \cstoidx{nopagebreak}, both
with optional numerical parameter, to discourage breaking.

\Level 2 {Manual spacing}
\label{sec:hvspace}

Most of the time you should leave spacing decisions to \LaTeX, but
for custom designs it is good to know the commands.
\begin{verbatim}
\hspace{1cm} \hspace*{1in} \hspace{\fill}
\vspace{1cm} \vspace*{1in} \vspace{\fill}
\end{verbatim}
\begin{itemize}
\item The \n{*}-variants give space that does not disappear at the
  beginning or end of a line (for horizontal) or page (vertical).
\item A space of size \cs{fill} is infinite: this means it will
  stretch to take up however much space is needed to the end of the
  line or page.
\end{itemize}

\Level 2 {Drawing lines}

Let us get one thing out of the way: \index{underlining}underlining is
a typewriter way of emphasizing text. It looks bad in typeset text,
and using italics or slanted text is a much better
way. Use~\cstoidx{emph}.

Lines can be used a typographical decorations, for instance drawn
between the regular text and the footnotes on a page, or as part of
chapter headings. The command is\indexcs{rule}
\begin{verbatim}
\rule[lift]{width}{height}
\end{verbatim}
Example
\begin{examplewithcode}
1\ \rule{2cm}{\fboxrule}\ The title
\end{examplewithcode}
You can draw a whole box around text: \verb+\fbox{text}+ gives
\fbox{text}. The thickness of the line is \cstoidx{fboxrule}.

\Level 2 {Horizontal and vertical mode}
\label{sec:hvmode}

\TeX\ is in horizontal or vertical mode\index{horizontal
  mode}\index{vertical mode} while it is
processing\footnote{The story is actually more complicated; for the
  whole truth see the notes about \TeX.}. In horizontal mode, elements
-- typically letters --
are aligned next to each other; in vertical mode elements are stacked
on top of one another. Most of the time you do
not have to worry about that. When \TeX\ sees text, it switches to
horizontal mode, and \LaTeX\ environments will briefly switch to
vertical mode so that they start on a new line.

In certain cases you want to force vertical mode; for that you can
use~\cstoidx{par}. You can force things into a line with
\cstoidx{mbox} (section~\ref{sec:textbox}). In rare cases, \cs{leavevmode}.

%\end{multicols}

\Level 0 {Tables and figures}
\label{sec:float}

Tables and figures are objects that typically do not appear in the
middle of the text. At the very least they induce a paragraph break,
and often they are placed at the top or bottom of a page. Also, some
publishers' styles demand that a document have a list of tables and a
list of figures. \LaTeX\ deals with this by having environments
\begin{verbatim}
\begin{<table or figure>}[placement]
... table or figure material ...
\caption{Caption text}\label{tabfig:label}
\end{<table or figure>}
\end{verbatim}
In this,
\begin{itemize}
\item The `placement' specifier is a combination of the letters
  \n{htbp} for `here', `top', `bottom', and `page', telling
  \LaTeX\ where to place the material, if possible. Suppose a
  placement of \n{[ht]} is given, then the material is placed `right
  here', unless there is not enough space on the page, in which case
  it will be placed on top of the page (this one or the next).
\item Table material is given by a \n{tabular} environment; see
  section~\ref{sec:tabular}.
\item Figure material needs some extra mechanism, typically provided
  by another package; see section~\ref{sec:graphics}.
\item The caption goes into the list of tables/figures.
\item The label will refer to the number, which is automatically
  generated.
\end{itemize}
The list of tables/figures is generated by the command
\cstoidx{listoftables} or \cstoidx{listoffigures}.

\Level 0 {Math}
\label{sec:math}

\TeX\ was designed by a computer scientist to typeset some books with
lots of mathematics. As a result, \TeX, and with it \LaTeX's, math
capabilities are far better than those of other typesetters.
 
\Level 1 {Math mode}

You can not just write formulas in the middle of the text. You have to
surround them with \verb+$<formula>$+ or \verb+\(<stuff>\)+ for
inline formulas, or
\begin{verbatim}
\begin{displaymath} ... \end{displaymath}
\[ ... \]
\end{verbatim}
for unnumbered and
\begin{verbatim}
\begin{equation} ... \end{equation}
\end{verbatim}
for numbered displayed equations respectively. You can refer to an
equation number by including a~\verb+\label+ statement.

In math mode, all sorts of rules for text typesetting are changed. For
instance, all letters are considered variables, and set italic:
\verb+$a$+~gives~`$a$'. Roman text is either for names of functions,
for which there are control sequences --
\verb+\sin(x)+ gives `$\sin(x)$' -- or for connecting text, which has
to be marked as such:
\begin{mathexamplewithcode}
\forall x \in \mathbf{R} 
\quad \mathrm{(sufficiently large)} \quad: \qquad x>5
\end{mathexamplewithcode}

A formula is limited to one line; if you want to break it, or if you
need several formulas vertically after one another, you have to do it
yourself. The \n{eqnarray} environment is
useful here. It acts as a three-column alignment.
\begin{examplewithcode}
\begin{eqnarray}
\sin x&=&x-\frac{x^3}{3!}+\frac{x^5}{5!}- \nonumber \\ 
      &&{}-\frac{x^7}{7!}+\cdots
\end{eqnarray}
\end{examplewithcode}
Note the use of \cstoidx{nonumber} here; with the \n{eqnarray*} all
lines would be unnumbered by default.

In AMS \LaTeX\ there is an \n{align} environment which looks better
than \n{eqnarray}.

%\begin{multicols}{2}
\Level 1 {Super and subscripts}

In math mode, the character \verb+^+ denotes a superscript, and
\verb+_+~denotes a subscript: \verb+x_i^2+~is~$x_i^2$. (Outside of
math mode these characters give an error.) Sub and superscripts of
more than one character have to be grouped.

\Level 1 {Grouping}

Grouping, in math mode as outside, is done with braces:
\verb+x_{i-1}^{n^2}+~looks like~$x_{i-1}^{n^2}$.

\Level 1 {Display math vs inline}

Math looks different when used inline in a paragraph from that used as
display math. This is mostly clear for operators with `limits':
\[ \textrm{text mode:}\,
   \hbox{$\sum_{i=1}^\infty$} \quad
   \mathrm{display mode:}\, \sum_{i=1}^\infty
\]

\Level 1 {Delimiters, matrices}

Delimiters are \verb+()[]\{\}+. You can prefix them with
\cstoidx{big}, \cstoidx{Big} and such, but \TeX\ can resize them
automatically:
\begin{mathexamplewithcode}
\left( \frac{1}{1-x^2} \right)
\left\{ \begin{array}{ccc}
    \mathrm{(a)}&\Rightarrow&x>0\\
    \mathrm{(b)}&\Rightarrow&x=0\\
    \mathrm{(c)}&\Rightarrow&x<0
        \end{array} \right.
\end{mathexamplewithcode}
Note that with \verb+\right.+ you get a omitted right delimiter.

In the above example you also saw the \n{array} environment, which can
be used for anything tabular in math mode, in particular matrices.
Here is a good example of a matrix. Note the different kinds of dots:
\begin{mathexamplewithcode}
 A = \left( \begin{array}{cccccc}
       a_{11}&0&&\ldots&0&a_{1n}\\
       &a_{22}&0&\ldots&0&a_{2n}\\
       &&\ddots&\ddots&\vdots&\vdots\\
       &&&a_{n-2n-2}&0&a_{n-2n}\\ 
       &\emptyset&&&a_{n-1n-1}&a_{n-1n}\\
       &&&&&a_{nn}
     \end{array} \right)
\end{mathexamplewithcode}

\Level 1 {There is more}

See a good book for the whole repertoire of symbols. If what
\LaTeX\ has is not enough, you can also get AMS \LaTeX, which has even
more fonts and tricky constructs.

%\end{multicols}

\Level 0 {References}

\Level 1 {Referring to document parts}

One of the hard things in traditional typesetting is to keep
references such as `see also section~3' in sync with the text.
This is very easy in \LaTeX. You write
\begin{verbatim}
\section{Results}\label{section:results}
\end{verbatim}
after which you can use this as
\begin{verbatim}
see also section~\ref{section:results}
on page~\pageref{section:results}.
\end{verbatim}
The \cstoidx{label} command can appear after headings, or in general
every time some counter has been increased, whether that's a section
heading or a formula number.

\LaTeX\ implements this trick by writing the information to an
auxiliary file -- it has extension \n{.aux} -- and reading it in next
run. This means that a \LaTeX\ document usually has to be typeset
twice for all references to be both defined and correct. You get a
reminder after the first run if a second one is needed, or if there
are missing or duplicately defined labels.

\begin{594exercise}
A document with references usually takes two passes to get
  right. Explain why a table of contents can increase this number to
  three.
\end{594exercise}
\begin{answer}
Consider the case where the toc is at the start of the document.
The first pass generates the toc and the labels. In the second pass
labels are used, but because the toc is inserted, the page references
are off again and another pass is needed.
\end{answer}

\Level 1 {Table of contents}
\label{sec:toc}

Something that typically goes into the front or back matter is the
table of contents. This gets inserted automatically by the command
\cstoidx{tableofcontents}. No other actions required. You can add your
own material to the contents with \cstoidx{addcontentsline} or
\cstoidx{addtocontents}.

\Level 1 {Bibliography references}

Another kind of the reference is that to external bibliographies. This
needs a bit more work. 
\begin{itemize}
\item You write \verb+\cite{Knuth:1978}+ where you want the citation.
\item At the end of your document you write
\begin{verbatim}
\bibliographystyle{plain}
\bibliography{cs}
\end{verbatim}
to get the bibliography included.
\item The bibliography references have to be defined in a file
  \n{cs.bib}.
\item After running \LaTeX\ once, you need to invoke \n{bibtex
  <yourfile>}, which creates another auxiliary file, this time with
  \n{.bbl} extension, and run \LaTeX\ once or twice more.
\end{itemize}
The bibliography files have a syntax of their own, but you can figure
that out from looking at some examples.

\Level 1 {Index}

Finally, a document can have an index. For this you need to have a
statement \verb+\usepackage{makeidx}+ in the preamble,
\cstoidx{printindex} wherever you want the index, and commands
\cstoidx{index}\verb+{<some term>}+ throughout your document. Additionally, as
with \n{bibtex}, you need to run the program \n{makeindex} to generate
the external \n{.ind} file.

Further indexing commands: \verb+\index{name!sub}+ for subentry;
\verb+\index{foo@\textit{foo})+ for sorting under `foo' but formatted
differently.

\Level 0 {Some \TeX{}nical issues}

\Level 1 {Commands inside other commands}

For deep technical reasons you can get completely incomprehensible
error messages by writing things like
\begin{verbatim}
\section{My first section \footnote{and not my last}}
\end{verbatim}
Remedy that by writing
\begin{verbatim}
\section{My first section \protect\footnote{and not my last}}
\end{verbatim}

\Level 1 {Grouping}

Most modern programming languages have a block structure of some sort,
where variables can be declared inside a block, and are no longer
known after the block ends. \TeX\ has a stronger mechanism, where
assignments to a  variable made inside a block are reverted at
the end of that block.

In \LaTeX\ you notice that only \cs{newcommand} and
\cs{newenvironment} declarations are local; \cs{newcounter}s are
global, as are \cs{setcounter} assignments. However, \cs{savebox}
assignments are local.

\Level 0 {Customizing \LaTeX}

\LaTeX\ offers a number of tools (on top of the possibility of doing
straight \TeX\ programming) for customizing your document. The easiest
customization is to change already defined parameters. However, you
can also define new commands and environments of your own.

In fact, several of the customization we will see in this section are
not part of standard \LaTeX, but have been written by other users. If
they do not come with your installation, you can download them from
the Central \TeX\ Archive Network; see section~\ref{sec:ctan}.

\Level 1 {Page layout}

\Level 2 {Layout parameters}
\label{sec:page-layout}

Page layout is controlled by parameters such as \cs{textheight},
\cs{textwidth}, \cs{topmargin} (distance to the running head, not to
the first text line), and \cs{odd/evensidemargin} (distance to the
`spine' of the document). These are set with commands like
\begin{verbatim}
\setlength{\textwidth}{10in}
\addtolength{\oddsidemargin}{-1cm}
\end{verbatim}
Some lengths are `rubber length'
\begin{verbatim}
\setlength{\parskip}{10pt plus 3pt minus 2pp}
\end{verbatim}

\Level 2 {Page styles}
\label{sec:page}

Use the commands\indexcs{pagestyle}
\begin{verbatim}
\pagestyle{<style>}
\end{verbatim}
and\indexcs{thispagestyle}
\begin{verbatim}
\thispagestyle{<style>}
\end{verbatim}
to change the style of all pages or
one page. Available styles are \n{empty} (no page numbers), \n{plain}
(the default), and \n{headings} (page numbers and running
headers). See also section~\ref{sec:fancy-head} for many more options.

For two-sided printing, use the \n{twoside} option for the
document class.

\begin{594exercise}
Take a look at the headers and footers in Oetiker's `Not so short
introduction' and `\TeX\ by Topic' (the \LaTeX\ and \TeX\ part
of the handout). Can you find a reason to prefer one over the other
from a point of usability?
In both books, what is the rationale behind the header on the odd
pages? See in particular page~35 of the former and~77 of the
latter. Do you agree with this design?
\end{594exercise}
\begin{answer}
In \TeX\ by Topic no information is against the spine of the
book. That makes it easier to leaf through it searching for both
section titles and page numbers.
The right running head seems to the number and title of the first
section that starts on that page, if any. Having the title of the
\emph{last} section would be better.
\end{answer}

\Level 2 {Running page headers}
\label{sec:running-head}

The \n{headings} page style (section~\ref{sec:page}) uses running
heads that can change through the document. For instance it would have
chapter titles in the left page head and section titles in the right
head. You can achieve this effect yourself by using the \n{myheadings}
page style, and using the \indexcs{markright}\indexcs{markboth}
\begin{verbatim}
\markright{<right head>}
\markboth{<left>}{<right>}
\end{verbatim}
You have access to these texts as \cstoidx{rightmark} and
\cstoidx{leftmark}; this is needed in the \n{fancyhdr} style.

\Level 2 {Multicolumn text}

Load 
\begin{verbatim}
\usepackage{multicol}
\end{verbatim}
and write
\begin{verbatim}
\begin{multicol}{3}
text in three column mode
\end{multicol}
\end{verbatim}

\Level 1 {New commands}
\label{sec:newcommand}

You can define your own commands in \LaTeX. As example of a a~simple
command, consider an often used piece of text
\begin{verbatim}
\newcommand{\IncrByOne}{increased by~$1$}
\end{verbatim}
The replacement text can have parameters:
\begin{verbatim}
\newcommand{\IncrDecrBy}[2]{#1creased by~$#2$}
\end{verbatim}
In this definition, the number of arguments is listed after the
command name:~\n{[2]}, and occurrences of the arguments in the
replacement text are indicated by \verb+#1+, \verb+#2+~etc. Example
calls: \verb+\IncrDecrBy{in}{5}+, \verb+\IncrDecrBy{de}{2}+.

The declaration of a new command can specify an optional argument.
If we define
\begin{verbatim}
\newcommand{\IncrDecrBy}[2][in]{#1creased by~$#2$}
\end{verbatim}
the \n{[in]} specification means that the first argument is optional
(only the first argument can ever be optional) with a default value
of~\n{in}. Example calls:
\begin{examplewithcode}
\newcommand{\IncrDecrBy}[2][in]{#1creased by~$#2$}
\IncrDecrBy[de]{1}, \IncrDecrBy{5}.
\end{examplewithcode}

To redefine an existing command, use \cs{renewcommand}.

\Level 1 {New environments}

It is possible to define environments, by specifying the commands to
be used at their start and end\indexcs{newenvironment}:
\begin{verbatim}
\newenvironment{example}%
  {\begin{quote}\textbf{Example.}}%
  {\end{quote}}
\end{verbatim}
which, used as \verb+\begin{example}...\end{example}+ gives a
\n{quote} environment that starts with the word `Example' in bold
face. While defining that environment does not save a lot of typing,
it is a good idea nevertheless from a point of view of logical
markup. Using the example environment throughout ensures a uniform
layout, and makes design changes easy if you ever change your mind.

Special case:\indexcs{newtheorem} defining mathematical statements with
\begin{verbatim}
\newtheorem{majorfact}{Theorem}
\newtheorem{minorfact}[majorfact]{Lemma}
\begin{minorfact}Small fact\end{minorfact}
\begin{majorfact}Big fact\end{majorfact}
\end{verbatim}
giving
\begin{quote}
\newtheorem{majorfact}{Theorem}
\newtheorem{minorfact}[majorfact]{Lemma}
\begin{minorfact}Small fact\end{minorfact}
\begin{majorfact}Big fact\end{majorfact}
\end{quote}
The optional argument in the definition of \n{lemma} makes it use the
\n{theorem} counter.

\begin{594exercise}
  Why does this not work:
\begin{verbatim}
\newenvironment{examplecode}%
 {\textbf{Example code.}\begin{verbatim}}{\end{verbatimm}}
\end{verbatim}
\end{594exercise}
\begin{answer}
The environment given does not work, because after the verbatim begin,
\LaTeX\ looks for a literal end verbatim line.
\end{answer}

\begin{594exercise}
Write macros for homework typesetting; make one master document
  that will contain all your homework throughout this course.
\begin{enumerate}
\item Define an environment \n{exercise} so that
\begin{verbatim}
\begin{exercise}
My answer is...
\end{exercise}
\end{verbatim}
gives
\begin{quote}
\textbf{Problem 5.} My answer is\dots
\end{quote}
The counter should be incremented automatically. List your solution in
your answer, and find a way that the listing is guaranteed to be the
code you actually use.
\item Write a macro \cs{Homework} that will go to a new page, and
  output
\begin{quote}
\textbf{Answers to the exercises for chapter 3}
\end{quote}
at the top of the page. The \n{exercise} environment should now take
the question as argument:
\begin{verbatim}
\begin{exercise}{Here you paraphrase the question that was asked}
My answer is...
\end{exercise}
\end{verbatim}
and this outputs
\begin{quote}
\begin{exerciseB}{Here you paraphrase the question that was asked}
My answer is\dots
\end{exerciseB}
\end{quote}
(Hint: read the section on text boxes. Also be sure to use \cs{par} to
get \LaTeX\ to go to a new line.)
Allow for the question to be more than one line long. Unfortunately
you can not get verbatim text in the question. Find a way around that.
\end{enumerate}
\end{594exercise}
\begin{answer}
First part
\begin{verbatim}
\newcounter{answer}
\newenvironment{answer}
 {\refstepcounter{answer}
  \par \textbf{Problem \arabic{answer}.}\ }
 {\par}
\end{verbatim}
Put any code snippets in include files, which you include
with \cs{input} and \cs{verbatiminput}.
\end{answer}

\Level 1 {Counters}
\label{sec:counter}

\LaTeX\ has a number of counters defined, for sections and such. You
can define your own counters too. Here are the main commands:
\begin{description}
\item[create] A new counter is created with
\begin{verbatim}
\newcounter{<name>}[<other counter>]
\end{verbatim}
where the name does {\em not} have a backslash. The optional \n{other
  counter} indicates a counter (such as \n{chapter}) that resets the
new counter every time it is increased. (To do this reset for an
already existing counter, check out the \n{chngcntr} package.)
\item[change values] A counter can be explicitly set or changed as
\begin{verbatim}
\setcounter{<name>}{<value>}
\addtocounter{<name>}{<value>}
\end{verbatim}
The command \cs{refstepcounter} also make the new value the target for
a~\cs{label} command.
\item[use] To get a counter value numerically, use \cs{value}. To
  print the value, use
\begin{verbatim}
\arabic{<name>}, \roman{<name>}, \Roman{<name>}
\end{verbatim}
et cetera.
\end{description}

\Level 1 {Lengths}

Parameters such as \cs{textwidth} (section~\ref{sec:page-layout}) are
called `lengths'. You can define your own with
\begin{verbatim}
\newlength{\mylength}
\setlength{\mylength}{5in}
\end{verbatim}
These lengths can be used in horizontal or vertical space commands
(section~\ref{sec:hvspace}) for your own designs.

\Level 1 {The syntax of \cs{new...} commands}

Have you noticed by now that the name you define starts with a
backslash in \cs{newcommand} and \cs{newlength}, but not in
\cs{newenvironment} or \cs{newcounter}? Confusing.

\Level 0 {Extensions to \LaTeX}

\LaTeX\ may be a package on top of \TeX, but that doesn't mean that the
programming power of \TeX\ is no longer available. Thus, many people
have written small or large extensions to be loaded in addition to
\LaTeX. We will discuss a couple of popular ones here, but first we'll
see how you can find them.

\Level 1 {How to find a package, how to use it}
\label{sec:ctan}

Packages are typically loaded in the file preamble with
\begin{verbatim}
\usepackage{pack1,pack2,...}
\end{verbatim}
(These course notes load about a dozen packages.)

Many popular packages are already part of the standard
\LaTeX\ distribution, but you will have to search to find where they
are stored on your computer. Make a document that uses a common
package, say \n{fancyhdr}, and see in the log output on the screen or
in the log file where the file is loaded from. A~typical location is
\verb+/usr/share/texmf/...+. With a bit of searching you can also
find\footnote{For instance using the Unix command `\n{find}'.} the
documentation, which can be a \n{dvi}, \n{ps}, or \n{pdf} file.

If you have heard of a package and it is not on your system, go to the
\index{CTAN}`Comprehensive \TeX\ Archive Network' (CTAN for short) and
download it from there: \url{http://www.ctan.org/}.

\Level 1 {Fancy page headers and footers}
\label{sec:fancy-head}

The \n{fancyhdr}\footnote{This supersedes the \n{fancyheadings}
package.} package provides customized headers and footers. The simple
interface is 
\begin{verbatim}
\lhead{<text>}  \chead{<text>}  \rhead{<text>}
\end{verbatim}
for specifying text to get left, center, and right in the
header. Likewise \cs{lfoot} and such for the footer.

This is assuming that all pages are the same. If you format for
two-sided printing (section~\ref{sec:page}), you can specify different
text for odd and even pages:
\begin{verbatim}
\fancyhead[LE,RO]{<text>}
\end{verbatim}
for text that is Left on Even and Right on Odd pages. Typically, you
specify text for \n{[LE,RO]} and \n{[RE,LO]}, for instance
\begin{verbatim}
\fancyhead[EL,OR]{\textsl{\rightmark}}
\end{verbatim}
(see section~\ref{sec:running-head}).

\Level 1 {Pdf file generation}
\label{sec:hyperref}

Making beautiful pdf documents, complete with hyperlinks and table of
contents, from your \LaTeX\ files is simplicity itself. Insert
\begin{verbatim}
\usepackage[pdftex]{hyperref}
\end{verbatim}
in the preamble, and format with \n{pdflatex}. That's it. Do see
section~\ref{sec:graphicsx} about including pictures.

\Level 1 {Graphics}
\label{sec:graphics}

Because of \TeX's ancient origins -- and its desire to be
machine-independent -- graphics is not standard, and frankly a bit of
a hassle. The basic \TeX\ system knows nothing about graphics, it just
keeps some space open. An extension mechanism (`specials') then puts
information in the output file that the printer driver can use to
place graphics. With \n{pdflatex} this situation has become a bit less
messy: now any graphics go straight into the pdf output file.

\Level 2 {The \protect\n{picture} environment}

There is a way to generate graphics from inside \LaTeX, using some
graphics fonts rather than a full drawing mode. While primitive and
limited, the \n{picture} environment has two advantages:
\begin{itemize}
\item It is easier to get the fonts for labels to be the same as the
  text font.
\item Since it involves explicit drawing instructions, you can
  automatically draw bar charts and such.
\end{itemize}

\Level 2 {Including external graphics}
\label{sec:graphicsx}

Most of the time, you will have graphics to include that come from
some drawing package. Using the \n{graphicx} package, you write
\begin{verbatim}
\includegraphics[key=value,...]{<file name>}
\end{verbatim}
where the file name can refer to any format, but if you use pdflatex,
Postscript can not be used; if your
picture is in Postscript, you can convert it with \n{ps2pdf}.

Commands such as \cs{includegraphics}, as well as similar commands in
other packages, leave space in your document for the graphic. Now you
have to be careful: you can not leave space for a 3~inch picture, an
inch from the bottom of the page. Here are two approaches
for placing a picture:
\begin{itemize}
\item Just place it where you want it, and if it falls on a page
  break, deal with it later by moving it.
\item Put the figure in a floating figure object
  (section~\ref{sec:float}) and let \LaTeX\ sort out the placement.
\end{itemize}

You can also have text wrap around a figure, by using the \n{wrapfig}
package.

There is a package \n{color} for colour output.

\Level 1 {Other languages than English}
\label{sec:babel}

The fact that \TeX\ and \LaTeX\ were written by Americans becomes
obvious in a couple of places.
\begin{itemize}
\item Various typographical conventions are geared towards American
  English.
\item Words like `Chapter' are the default in the style
  files\footnote{They used to be hard-wired, so the situation is
    improved.}.
\end{itemize}
To address this and make \LaTeX\ easier to use with other languages,
there is a package \n{babel}.
