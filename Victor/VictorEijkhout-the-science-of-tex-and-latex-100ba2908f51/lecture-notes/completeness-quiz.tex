\documentclass{artikel3}
\usepackage{times}

\begin{document}
\pagestyle{empty}
\title{Quiz over the Dynamic Programming and NP-completeness chapters}
\author{}\date{}\maketitle

\section*{Your name:}

\vskip1cm

Consider the problem of assigning sales people to regions in such a
way as to maximize yield. There are 6
sales people and 3 regions, each region has to be covered by at least
one salesperson, and the following are the expected yields of
assigning more or less people to regions.

\begin{tabular}{r|rrr}
number of&\multicolumn{3}{c}{region}\\
people&1&2&3\\
\hline
1&4&3&5\\
2&6&6&7\\
3&9&8&10\\
4&11&10&12
\end{tabular}

Write out the principle of optimality as it applies to this problem,
and solve the problem by dynamic programming.
\vskip 1cm

Which of the following can we infer from the fact that the traveling
    salesperson problem is NP-complete, if we assume that P is not equal to NP?
\begin{enumerate}
\item There does not exist an algorithm that solves arbitrary instances of
        the TSP problem.
\item There does not exist an algorithm that efficiently solves arbitrary
        instances of the TSP problem.
\item There exists an algorithm that efficiently solves arbitrary instances
        of the TSP problem, but no one has been able to find it.
\item The TSP is not in P.
\item All algorithms that are guaranteed to solve the TSP run in polynomial
        time for some family of input points.
\item All algorithms that are guaranteed to solve the TSP run in exponential
        time for all families of input points.
\end{enumerate}

\end{document}

 2. Which of the following can we infer from the fact that PRIMALITY is in NP
    but not known to be NP-complete, if we assume that P is not equal to NP?

    (a) There exists an algorithm that solves arbitrary instances of PRIMALITY.
    (b) There exists an algorithm that efficiently solves arbitrary instances
        of PRIMALITY.
    (c) If we found an efficient algorithm for PRIMALITY, we could 
        immediately use it as a black box to solve TSP.
\end{document}

1.  We can infer (b) and (d) only.

     (a)  The brute force TSP algorithm always works - it's just inefficient.
     (b)  If P != NP, then there does not exist an efficient algorithm for
          any NP-complete problem, including TSP.
     (c)  We could infer this if P = NP.
     (d)  If P != NP, then no NP-complete problem can be in P.
     (e)  The brute force TSP algorithm always takes N! steps to solve a
          problem with N points. This is not polynomial.
     (f)  There may be easy instances. E.g., if all the TSP points lie on
          a line (or the boundary of a circle).

 2.  We can infer only (a).

     (a)  All problem in NP are solvable.
     (b)  There are problems in NP that are neither in P or NP-complete
          (assuming P != NP). PRIMALITY could be one of them.
     (c)  This cannot be inferred since we don't know if PRIMALITY is
          NP-complete. (Note that the discovery of an efficient algorithm
          for TSP would immediately imply the discovery of an efficient
          algorithm for PRIMALITY.)

