\project What is the paragraph breaking algorithm of OpenOffice?
Replace by \TeX's algorithm.
\project Write a paragraph breaking algorithm that prevents `rivers'
in the text.
\project \TeX's line breaking algorithm is not just for good looks.
Many aesthetic decisions in typography actually influence readability
of the document.
Read `Digital Typography' by Rubinstein~\cite{Rubinstein:digital-typography}
and find an issue to investigate.
\project Many page layout parameters 
(Rubinstein~\cite{Rubinstein:digital-typography} and the references 
therein) have an influence on legibility. Has typographic design deteriorated
in that sense now that authors publish their own works?
Do a study, using various books in the library.
\project The following sentence
\begin{quote}
Only the fool would take trouble to verify that this sentence was
composed of ten a's, three b's, four c's, four d's, forty-six e's,
sixteen f's, four g's, thirteen h's, fifteen i's, two k's, nine l's,
four m's, twenty-five n's, twenty-four o's. five p's, sixteen r's,
forty-one s's, thirty-seven t's, ten u's, eight v's, eight w's, four
x's, eleven y's, twenty-seven commas, twenty-three apostrophes, seven
hyphens and, last but not least, a single !
\end{quote}
is called a pangram. (There are other pangrams.
Google for the combination of `pangram' and `Lee Sallows'
for this particular type.)
Given a beginning of the sentence (`Only the fool\dots'), solve the rest of the 
sentence by dynamic programming.
