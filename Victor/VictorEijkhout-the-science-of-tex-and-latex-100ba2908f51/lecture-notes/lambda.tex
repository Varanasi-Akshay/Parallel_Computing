\input logicmacs

\Level 0 {Logic with \TeX}

\Level 1 {Truth values, operators}

We start by defining a couple of simple tools.
\begin{inputwithcode}
\def\Ignore#1{}
\def\Identity#1{#1}
\def\First#1#2{#1}
\def\Second#1#2{#2}
\end{inputwithcode}

For example:
\begin{logix}
\gtest{Taking first argument}{\First{first}{second}}
\gtest{Taking second argument}{\Second{first}{second}}
\end{logix}

We define truth values:
\begin{inputwithcode}
\let\True=\First
\let\False=\Second
\end{inputwithcode}
and logical operators:
\begin{inputwithcode}
\def\And#1#2{#1{#2}\False}
\def\Or#1#2{#1\True{#2}}
\def\Twiddle#1#2#3{#1{#3}{#2}}
\let\Not=\Twiddle
\end{inputwithcode}
Explanation: \n{And $x$ $y$} is $y$ if $x$~is true, false is $x$~is
false. Since True and False are defined as taking the first and second
component, that gives the definition of \n{And} as above. Likewise \n{Or}.

To test logical expressions, we attach \n{TF} to them before
evaluting; that was \verb+\True TF+ will print~\n{T}, and
\verb+\False TF+ will print~\n{F}.

Let us test the truth values and operators:
\begin{logixx}
\test{True takes first of TF}{\True}
\test{False takes second of TF}{\False}
\test{Not true is false}{\Not\True}

\test{And truth table TrueTrue}{\And\True\True}
\test{And truth table TrueFalse}{\And\True\False}
\test{And truth table FalseTrue}{\And\False\True}
\test{And truth table FalseFalse}{\And\False\False}

\test{Or truth table TrueTrue}{\Or\True\True}
\test{Or truth table TrueFalse}{\Or\True\False}
\test{Or truth table FalseTrue}{\Or\False\True}
\test{Or truth table FalseFalse}{\Or\False\False}
\end{logixx}

\Level 1 {Conditionals}

Some more setup. We introduce conditionals
\begin{inputwithcode}
\def\gobblefalse\else\gobbletrue\fi#1#2{\fi#1}
\def\gobbletrue\fi#1#2{\fi#2}
\def\TeXIf#1#2{#1#2 \gobblefalse\else\gobbletrue\fi}
\def\IfIsPositive{\TeXIf{\ifnum0<}}
\end{inputwithcode}
with the syntax
\begin{verbatim}
\TeXIf <test> <arg>
\end{verbatim}
We test this:
\begin{logix}
\test{Numerical test}{\IfIsPositive{3}}
\test{Numerical test}{\IfIsPositive{-2}}
\end{logix}

\Level 1 {Lists}

A list is defined as a construct with a head, which is an element, and
a tail, which is another list. We will denote the empty list by~\n{Nil}.
\begin{inputwithcode}
\let\Nil=\False
\end{inputwithcode}
We implement a list as an operator with two arguments:
\begin{itemize}
\item If the list is not empty, the first argument is applied to the
  head, and the tail is evaluated;
\item If the list is empty, the second argument is evaluated.
\end{itemize}
In other words
\[ L \, a_1 \, a_2 =
      \left\{\begin{array}{rl}a_2&\mbox{if $L=()$}\\
             a_1(x) \, Y&\mbox{if $L=(x,Y)$}
      \end{array}\right. \]
In the explanation so far, we only know the empty list \n{Nil}. Other
lists are formed by taking an element as head, and another list as
tail. This operator is called \n{Cons}, and its result is a
list. Since a~list is a two argument operator, we have to make
\n{Cons} itself a four argument operator:
\begin{inputwithcode}
% \Cons <head> <tail> <arg1> <arg2>
\def\Cons#1#2#3#4{#3{#1}{#2}}
\end{inputwithcode}
Since \verb+Cons#1#2+ is a list, applied to \verb+#3#4+ it should
expand to the second clause of the list definition, meaning it applies
the first argument~(\verb+#3+) to the head~(\verb+#1+), and evaluates the
tail~(\verb+#2+).

The following definitions are typical for list operations: since a
list is an operator, applying an operation to a list means applying
the list to some other objects.
\begin{inputwithcode}
\def\Error{{ERROR}}
\def\Head#1{#1\First\Error}
\def\Tail#1{#1\Second\Error}
\end{inputwithcode}

Let us take some heads and tails of lists.
As a convenient shorthand, a singleton is a list with an empty tail:
\begin{inputwithcode}
\def\Singleton#1{\Cons{#1}\Nil}
\end{inputwithcode}
\begin{logix}
\test{Head of a singleton}{\Head{\Singleton\True}}
\test{Head of a tail of a 2-elt list}%
     {\Head{\Tail{\Cons\True{\Singleton\False}}}}
\end{logix}

We can also do arithmetic tests on list elements:
\begin{logix}
\test{Test list content}{\IfIsPositive{\Head{\Singleton{3}}}}
\test{Test list content}{\IfIsPositive{\Head{\Tail{\Cons{3}{\Singleton{-4}}}}}}
\end{logix}

\begin{594exercise}
\item Write a function \cs{IsNil} and test with
\begin{verbatim}
\test{Detect NIL}{\IsNil\Nil}
\test{Detect non-NIL}{\IsNil{\Singleton\Nil}}
\end{verbatim}
\end{594exercise}
\begin{answer}
\begin{examplewithcode}
\def\IsNil#1{#1{\expandafter\False\gobbletwo}{\True}}
\test{Detect NIL}{\IsNil\Nil}
\test{Detect non-NIL}{\IsNil{\Singleton\Nil}}
\end{examplewithcode}
\end{answer}

\def\IsNil#1{#1{\expandafter\False\gobbletwo}{\True}}
\let\IsZero\IsNil
%\test{Zero (true result)}{\IsZero\Zero}
%\test{Zero (true result)}{\IsZero\Two}

\Level 2 {A list visualization tool}

If we are going to be working with lists, it will be a good idea to
have a way to visualize them. The following macros print a~`1' for
each list element.
\begin{inputwithcode}
\def\Transcribe#1{#1\TranscribeHT\gobbletwo}
\def\TranscribeHT#1#2{1\Transcribe{#2}}
\end{inputwithcode}

\Level 2 {List operations}

Here are some functions for
manipulating lists. We want a mechanism that takes a function~$f$, an
initial argument~$e$, and a list~$X$, so that
\[ \mathtt{Apply}\, f\, e\, X \Rightarrow
     f\, x_1\,(f\, x_2\, (\ldots (f\, x_n\, e)\ldots)) \]
\begin{inputwithcode}
% #1=function #2=initial arg #3=list
\def\ListApply#1#2#3{#3{\ListApplyp{#1}{#2}}{#2}}
\def\ListApplyp#1#2#3#4{#1{#3}{\ListApply{#1}{#2}{#4}}}
\end{inputwithcode}
This can for instance be used to append two lists:
\begin{inputwithcode}
\def\Cat#1#2{\ListApply\Cons{#2}{#1}}
\end{inputwithcode}
For example:
\begin{logix}
\test{Cat two lists}%
  {\Transcribe{\Cat{\Singleton\Nil}{\Cons\Nil{\Singleton\Nil}}}}
\end{logix}
From now on the \cs{Transcribe} macro will be implicitly assumed; it
is no longer displayed in the examples.

\Level 1 {Numbers}

We can define integers in terms of lists: zero is the empty list, and
to add one to a number is to \n{Cons} it with an empty list as head
element. In other words, \[n+1\equiv(0,n).\]
This defines the `successor' function on the integers.
\begin{inputwithcode}
\let\Zero\Nil
\def\AddOne#1{\Cons\Nil{#1}}
\end{inputwithcode}
Examples:
\begin{logix}
\ttest{Transcribe zero}{\Zero}
\ttest{Transcribe one}{\AddOne\Zero}
\ttest{Transcribe three}{\AddOne{\AddOne{\AddOne\Zero}}}
\end{logix}

Writing this many \cs{AddOne}s get tiring after a while, so here is a
useful macro:
\begin{inputwithcode}
\newtoks\dtoks\newcount\nn
\def\ndef#1#2{\nn=#2 \dtoks={\Zero}\nndef#1}
\def\nndef#1{
  \ifnum\nn=0 \edef\tmp{\def\noexpand#1{\the\dtoks}}\tmp
  \else \edef\tmp{\dtoks={\noexpand\AddOne{\the\dtoks}}}\tmp
        \advance\nn by -1 \nndef#1
  \fi}
\end{inputwithcode}
which allows us to write
\begin{inputwithcode}
\ndef\One1 \ndef\Two2 \ndef\Three3 \ndef\Four4 \ndef\Five5
\ndef\Seven7\ndef\Six6
\end{inputwithcode}
et cetera.

It is somewhat surprising that, even though the only thing we can do
is compose lists, the predecessor function is just as computable as
the successor:
\begin{inputwithcode}
\def\SubOne#1{#1\Second\Error}
\end{inputwithcode}
\begin{logix}
\ttest{Predecessor of two}{\SubOne{\AddOne{\AddOne\Zero}}}
\end{logix}
(If we had used \cs{Ignore} instead of \cs{Second} a~subtle \TeX
nicality would come into play: the list tail would be inserted
as~\verb+{#2}+, rather than~\verb+#2+, and you would see an
\verb+Unexpected }+ error message.)

Some simple arithmetic: we test if a number is odd or even.
\begin{inputwithcode}
\def\IsEven#1{#1\IsOddp\True}
\def\IsOddp#1#2{\IsOdd{#2}}
\def\IsOdd#1{#1\IsEvenp\False}
\def\IsEvenp#1#2{\IsEven{#2}}
\end{inputwithcode}
\begin{logixx}
\test{Zero even?}{\IsEven\Zero}
\test{Zero odd?}{\IsOdd\Zero}
\test{Test even}{\IsEven{\AddOne{\AddOne{\AddOne\Zero}}}}
\test{Test odd}{\IsOdd{\AddOne{\AddOne{\AddOne\Zero}}}}
\test{Test even}{\IsEven{\AddOne{\AddOne{\AddOne{\AddOne{\Zero}}}}}}
\test{Test odd}{\IsOdd{\AddOne{\AddOne{\AddOne{\AddOne{\Zero}}}}}}
\end{logixx}
\def\IsOne#1{#1\IsOnep\False}
\def\IsOnep#1{\IsZero}

\begin{594exercise}
Write a test \cs{IsOne} that tests if a number is one.
\begin{logix}
\test{Zero}{\IsOne\Zero}
\test{One}{\IsOne\One}
\test{Two}{\IsOne\Two}
\end{logix}
\end{594exercise}
\begin{answer}
\begin{verbatim}
\def\IsOne#1{#1{\IsZero{\Tail{#1}}}\False}
\end{verbatim}
\end{answer}

\Level 2 {Arithmetic: add, multiply}

Above, we introduced list concatenation with~\cs{Cat}.
This is enough to do addition. To save typing we will make macros
\cs{Three} and such that stand for the usual string of \cs{AddOne}
compositions:
\begin{inputwithcode}
\let\Add=\Cat
\end{inputwithcode}
\begin{logix}
\ttest{Adding numbers}{\Add{\Three}{\Five}}
\end{logix}
Instead of adding two numbers we can add a whole bunch
\begin{inputwithcode}
\def\AddTogether{\ListApply\Add\Zero}
\end{inputwithcode}
For example:
\begin{logix}
\ttest{Adding a list of numbers}%
  {\AddTogether{\Cons\Two{\Singleton\Three}}}
\ttest{Adding a list of numbers}%
  {\AddTogether{\Cons\Two{\Cons\Three{\Singleton\Three}}}}
\end{logix}
This is one way to do multiplication: to evaluate~$3\times5$ we make
a list of 3~copies of the number~5.
\begin{inputwithcode}
\def\Copies#1#2{#1{\ConsCopy{#2}}\Nil}
\def\ConsCopy#1#2#3{\Cons{#1}{\Copies{#3}{#1}}}
\def\Mult#1#2{\AddTogether{\Copies{#1}{#2}}}
\end{inputwithcode}
Explanation:
\begin{itemize}
\item If \verb+#1+ of \cs{Copies} is empty, then \n{Nil}.
\item Else, \cs{ConsCopy} of \verb+#2+ and the head and tail
  of~\verb+#1+.
\item The tail is one less than the original number, so \cs{ConsCopy}
  makes that many copies, and conses the list to it.
\end{itemize}
For example:
\begin{logix}
\ttest{Multiplication}{\Mult{\Three}{\Five}}
\end{logix}
However, it is more elegant to define multiplication recursively.
\begin{inputwithcode}
\def\MultiplyBy#1#2{%
  \IsOne{#1}{#2}{\Add{#2}{\MultiplyBy{\SubOne{#1}}{#2}}}}
\end{inputwithcode}
\begin{logix}
\ttest{Multiply by one}{\MultiplyBy\One\Five}
\ttest{Multiply bigger}{\MultiplyBy\Three\Five}
\end{logix}

\Level 2 {More arithmetic: subtract, divide}

\def\Sub#1#2{#1{\SubFrom{#2}}{#2}}
\def\SubFrom#1#2#3{#1{\SubMinOne{#3}}\Error}
\def\SubMinOne#1#2#3{\Sub{#1}{#3}}

The recursive definition of subtraction is
\[ m-n=\left\{\begin{array}{ll}m&\mbox{if $n=0$}\\
    (m-1)-(n-1)&\mbox{otherwise}\end{array}\right. \]
\begin{594exercise}
Implement a function \cs{Sub} that can subtract two numbers.
Example:
\begin{logix}
\ttest{Subtraction}{\Sub\Three\Five}
\end{logix}
\end{594exercise}
\begin{answer}
\begin{inputwithcode}
\def\Sub#1#2{#1{\SubFrom{#2}}{#2}}
\def\SubFrom#1#2#3{#1{\SubMinOne{#3}}\Error}
\def\SubMinOne#1#2#3{\Sub{#1}{#3}}
\end{inputwithcode}
Explanation:
\begin{quote}\begin{ttfamily}\begin{tabular}{r@{$\Rightarrow$}l}
Sub <total> <term> & <term> SubFrom{<total>} Zero\\
if <term> is empty & then <total>\\
otherwise &SubFrom<total> <term-head> <term-tail>\\
&<total> SubMinOne<term-tail> Error\\
&SubMinOne<term-tail> <total-head> <total-tail>\\
&Sub <total-tail> <term-tail>
\end{tabular}\end{ttfamily}\end{quote}
\end{answer}

\Level 2 {Continuing the recursion}

The same mechanism we used for defining multiplication from addition
can be used to define taking powers:
\begin{inputwithcode}
\def\ToThePower#1#2{%
  \IsOne{#1}{#2}{%
    \MultiplyBy{#2}{\ToThePower{\SubOne{#1}}{#2}}}}
\end{inputwithcode}
\begin{logix}
\ttest{Power taking}{\ToThePower{\Two}{\Three}}
\end{logix}

\Level 2 {Testing}

Some arithmetic tests. Greater than: if
\[ X = (x,X'), \quad Y = (y,Y') \]
then $Y>X$ is false if $Y\equiv0$:
\begin{inputwithcode}
\def\GreaterThan#1#2{#2{\GreaterEqualp{#1}}\False}
\end{inputwithcode}
Otherwise, compare $X$ with $Y'=Y-1$: $Y>X\Leftrightarrow Y'\geq X$;
this is true if $X\equiv0$:
\begin{inputwithcode}
\def\GreaterEqualp#1#2#3{\GreaterEqual{#1}{#3}}
\def\GreaterEqual#1#2{#1{\LessThanp{#2}}\True}
\end{inputwithcode}
Otherwise, compare $X'=X-1$ with $Y'=Y-1$:
\begin{inputwithcode}
\def\LessThanp#1#2#3{\GreaterThan{#3}{#1}}
\end{inputwithcode}
\begin{logixx}
\test{Greater (true result)}{\GreaterThan\Two\Five}
\test{Greater (false result)}{\GreaterThan\Three\Two}
\test{Greater (equal case)}{\GreaterThan\Two\Two}
\test{Greater than zero}{\GreaterThan\Two\Zero}
\test{Greater than zero}{\GreaterThan\Zero\Two}
\end{logixx}
Instead of just printing `true' or `false', we can use the test to
select a number or action:
\begin{logix}
\ttest{Use true result}{\GreaterThan\Two\Five\Three\One}
\ttest{Use false result}{\GreaterThan\Three\Two\Three\One}
\end{logix}
Let's check if the predicate can be used with arithmetic.
\begin{logix}
\test{$3<(5-1)$}{\GreaterThan\Three{\Sub\One\Five}}
\test{$3<(5-4)$}{\GreaterThan\Three{\Sub\Four\Five}}
\end{logix}

Equality:
\begin{inputwithcode}
\def\Equal#1#2{#2{\Equalp{#1}}{\IsZero{#1}}}
\def\Equalp#1#2#3{#1{\Equalx{#3}}{\IsOne{#2}}}
\def\Equalx#1#2#3{\Equal{#1}{#3}}
\end{inputwithcode}
\begin{logixx}
\test{Equality, true}{\Equal\Five\Five}
\test{Equality, true}{\Equal\Four\Four}
\test{Equality, false}{\Equal\Five\Four}
\test{Equality, false}{\Equal\Four\Five}
\test{$(1+3)\equiv5$: false}{\Equal{\Add\One\Three}\Five}
\test{$(2+3)\equiv(7-2)$: true}{\Equal{\Add\Two\Three}{\Sub\Two\Seven}}
\end{logixx}

Fun application: \ndef\TwentySeven{27}
\begin{inputwithcode}
\def\Mod#1#2{%
  \Equal{#1}{#2}\Zero
    {\GreaterThan{#1}{#2}%
       {\Mod{#1}{\Sub{#1}{#2}}}%
       {#2}%
    }}
\end{inputwithcode}
\begin{logix}
\ttest{$\mathop{\mathrm{Mod}}(27,4)=3$}{\Mod\Four\TwentySeven}
\ttest{$\mathop{\mathrm{Mod}}(6,3)=0$}{\Mod\Three\Six}
\end{logix}

With the modulo operation we can compute greatest common divisors:
\begin{inputwithcode}
\def\GCD#1#2{%
   \Equal{#1}{#2}%
      {#1}%
      {\GreaterThan{#1}{#2}%             % #2>#1
         {\IsOne{#1}\One
            {\GCD{\Sub{#1}{#2}}{#1}}}%   % then take GCD(#2-#1,#1)
         {\IsOne{#2}\One
            {\GCD{\Sub{#2}{#1}}{#2}}}}}  % else GCD(#1-#2,#2)
\end{inputwithcode}
\begin{logix}
\ttest{GCD(27,4)=1}{\GCD\TwentySeven\Four}
\ttest{GCD(27,3)=3}{\GCD\TwentySeven\Three}
\end{logix}
and we can search for multiples:
\begin{inputwithcode}
\def\DividesBy#1#2{\IsZero{\Mod{#1}{#2}}}
\def\NotDividesBy#1#2{\GreaterThan\Zero{\Mod{#1}{#2}}}
\def\FirstDividesByStarting#1#2{%
  \DividesBy{#1}{#2}{#2}{\FirstDividesByFrom{#1}{#2}}}
\def\FirstDividesByFrom#1#2{\FirstDividesByStarting{#1}{\AddOne{#2}}}
\end{inputwithcode}
\ndef\TwentyFive{25}
\begin{logix}
\test{$5|25$}{\DividesBy\Five\TwentyFive}
\test{$5\not|27$}{\DividesBy\Five\TwentySeven}
\test{$5\not|27$}{\NotDividesBy\Five\TwentySeven}
\ttest{$10=\min\{i:i\geq7\wedge 5|i\}$}{\FirstDividesByFrom\Five\Seven}
\end{logix}

\Level 1 {Infinite lists}

So far, we have dealt with lists that are finite, built up from an
empty list. However, we can use infinite lists too.
\begin{inputwithcode}
\def\Stream#1{\Cons{#1}{\Stream{#1}}}
\end{inputwithcode}
\begin{logix}
\gtest{Infinite objects}{\Head{\Tail{\Stream3}}}
\gtest{Infinite objects}{\Head{\Tail{\Tail{\Tail{\Tail{\Tail{\Stream3}}}}}}}
\end{logix}
Even though the list is infinite, we can easily handle it in finite
time, because it is never constructed further than we ask for it. This
is called \index{lazy evaluation}`lazy evaluation'.

We can get more interesting infinite lists by applying successive
powers of an operator to the list elements. Here is the definition of
the integers by applying the \n{AddOne} operator a number of times to zero:
\begin{inputwithcode}
% \StreamOp <operator> <initial value>
\def\StreamOp#1#2{\Cons{#2}{\StreamOp{#1}{#1{#2}}}}
\def\Integers{\StreamOp\AddOne\Zero}
\def\PositiveIntegers{\Tail\Integers}
\def\TwoOrMore{\Tail\PositiveIntegers}
\end{inputwithcode}
Again, the \n{Integers} object is only formed as far as we need it:
\begin{logix}
\ttest{Integers}{\Head{\Tail{\Integers}}}
\ttest{Integers}%
  {\Head{\Tail{\Tail{\Tail{\Tail{\Tail{\Integers}}}}}}}
\end{logix}

Let us see if we can do interesting things with lists. We want to make
a list out of everything that satisfies some condition.
\ndef\TwentyOne{21}
\begin{inputwithcode}
\def\ConsIf#1#2#3{#1{#2}{\Cons{#2}{#3}}{#3}}
\def\Doubles{\ListApply{\ConsIf{\DividesBy\Two}}\Nil\PositiveIntegers}
\def\AllSatisfy#1{\ListApply{\ConsIf{#1}}\Nil\PositiveIntegers}
\def\FirstSatisfy#1{\Head{\AllSatisfy{#1}}}
\end{inputwithcode}
\begin{logix}
\ttest{third multiple of two}{\Head{\Tail{\Tail\Doubles}}}
\ttest{old enough to drink}{\FirstSatisfy{\GreaterThan\TwentyOne}}
\end{logix}
We add the list in which we test as a parameter:
\begin{inputwithcode}
\def\AllSatisfyIn#1#2{\ListApply{\ConsIf{#1}}\Nil{#2}}
\def\FirstSatisfyIn#1#2{\Head{\AllSatisfyIn{#1}{#2}}}
\end{inputwithcode}
\begin{logix}
\ttest{}{\FirstSatisfyIn
   {\NotDividesBy{\FirstSatisfyIn{\NotDividesBy\Two}\TwoOrMore}}
   {\AllSatisfyIn{\NotDividesBy\Two}\TwoOrMore}
   }
\end{logix}
And now we can repeat this:
\begin{inputwithcode}
\def\FilteredList#1{\AllSatisfyIn{\NotDividesBy{\Head{#1}}}{\Tail{#1}}}
\def\NthPrime#1{\Head{\PrimesFromNth{#1}}}
\def\PrimesFromNth#1{\IsOne{#1}\TwoOrMore
  {\FilteredList{\PrimesFromNth{\SubOne{#1}}}}}
\end{inputwithcode}
\begin{logix}
\ttest{Third prime; spelled out}{\Head{\FilteredList{\FilteredList\TwoOrMore}}}
\ttest{Fifth prime}{\NthPrime\Four}
\end{logix}
However, this code is horrendously inefficient. To get the 7th prime
you can go make a cup of coffee, one or two more and you can go pick
the beans yourself.
\begin{inputwithcode}
%\def\FilteredList#1{\AllSatisfyIn{\NotDividesBy{\Head{#1}}}{\Tail{#1}}}
\def\xFilteredList#1#2{\AllSatisfyIn{\NotDividesBy{#1}}{#2}}
\def\FilteredList#1{\xFilteredList{\Head{#1}}{\Tail{#1}}}
\end{inputwithcode}
\begin{logix}
%\tracingmacros=2
\ttest{Fifth prime}{\NthPrime\Five}
\end{logix}

