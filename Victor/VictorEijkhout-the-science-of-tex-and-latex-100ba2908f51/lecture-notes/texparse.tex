A number of projects involve parsers or code generators for (parts of)
\TeX\ or \LaTeX.

\begin{description}
\item[formulas]
Reinhold Heckmann and Reinhard Wilhelm. 1997.  A functional
description of TeX's formula layout.  Journal of Functional
Programming 7(5):451-485. Available online at
\url{http://rw4.cs.uni-sb.de/users/heckmann/doc.html}.
For software, see \url{ftp://ftp.cs.uni-sb.de/formulae/}.

Preview-LaTeX (\url{http://preview-latex.sourceforge.net/}) displays
formulas right in the emacs edit buffer.
\item[math on web pages]see \url{http://www.forkosh.com/mimetex.html}.
\item[\LaTeX\ to HTML] HeVeA, TtH, TeX4ht and LaTeX2HTML.
\item[front end for \LaTeX] \url{http://www.lyx.org/}
Ages ago there was `griff'. Scientific Word.
\item[reimplementation of \TeX] \TeX\ in Python: \url{http://www.pytex.org/}
\end{description}

