No separate handout for this chapter; see the book `\TeX\ by Topic'.

\begin{594exercise}
Write a macro \cs{intt} (`in typewriter type')
such that \verb+\intt{foo}+ and
\verb+\intt{foo_bar}+ are output as \n{foo} and \n{foo_bar},
 in typewriter type. 
\end{594exercise}
\begin{answer}
\begin{examplewithcode}
\def\n{\bgroup\catcode`\#=12 \catcode`\_=12 \ttfamily
  \let\next=}
This is a \n{short} piece of \n{test_text}.
\end{examplewithcode}
\end{answer}

\begin{594exercise}
Write a macro that constructs another macro: \verb+\tees\three3+
should be equivalent to \verb+\def\three{TTT}+, \verb+\tees\five5+
equivalent to \verb+\def\five{TTTTT}+ et cetera. In other words, the
first argument of \cs{tees} is the name of the macro you are defining,
the second is the number of letters~`T' the defined macro expands to.
To make sure that your solution really expands to that string of `T's,
and not some code that generates it when the macro is called, 
do \verb+\show\five+ and check the screen output.
\end{594exercise}
\begin{answer}
\begin{examplewithcode}
\newcount\ntees
\def\tees#1#2{\ntees=#2 \def #1{}\ttees#1}
\def\ttees#1{
   \ifnum\ntees>0
      \edef\tmp{\def\noexpand#1{#1T}}\tmp
      \advance\ntees by -1 \ttees#1 \fi}
\tees\sept7 \sept
\end{examplewithcode}
\end{answer}

\begin{594exercise}
\TeX\ natively has addition, multiplication, and division arithmetic. 
Write a square root routine in \TeX. Hint: Use Newton's method.
\end{594exercise}

\begin{594exercise}
\def\DefineWithDelims#1%
   {\edef\tmp{\def\expandafter\noexpand\csname#1\endcsname
                  \LeftDelim ####1\RightDelim}
    \tmp}
Make this work:
\begin{examplewithcode}
\def\LeftDelim{(}\def\RightDelim{)}
\DefineWithDelims{foo}{My argument is `#1'.}
\def\LeftDelim{<}\def\RightDelim{>}
\DefineWithDelims{bar}{But my argument is `#1'.}
\foo(one)\par
\bar<two>
\end{examplewithcode}
In other words, \cs{DefineWithDelims} defines a macro --~in
this case \cs{foo}~-- and this macro has one argument, delimited by
custom delimiters. The delimiters can be specified for each macro separately.

Hint: \cs{DefineWithDelims} is actually a macro with only one argument.
Consider this code snippet:
\begin{verbatim}
\Define{foo}{ ... #1 ...}
\def\Define#1{
  \expandafter\def\csname #1\endcsname##1}
\end{verbatim}
\end{594exercise}
\begin{answer}
\begin{examplewithcode}
\def\LeftDelim{(}\def\RightDelim{)}
\def\DefineWithDelims#1%
   {\edef\tmp{\def\expandafter\noexpand\csname#1\endcsname
                  \LeftDelim ####1\RightDelim}
    \tmp}
\DefineWithDelims{foo}{The argument is `#1'.}
\foo(bar) % note the parentheses!
\end{examplewithcode}
\end{answer}

\endinput 

\begin{594exercise}
The `fake small caps' macro was inelegant, because it required the
period at the end of the sentence to be separated with a space. Repair
that macro so that it stops when there is a period at the end of a word.
\end{594exercise}
\begin{answer}
\begin{verbatim}
\def\periodstop{.}
\def\FakeSC#1#2 {
   {\uppercase{#1}\footnotesize\uppercase{#2}\ }%
   \ifHasPeriod#2.; \def\next{}%
   \else        \def\next{\FakeSC}%
   \fi \next}
\def\ifHasPeriod#1{\xTestPeriod#1.}
\def\xTestPeriod#1#2.{}
\end{verbatim}
\end{answer}
