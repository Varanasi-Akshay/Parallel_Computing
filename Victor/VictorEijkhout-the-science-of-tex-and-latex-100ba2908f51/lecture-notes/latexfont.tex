\Level 0 {The \protect\n{fontenc} package}

Traditionally, in \TeX\ accented characters were handled with control
characters, such as in~\verb+\'e+. However, many keyboards -- and this
should be understood in a software sense -- are able to generate
accented characters, and other non-latin characters,
directly. Typically, this uses octets with the high bit set.

As we have seen, the interpretation of these octets is not clear. In
the absense of some Unicode encoding, the best we can say is that it
depends on the code page that was used. This dependency could be
solved by having the \TeX\ software know, on installation, what code
page the system is using. While this may be feasible for one system,
if the input files are moved to a different system, they are no longer
interpreted correctly. For this purpose the \n{inputenc} package was
developed.

An input encoding can be stated at the load of the package:
\begin{verbatim}
\usepackage[cp1252]{inputenc}
\end{verbatim}
or input encodings can be set and switched later:
\begin{verbatim}
\inputencoding{latin1}
\end{verbatim}
With this, a (part of~a) file can be written on one machine, using
some code page, and still be formatted correctly on another machine,
which natively has a different code page.

These code pages are all conventions for the interpretation of singly
octets. The \n{inputenc} package also has limited support for
\n{UTF-8}, which is a variable length (up to four octets) encoding of
Unicode.
