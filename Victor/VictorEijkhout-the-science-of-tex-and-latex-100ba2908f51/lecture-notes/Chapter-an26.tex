First we observe that through the inclusion of C~code, all these
languages, whether regular, context-free, context-sensitive, can be
parsed in \lex. This means that \lex\ and \yacc\ are theoretically not
restricted to regular and context-free languages, even though their
basic mechanism is a FSA and PDA respectively.

Let us now look at solutions that use both \lex\ and \yacc.

For these languages, the \lex\ program can be very simple:
\verbatiminput{anbn-lex.l}
The problem with the \yacc\ code is how to give error messages for
ungrammatical strings. This program recognizes the language, and lets
\yacc\ give its default error on ungrammatical strings:
\verbatiminput{anbn1.y}
Attempts to parse unbalanced strings of $a$s and $b$s such as
\verbatiminput{anbn2.y}
invariably lead to conflicts, because \yacc\ can not decide which rule
to match on an $a$~input. An unbalanced amounts of $b$s can be
handled:
\verbatiminput{anbn3.y}
but for a general solution we really need to recognize~$\{a^mb^n\}$
and impose the restriction~$m\equiv\nobreak n$ through included
C~code:
\verbatiminput{anbn4.y}
Without the error clauses, this recognizes $a^mb^n$, and it is easy to
extend this program to~$a^nb^nc^n$.
