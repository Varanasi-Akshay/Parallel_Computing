\documentclass[twoside,letterpaper,11pt]{boek3}

% general packages
\usepackage{hyperref}
\usepackage{verbatim,fleqn,times,multicol,amssymb}
\usepackage[pdftex]{graphicx}
\usepackage[noeepic]{qtree}
\input supp-pdf.tex

% my packages
\usepackage{comment,outliner}

\input idxmacs

%\addtocounter{tocdepth}{-2}
\setcounter{tocdepth}{1}

%
% pdf formatting for screen or paper
%
\begin{comment}
\includecomment{print}
% no edits below this line.
\includecomment{screen}
\begin{print}
\excludecomment{screen}
\end{print}
\input paperscreen
\end{comment}

\input tutmacs
\input exmacs

%
% Specific macros for this document
%
\newcommand\checkthis{{\bf CHECKTHIS}}
\newcommand{\PageTitle}[1]{
    \vfill \pagebreak
    \bigskip {\bf \Large #1.}\nopagebreak\medskip
    \addcontentsline{toc}{section}{\protect\numberline{}{#1}}
    }
\renewenvironment{exercises}%
    {\PageTitle{Exercises and projects}
     \begin{enumerate}}%
    {\end{enumerate}}
\newcounter{projects}[chapter]
%\counterwithin{projects}{chapter}
\renewcommand{\project}{%
    \stepcounter{projects}
    \item[{\bfseries Project \arabic{chapter}.\arabic{projects}.}]%
    }
\newcommand{\reserve}[3]{%
    \textsl{#3}\addtocontents{rsv}{#1. #2 -- \textsl{#3}}}
\newcommand{\projects}[1]{%
    \PageTitle{Projects for this chapter}
    \begin{projectslist}\input #1-projects.tex
    \end{projectslist}
    }
\newenvironment{projectslist}{\begin{enumerate}}{\end{enumerate}}

%
% teacher's vs students' version
%

% UT use versus public
\excludecomment{ut-only}
\includecomment{external}

% class room use
\excludecomment{instructor}
\includecomment{fullversion}
\excludecomment{problem}
\excludecomment{answers}
% instructor version
\excludecomment{quiz}
\begin{instructor}
\specialcomment{quiz}{\begin{description}\item[Quiz]}{\end{description}}
\end{instructor}
% full and short version
\newcounter{sessions}
\def\includechapter#1#2#3{\addtocounter{sessions}{#2}}
\begin{fullversion}
\def\includechapter#1#2#3{
  \PageTitle{#1}
  \addtocounter{sessions}{#2}
  \SetBaseLevel{1}
  \input #3.tex
  \SetBaseLevel{0}
  }
\end{fullversion}
\def\chapteranswers#1#2{}
\begin{instructor}
\def\chapteranswers#1#2{
  \PageTitle{Answers for chapter #1}
  \begingroup \def\chaptername{#2}
    \TheAnswers
  \endgroup}
\end{instructor}

\newtheorem{theorem}{Theorem}
\newtheorem{lemma}{Lemma}
\newenvironment{material}%
    {\bigskip {\bf Handouts and further reading for this
        chapter}\par}%
    {\bigskip}

\OutlineLevelStart0{\cleardoublepage\chapter{#1}}
\OutlineLevelStart1{\section{#1}}
\OutlineLevelStart2{\subsection{#1}}
\setcounter{secnumdepth}{3}
\OutlineLevelStart3{\subsubsection{#1}}
\OutlineLevelStart4{\paragraph{#1}}

%
% page layout
%
\usepackage{fancyhdr}
\pagestyle{fancy}\fancyhead{}\fancyfoot{}
\fancyhead[RO,LE]{\thepage}
\fancyhead[LO]{\rightmark}
\fancyhead[RE]{\leftmark}
\fancyfoot[RE]{\footnotesize\sl \TeX\ -- \LaTeX\ -- CS 594}
\fancyfoot[LO]{\footnotesize\sl Victor Eijkhout}

\begin{document}
%\frontmatter

\title{The Computer Science of \TeX\ and \LaTeX;\\
based on CS 594, fall 2004, University of Tennessee}
\author{Victor Eijkhout\\ Texas Advanced Computing Center\\
 The University of Texas at Austin}
\date{2012}
\begin{instructor}
\date{{\Large\bf Instructor's version}}
\end{instructor}
\maketitle

\begin{ut-only}
\emptypage
\PageTitle{CS 594 -- Practical Matters}
\begin{description}
\item[aim of this course] This course will teach you the fundamentals
  of \TeX\ and \LaTeX\ use. However, you will pick up more than just a
  few practical skills. We will see all sorts of mathematical and
  computer science topics that are (sometimes maybe only marginally)
  related to \TeX\ and \LaTeX. So, \TeX/\LaTeX\ is here really just an
  excuse for teaching cool CS subjects.
\item[homework and exams] There will be regular homework and
  occasional pop quizzes. Homework needs to be done in \LaTeX: mail
  both source and pdf output to \n{cs594tex}.
\item[final project] The final
  exam will be in the form of a project that can be done
  individually or with 2~or~3 people. You can find the project
  descriptions at the end of each chapter. If you have an idea for a
  project, feel free to talk to me about it.
\item[course materials] The main handout can be bought from Graphic
  Creations (1809 Lake Avenue, 865-522-6221). It is very much
  recommended that you buy the {\slshape Guide To \LaTeX} (fourth
  edition) by Kopka and
  Daly. (Order from Amazon, Ecampus, Bookpool, et cetera; the
  university bookstore has a small number of copies.) Certain
  reference books have been put on reserve in the library (2nd floor).
\item[lecture notes] Further material can be downloaded from
  \url{http://www.cs.utk.edu/~eijkhout/594-LaTeX}. Files with
  \n{tutorial} in the name are written out lecture notes; you can
  download and print them. Files with
  \n{slides} in the name are slides; there should be no need for
  printing them.
\item[classes] We meet every tuesday and thursday 2:10--3:25pm. No class on
  sep~2, oct~14 (fall break), nov~25 (thanksgiving).
\end{description}

\bigskip
Victor Eijkhout\\
eijkhout@cs.utk.edu\\
329 Claxton, daily 11--12 and by appointment

\bigskip
Lan Lin\\
lin@cs.utk.edu\\
110E Claxton, mon/tue/wed 3:30--5:30
\end{ut-only}

\begin{external}

\hbox{}\vfill
\pagebreak

\PageTitle{About this book}

These are the lecture notes of a course I taught in the fall of 2004.
This was the first time I taught the course, and the first time this
course was taught, period. It has also remained the \emph{only} time
the course was taught. These lecture notes, therefore, are
probably full of inaccuracies, mild fibs, and gross errors. Ok, make
that: are \emph{definitely} full of at the least the first two
categories,
because I~know of several
errors that time has prevented me from addressing.

My lack of time being what it is, this unfinished book wil now remain
as is. The reader is asked to enjoy it, but not to take it for gospel.

\bigskip
Victor Eijkhout\\
eijkhout@tacc.utexas.edu\\
Knoxville, TN, december 2004;\\
Austin, TX, january 2012.

\bigskip
copyright 2012 Victor Eijkhout\\
ISBN 978-1-105-41591-3\\
distributed under a Creative Commons Attribution 3.0
Unported (CC BY 3.0) license
\end{external}

\bigskip
Enjoy!

\clearpage
\pagestyle{fancy}
\begin{multicols}{2}
\tableofcontents
\end{multicols}

%\mainmatter
\begin{Outline}

%%
%% TeX and LaTeX
%%
\Level 0 {\TeX\ and \LaTeX}

In this chapter we will learn 
\begin{itemize}
\item The use of \LaTeX\ for document preparation,
\item \LaTeX\ style file programming,
\item \TeX\ programming.
\end{itemize}

\begin{material}
For \LaTeX\ use the `Not so short introduction to \LaTeX' by Oetiker
{\it et al.} For further reading and future reference, it is highly
recommended that you get `Guide to \LaTeX' by Kopka and
Daly~\cite{KopkaDaly}. The original reference is the book by
Lamport~\cite{Lamport:LaTeX}. While it is a fine book, it has not kept
up with developments around \LaTeX, such as contributed graphics and
other packages. A~book that does discuss extensions to \LaTeX\ in
great detail is the `\LaTeX\ Companion' by Mittelbach {\it et
  al.}~\cite{LaTeXcompanion}.

For the \TeX\ system itself, consult `\TeX\ by Topic'. The original
reference is the book by Knuth~\cite{Knuth:TeXbook}, and the ultimate
reference is the published source~\cite{Knuth:TeXprogram}.
\end{material}

\includechapter{\LaTeX}{3}{latex}
\includechapter{\TeX\ programming}{2}{tex1}
\includechapter{\TeX\ visuals}{1}{tex2}
%\chapteranswers{\LaTeX}{latex}
\chapteranswers{\TeX\ programming}{tex1}
\chapteranswers{\TeX\ visuals}{tex2}
\projects{tex}

\begin{quiz}
\TeX\ is like assembly language, and \LaTeX\ is like a higher level
programming language. Do you agree?
\end{quiz}

%%
%% Parsing
%%
\Level 0 {Parsing}

The programming language part of \TeX\ is rather unusual. In this
chapter we will learn the basics of language theory and parsing, and
apply this to parsing \TeX\ and \LaTeX. Although \TeX\ can not be
treated like other programming languages, it is interesting to see how
far we can get with existing tools.

\begin{material}
The theory of languages and automata is discussed in any number of
books, such as the Hopcroft and Ulman one. For a discussion that is
more specific to compilers, see the compilers book by Aho and Ulman or
Aho, Seti, and Ulman.

The tutorials on \lex\ and \yacc\ should suffice you for most
applications.  The O'Reilly book by Levine, Mason, and Brown is
probably the best reference on \lex\ and \yacc. A~copy of it is on
reserve in the library, \reserve{Lex and Yacc}{Levine, Mason, and
  Brown}{QA76.76.U84M37}.

The definitive reference on hashing is Knuth's volume~3 of The Art of
Computer Programming~\cite{Knuth:aocp3}, section~6.4. This is on
reserve, \reserve{Sorting and searching}{Knuth}{QA76.5.K57}.
\end{material}

\includechapter{Parsing theory}{2}{parsing}
\includechapter{Lex}{2}{lex}
\includechapter{Yacc}{2}{yacc}
\includechapter{Hashing}{2}{hashing}
\projects{parsing}

%%
%% Line and page breaking
%%
\Level 0 {Breaking things into pieces}

The line breaking algorithm of \TeX\ is interesting, in that it
produces an aesthetically optimal solution in very little time.

\begin{material}
If you still have the book `Introduction to Algorithms' by Cormen
\etal, you can find a discussion of Dynamic Programming and
NP-completeness there.
The books by Bellman are the standard works in this field. Bellman's
`Applied Dynamic Programming'~\cite{Bellman:AppliedDynamic}
has been put on reserve,
\reserve{Applied Dynamic Programming}{Bellmand and
  Stuarts}{QA264.B353}.
The \TeX\ line breaking algorithm is described in an article by Knuth
and Plass~\cite{K:break}, reprinted in~\cite{Knuth:digitaltypography}.

The standard work on Complexity Theory, including NP-completeness, is
Garey and Johnson `Computers and intractibility'~\cite{Garey:Intractibility}. 
There is excellent online material about this subject on Wikipedia,
for instance \url{http://en.wikipedia.org/wiki/Complexity_classes_P_and_NP}.
Issues in page breaking are discussed in Plass' thesis~\cite{Plass:thesis}.
\end{material}

\includechapter{Dynamic Programming}{2}{dynamic}
\includechapter{\TeX\ paragraph breaking}{1}{paragraph}
\includechapter{NP completeness}{1}{completeness}
\includechapter{Page breaking}{1}{page}
\projects{dynamic}

%%
%% Fonts
%%
\Level 0 {Fonts}

Knuth wrote a font program, Metafont, to go with \TeX. The font
descriptions involve some interesting mathematics.

\begin{material}
Bezier curves and raster graphics are standard topics in computer
graphics. The book by Foley and Van Dam (section~11.2 about Splines)
has been placed on reserve,
\reserve{Computer Graphics: principles and practice}{Foley
  \etal}{T385�.C587}. More theoretical information can be found de
Boor's Practical Guide to Splines~\cite{deBoor:splines}, which
unfortunately limits itself to spline functions.

Digital typography is a very wide area, spanning from perception
psychology and physiology to the electronics and mathematics of
display equipment. The book by
Rubinstein~\cite{Rubinstein:digital-typography} is a good
introduction. This book has been placed on reserve, \reserve{Digital
  Typography}{Rubinstein}{Z253.3.R8}.

The relation between Bezier curves and aesthetics is explicitly
discussed in~\url{http://www.webreference.com/dlab/9902/examples-ii.html}.

\begin{comment}
The material on raster graphics is based on Knuth's `A~note on
digitized angles'~\cite{Knuth:angles}, and chapter~24 of the Metafont
book~\cite{Knuth:metafont}.
\end{comment}
\end{material}

\includechapter{Bezier curves}{3}{bezier}
\includechapter{Curve plotting with \protect\n{gnuplot}}{1}{gnuplot}
\includechapter{Raster graphics}{1}{raster}
\projects{bezier}

%%
%% Lambda calculus
%%
\Level 0 {\TeX's macro language -- unfinished chapter}

The programming language of \TeX\ is rather idiosyncratic. One notable
feature is the difference between expanded and executed commands. The
expansion mechanism is very powerful: it is in fact possible to
implement lambda calculus in it.

\begin{material}
The inspiration for this chapter was the article about lists by Alan
Jeffrey~\cite{Jeffrey:lists}.
\end{material}

\includechapter{Lambda calculus in \TeX}{2}{lambda}


%%
%% Software engineering 4: character encoding
%%
\Level 0 {Character encoding}

This chapter is about how to interpret the characters in an input file
-- no there ain't such a thing as a plain text file -- and how the
printed characters are encoded in a font.

\begin{material}
There is very little printed material on this topic. A~good
introduction is
\url{http://www.joelonsoftware.com/articles/Unicode.html}; after that,
\url{http://www.cs.tut.fi/~jkorpela/chars.html} is a good
tutorial for general issues, and
\url{http://en.wikipedia.org/wiki/Unicode} for Unicode.

For the technical details on Unicode consult
\url{http://www.unicode.org/}. An introduction to ISO~8859:
\url{http://www.wordiq.com/definition/ISO_8859}.
\end{material}

\includechapter{Input file encoding}{1}{encoding}
\includechapter{Font encoding}{1}{fonts}
\includechapter{Input and output encoding in \LaTeX}{1}{latexfont}
\projects{encoding}

\begin{quiz}
I type a key~`A' into an \n{html} file. An~`A' appears on my
screen. If you browse the file, you see an~`A'. What translation
stages are happening here?
\end{quiz}

%%
%% Software engineering 1: literate programming
%%
\Level 0 {Software engineering}

In the course of writing \TeX\ and Metafont, Knuth developed some
interesting ideas about software engineering. We will put those in the
context of other writings on the subject. One of the by-products of \TeX\
is the Web system for `literate programming'. We will take a
look at that, as well as at the general idea of markup.

\begin{material}
Knuth wrote a history of the \TeX\ project in~\cite{Knuth:TeXerrors},
reprinted in `Literate Programming', which is on reserve in the library,
\reserve{Literate Programming}{D.E. Knuth}{QA76.6.K644 1992}.

One of the classics of software engineering is Frederick Brooks' `The
Mythical Man-Month'~\cite{Brooks:mythical}.

For software engineering research, consult the following journals:
\begin{itemize}
\item Software practice and experience
\item Journal of systems and software
\item ACM Transactions on Software Engineering and Methodology
\item IEEE Transactions on Reliability
\end{itemize}
Some of these the library has available online.
\end{material}

\includechapter{Literate programming}{1}{web}
\includechapter{Software engineering}{1}{software}
\includechapter{Markup}{1}{markup}
\projects{software}

\bibliographystyle{plain}
\bibliography{math,cs,tex}

\printindex

\begin{instructor}
%\backmatter
\PageTitle{Course summary}

Number of projects: \arabic{projects}

Number of sessions: \arabic{sessions} (out of 29)
\end{instructor}

\end{Outline}

\begin{instructor}
\message{>>>>>>>> This version is for the instructor only <<<<<<<<<<<}
\message{>>>>>>>> This version is for the instructor only <<<<<<<<<<<}
\message{>>>>>>>> This version is for the instructor only <<<<<<<<<<<}
\message{>>>>>>>> This version is for the instructor only <<<<<<<<<<<}
\end{instructor}

\end{document}
