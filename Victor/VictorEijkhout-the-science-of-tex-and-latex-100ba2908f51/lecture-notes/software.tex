(Quotes by Knuth in this chapter taken from~\cite{Knuth:TeXerrors}.)

\Level 0 {Extremely brief history of \TeX}

Knuth wrote a first report on \TeX\ in early 1977, and left it to
graduate students Frank Liang and Michael Plass to implement it over
the summer. Starting from their prototype,
he then spent the rest of 1977 and early 1978 implementing \TeX\ and
producing fonts with the first version of \metafont. 

\TeX\ was used by Knuth himself in mid 1978 to typeset volume~2 of The
Art of Computer Programming; it was in general
use by August of 1978. By early 1979, Knuth had written a system
called Doc that was the precursor of \web, and that produced both
documentation and a portable Pascal version of the source; the
original program was written in Sail.

In 1980 Knuth decided to rewrite \TeX\ and \metafont. He started on
this in 1981, and finished, including producing the five volumes of
\textsl{Computer and Typesetting}, in 1985.

\Level 0 {\TeX's development}

\Level 1 {Knuth's ideas}

Inspecting the work of his students, Knuth found that they had had to
make many design decisions, despite his earlier conviction to have
produced `a~reasonably complete specification of a language for
typesetting'.
\begin{quote}
The designer of a new kind of system must participate fully in the
implementation.
\end{quote}

Debugging happened in about 18 days in March 1978. Knuth contrasts
that with 41 days for writing the program, making debugging about 30\%
of the total time, as opposed to 70\% in his earlier experience. The
whole code at that time was under 5000 statements. The rewritten \TeX
82 runs to about $14\,000$ statements, in $1400$ modules of \web.

He considered
this his first non-trivial program written using the \index{structured
  programming}structured programming methodology of Dijkstra, Hoare,
Dahl, and others. Because of the confidence this gave him in the
correctness of the program, he did not test \TeX\ until both the whole
program and the fonts were in place. `I~did not have to prepare dummy
versions of non-existent modules while testing modules that were
already written'.

By mid 1979, Knuth was using \TeX, and was improving \TeX\ `at a
regular rate of about one enhancement for every six pages
typed'.
\begin{quote}
Thus, the initial testing of a program should be done by the
designer/implementor.
\end{quote}
Triggered by a challenge of John McCarthy, Knuth wrote a manual, which
forced him to think about \TeX\ as a whole, and which led to further
improvements in the system.
\begin{quote}
The designer should also write the first user manual.
\end{quote}
`If I had not participated fully in all these activities, literally
hundreds of improvements would never have been made, because I would
never have thought of them or perceived why they were important.'

Knuth remarks that testing a compiler by using it on a large, real,
input typically leaves many statements and cases unexecuted. He
therefore proposes the `\index{Torture test}\index{Trip test}torture
test' approach. This consists of writing input that is as far-fetched
as possible, so that it will explore many subtle interactions between
parts of the compiler. He claims to have spent 200 hours writing and
maintaining the `trip test'; there is a similar `trap test' for
\metafont.

\Level 1 {Context}

Software engineering is not an exact science. Therefore, some of
Knuth's ideas can be fit in accepted practices, others are counter. In
this section we will mention some schools of thought in software
engineering, and see how Knuth's development of \TeX\ fits in it.

\Level 2 {Team work}

The upshot of the development of \TeX\ seems to be that software
development is a one-man affair. However, in industry, programming
teams exist. Is the distinction between \TeX\ and commercial products
then that between academic and real-world? 

Knuth's ideas are actually not that unusual. Programming productivity
is not simply expressible as the product of people and time, witness
the book `The Mythical man-Month'. However, some software projects are
too big, even for one \emph{really} clever designer~/ programmer~/ tester~/
manual writer.

Dividing programming work is tricky, because of the interdependence of
the parts. The further you divide, the harder coordination becomes,
and the greater the danger of conflicts.

Harlan Mills proposed that software should be written by groups, where
each group works like a \index{surgical team}surgical team: one chief
surgeon who does all the real work, with a team to assist in the more
mundane tasks. Specifically:
\begin{itemize}
\item The Chief Programmer designs the software, codes it, tests, it,
  and writes the documentation.
\item The Co-Pilot is the Chief Programmer's alter ego. He knows all
  the code but writes none of it. He thinks about the design and
  discusses it with the Chief Programmer, and is therefore insurance
  against disaster.
\item The Administrator takes care of the mundane aspects of a
  programming project. This can be a part-time position, shared
  between teams.
\item The Editor oversees production of documentation.
\item Two Secretaries, one each for the Administrator and Editor.
\item The Program Clerk is responsible for keeping records of all code
  and the test runs with their inputs. This post is also necessary
  because all coding and testing will be matter of public record.
\item The Toolsmith maintains the utilities used by the other team
  members.
\item The Tester writes the test cases.
\item The Language Lawyer investigates different constructs that can
  realize the Chief Programmer's algorithms.
\end{itemize}

With such teams of 10 people, coordination problems are divided
by~10. For the overall design there will be a system architect, or a
small number of such people.

Recently, a development methodology name `\index{Extreme
  programming}Extreme Programming' has become popular. One aspect of
this is pair programming: two programmers share one screen, one
keyboard. The advantage of this is that all code is immediately
reviewed and discussed.

\Level 2 {Top-down and bottom-up}

Knuth clearly favours the top-down approach that was proposed by
Nicklaus Wirth in `Program Development by Stepwise
Refinement'~\cite{wirth:refinement}, and by Harlan Mills, who
pioneered it at IBM. The advantage of top-down programming is that the
design of the system is set from the beginning. The disadvantage is
that it is hard to change the design, and testing that shows
inadequacies can only start relatively late in the development
process.

Bottom-up programming starts from implementing the basic blocks of a
code. The advantage is that they can immediately be tested; the
disadvantage is that the design is in danger of becoming more ad
hoc.

An interesting form of bottom-up programming is `\index{Test-driven
  development}test-driven development'. Here, first a test is written
for a unit (a~`\index{Unit test}unit test'), then the code. At all
times, all tests need to be passed. Rewriting code, usually to
simplify it, with preservation of functionality as defined by the
tests, is known as `\index{Refactoring}refactoring'.

\Level 2 {Program correctness}

The Trip test is an example of `\index{Regression testing}regression
testing': after every change to the code, a batch of tests is run to
make sure that previous bugs do not reappear. This idea dates back to
Brooks; it is an essential part of Extreme Programming.

However, the Trip test only does regression testing of the whole
code. TDD uses both Unit tests and Integration tests. A~unit is a
specific piece of code that can easily be tested since it has a clear
interface. In testing a unit, the code structure can be used to design
the tests. Integration testing is usually done as Black Box testing:
only the functionality of an assemblage of units is known and tested,
rather than the internal structure. One way of doing integration
testing is by `\index{Equivalence partitioning}equivalence
partitioning': the input space is divided into classes such that
within each classes the input are equivalent in their
behaviour. Generating these classes, however, is heuristic, and it is
possible to overlook cases.

On the opposite side of the testing spectrum is program
proving. However, as Knuth wrote in a memo to Peter van Emde Boas:
`Beware of bugs in the above code; I have only proved it correct, not
tried it.'
