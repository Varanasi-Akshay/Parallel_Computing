% -*- latex -*-
%%%%%%%%%%%%%%%%%%%%%%%%%%%%%%%%%%%%%%%%%%%%%%%%%%%%%%%%%%%%%%%%
%%%%
%%%% This TeX file is part of the course
%%%% Introduction to Scientific Programming in C++/Fortran2003
%%%% copyright 2017 Victor Eijkhout eijkhout@tacc.utexas.edu
%%%%
%%%% infect.tex : exercises for infectious disease simulation
%%%%
%%%%%%%%%%%%%%%%%%%%%%%%%%%%%%%%%%%%%%%%%%%%%%%%%%%%%%%%%%%%%%%%

This section contains a sequence of exercises that builds up to a
somewhat realistic simulation of the spread of infectious
diseases. 

\Level 0 {Model design}

It is possible to model disease propagation statistically, but here we
will build an explicit simulation: we will maintain an explicit
description of all
the people in the population, and track for each of them their status.

We will use a simple model where a person can be:
\begin{itemize}
\item sick: when they are sick, they can infect other people;
\item susceptible: they are healthy, but can be infected;
\item recovered: they have been sick, but no longer carry the disease,
  and can not be infected for a second time;
\item inocculated: they are healthy, do not carry the disease, and can
  not be infected.
\end{itemize}
In more complicated models a person could be infectious during only
part of their illness, or there could be secondary infections with
other diseases, et cetera. We keep it simple here:
any sick person is infectious.

In the exercises below we will gradually develop a somewhat realistic
model of how the disease spreads from an infectious person. We always
start with just one person infected.
The program will then track the population from day to day,
running indefinitely until none of the population
is sick. Since there is no re-infection, the run will always end.

\Level 1 {Other ways of modeling}

Instead of capturing every single person in code, a~`contact network'
model,
it is possible to use an
\ac{ODE} approach to disease modeling. You would then model the
percentage of infected persons by a single scalar, and derive
relations for that and other scalars~\cite{Anderson:population1979}.

\url{http://mathworld.wolfram.com/Kermack-McKendrickModel.html}

This is known as a `compartmental model'. Both the contact network and
the compartmental model capture part of the truth. In fact, they can
be combined. We can consider a country as a set of cities, where
people travel between any pair of cities. We then use a compartmental
model inside a city, and a contact network between cities.

\Level 0 {Coding up the basics}

\prerequisite{\ref{ch:object}}

We start by writing code that models a single person. The main methods
serve to infect a person, and to track their state. We need to have
some methods for inspecting that state. 

The intended output looks something like:
\begin{verbatim}
On day 10, Joe is susceptible
On day 11, Joe is susceptible
On day 12, Joe is susceptible
On day 13, Joe is susceptible
On day 14, Joe is sick (5 to go)
On day 15, Joe is sick (4 to go)
On day 16, Joe is sick (3 to go)
On day 17, Joe is sick (2 to go)
On day 18, Joe is sick (1 to go)
On day 19, Joe is recovered
\end{verbatim}

\begin{exercise}
  \label{ex:infect:person}
  Write a \n{Person} class with methods:
  \begin{itemize}
  \item \n{status_string()} : returns a description of the
    person's state;
  \item \n{update()} : update the person's status to the next day;
  \item \n{infect(n)} : infect a person, with the disease to run for
    $n$ days;
  \item \n{is_stable()} : report whether the person has been sick and
    is recovered.
  \end{itemize}
  The main program should execute these statements in a loop:
  \verbatimsnippet{infect0}
\end{exercise}

Your main program could for instance look like:
\verbatimsnippet{infect0}

Here is a suggestion how you can model 
disease status. Use a single integer with the following interpretation:
\begin{itemize}
\item healthy but not inoculated, value~$0$,
\item recovered, value~$-1$,
\item inocculated, value~$-2$,
\item and sick, with $n$ days to go before recovery,  modeled by value~$n$.
\end{itemize}
The \n{Person::update} method then updates this integer.

\Level 0 {Population}

\prerequisite{\ref{ch:array}}

Next we need a \n{Population} class. Implement a population as a \n{vector}
consisting of \n{Person} objects. Initially we only infect one person, and there
is no transmission of the disease.

The trace output should look something like:
\begin{verbatim}
Size of
population?
In step   1 #sick:    1 :  ? ? ? ? ? ? ? ? ? ? + ? ? ? ? ? ? ? ? ?
In step   2 #sick:    1 :  ? ? ? ? ? ? ? ? ? ? + ? ? ? ? ? ? ? ? ?
In step   3 #sick:    1 :  ? ? ? ? ? ? ? ? ? ? + ? ? ? ? ? ? ? ? ?
In step   4 #sick:    1 :  ? ? ? ? ? ? ? ? ? ? + ? ? ? ? ? ? ? ? ?
In step   5 #sick:    1 :  ? ? ? ? ? ? ? ? ? ? + ? ? ? ? ? ? ? ? ?
In step   6 #sick:    0 :  ? ? ? ? ? ? ? ? ? ? - ? ? ? ? ? ? ? ? ?
Disease ran its course by step 6
\end{verbatim}

\begin{exercise}
  \label{ex:infect:notransfer}
  Program a population without infection.
  \begin{itemize}
  \item Write the \n{Population} class. The constructor takes the number of people:
\begin{verbatim}
Population population(npeople);  
\end{verbatim}
  \item Write a method that infects a random person:
\begin{verbatim}
population.random_infection();
\end{verbatim}
  \item Write a method \n{count_infected} that counts how many people are infected.
  \item Write an \n{update} method that updates all persons in the population.
  \item Loop the \n{update} method until no people are infected: the
    \n{Population::update} method should apply \n{Person::update} to
    all person in the population.
  \end{itemize}
\item Write a routine that displays the state of the popular, using
  for instance: \n{?}~for susceptible, \n{+}~for infected, \n{-}~for recovered.
\end{exercise}

\Level 0 {Contagion}

This past exercise was too simplistic: the original patient zero was
the only one who ever got sick.
Now we incorporate contagion, seeing how far the disease can spread
from a single infected person.

We start with a very simple model of infection.

\begin{exercise}
  \label{ex:infect:1}
  Read in a number $0\leq p \leq 1$ representing the probability of
  disease transmission upon contact. Incorporate this into the
  program: in each step the direct neighbours of an infected person
  get sick themselves.
\begin{verbatim}
population.set_probability_of_transfer(probability);  
\end{verbatim}
  Run a number of simulations with population sizes and contagion
  probabilities. Are there cases where people escape getting sick?
\end{exercise}

\begin{exercise}
  \label{ex:infect:2}
  Incorporate inoculation: read another number representing the
  percentage of people that has been vaccinated. Choose those members
  of the population randomly.

  Describe the effect of vaccinated people on the spread of the
  disease. Why is this model unrealistic?
\end{exercise}

\Level 0 {Spreading}

To make the simulation more realistic, we let every sick person come
into contact with a fixed number of random people every day. This
gives us more or less the \indexterm{SIR model};
\url{https://en.wikipedia.org/wiki/Epidemic_model}.

Set the number of people that a person comes into contact with, per
day, to~6 or so. You have already programmed the probability that a
person who comes in contact with an infected person gets sick themselves.
Again start the simulation with a single infected person.

\begin{exercise}
  \label{ex:infect:3}
  Code the random interactions. Now run a number of simulations varying
  \begin{itemize}
  \item The percentage of people inoculated, and
  \item the chance the disease is transmitted on contact.
  \end{itemize}
  Record how long the disease runs through the population. With a
  fixed number of contacts and probability of transmission, how is
  this number of function of the percentage that is vaccinated?

  Investigate the matter of `herd immunity': if enough people are
  vaccinated, then some people who are not vaccinated will still never get
  sick. Let's say you want to have this probability over 95
  percent. Investigate the percentage of inoculation that is needed
  for this as a function of the contagiousness of the disease.
\end{exercise}

\Level 0 {Project writeup and submission}

\Level 1 {Program files}

In the course of this project you have written more than one main
program, but some code is shared between the multiple programs.
Organize your code with one file for each main program, and a single
`library' file with the class methods. This requires you to use
\indextermsub{separate}{compilation} for building the program, and you
need a \indexterm{header} file; section~\ref{sec:hfile}.

Submit all source files with instructions on how to build all the main
programs. You can put these instructions in a file with a descriptive
name such as \n{README} or \n{INSTALL}, or you can use a
\indexterm{makefile}.

\Level 1 {Writeup}

In the writeup, describe the `experiments' you have performed and the
conclusions you draw from them. The exercises above give you a number
of questions to address.

For each main program, include some sample output, but note that this
is no substitute for writing out your conclusions in full sentences.

The last exercise asks you to explore the program behaviour as a
function of one or more parameters. Include a table to report on the
behaviour you found. You can use Matlab or Matplotlib in Python (or
even Excell) to plot your data, but that is not required.

\endinput

\begin{exercise}
  You can make the model more realistic by letting innoculation be
  only partly effective. For instance, 50\% of people got the flu
  vaccine, but it was only 40\% effective; 90\%~of people have the
  measles vaccine, and it is about 97\%
  effective. (\url{https://www.cdc.gov/nchs/fastats/immunize.htm}) How
  does your model function in this case? Keep in mind that different
  diseases have different degrees of infectuousness
  (\url{https://en.wikipedia.org/wiki/Basic_reproduction_number}).
\end{exercise}
