% -*- latex -*-
%%%%%%%%%%%%%%%%%%%%%%%%%%%%%%%%%%%%%%%%%%%%%%%%%%%%%%%%%%%%%%%%
%%%%
%%%% This TeX file is part of the course
%%%% Introduction to Scientific Programming in C++/Fortran2003
%%%% copyright 2017 Victor Eijkhout eijkhout@tacc.utexas.edu
%%%%
%%%% iff.tex : conditionals in Fortran
%%%%
%%%%%%%%%%%%%%%%%%%%%%%%%%%%%%%%%%%%%%%%%%%%%%%%%%%%%%%%%%%%%%%%

\Level 0 {Boolean variables}

The fortran type for booleans is \indextermfort{Logical}.

The two literals are \n{.true.} and \n{.false.}

\Level 1 {Operators and such}

Operators: \n{.and.} \n{.or.} \n{.not.}

Equivalence between logical expressions:
\begin{itemize}
\item \n{.eqv.} : $ (x\wedge y)\vee (\neg x\wedge \neg y)$, with
  negation
\item \n{.neqv.} : $ (x\wedge \neg y)\vee (\neg x\wedge  y)$.
\end{itemize}


\Level 0 {Conditionals}
\label{sec:iff}

The Fortran conditional statement uses the \indextermfort{if} keyword:

\begin{block}{Conditionals}
  \label{sl:fconditional}
Single line conditional:
\begin{verbatim}
if ( test ) statement
\end{verbatim}
The full if-statement is:
\begin{verbatim}
if ( something ) then
  do something
else
  do otherwise
end if
\end{verbatim}
The `else' part is optional; you can nest conditionals.
\end{block}

You can label conditionals, which is good for readability but adds no functionality:
\begin{verbatim}
checkx: if ( ... some test on x ... ) then
checky:   if ( ... some test on y ... ) then
               ... code ...
          end if checky
        else checkx
             ... code ...
        end if checkx   
\end{verbatim}

\Level 1 {Operators}

\begin{block}{Comparison and logical operators}
  \label{sl:foperators}
  \begin{tabular}{|l|l|l|}
    \hline
    Operator&meaning&example\\ \hline
    \texttt{==}&equals&\texttt{x==y-1}\\
    \texttt{/=}&not equals&\texttt{x*x*!=5}\\
    \texttt{>}&greater&\texttt{y>x-1}\\
    \texttt{>=}&greater or equal&\texttt{sqrt(y)>=7}\\
    \texttt{<},\texttt{<=}&less, less equal&\texttt{}\\
    \n{.and.} \n{.or.}&and, or&\n{x<1 .and. x>0}\\
    \texttt{!}&not&\n{.not.( x>1 .and. x<2 )}\\
    \hline
  \end{tabular}
\end{block}

The logical operators such as \n{.AND.} are not short-cut as
in~C++. Clauses can be evaluated in any order.

\Level 0 {Select statement}

The Fortran equivalent of the C++ \n{case} statement is \indextermfort{select}. It takes
single values or ranges; works for integers and characters.

\begin{block}{Select statement}
  \label{sl:fswitch}
  Test single values or ranges, integers or characters:
  %
  \verbatimsnippet{casef}
  %
  Compiler does checking on overlapping cases!
\end{block}

\Level 0 {Review questions}

\begin{exercise}
  \label{ex:select-vs-switch}
  What is a conceptual difference between the C++ \n{switch} and the
  Fortran \n{Select} statement?
\end{exercise}
