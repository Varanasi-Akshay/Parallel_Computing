% -*- latex -*-
%%%%%%%%%%%%%%%%%%%%%%%%%%%%%%%%%%%%%%%%%%%%%%%%%%%%%%%%%%%%%%%%
%%%%
%%%% This TeX file is part of the course
%%%% Introduction to Scientific Programming in C++/Fortran2003
%%%% copyright 2017/8 Victor Eijkhout eijkhout@tacc.utexas.edu
%%%%
%%%% ccpp.tex : differences between C++ and C
%%%%
%%%%%%%%%%%%%%%%%%%%%%%%%%%%%%%%%%%%%%%%%%%%%%%%%%%%%%%%%%%%%%%%

C++ is, to a large extent, a superset of~C. But that doesn't mean that
you should look at C++ as `C~with some extra mechanisms'. C++~has a
number of new mechanisms that offer the same functionality as older
C~mechanisms, and \textbf{that should replace them}.

An entertaining talk that makes this point:
\url{https://www.youtube.com/watch?v=YnWhqhNdYyk}.

Here are some topics.

\Level 0 {I/O}

There is little employ for \indextermtt{printf} and
\indextermtt{scanf}. Use \indextermtt{cout} (and~\indextermtt{cerr})
and \indextermtt{cin} instead.

Chapter~\ref{ch:io}

\Level 0 {Arrays}

Arrays through square bracket notation are unsafe. They are basically
a pointer, which means they carry no information beyond the memory location.

It is much better
to use \indextermtt{vector}. Use range-based loops, even if you use
bracket notation.

Chapter~\ref{ch:array}

\Level 0 {Strings}

No more hassling with \indexterm{null
  terminator}s. A~\indextermtt{string} is an object with operations
defined on it.

Chapter~\ref{ch:string}.

\Level 0 {Pointers}

There is very little need for pointers.
\begin{itemize}
\item Strings are done through \n{std::string}, not character arrays;
  see above.
\item Arrays can largely be done through \n{std::vector}, rather than
  \indextermtt{malloc}; see above.
\item Traversing arrays and vectors can be done with ranges;
  section~\ref{sec:arrayrange}.
\item To pass an argument
  \emph{by reference}\index{parameter!passing!by reference},
  use a \emph{reference}\index{reference!argument}.
  Section~\ref{sec:passing}.
\item Anything that obeys a scope should be created through a
  \indexterm{constructor}, rather than using \indextermtt{malloc}.
\end{itemize}

Legitimate needs:
\begin{itemize}
\item Objects on the heap. Use \indextermtt{shared_ptr} or
  \indextermtt{unique_ptr}; section~\ref{sec:shared_ptr}.
\item Use \indextermtt{nullptr} as a signal.
\end{itemize}

\Level 1 {Parameter passing}

No longer by address: now true references! Section~\ref{sec:passing}.

\Level 0 {Objects}

Objects are structures with functions attached to
them. Chapter~\ref{ch:object}.

\Level 1 {Namespaces}

No longer name conflicts from loading two packages: each can have its
own namespace. Chapter~\ref{ch:namespace}.

\Level 1 {Templates}

If you find yourself writing the same function for a number of types,
you'll love templates. Chapter~\ref{ch:template}.

\Level 0 {Obscure stuff}

\Level 1 {Const}

Functions and arguments can be declared const. This helps the
compiler. Section~\ref{sec:constparam}.

\Level 1 {Lvalue and rvalue}

Section~\ref{sec:lrvalue}

\endinput

You’ll have no destructors, so cleanup is manual. This is most fun
with early-return functions, but it can keep you entertained for all
cases. File handles, memory, and other resources (thread locks,
anyone) are all waiting patiently and silently for you to forget them.

Initialization has be be explicitly called. No constructors either.

Want inheritance? Sure. Write your own vtable (often done with function pointers in a struct).
Instead of templates, you’ll need to abandon type safety and cast back and forth to (void*). Don’t explicitly cast to (void *), because the compiler never warns about explicit or implicit casts to and from (void *).

You’ll also need to make sure you’re using the right library calls - snprintf versus sprintf etc. Hopefully an existing project will be using the right ones.

On the plus side, you’re moving to Linux, and a lot of the tooling available is - while very command-line oriented - very good indeed.

For an IDE, I’d recommend CLion from JetBrains, but I’m told that with sufficient patience, Atom can be encouraged into doing useful stuff.

You’ll find that while the command-line of GDB, the debugger, isn’t very easy to learn to begin with, it’s very powerful, allowing you to do conditional breakpoints with comparative ease.

Valgrind is amazing. Voodoo. It’ll find uninitialized memory, allocation errors, overflows, and leaks - all common and hard to debug issues in C.

The CLang static analyzer is pretty impressive, too.

(copied from
\url{https://www.quora.com/How-should-a-C++-programmer-learn-Linux-C/answer/Dave-Cridland})
