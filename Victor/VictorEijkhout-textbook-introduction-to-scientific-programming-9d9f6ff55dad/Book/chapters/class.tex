% -*- latex -*-
%%%%%%%%%%%%%%%%%%%%%%%%%%%%%%%%%%%%%%%%%%%%%%%%%%%%%%%%%%%%%%%%
%%%%
%%%% This TeX file is part of the course
%%%% Introduction to Scientific Programming in C++/Fortran2003
%%%% copyright 2017 Victor Eijkhout eijkhout@tacc.utexas.edu
%%%%
%%%% scope.tex : scope issues, functions, classes
%%%%
%%%%%%%%%%%%%%%%%%%%%%%%%%%%%%%%%%%%%%%%%%%%%%%%%%%%%%%%%%%%%%%%

\Level 0 {Classes and objects}
\label{sec:class}

Everything you have learned up till now was (more or less) part of the
C~language, the predecessor of~C++. (The one clear exception was
\n{cin}/\n{cout}.) You will now learn one of the major distinguishing
features that makes C++ an `object oriented language':
\emph{classes}\index{class|textbf} and
\emph{objects}\index{object}.

A program naturally contains objects: the combination of data
and functions operating on that data.
%
\verbatimsnippet{intclass}

This looks a little like a structure, but a structure only had data
elements, and no functions. Also note that unlike declaring a
structure, a class definition  does not itself make any data: it
defines the structure of objects of that class. The line in the main
program is what creates an object with the structure as defined in the
class definition.

You can have more than one version of a function. This is known as
\indexterm{polymorphism}.
%
\verbatimsnippet{intclasspoly}

You can even redefine arithmetic operators such as~\n{+*/\%}.

\Level 0 {Exercises}

\begin{exercise}
  Write a fraction class which stores non-integer numbers as a
  numerator/denominator pair. Implement addition and multiplication
  functions. Make sure to simplify fractions!

  Can you print fractions so that $5/3$ is displayed as~\hbox{\n{1 2/3}}?
\end{exercise}
