% -*- latex -*-
%%%%%%%%%%%%%%%%%%%%%%%%%%%%%%%%%%%%%%%%%%%%%%%%%%%%%%%%%%%%%%%%
%%%%
%%%% This TeX file is part of the course
%%%% Introduction to Scientific Programming in C++/Fortran2003
%%%% copyright 2017 Victor Eijkhout eijkhout@tacc.utexas.edu
%%%%
%%%% google.tex : exercises for pagerank
%%%%
%%%%%%%%%%%%%%%%%%%%%%%%%%%%%%%%%%%%%%%%%%%%%%%%%%%%%%%%%%%%%%%%

\Level 0 {Basic ideas}

Assuming you have learned about arrays~\ref{ch:array}, in particular
the use of \n{std::vector}.

The web can be represented as a matrix~$W$ of size~$N$, the number of web
pages, where $w_{ij}=1$ if page~$i$ has a link to page~$j$ and zero
otherwise. However, this representation is only conceptual; if you
actually stored this matrix it would be gigantic and largely full of
zeros. Therefore we use a \indextermsub{sparse}{matrix}: we store only
the pairs $(i,j)$ for which $w_{ij}\not=0$. (In this case we can get
away with storing only the indices; in a general sparse matrix you
also need to store the actual~$w_{ij}$ value.)

\begin{exercise}
  Store the sparse matrix representing the web as a
\begin{verbatim}
vector< vector<bool> > web;
\end{verbatim}
  structure.
  \begin{enumerate}
  \item At first, assume that the number of web pages is given and reserve the outer
    vector. Read in values for nonzero indices and add those to the
    matrix structure.
  \item Then, assume that the number of pages is not pre-determined. Read in
    indices; now you need to create rows as they are needed. Suppose
    the requested indices are
\begin{verbatim}
5,1
3,5
1,3
\end{verbatim}
    Since your structure has only three rows, you also need to remeber
    their row numbers.
  \end{enumerate}
\end{exercise}

Now we want to model the behaviour of a person clicking on links.

\begin{exercise}
  Track the probability that a user is on a certain page with a
  \indextermsub{probability}{vector}:
\begin{verbatim}
vector<float> state;
\end{verbatim}
  The sum of the entries needs to be~$1$. While the algorithms will
  mathematically guarantee this, it may be a good idea to test for it
  every once in a while.

  Now model the user clicking on a link by computing a new probability
  vector:
  \begin{itemize}
  \item if for some page~$p$ \n{state[p]} is positive,
  \item then for all links~$q$, where \n{web[p][q]} is true,
  \item the new state gets a contribution of \n{state[p]} divided by
    the number of $q$-links.
  \end{itemize}
  Do you recognize a linear algebra algorithm?
\end{exercise}

Together this gives an approximation of Google's \indexterm{PageRank}
algorithm.
