% -*- latex -*-
%%%%%%%%%%%%%%%%%%%%%%%%%%%%%%%%%%%%%%%%%%%%%%%%%%%%%%%%%%%%%%%%
%%%%
%%%% This TeX file is part of the course
%%%% Introduction to Scientific Programming in C++/Fortran2003
%%%% copyright 2017 Victor Eijkhout eijkhout@tacc.utexas.edu
%%%%
%%%% structf.tex : types Fortran
%%%%
%%%%%%%%%%%%%%%%%%%%%%%%%%%%%%%%%%%%%%%%%%%%%%%%%%%%%%%%%%%%%%%%

\begin{block}{Structures: \noexpand\texttt{type}}
  \label{sl:ftype}
  The Fortran name for structures is \indextermtt{type} or
  \indextermsub{derived}{type}.
\end{block}

\begin{block}{Type definition}
  \label{sl:ftype-def}
  \n{Type name} / \n{End Type} block.
  Variable declarations inside the block
\begin{verbatim}
type mytype
  integer :: number
  character :: name
  real(4) :: value
end type mytype
\end{verbatim}
\end{block}

\begin{block}{Creating a type structure}
  \label{sl:ftype-set}
  Declare a typed object in the main program:
\begin{verbatim}
Type(mytype) :: typed_object,object2
\end{verbatim}
 Initialize with type name:
\begin{verbatim}
typed_object = mytype( 1, 'my_name', 3.7 )
object2 = typed_object
\end{verbatim}
\end{block}

\begin{block}{Member access}
  \label{sl:ftype-access}
  Access structure members with \verb+%+
\begin{verbatim}
Type(mytype) :: typed_object
type_object%member = ....  
\end{verbatim}
\end{block}

\begin{block}{Example}
  \label{sl:ftype-ex}
\verbatimsnippet{ftype}
\end{block}

\begin{verbatim}
type(my_struct) :: data
type(my_struct),dimension(1) :: data_array
\end{verbatim}

\begin{block}{Types as subprogram argument}
  \label{sl:ftype-pass}
  \verbatimsnippet{ftypepass}  
\end{block}
